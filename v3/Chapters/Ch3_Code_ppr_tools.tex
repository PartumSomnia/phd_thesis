% Chapter Template

\chapter{Evolution setup \& post-processing tools} % Main chapter title

\label{Chapter3} % Change X to a consecutive number; for referencing this chapter elsewhere, use \ref{ChapterX}


%% \section{Used codes and setup}
%% \section{Modeling binary neutron star merger with \texttt{WhiskyTHC}}

\section{Initial Data}

For Each EOS considered, we compute the irrotational BNS configurations in quasi-circular orbit.
We employ the pseudo-spectral code \texttt{Lorene} \citep{Gourgoulhon:2000nn}, that 
solves the general relativistic initial data problem.
The initial separation (of the qusi-circular orbit) is chosen $\sim40$~km and that corresponds to $~2-3$ orbits before merger.

The EOS table used for the initial data computation is the minimum temperature slice
$(T\sim 0.5 - 0.1)$~MeV of the finite temperature EOS table used for the evolution.
The assumption of the neutrino-less beta-equilibrium is made.
At constant temperature, at lowest densities, the photon energy (radiation) is a dominant contribution to 
pressure. Thus, we substruct this contribution from the tables.
%Addiitionally, the assuming constant temperature, we also remove the photon (radiation) energy contribution to the pressure (which dominates at the lowest densities)

The EOS table for the minimum temperature slice of the EOS table used for the evolution assuming neutrino-less beta-equilibrium.
Assuming constant temperature, we also remove the photon energy contribution to the pressure.

%In the evolution code, passing the initial data, the mapping is done from the zero temerature
In the evolution code, the electron fraction is set by the beta equilibrium condition. 
The specific internal energy is reset in accordance with minimum temperature slice of the EOS table used for evolution.

Errors present in the initial data in introduced during the mapping result in a small oscillations of netron stars.
In terms of relative changes in central density these amounts to $\sim2-3\%$ \cite{Radice:2018pdn}



\section{Evolution with \texttt{WhiskyTHC}}



In this part we review the numerical relativity code that was used to model binary neutron star mergers analyzed in this thesis.
The part is based on the David Radice PhD thesis where most of the methods are discussed \cite{Radice:2013apa}, and on the published results and regarding code structure and \cite{Radice:2012cu,Radice:2013xpa,Radice:2013hxh,Radice:2015nva}.

In order to perform numerical simulations of fluid flow, accurate numerical codes are essencial. Codes that flux-conservative finite-difference HRSC schemes offer a ciertain degree of simplicity, while high-order finite volume schemes are more computationally expensive (as they require solution of multiple Riemann problems at the interface between regions) \cite{Reisswig:2009us,Shu:2001rep} as well as complex averaging and de-averaging procedures \cite{Tchekhovskoy:2007zn}.

The \texttt{THC} code is the Templated-Hydrodynamics Code developed using the \texttt{Cactus} framework \cite{Goodale:2003}. In \texttt{THC}, the state-of-the-art flux-vector splitting scheme are employed. The "templated" in the code name stands for a modern paradigm in C++ programming, the templated programming, where a part of the code can be generated from the prescribed templates at compiling time (\textit{e.g.,} \cite{Yang:2001})

The \texttt{THC} has several primitive cariable reconstruction schemes implemented, such as MP5, classical monotonicity preserving \cite{Suresh:1997,Mignone:2010} the weighted essentially non oscillatory (WENO) schemes WENO5 and WENO7 \cite{Liu:1994,Jiang:1996,Shu:1997} and two bandwidth-optimized WENO schemes WENO3B and WENO4B \cite{Martin:2006,Taylor:2007}, contracted for modeling the compressible turbulence. Note, that the number in scheme name stands for a formal order of accuracy.

\texttt{WhiskyTHC} is a result of combination of two \texttt{Whisky} \cite{Baiotti:2004wn} and \texttt{THC} \cite{Radice:2012cu}. High-order flux-vector splitting finite-differencing techniques came from the former, while the module for the recovery of the primitive quantities as well as the equation of state framework from the latter \cite{Galeazzi:2013mia}. Tabulated temperature and composition dependent equation of states can be used.

Overall, \texttt{WhiskyTHC} solves the equations of general-relativistic hydrodynamics in conservation form \ref{eq:theory:grhdeq_thc} using a finite difference scheme. \red{tripple check the FD is used for space time evol and central FV method is for hydro}
The flux reconstruction is done in local-characteristic variables using the MP5 scheme, see \textit{e.g.,} \cite{Rezzolla:2013} \red{CHECK}.
The space-time is evolved using the CCZ4 formulation \ref{eq:theory:ccz4equations}, solved via finite difference code publicly available through \texttt{Einstein Toolkit}, \cite{McLachlan,Loffler:2011ay}.\red{CHECK}.
There, the central stencil is used throughout, and only terms associated with the advection along the shift vector are treated using the upwinded by one grid point stencil. The accuracy of the scheme is available at 6th and 8th order, while 4th is commonly employed. In addition, the fifth order Kreiss-Oliger style artificial dissipation \cite{Kreiss:1973} is added to aid with non-linear stability.

The code is build on the \texttt{Carpet} AMR driver \cite{Schnetter:2003rb} from the \texttt{Cactus} computational toolkit \cite{Goodale:2003}, incorporating a provided by \texttt{Carpet} Berger-Oliger-style mesh refinement \cite{Berger:1989,Berger:1984} with subcycling in time and refluxing. \textcolor{red}{in Thesis it is said, -- no refluxing was done yet}

To treat the vacuum region, the code utilizes common, 'atmosphere' approach. The atmosphere is referred to an artificial density floor in the simulation domain. It is introduced in order to tackle the challenges arising when considering boundary between the fluid and vacuum in Eulerian (relativistic) hydrodynamics codes \cite{Galeazzi:mThesis:2008,Kastaun:2006,Millmore:2009dk}. 
The defining property of the atmosphere is that the rest mass density and coordinate velocity are reset to a floor values once the former falls below a certain threshold value during the evolution \cite{Font:2001ew,Baiotti:2004wn}. While showing a reasonable results in second order codes, in higher order ones the numerical oscillations lead to the creation of vacuum nonetheless, that in light of the aforemention atmosphere effect result in the mass and energy violation \cite{Radice:2011qr}. For codes that rely on characteristic variables, the degeneracy in low-density, low-temperature limits also plagues the computation. This problem is the main reason behind the popularity of robust shock capturing codes, even though they are of first order in the general-relativistic hydrodynamics codes. Vacuum treatment for higher order codes is of main challenges to overcome.

There are several approaches in how to treat atmosphere. 

\textit{Standard atmosphere Treatment} or \textit{"ordinary MP5 approach"} is based on setting density that falls below $(1+\epsilon)\rho_{\text{atmo}}$ to the atmosphere density, velocity to zero and internal energy to the one prescubed by the polytropic EoS. The $\rho_{\text{atmo}}$ is usually related to a certain characteristic density, \textit{e.g.,} maximum density at the beginning of the simulation as $\rho_{\text{atmo}} = 10^{-7,-9}\rho_{\text{max}}$. The tolerance parameter $\epsilon$ is usually set to $10^{-2}$ and accounts for excessive oscillations of the fluid–vacuum interface. 

\textit{An Improved Atmosphere Treatment} or \textit{"MP5+LF"} In this approach the component-wise Lax-Friedrichs flux split is turned on when a certain density is reached. This increases the dissipation of the scheme and allows to avoid problems arising in characteristic reconstruction, associated with the degeneracy of the characteristic variables close to vacuum. Unfortunately, if the ejection of low velocity and density matter is concerned, this approach may yield oscillatory solutions and thus creates artifacts. 

\textit{Positivity Preserving Limiter} is an approach discussed in \cite{Radice:2013apa} based on the use of PPL proposed in \cite{Hu:2013}. \gray{Here we provide a brief overview}. \red{No, we not}.
\gray{While for a simplified case of classical gas dynamics it might require a lower timestep, in the general relativistic case and general tabulated EOS, the positivist of pressure is difficult to assure due to complexity of the energy source terms. It can be mitigated by enforcing a floor value on the pressure}. 
Note, that adopting a positivity preserving limiter to treat the transition between matter and vacuum, still implies replacing the vacuum with low density fluid at rest, is not a physically accurate approach. That would rely on treating the transition as a free boundary (see \textit{e.g.,} \cite{Kastaun:2006}) The advantage of positivity preserving limiter with respect to a classical atmosphere treatment, is that it allows to have a value of $\rho_{\text{atmo}}$ that does not require further tuning and can be arbitrary small, and assure that the solution is locally conserved. 
\textcolor{red}{In our models} we employ this approach as follows, at the beginning of the simulations we set the floor density, relying in the subsequent evolution on a positivity preserving limiters to ensure the atmosphere well behavior. Due to negligible density of the atmosphere its accretion has a negligible effect on the evolved object. 

For the extensive tests of different reconstruction methods and atmosphere treatment we refer to the \cite{Radice:2013apa}.


%% \cite{Radice:2012cu,Radice:2013xpa,Radice:2013hxh,Radice:2015nva}


\subsection{Hydrodynamics}



\red{THE IDEA IS THAT YOU PUT MAIN EQ AND THEORY INTO THEORY AND HERE JUST USE THE 'PAPER' STUFF}

The code evolves the proton and neutron number densities, $n_n$ and $n_p$
respectively, as 

\begin{equation}
\label{eq:wthc:pndens}
\nabla_\nu (n_p u^\mu) = R_p^\mu \ \ , \ \ 
\nabla_\nu (n_n u^\mu) = R_n^\mu \ .
\end{equation}

\gray{in Radice2016dwd it is $\nabla_{\alpha}(n_e u^{\alpha}) = R$}

\gray{in Galezzi2013, for Whisky, the equations are separate for baryon and leptons: $\nabla_{\alpha}(n_bu^{\alpha})=0$ and $\nabla_{\alpha}(n_eu^{\alpha})=N$, where the $n_b$ and $n_e$ are the baryon and electron number densities respectively.}

Here $u^{\mu}$ is the fluid four-velocity, $R_p = -R_n$ is the net
lepton number deposition rate due to the absorption and emission of neutrinos 
and antineutrinos (\red{see Section XXX})

The $R_{p,n}$ is computed according to the neutrino M0 scheme \cite{Radice:2016dwd,Radice:2018pdn}

The number densities are related as $n_p=Y_e n$ where $n = n_p + n_e$ is the baryon 
number density and $Y_e$ is electron fraction.

The matter of a neutron star is approximated with ideal fluid with stress-energy tensor

\begin{equation}
T_{\mu\nu} = \rho h u_{\mu} u_{\nu} + Pg_{\mu\nu}
\end{equation}

where $\rho=m_{\text{b}} n$ is the baryon rest-mass density, 
$n$ the baryon number density, $m_{\text{b}} \simeq 10^{-24}\,$g 
the neutron mass, 
\gray{if Galezzi13 it is nucleon mass which is actrually related to the EOS.}
$h=1+\epsilon + P/\rho$ the specific enthalpy, 
$\epsilon$ the specific internal energy (energy density),
and $P$ is \gray{total isotropic} pressure.

Written in a covariant form, the Euler equation for balance of energy and momentum reads

\begin{equation}
\label{eq:wthc:euler}
\nabla_\nu T^{\mu\nu} = Q u^{\mu} \ ,
\end{equation}

\gray{in Radice2016dwd it is $\nabla_{\beta}T^{\alpha\beta}=\Psi^{\alpha}$
    with $\Psi^{\alpha} = Q u^{\alpha}$.
}
\gray{In the Galezzi:2013 it is $\nabla_{\alpha}T^{\alpha\beta}=\Psi^{\beta}$.
    There the $T^{\alpha\beta}$ accounts for the ordinary matter and for trappend neutrinos and photons, but it does not include free-streaming neutrinos. Assumed to be similar to the 'test-fluid' they are neglected in constracting RHS of the Eistein equations.
}

where $Q$ is the net energy deposition rate doe to absorption
and emission of neutrinos also treated with the M0 scheme.
\red{JUST PUT REFS TO THE EQs THAT ARE IN THE 'nuetrino' AND 'viscosity' sections}


\subsection{Numerical methods}


High resolution shock capturing methods are used to discritize equations 
\eqref{eq:wthc:euler} and \eqref{eq:wthc:pndens}.
Specifically, central Kurganov-Tadmor type scheme \cite{Kurganov:2000} with 
HLLE flux formula \cite{Einfeldt:1988}
and non-oscillatory reconstruction of the primitive variables with the MP5 scheme of
\cite{Suresh:1997}.

Shock capturing schemes require the presence of a low density atmosphere around neutron stars.
The constant value of $\rho_0 = m_p n \approx 6\times 10^4$~\gcm.

The rest-mass consirvation in the presence of artificial atmosphere is assured via 
positivity-preserving limiter from \cite{Radice:2013xpa}

The local number densities of neutrons and protons separately, are assured via 
multi-fluid advection method of \cite{Plewa:1998nma}

The outflow properties are extracted when the density exceeds the atmosphere density
by several orders of magnitude.

%% Spacetime evolution
The spacetime is evolved using the Z4c formulation of Einstein's equations
\cite{Bernuzzi:2009ex,Hilditch:2012fp} as implemented in the \texttt{CTGamma} code
\cite{Pollney:2009yz,Reisswig:2013sqa} which is part of the \texttt{Einstein Toolkit} 
\cite{Loffler:2011ay}.

The non-linear stability of evolution is assured via Kreiss-Oliger dissipation. 
The spacial discritisation is done via fourth-order finite-differencing implemented in \texttt{CTGamma}.

The method of lines, MOL, couples the space-time evolution and hydrodynamics. 

Time integrator of choice is strongly-stability preserving third-order Runge-Kutta scheme \cite{Gottlieb:2009}.
The timestep is regulated by the Courant-Friedrichs-Lewy (CFL) condition, that required CFL factor 
to be $<0.25$ for numerical stability. To assure taht the positivity-preserving limiter implemented in \texttt{WhiskyTHC} maintains the density positive, the CFL factor is set to $0.15$.


\subsection{AMR}


The code uses the Berger-Oliger conservative adaptive mesh renement (AMR) \cite{Berger:1984} with 
sub-cycling in time and \red{refluxing (Davids thesis does not have refluxing)} \cite{Berger:1989,Reisswig:2012nc} as provided by the \texttt{Carpet module} of the \texttt{Einstein Toolkit} 
\cite{Schnetter:2003rb}. 


\subsection{Neutrino scheme}


\begin{table}
    \caption{
        Weak reactions employed in our simulations and references for their implementation.
        In the left column, $\nu \in \{\nu_e, \bar{\nu}_e, \nu_{x}\}$ denotes any neutrino species, 
        $\nu_{x}$ any heavy-lepton neutrinos, $N \in\{n, p\}$ a nucleon, and $A$ any nucleus.
        In the central column the role of each reaction is highlighted, with "P" standing for 
        production, "A" for absorption opacity and "S" for scattering opacity. When two roles are
        indicated, the second refers to the inverse ($\leftarrow$) reaction.
        Table is taken from \cite{Radice:2018pdn}.
    }
    \label{tab:leakage}
    \begin{center}
        \begin{tabular}{lll}
            \hline\hline
            Reaction & Role &  Ref. \\ 
            \hline
            $p + e^- \leftrightarrow \nu_e + n $          & P,A & \cite{Bruenn:1985}  \\
            $n + e^+ \leftrightarrow \bar{\nu}_{e} + p $  & P,A & \cite{Bruenn:1985}  \\
            $e^+ + e^- \rightarrow \nu + \bar{\nu}$       & P & \cite{Ruffert:1995fs} \\
            $\gamma + \gamma \rightarrow \nu + \bar{\nu}$ & P & \cite{Ruffert:1995fs} \\
            $N + N \rightarrow \nu + \bar{\nu} + N  + N$  & P & \cite{Burrows:2004vq} \\
            $\nu + N \rightarrow \nu + N$                 & S & \cite{Ruffert:1995fs} \\
            $\nu + A \rightarrow \nu + A$                 & S & \cite{Shapiro:1983du} \\
            \hline\hline
        \end{tabular}
    \end{center}
\end{table}

\red{JUST Name the chemes and reference the equations}
\red{JUST PUT REFS TO THE EQs THAT ARE IN THE 'nuetrino' AND 'viscosity' sections}


\section{General Setup}



The simulation domain is a cube of $3.024$~km each side, whose center is at the center of mass pf the binary.
The AMR structure has $7$ refinemnt levels, with the finest convering both compact objects during the inspiral and the remnant postmerger.

We consider several resolution setups. Low resolution (LR) simulations have $h=246$~m, standard resolution (SR) 
have $h=185$~m and high resolution (HR) $h=123$~m for the final refinemnt level.

In the simulations where the neutirno M0 scheme is included, it is switched on shortly before the merger. 
The equations \eqref{eq:method:whisky:eq7} and \eqref{eq:method:whisky:eq9} are solved on the uniform spherical grid
with radius $\approx 756$~km, and resolution $n_r\times n_{\theta}\times n_{\phi} = 3096 \times 32 \times 64$
grid points.

A subset of models discussed in this thesis include the effective treatment of viscosity. 

We consider \red{$33$} distinct binary with total masses $\red{[None,None]}$ and mass-ratio $q\in[1.00,1.82]$.
In all models the neutrino leackage plus M0 scheme. Most models were computed at at least two resolutions. 
Most our models also include the effect of subgrid turbulence, viscosity.

\gray{Summary of all results in given in the table...}

\gray{Each run is nameed as}

\gray{We simulate each model for at least $\red{None}$~ms after the merger or a few milliseconds after BH formation}



\section{Postprocessing tools and methods}


In order to investigate the neutron star merger dynamics and outflowing material we imploy the following methods and tools.

In order to study and compara on a quantiative level properties of outflow, disk and remnant we employ the mass-averaged quantities and for a quantity $f$ they are computed as 
\begin{equation}
\langle f \rangle = \frac{\sum_i f(m_i)m_i}{\sum_i m_i}
\end{equation}
where $m_i$ is the mass contained in the $i$-th bin.


\subsection{Disk \& Remnant}

It is common to discuss the post-merger state of the binary neutron star systems in terms of the remnant, a neutron star or a black hole, and a disk or torus. However there is not unified convention in how to define the latter and separate it from the former. 

In the case where the remnant is a BH, the disk common disk definition is the matter outside the apparent horizon, (\eg, \cite{Dietrich:2015iva,Dietrich:2016hky}). 
However, owing to the disk accretion onto a black hole, the extraction time is a crutual paramter, and unfortunately is not consistent in the literature, (\eg $\sim1$~ms in \cite{Dietrich:2015iva,Dietrich:2016hky} and $\sim30$~ms in \cite{Sekiguchi:2016bjd}).

In the case where the remnant is a neutron star, the disk definition usually includes the density cut. For instance, in \cite{Radice:2018pdn,Kiuchi:2019lls,Vincent:2019kor} the disk is assumed to encompass the matter with $\rho < 10^{13}$~\gcm. 
The threshold $\rho\sim 10^{13}$~\gcm corresponds to the point in the remnant where
the angular velocity profiles becomes approximately Keplerian, \citep[\eg][]{Shibata:2005ss,Shibata:2006nm,Hanauske:2016gia,Kastaun:2016elu}.
The extraction time here is also important, but less so, as we find that the accretion on the NS is considerably slower.

Overall, we estimate that these differences can amount to a systematic factor of a few,
which we employ for the statistical analysis in section \ref{sec:stat:anal}

% In this thesis we define the disk as a matter that satisfies two criteria $\alpha > 0.15$ and $\tho < 10^{13}\gcm$, where $\alpha$ is the lapse function (see section \ref{sec:theory:gr3p1}).

We compute the baryonic mass of the disks is computed as the volume integral of the conserved rest-mass density $D=\sqrt{\gamma}~W\rho$,

\begin{equation}
\label{eq:method:mdisk}
M_{\text{disk}} = \int D \dd^3 x
\end{equation}

from 3D snapshots of the simulations in postprocessing.



%% In \cite{Sekiguchi:2016bjd}, the disk mass is extracted at
%% ${\approx} 30$~ms outside the AH. In \cite{Radice:2018pdn}, the disk mass is computed
%% as the baryonic mass outside the AH at BH formation, while for NS
%% remnants the criterion $\rho < 10^{13}$ g cm$^{-3}$ is used. 
%% In \cite{Kiuchi:2019lls} for both BH and NS outcome the $\rho < 10^{13}$ g cm$^{-3}$ 
%% criterion is used and time of the extraction is not specified. 
%% In \cite{Vincent:2019kor} the density criterion is the same, however the simulations 
%% are significantly shorter (${~\sim 7.5}$~ms) than in other
%% works. Overall, we estimate that these differences can amount to a
%% systematic factor of a few.



\subsection{Density modes}

%% FROM THE LETTER 

The hydrodynamic instability is monitored by a decomposition in Fourier modes
$e^{-\i m\phi}$ of the Eulerian rest-mass density on the equatorial plane 
[see Eq.~(1) of \citep{Radice:2016gym}] and characterized by the
development of a $m=2$ followed by a $m=1$ mode 
\citep{East:2015vix,Paschalidis:2015mla,Radice:2016gym,Lehner:2016wjg,Bernuzzi:2013rza,Kastaun:2014fna}.
In the short-lived remnant (LS220) the $m=1$ mode
is subdominant with respect to the $m=2$, and it reaches a maximum close to the collapse
\citep{Bernuzzi:2013rza}. Instead, in the long-lived remnant (DD2) the $m=1$
becomes the dominant mode at $\sim$20~ms and persists throughout the
remnant's lifetime, while the $m=2$ efficiently dissipates via
gravitational-wave emission \citep{Bernuzzi:2015opx,Radice:2016gym}.

%% FROM THE PAPER 

During the post-merger evolution the neutron star oscillates. The most prominant modes are quasiradial mode $F$ ($m=0$), the $m=2$ $f$-model and non-linear combinations of them \citep[\eg][]{Shibata:2000jt,Stergioulas:2011gd}.

It has also been shown that the $m=1$, one-armed spiral instability, is present in the remnant of neutron star mergers \citep{Paschalidis:2015mla,Radice:2016gym,East:2016zvv}.

In order to investigate the dynamical instabilities in our simulations we 
project the rest-mass density onto spherical harmonics,
or, in other words, we perform the complex azimuthal mode decomposition of,
the conserved rest-mass density.
For simplicity we consider only $\rho(x,y,z=0,t)$, 
\ie restrict our analysis to the orbital plane $z=0$

\red{DOUBLE check the presence of Gamma and W}
%% \begin{equation}
%% \label{eq:modes}
%% C_m = \int \rho(x,y,z=0,t) W e^{-i m \phi} \sqrt{\gamma} %% \text{d}x \text{d} y \, ,
%% \end{equation}
\begin{equation}
\label{eq:modes}
C_m(t) = \int \rho(x,y,z=0,t) e^{-i m \phi(x,y)} \text{d}x \text{d} y \, ,
\end{equation}
(see \eg~\citet{Baiotti:2009gk}).

% where %% $\rho$ is the rest mass density, 
% $\gamma$ is the determinant of the three-metric and $W$ is the
% Lorentz factor between the fluid and the Eulerian observers. 

Note that the above quantities are gauge dependent.

\red{Dietrich in his thesis thinks that the growing m=1 mode is "We can not
    exclude the possibility that the growing m = 1 mode is triggered by numerical effects,
    but we think that it is a physical hydrodynamical effect due to mode couplings."
    Page 53 of the thesis
}

In addition, friequencies of the modes can be computed with the Fourier analysis of the $\rho_{\text{max}}$ and projections $C_{m}$. We resort it for the future work, as in this work we are interested inly in their magnitude. 





\subsection{Angular momentum}

\red{First, white that in GR the angular momentum is not clearly defined}

%% FROM LETTER 

From the fluid's stress energy tensor,
we compute the angular momentum density flux $J_r = T_{ra}(\partial_\phi)^a$,
where $\phi$ is the cylindrical angular coordinate;
angular momentum is conserved if $(\partial_\phi)^a$ is a Killing vector.

%% FROM PAPER 

The fluid's angular momentum analysis in the remnant and disk is performed
assuming axisymmetry (see Appendix~\ref{app:ang} for derivation).
That is, we assume $\phi^{\mu} = (\partial_{\phi})^{\mu}$ to be a Killing
vector. Accordingly, the conservation law
\begin{equation}
\partial_t(T^{\mu\nu}\phi_{\nu}n_{\nu}\sqrt{\gamma}) -
\partial_i(\alpha T^{i \nu}\phi_{\nu}\sqrt{\gamma}) = 0 \ ,
\end{equation}
where $n^\mu$ is the normal vector to the spacelike hypersurfaces of
the spacetime's $3+1$ decomposition, 
implies the conservation of the angular momentum
\begin{equation}
J = % \int j dV = 
%- \int \, T_{\mu\nu}n^{\mu}\phi^{\nu}\,\dd ^3x = 
-\int \,
T_{\mu\nu}n^{\mu}\phi^{\nu}\,\sqrt{\gamma}\, \dd^3 x\ .
\end{equation}
In the cylindrical coordinates $x^i=(r,\phi,z)$ adapted to the symmetry
the angular momentum density is  
\begin{equation}
j = %-
\rho h W^2 v_{\phi} \ ,
\label{eq:method:ang_mom}
\end{equation}
and the angular momentum flux is 
\begin{equation}
\alpha\sqrt{\gamma}T^r _{\nu}\phi^{\nu} =
\alpha\sqrt{\gamma}\rho h W^2 (v^{r}v_{\phi}) .
\end{equation}


\subsection{Ejecta}


To model and study the electromagnetic counterparts to mergers, the amount and properties of the material
leaving the system are needed.
The matter expelled at high velocity may ultimately become unbound from the central gravitational
potential. There are two indicators commonly adopted to mark the unbound matter.

\subsubsection{The Geodesic criterion}

Assuming that the spacetime is stationary, the $\partial_t$ is the killing vectory \red{confirm}, 
the four-velocity, $u_t$, (along the time-like killing vector), is a constant of motion for geodesics. 
Additionally, if the space is asymptotically flat, at infinity the $u_t = -W$, where $W$ is the fluid element Lorentz factor. Then, if a fluid element has $u_t < -1$, it may be considered unbound \red{as it will retain the non-zero positive velocity at infinity}. 
The fluid reaches an asymptotic velocity 
\begin{equation}
\upsilon_{\infty} \simeq \sqrt{2E_{\infty}} = \sqrt{(1-u_t ^2)}.
\end{equation}
The criterion can be through of as considering the fluid to be made of isolated particles that follow the geodesics. Indeed, the effects of equation of state, fluids pressure gradient, internal energy and heating (\eg, due to an $r$-process (\red{see section XXX})) are neglected. The space time is also assumed to be static.
Strictly speaking, none of these assumptions is fulfilled in the BNS post-merger environment. However, this criterion is widely used in the literature \citep[\eg][]{Radice:2018pdn,Vincent:2019kor}.
Note that the geodesic criterion above neglects the fluid's pressure and might underestimate the ejecta mass.

\subsubsection{The Bernoulli criterion}

From the relativistic Bernoulli equation \citep{Rezzolla:2013}, it follows
that for a stationary relativisitc flow, the $hu_t$ is constnat along the 
fluid worldliens. Here $h$ is the (relativistic) enthalpy, which is
defined up to a constant factor. 
If at the spatial infinity the enthalpy is set so $h\rightarrow-1$, 
\red{in Vincent it is $h\leftarrow 1$}
the condition $hu_t < -1$ would mark the unbound matter 
(as in the assymptotically flat space-time the $u_t = -W$ for the flow particles following geodesics).

The associated asymptotic velocity is calculated as 
\begin{equation}
\upsilon_{\infty} \simeq \sqrt{2 (h (E_{\infty}+1)-1)}. 
\end{equation}

The criterion can be regarded as assuming all the internal energy of the fluid 
gets added to the fluid kinetic energy, as the fluid decompresses \red{(pressure drops?)}.

The $r$-process nucleosynthesis that occurs in the outflow deposits the energy.
\red{In Vincent it is assumed that the difference in binding energy between the 
    particles in NSE at a given $\rho$, $T$ and $Y_e$, and their binding energy at
    the same $Y_e$ but low $\rho$ and $T$ is added/substracted from the fluid's kinetic energy.
    Out-of-NSE evolution and effects this neglected \citep[][see]{Foucart:2016vxd}
    [BUT I AM NOT SURE IF THIS IS THE CASE FOR OUR SIMULATIONS! MAYBE NOT!]}

This criterion has been found to estimate more accurately the amount of unbound material \citep{Foucart:2015gaa}
However, as was also found that the Bernoulli criterion leads to up to twice the amount of ejecta detected in comparion with the geodensic criterion, if the estimation is done within a given volume \citep{Kastaun:2014fna}.

We adopt the geodesic criterion to study the "burst-like", short outflows,
such as dynamical ejecta, where the pressure gradient is not expected to make a significant contribution.
For the steady-state outflows, like postmerger winds we adopt the Bernoulli criterion.

The term ejecta would refer to the material gravitationaly unbound according to eather of the criteria.




All considered mass ejecta are calculated on a coordinate sphere at $R \simeq 294$km. 
\red{untill afterglow}

