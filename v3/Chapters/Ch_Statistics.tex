% Chapter Template

\chapter{\ac{BNS} ejecta and disk mass statistics} % Main chapter title

\label{ch:stat} % Change X to a consecutive number; for referencing this chapter elsewhere, use \ref{ChapterX}

One of the important aspects of \ac{NR} simulations is to provide a link between the intrinsic binary parameters and properties of the electromagnetic signal.

In this chapter we investigate different relations between the \ac{BNS} parameters 
and the properties of the ejecta and the disk mass.
For that we employ the largest-to-date compiled dataset of \ac{NR} \ac{BNS} models 
available in the literature, some of which include the effects of the 
microphysical nuclear equations of state (EOS) and neutrino transport.

We report the statistical properties of the dataset and determine the 
best fitting formulae/ 


\section{Introduction}

The rapid growth of \ac{MM} astronomy facilitates the need in methods of quickly inferring the properties 
of the binary from the observations, or predicting the \ac{EM} signal from the \ac{GW} detection.

while \ac{NR} simulations are the most robust way to obtain the properties of the material
ejected in mergers, they are computationally long and very expensive. 

Simple fitting formulae, that map the binary proprties onto the ejecta properties is a sensible
compromise that, while allowing for the fast parameter estimation studies, provide a reasonable 
degree of accuracy for the first order studies 

\red{refertence many MM parpers and studies}
\red{Discuss the Tim's paper, the benefits, popularity and negatives of this study}
\red{Note how this can be improved with the increasing availability of the high resolution 
simulations with advavced physics}


\section{Method} 

