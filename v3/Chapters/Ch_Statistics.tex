% Chapter Template

\chapter{\ac{BNS} ejecta and disk mass statistics} % Main chapter title

\label{ch:stat} % Change X to a consecutive number; for referencing this chapter elsewhere, use \ref{ChapterX}

%% Our models
\newcommand{\DSrefset}{\texttt{M0RefSet}} 
%% large leackage dataset that we usually compare to
\newcommand{\DSheatcool}{\texttt{M0/M1Set}} 
%% all datasets with neutrino absorption
\newcommand{\DScool}{\texttt{LeakSet}}
%% all datasets with leaakge
\newcommand{\DSnone}{\texttt{NoNusSet}}

\def\chid{\chi_{\nu}^2}
\def\non{\nonumber}


One of the important aspects of \ac{NR} simulations is to provide a link between the intrinsic binary parameters and properties of the electromagnetic signal.

In this chapter we investigate different relations between the \ac{BNS} parameters 
and the properties of the ejecta and the disk mass.
For that we employ the largest-to-date compiled dataset of \ac{NR} \ac{BNS} models 
available in the literature, some of which include the effects of the 
microphysical nuclear equations of state (EOS) and neutrino transport.

We report the statistical properties of the dataset and determine the 
best fitting formulae/ 


\section{Introduction}

The rapid growth of \ac{MM} astronomy facilitates the need in methods of quickly inferring the properties 
of the binary from the observations, or predicting the \ac{EM} signal from the \ac{GW} detection.

while \ac{NR} simulations are the most robust way to obtain the properties of the material
ejected in mergers, they are computationally long and very expensive. 

Simple fitting formulae, that map the binary proprties onto the ejecta properties is a sensible
compromise that, while allowing for the fast parameter estimation studies, provide a reasonable 
degree of accuracy for the first order studies 

\red{refertence many MM parpers and studies}
\red{Discuss the Tim's paper, the benefits, popularity and negatives of this study}
\red{Note how this can be improved with the increasing availability of the high resolution 
simulations with advavced physics}

%% fitpaper intro
The ejection of mass during the \ac{BNS} mergers can be set off by a variety of mechanisms acting
on a different timescales 
(see
\citet{Metzger:2019zeh,Shibata:2019wef,Radice:2020ddv,Bernuzzi:2020tgt} for reviews on various aspects of the problem and the Ch.~\ref{ch:bns_sims}).
Most \ac{NR} simulations of mergers show the presence of the \ac{DE} with masses 
$\md\sim\O(10^{-4}-10^{-2})\,\Msun$ and average velocities $\avd\sim0.1-0.3\,$c, 
\citep[see \eg][]{Rosswog:1998hy,Rosswog:2005su,Hotokezaka:2013iia,Bauswein:2013yna,Wanajo:2014wha,Sekiguchi:2015dma,Radice:2016dwd,Sekiguchi:2016bjd,Vincent:2019kor}. See also Sec.\ref{sec:res_dyn_ej}.
Longer simulations also show the development of more massive, slower winds 
\citep[see \eg][]{Dessart:2008zd,Fernandez:2014bra,Perego:2014fma,Just:2014fka,Kasen:2014toa,Metzger:2014ila,Martin:2015hxa,Wu:2016pnw,Siegel:2017nub,Fujibayashi:2017puw,Fahlman:2018llv,Metzger:2018uni,Fernandez:2018kax,Miller:2019dpt}. See also Sec.~\ref{sec:res_sww}.
The ejecta properties are vital to model the \ac{EM} signal to mergers. 
The most accurate and direct way of linking the properties of the ejecta to the parameters of the binary 
is to consider the ab-initio 3+1 simulations in numerical relativity
\citep[\eg][]{Hotokezaka:2012ze,Hotokezaka:2013iia,Wanajo:2014wha,Sekiguchi:2015dma,Dietrich:2015iva,Palenzuela:2015dqa,Bernuzzi:2015opx,Radice:2016dwd,Lehner:2016lxy,Sekiguchi:2016bjd,Radice:2018pdn,Vincent:2019kor,Perego:2019adq,Kiuchi:2019lls,Endrizzi:2019trv,Bernuzzi:2020txg}.
However, the growing sample size of \ac{BNS} merger simulations and published ejecta properties 
allows us to asses the dependencies of ejecta and remnant properties on the binary parameters.
%% ---
Such studies of published \ac{NR} \ac{BNS} merger simulations have been done before, providing the 
fitting formulae to the ejecta and \pmerg{} remnants properties of various complexities 
\cite{Dietrich:2016fpt,Radice:2018pdn,Kruger:2020gig}. 
However, a continuous update and reassessment are reuqired as more, higher resolution simulations
with more accurate and complete physical treatment become available. 
%% ---
The motivation for these studies lie in their importance for (i) the parameter estimation studies,
infurring the binary properties from \ac{EM} observations,
\citep[\eg][]{Radice:2017lry,Perego:2017wtu,Coughlin:2018fis,Coughlin:2019zqi} 
(ii) predicting the properties of the \ac{EM} counterparts from the \ac{GW} observations 
\citep[\eg][]{Stachie:2021noh}, and (iii) consmic chemical evolution models 
\citep[\eg][]{Bonetti:2019fxj}, as they allow 
to predict how much heavy elements have been injected into \ac{ISM} from mergers.

In this chapter we consider the largest-to-date combined set of \ac{NR} \ac{BNS} models, that 
include our simulations discussed in Ch.~\ref{ch:bns_sims}. 
We perform statistical analysis of the ejecta and remnant properties and 
asses the systematic uncertainties introduced by different 
physics in simulations.
Additionally, we re-calibrate the fit models proposed in the literature, and propose  
simple polynomial fits to the data.



In this chapter we label the \acp{NS} of the binary with subscripts $A$, $B$.
The individual gravitational masses are indicated as $M_A$, $M_B$, 
the baryonic masses as $M_{b~A}$, $M_{b~B}$, 
the total mass as $M = M_A + M_B$, 
and the mass ratio $q=M_A/M_B\geq1$. 
\red{perhaps the Lambda definition has to be moved to the bns-sims chapter}
We define the quadrupolar tidal parameters as
$\Lambda_i \equiv 2/3\, C_i^{-5} k^{(2)}_i$
where $k_i^{(2)}$ is the dimensionless gravitoelectric Love number \cite{Damour:2009vw}, 
$C_i \equiv GM_A/(c^2R_A)$ the compactness parameter, and $i=A,B$.
The reduced tidal parameter \cite{Favata:2013rwa} is:
\begin{equation}
\tilde\Lambda = \frac{16}{13}\frac{(M_A+12 M_B)M_A^4 \Lambda_A}{M^5}+(A\leftrightarrow B)\,.
\label{eq:Lambda_tilde}
\end{equation}
We use CGS units except for masses and velocities, given in units of $\Msun$ and $c$, respectively.

%% =================================================================================
%%
%%                          D A T A
%%
%% =================================================================================

\section{Data} 

%% --- data available

\begin{table*}[t]
    \caption{
        Datasets with the dynamical ejecta data and disk masses
        employed in this work. The available data is shown in the columns
        starting from the fourth, that contain: gravitational mass of the binary, baryonic mass of
        the binary, reduced tidal parameter, ejecta mass, ejecta
        velocity, ejecta electron fraction, disk/torus mass. EOS are
        either microphysical or piecewise polytropic (PWP). Neutrino
        schemes are: leakage, leakage + M0 or M1 for free streaming
        neutrinos, or M1. 
        The compiled data are available online at \citep{vsevolod_nedora_2020_4283517}.
        (Adapted from \citet{Nedora:2020qtd})
    }
    \label{tab:data}
    \begin{tabular}{ccccccccccc}
        \hline\hline
        Ref.  & EOS  & Neutrinos & $M$  & $M_b$  & $\tilde{\Lambda}$ & $M_{\rm ej}$ & $\upsilon_{\rm ej}$ & $Y_e$  & $M_{\rm disk}$ & Dataset
        \\ \hline \hline
        \multicolumn{1}{c|}{\citep{Perego:2019adq}}     & \multicolumn{1}{c|}{Micro} & Leak+M0    & \cmark & \cmark & \cmark & \cmark & \cmark  & \cmark & \cmark &  \DSrefset{} \& \DSheatcool \\
        \multicolumn{1}{c|}{\citep{Nedora:2019jhl}}     & \multicolumn{1}{c|}{Micro} & Leak+M0    & \cmark & \cmark & \cmark  & \cmark  & \cmark & \cmark & \cmark  & \DSrefset{} \& \DSheatcool  \\
        \multicolumn{1}{c|}{\citep{Bernuzzi:2020txg}}   & \multicolumn{1}{c|}{Micro} & Leak+M0    & \cmark & \cmark & \cmark & \cmark & \cmark & \cmark & \cmark & \DSrefset{} \& \DSheatcool   \\
        \multicolumn{1}{c|}{\citep{Nedora:2020pak}}        & \multicolumn{1}{c|}{Micro} &   Leak+M0   & \cmark & \cmark & \cmark & \cmark & \cmark & \cmark & \cmark   & \DSrefset{} \& \DSheatcool   \\
        \hline
        \multicolumn{1}{c|}{\citep{Vincent:2019kor}}    & \multicolumn{1}{c|}{Micro}  &  M1  & \cmark & \cmark & \cmark & \cmark & \cmark  & \cmark & \xmark   &  \DSheatcool{} \\
        \multicolumn{1}{c|}{\citep{Sekiguchi:2015dma}}  & \multicolumn{1}{c|}{Micro} &  Leak+M1   & \cmark & \xmark & \xmark  & \cmark & \xmark & \cmark & \xmark &  \DSheatcool \\
        \multicolumn{1}{c|}{\citep{Sekiguchi:2016bjd}}  & \multicolumn{1}{c|}{Micro} &  Leak+M1   & \cmark & \xmark & \xmark & \cmark & \cmark & \cmark & \cmark & \DSheatcool \\
        \multicolumn{1}{c|}{\citep{Radice:2018pdn}~(M0)} & \multicolumn{1}{c|}{Micro} & Leak+M0    & \cmark & \cmark & \cmark & \cmark & \cmark & \cmark & \cmark  &  \DSheatcool   \\
        \hline
        \multicolumn{1}{c|}{\citep{Lehner:2016lxy}}     & \multicolumn{1}{c|}{Micro} & Leak    & \cmark & \cmark & \xmark & \cmark & \cmark & \xmark & \xmark   &  \DScool \\
        \multicolumn{1}{c|}{\citep{Radice:2018pdn}~(LK)} & \multicolumn{1}{c|}{Micro} & Leak    & \cmark & \cmark & \cmark & \cmark & \cmark & \cmark & \cmark  & \DScool   \\
        \hline
        \multicolumn{1}{c|}{\citep{Kiuchi:2019lls}}     & \multicolumn{1}{c|}{PWP}   &  -    & \cmark & \cmark & \cmark & \cmark & \xmark  & \xmark & \cmark    & \DSnone \\
        \multicolumn{1}{c|}{\citep{Dietrich:2016hky}}   & \multicolumn{1}{c|}{PWP}  &  -     & \cmark & \cmark & \cmark & \cmark & \cmark  & \xmark & \cmark  &  \DSnone \\
        \multicolumn{1}{c|}{\citep{Dietrich:2016hky}}   & \multicolumn{1}{c|}{PWP}  &   -    & \cmark & \cmark & \cmark & \cmark & \cmark & \xmark & \cmark  & \DSnone \\
        \multicolumn{1}{c|}{\citep{Hotokezaka:2012ze}}  & \multicolumn{1}{c|}{PWP}  &    -   & \cmark & \xmark & \xmark & \cmark & \cmark & \xmark & \xmark  &  \DSnone \\
        \multicolumn{1}{c|}{\citep{Bauswein:2013yna}}   & \multicolumn{1}{c|}{Micro}&  - & \cmark & \xmark & \xmark & \cmark & \cmark & \xmark & \xmark  &  \DSnone \\
        \hline\hline
    \end{tabular}
\end{table*}


\begin{figure*}[t]
    \centering 
    \includegraphics[width=0.32\textwidth]{statistics/ej_mej_vej_groups.pdf}
    \includegraphics[width=0.32\textwidth]{statistics/ej_mej_yeej_groups.pdf}
    \includegraphics[width=0.32\textwidth]{statistics/ej_vej_yeej_groups.pdf}
    \caption{Summary of dynamical ejecta properties used in this work.
        Blue circles represent models of \DSrefset{}, 
        red diamonds stands for models from \DSheatcool{}, 
        green crosses are models from \DScool{}
        and gray squares stand for models from \DSnone{}, 
        %% 
        We show for comparison the two-component fit to AT2017gfo as
        colored patches from \cite{Villar:2017wcc,Siegel:2019mlp}.
        (Adapted from \citet{Nedora:2020qtd})
    }
    \label{fig:ejecta:dyn:ds}
\end{figure*}

The datasets used in this paper are summarized in Tab.~\ref{tab:data}.
We group them with respect to the employed neutrino treatment:

\begin{itemize}
    %% ---
    \item \DSheatcool{} comprises a set of models with neutrino emission 
    and absorption and microphysical \ac{EOS}. It includes 
    $8$ models with leakage+M0 of \cite{Radice:2018pdn} and models 
    of \cite{Sekiguchi:2015dma,Sekiguchi:2016bjd,Vincent:2019kor}
    in which a leakage+M1 scheme or a M1 gray scheme are employed for the neutrino transport. 
    Models reported in these works span 
    $q\in[1, 1.30]$, 
    $\tilde{\Lambda}\in[340, 1437]$, 
    $M_{\rm tot}\in[2.52,2.88]$, 
    and $M_{\rm chirp}\in[1.10,1.25]$.
    %% ---
    \item \DSrefset{} harbors models with the same physical setup as 
    those models with leakage+M0 of \cite{Radice:2018pdn}, that are 
    part of the \DSheatcool{}. However, these models, presented in 
    \cite{Perego:2019adq,Nedora:2019jhl,Bernuzzi:2020txg,Nedora:2020pak}, 
    and disucssed in Ch.~\ref{ch:bns_sims}, are uniform in turns of the 
    numerical setup, code and physics and have fixed chirp mass. 
    For that reason we group them into a separate, reference, dataset. 
    The models of this set span $q\in[1, 1.82]$, 
    $\tilde{\Lambda}\in[400, 850]$, 
    $M_{\rm tot}\in[2.73,2.88]$ with 
    the chirp mass $M_{\rm chirp}=1.19$.
    %% ---
    \item \DScool{} comprises models with leakage scheme as neutrino treatment and microphysical \ac{EOS}.
    The dataset includes a subset of models from \cite{Radice:2018pdn} ($35$ runs denoted as LK),
    and the set of models from \cite{Lehner:2016lxy}.
    The models in this dataset span $q\in[1, 1.31]$, 
    $\tilde{\Lambda}\in[116, 1688]$, 
    $M_{\rm tot}\in[2.40,3.42]$, 
    and $M_{\rm chirp}\in[1.04,1.49]$.
    %% ---
    \item \DSnone{} is composed of models with piecewise-polytropic \acp{EOS} 
    \cite{Hotokezaka:2012ze,Dietrich:2015iva,Dietrich:2016hky,Kiuchi:2019lls,Bauswein:2013yna},
    in which temperature effects are approximated by a
    gamma-law pressure contribution, while
    composition and weak effects are neglected.
    The models in this dataset span 
    $q\in[1, 2.06]$, 
    $\tilde{\Lambda}\in[50, 3196]$, 
    $M_{\rm tot}\in[2.4,4.0]$, 
    and $M_{\rm chirp}\in[1.04,1.74]$.
    %% 
\end{itemize}

%% --- Overal dataseamble
In total $324$ models are available.
For each of which we compute the reduced tidal parameter, $\tilde{\Lambda}$, 
solving the \ac{TOV} equations for the corresponding gravitational masses of the stars 
and \ac{EOS}. 
For a subset of models with polytropic \acp{EOS} of \citet{Bauswein:2013jpa}
and \citet{Kiuchi:2019lls}, however, the \ac{EOS} data are not available. and, the $\tilde{\Lambda}$ cannot be estimated. We exclude these models from the statistical analysis.
Overall, out of $324$, we consider $271$ models for which the required binary data is available/computed. For all $271$ of them the ejecta mass, $\amd$ is present. The average velocity $\avd$, is avaialble for only $246$ models, as a subset of models from \citet{Kiuchi:2019lls} does not contain this information. The electron fraction is found for $99$ models, as we exclude the subset of models with leakage scheme for which this data is not given in \citet{Lehner:2016lxy}. The \ac{RMS} half-opening angle around the orbital plane, $\athetarms$, of the ejecta is present for $76$ models and the mass of the disk, $M_{\rm disk}$, is given for $119$ models.

%% --- Uncertanty
The data from different datasets, performed with different numerical codes, at different resolution,
with idfferent physics input and extracted with different methods is subjected to many uncertanties.
For this reason and as the uncertainties introduced by the finite grid effects are not available for 
all datasets, we employ the following assumtions.

For the \ac{DE} mass we consider an uncertainty given by \citep{Radice:2018pdn}:
\begin{equation}
\Delta M_{\text{ej}} = 0.5M_{\text{ej}} + 5\times10^{-5}M_{\odot}.
\label{eq:ejecta:mej_err}
\end{equation}

For the ejecta velocity and for the electron fraction we consider 
$\Delta \upsilon_{\text{ej}} = 0.02$~c 
and $ \Delta Y_e = 0.01$ as fiducial uncertainties, respectively.
The latter value is justified by the robust behavior of the average electron 
fraction in simulations where multiple resolutions are available
Notably, it is possible the uncertainties are larger due to the approximate nature of current 
neutrino treatments (see \eg, \citep{Foucart:2016rxm,Foucart:2018gis}. 
%% However, due to the lack of extensive comparison studies, 
%% we consider only the numerical resolution error.
We leave the more accurate investigation to the future works, when mode simulations
with advanced neutrino treatment, such as M1 and \ac{MC} schemes become available. 

For the disk mass we assume \citep{Radice:2018pdn}
\begin{equation}
\Delta M_{\text{disk}} = 0.5M_{\rm disk} + (5\times
10^{-4})M_{\odot}\ .
\label{eq:disk:mdisk_err}
\end{equation}

%% =================================================================================
%%
%%                          M E T H O D
%%
%% =================================================================================


\section{Method}

We perform two types of analysis in this chapter.
First, (i), we asses the quality of different fitting formulae for a given
dataset and determine the best performing one.
Second, (ii), we evaluate the differences between datasets that 
simulate microphysics and neutrino in different ways (if at all).

For (i) we consider the fitting formulae available in the literature 
and new ones, based on the simple polynomials in the key \ac{BNS} parameters, \ie, 
the reduced tidal deformability, $\tilde{\Lambda}$, and \mr{} $q$
To assess their performance, we employ basic fitting procedure with least
square method, minimizing the $\chid$ (discussed below) or the residuals.
The $\chid$ statistics reads
\begin{equation}
\chi_{\nu}^{2} = \frac{\chi^2}{N - C} = \frac{1}{N-C}\sum\limits_{i=1}^{N}\Bigg(\frac{o_i - e_i}{o_i ^{\rm err}}\Bigg)^2,
\label{eq:theory:chi2dof}
\end{equation}
where $N$ is the number of points in the dataset, $C$ is the number of coefficients in the 
fitting model (thus $N-C$ defines the number of degrees of freedom),
$o_i$ are the dataset values and $o_i ^{\rm err}$ their errors,
$e_i$ are the values predicted by the fitting model, and 
$o_i - e_i$ are the residuals.
%% ---
The closer the value of $\chid$ given by the fitting to $1$, the better the fitting formulae performs.
%% ---
Additionally, we compute the coefficient of determination, $R^2$:
\begin{equation}
R^2 = 1 - \frac{\sum\limits_{i=1}^{N}(o_i - e_i)^2}{\sum\limits_{i=1}^{N}(o_i - \mu)^2},
\end{equation}
where $\mu$ is the mean value of $\left\{ o_i \right\}_{i=1,N}$.
%% ---
Here, again, the closer the $R^2$ to $1$, the better is the fit. 
%% ---
The fitting procedure is conducted first for the \DSrefset{} to establish
the baseline and repeated each time we add a new dataset until all models are included.
%% ---
We also perform the analysis independently for each dataset.
%% --- 

To evaluate the incluence of different physics input in simulations 
on the statistical behavior of the ensemble of the models we employ the following procedure.
%% ---
We begin with the dataset that is uniform in physics and code, the \DSrefset{},
all models in which have fixed chirp mass.
Than we add the models from the \DSheatcool{} that also include the effects of neutrino 
heating and cooling but not uniform in numerical setup and exact neutrino treatment.
To asses, how the statistical properties changed, we consider the mean value and standard deviation.
To investigate the effects of the absence of neutrino reabsorption, we add the \DScool{},  
where only neutrino cooling is present, and repeat the analysis.
Finally, to asses the effect of neutrinos and changes in the \ac{EOS} treatment 
we repeat the analysis with all datasets, including the \DSnone{}.

%% =================================================================================
%%
%%                          R E S U L T S
%%
%% =================================================================================

\section{Analysis of the \ac{DE}}
\label{sec:res_stat_dynej}

We discuss the mechanism of the dynamical ejecta in Ch.~\ref{ch:bns_sims} and Sec.~\ref{sec:res_dyn_ej},
see also \textit{e.g.,} \citet{Radice:2020ddv,Bernuzzi:2020tgt,Shibata:2019wef}. 
Here we focus on the global properties of the ejecta, \ie, mass, averaged velocity, 
electron fraction, and \ac{RMS} half-opening angle. 
The first three quantity are shown in Fig.~\ref{fig:ejecta:dyn:ds} for all datasets.
We note that the overall properties of the ejecta are similar between the \DSrefset{} and
\DSheatcool{}. This is expected, as these datasets include similar, albeit not the same, physics,
regarding \ac{EOS} and neutrino treatment.
The important exception are the high \mr{} models of \DSrefset{} as their \ac{DE} of tidal 
origin only \cite{Bernuzzi:2020txg}.
Comparing the properties of datasets with and without neutrino reabsorption, we observed that the 
the inclusion of this effect leads to the \ac{DE} of higher mass, in addition to the epxected increase 
in average electron fraction. This is especially noticeable when comparing the models of \DSrefset{} 
and a subset of \citet{Radice:2018pdn} with leakage scheme only.

%% --------------------------------------------------------------------------------
%%
%%   MASS
%%
%% --------------------------------------------------------------------------------

\subsection{\ac{DE} mass}

The mass of the \ac{DE} averaged over all the models of \DSrefset{} is 

\begin{equation}
\label{eq:ejecta:dyn:avg:M}
\overline{\amd} = (3.51 \pm 2.57)\times 10^{-3}M_{\odot}\ ,
\end{equation}

where we also report the standard deviation computed over the relevant simulation sample.
If we add other models with neutrino heating and cooling, \ie, the \DSheatcool{} models,
we observed that the mean value increases to $(4.17 \pm 3.65)\times 10^{-3}M_{\odot}$.
This is due to the models with M1 neutrino scheme of \citet{Vincent:2019kor} and \citet{Sekiguchi:2016bjd}
%% ---
As expected, the inclusion of models with neutrino cooling only, \DScool{}, leads to the decrease in 
$\amd$ to $2.91\times 10^{-3}M_{\odot}$.
%% --- 
Including the rest of the models, those without neutrinos, \DSnone{}, we observe that the 
$\amd$ rises to $5.56\times 10^{-3}M_{\odot}$. This is due to the addition of models computed with
polytropic \acp{EOS}, specifically, models from \citet{Dietrich:2016hky}, that display 
the largest ejecta masses among all the datasets.

\begin{figure}[t]
    \centering 
    \includegraphics[width=0.49\textwidth]{statistics/parspace_mej_m0m1.pdf}
    \includegraphics[width=0.49\textwidth]{statistics/parspace_mej_all.pdf}
    \caption{
        Comparsion between ejecta mass informed by the fit (counters filled colors), 
        and the simulation ejecta mass (marker colors) for $P_2^2(q,\tilde{\Lambda})$ 
        fitting model calibrated with advanced-physics datasets, 
        \DSrefset{} and \DSheatcool{}, (\textit{top panel}) and with all 
        avialable datasets (\textit{bottom panel}).
        The plot shows that while qulitatilve the fit is able to capture main trends 
        in data, on the model-per-model bases the systematic uncertanties domiante 
        especially for the fit, calibrated with all models (see text).
        (Adapted from \citet{Nedora:2020qtd})
    }
    \label{fig:mej_parspace}
\end{figure}

Next, we perform the fitting procedure to the total ejecta mass. 
We consider the widely used fitting formulae first, 
\citep{Kawaguchi:2016ana,Dietrich:2016fpt,Radice:2018pdn}, 

\begin{eqnarray}
\label{eq:fit_Mej}
&\left(\frac{\amd}{10^{-3}M_{\odot}}\right)_{\rm fit} =
\Big[\alpha\Big(\frac{M_B}{M_A}\Big)^{1/3}\Big(\frac{1-2C_A}{C_A}\Big)+  
\beta\Big(\frac{M_B}{M_A}\Big)^n \\
&+ \gamma\Big(1-\frac{M_A}{M_{b\,A}} \Big)\Big]M_{b\,A} + (A\leftrightarrow B) + \delta\non,
\end{eqnarray}
and the model presented in \cite{Kruger:2020gig}:
\begin{equation}
\label{eq:fit_Mej_Kruger}
\left(\frac{\amd}{10^{-3}M_{\odot}}\right)_{\rm fit} =
\left(\frac{\alpha}{C_A} + \beta\frac{M_B ^n}{M_A ^n} + \gamma
C_A\Bigg)M_A + (A\leftrightarrow B)\right. \ .
\end{equation}

We also employ simple second-order polynomials: 
the one-parameter formula $(\tilde\Lambda)$ and the two-parameter
formula in $(q,\tilde\Lambda)$, namely

\begin{align}\label{eq:polyfit2}
P_2 ^1(\tilde{\Lambda}) &= b_0 + b_1\tilde\Lambda + b_2 \tilde\Lambda^2, \\\label{eq:polyfit22}
P_2 ^2(q,\tilde\Lambda) &= b_0 + b_1q + b_2\tilde\Lambda + b_3q ^2 +  b_4 q \tilde\Lambda + b_5\tilde\Lambda^2 \, .
\end{align}

We limit the degree of the polynomial to the second order due to the intrinsic scatter 
in the data. We leave a more detailed investigation to future work as more \ac{NR} models at 
higher resolution become available.

The fitting procedure is performed considering the $\log_{10}(\md)$ instead of $\md$
for numerical reasons. This is motivated by the fact that, even withing \DSrefset{} the values 
of the $\md$ changes by an order of magnitude for a very similar values of $q$ and $\tilde{\Lambda}$
(\red{See \ref{fig:ejecta:dyn:dsfits}}).
Additionally, the error measure we consider for $\md$, Eq.~\eqref{eq:ejecta:mej_err}, is 
biased towards the data with smaller values of $\md$ (the lower error bar for the lower $\md$).
A possible alternative approach is to consider the residuals instead of $\chid$ for the 
minimization. In the case of $\md$, however, we find that two approaches lead to similar 
qualitative fit within the domain of calibration.

\begin{figure*}[t]
    \centering 
    \includegraphics[width=0.32\textwidth]{statistics/ej_q_mej_our_poly22_cc.pdf}
    \includegraphics[width=0.32\textwidth]{statistics/ej_q_vej_our_poly22_cc.pdf}
    \includegraphics[width=0.32\textwidth]{statistics/ej_q_yeej_our_poly22_cc.pdf}
    \caption{Dynamical ejecta properties as a function of mass ratio
        and reduced tidal parameter. The dependency on the latter is
        color coded. From left to right the main panels show the total
        mass, the mass-averaged velocity and the electron fraction.
        The bottom panels show the relative difference between the data
        and the fit polynomial fit discussed in the text.
        (Adapted from \citet{Nedora:2020pak})
    }
    \label{fig:ejecta:dyn:dsfits}
\end{figure*}

Considering the fitting formulae from the literature, Eq.~\eqref{eq:fit_Mej} and Eq.~\eqref{eq:fit_Mej_Kruger},
we find that the outcome of the fitting procedure depends strongly on the non linear fitting algorithm 
and on its initial guesses. This makes the fitting ill-constrained.
Moreover, in the Eq.~\eqref{eq:fit_Mej_Kruger} the compactness enters twice with opposite trends. 
The physical motivation of this formula is not clear.

We report all the fit calibrations in Appendix~\ref{app:fit}
with the calibration for the polynomials reported in Tab.~\ref{tab:dynfit:poly};
and for the Eqs.\eqref{eq:fit_Mej}-\eqref{eq:fit_Mej_Kruger} -- in Tab.~\ref{tab:dynfit:fit_form}.

%% TAB Chi2dof for all ejecta models ---------------------------
%% Ejecta 

\begin{table}[t]
    \begin{center}
    \caption{
        Reduced $\chi$-squared $\chi^2 _{\nu}$ for different
        fitting models for the dynamical ejecta properties. Mean is the simulation
        average, $P_n(x,y)$ is a polynomial of order $n$ in the variables $x,y$. Fits are performed for the data of this work and for an increasingly larger combined dataset from
        the literature. See text for discussion. 
        The best fitting model is characterized by the lowest value of $\chi_{\nu}^2$.
        (Adapted from \citet{Nedora:2020qtd})
    } \label{tbl:fit:ejecta:chi2dofsall}
    \scalebox{0.88}{
        \begin{tabular}{l|l|ccccc}
            \hline\hline
            $\log_{10}(\md)$ & Datasets & Mean & Eq.~\eqref{eq:fit_Mej} & Eq.~\eqref{eq:fit_Mej_Kruger} & $P_2^1(\tilde{\Lambda})$ & $P_2^2(q,\tilde{\Lambda})$ \\ \hline
            & \DSrefset{} & 3.84 & 2.23 & 1.58 & 3.03 & 1.55 \\ 
            & \& \DSheatcool{}  & 26.66 & 16.85 & 10.60 & 37.29 & 56.45 \\ 
            & \& \DScool{}  & 99.11 & 30.12 & 11.91 & 45.59 & 24.40 \\ 
            & \& \DSnone{}  & 196.52 & 84.81 & 39.88 & 123.56 & 44.36 \\ 
            \hline\hline
            $\langle v_{\rm ej}\rangle$ & Datasets & Mean & Eq.~\eqref{eq:fit_vej} & & $P_2^1(\tilde{\Lambda})$ & $P_2^2(q,\tilde{\Lambda})$ \\ \hline
            & \DSrefset{} & 3.76 & 1.51 & & 3.24 & 1.05 \\ 
            & \& \DSheatcool{}  & 4.03 & 2.42 & & 3.35 & 1.67 \\ 
            & \& \DScool{}  & 7.10 & 6.07 & & 6.34 & 5.09 \\ 
            & \& \DSnone{}  & 7.95 & 6.79 & & 7.64 & 6.83 \\ 
            \hline\hline
            $\langle Y_{\rm e}\rangle$ & datasets & Mean &  & & $P_2^1(\tilde{\Lambda})$ & $P_2^2(q,\tilde{\Lambda})$ \\ \hline
            & \DSrefset{} & 42.49 &  & & 43.69 & 9.07 \\ 
            & \& \DSheatcool{}  & 37.78 &  & & 38.62 & 9.68 \\ 
            & \& \DScool{}  & 35.80 &  & & 36.27 & 24.96 \\ 
            \hline\hline
            $\langle \theta_{\rm RMS}\rangle$ & datasets & Mean & & & $P_2^1(\tilde{\Lambda})$ & $P_2^2(q,\tilde{\Lambda})$ \\ \hline
            & \DSrefset{} & 20.68 & & & 21.66 & 4.55 \\ 
            & \& \DSheatcool{}  & 18.18 & & & 18.69 & 4.17 \\ 
            & \& \DScool{}  & 15.56 & & & 14.34 & 8.73 \\ 
            \hline\hline
        \end{tabular}
    }%scalebox
    \end{center}
\end{table}



 
%% -------------------------------------------------------------

The performance of different fitting formulae is reported in  
Tab.~\ref{tbl:fit:ejecta:chi2dofsall}.
in terms of the $\chid$.
%% ---
Starting with the \DSrefset{} we observe that the best fitting formula, that gives the 
lowest $\chid=1.55$, is the second order two-parameter polynomial $P_2^2(q,\tilde{\Lambda})$.
Notably, the Eq.~\eqref{eq:fit_Mej_Kruger} gives a very similar $\chid=1.58$.
%% ---
The performance of the $P_2^2(q,\tilde{\Lambda})$ for individual models of the \DSrefset{} 
is shown on the first panel of Fig.~\ref{fig:ejecta:dyn:dsfits}
%% --- 
\red{From Main paper}
We recall that the \DSrefset{} contains simulations Ch.~\ref{ch:bns_sism}, and presented 
in Tab~\ref{tab:sim}. 
The plot shows the \ac{DE} mass as a function of the \mr{} and (color coded) $\tilde\Lambda$.
and highlights the strong dependency of the $\md$ of these parameters in the set of targeted 
simulations. 
%% --- 
Adding the \DSheatcool{} we find a rise in $\chid$ evaluated for all fitting formulae. 
This can be attributed to the models of \DSheatcool{} spanning a significantly broader range 
in terms of $\tilde{\Lambda}$. Additionally, with an addition of models from \DSheatcool{} the 
systematic and methodological uncertainties enter the picture and as we shall see from the 
analysis, they will dominate the overall statistics.
%% ---
When all datasets are included the $P_2^2(q,\tilde{\Lambda})$ and Eq.~\eqref{eq:fit_Mej_Kruger},
remain the best fitting models albeit with $\chid$ of $44$ and $40$ respectively.
%% ---
The similar performance of these fitting formulae can be attributed to the fact that in 
both the \mr{} enters explicitly, allowing to capture the leading trends in data.
The Eq.~\eqref{eq:fit_Mej} also includes the \mr{} but has more degrees of freedom and 
thus less favorable by the $\chid$ analysis. Additionally, it was pointed out in \citet{Radice:2018pdn},
that this formula dies not reproduce well the systematic trends in the set of models with 
leakage neutrino scheme.
%% ---
Notably, the second order polynomial just in one quantity, $\tilde{\Lambda}$, is failing to 
capture the main trends in data, giving $\chid=123$ when all models from all datasets are considered.
Similarly, a fit with no free parameters, the mean value, does not perform well, giving $\chid=196$.
%% --- 
Thus, we conclude, that the dependency of \mr{} ought to be included into the fitting formula
to capture the leading trends in statistical behaviour of $\amd$.

\red{This paragraph is not approved at the time of writing}
We show how the $P_2^2(q,\tilde{\Lambda})$ performs when only the 
 \DSrefset{} \& \DSheatcool{} and all models 
are considered on Fig.~\ref{fig:mej_parspace} on the left and right panels respectively. 
We observe that the smooth polynomial fit cannot capture the oscillations id data, where 
the $\amd$ changes by up to an order of magnitude for a very similar values of \mr{} and 
$\tilde{\Lambda}$. Overall, while for the  \DSrefset{} \& \DSheatcool{}
 the leading trends are appears to be 
captured by the fit, for all the datasets the fits' predictive power reduces significantly.
\begin{figure}[t]
    \centering 
    \includegraphics[width=0.49\textwidth]{statistics/residuals_sets_mej.pdf}
    \caption{
        Relative differences between data and fits for the dynamical ejecta
        mass, $\Delta M_{\rm ej} = M_{\rm ej} - M^{\rm fit}_{\rm ej}$.
        We show polynomial fits and fitting formulae
        Eq.~\eqref{eq:fit_Mej} and Eq.~\eqref{eq:fit_Mej_Kruger}.
        From top to bottom the models arrange based on their $\chi_{\nu}^2$: from lowest to highest.
        The gray region represents the fit's $68\%$ confidence level.
        Note that while $P_2^2(q.\tilde{\Lambda})$ gives the second lowest $\chi_{\nu}^2$
        the plot shows that is has tighter residuals then the best fit Eq.\eqref{eq:fit_Mej_Kruger}.
        This is because Eq.\eqref{eq:fit_Mej_Kruger} fits better models with small $\amd$,
        with tighter error bars, Eq.~\eqref{eq:ejecta:mej_err}. Hence,
        the $\chi_{\nu}^2$ is smaller.
        Note that fitting was performed minimizing $\log_{10}(\md)$. See text for details.
        (Adapted from \citet{Nedora:2020qtd})
    }
    \label{fig:ejecta:dyn:m}
\end{figure}

%% --- residual plots
We display the relative differences between the model data and fit provided data in 
Fig.~\ref{fig:ejecta:dyn:m}. The plot shows that none of the fitting models can 
reproduce the $\amd$ of the \DScool{} models with the leakage scheme employed.
%% --
While the Eq.~\eqref{eq:fit_Mej_Kruger} and $P_2^2(q,\tilde{\Lambda})$ showed similar 
$\chid$, the plot shows that the latter reproduces the high ejecta masses better, 
truncating the distribution at $\sim10^{-2}M_{\odot}$. 
The poor performance of the single parameter polynomial, $P_2^1(\tilde{\Lambda})$,
is clear, as the fit gives an almost flat distribution around the mean value of $\amd$.
Similarly, the Eq.~\eqref{eq:fit_Mej} cannot reproduce the large masses of some 
models.

\red{This paragraph is not approved at the time of writing}
Overall we conclude that the intrinsic scatter in the data hinders performance 
of any smooth fitting formula. The inclusion \mr{} is required to capture the leading 
trends in data and the simple polynomial $P_2^2(q,\tilde{\Lambda})$ show a reasonably 
good statistical performance.
\gray{We recomment the calibration based on the models with }
%% --- 
The statistical analysis for considered datasets suggests that the $\amd$ depends 
sensibly on the physics input of the simulations: neutrino scheme and \ac{EOS} treatment.
The magnitude of systematic uncertainties reduces the ability of any fitting formulae 
to identify and capture leading trends.



\subsection{Mass-averaged velocity}

%% VELOCITY 
\begin{figure}[t]
    \centering 
    \includegraphics[width=0.49\textwidth]{./figs/residuals/residuals_sets_vej.pdf}
    \caption{
        Relative differences between ata and fits for the mass-averaged velocity of the dynamical ejecta, $\Delta \upsilon_{\rm ej} = \upsilon_{\rm ej} - \upsilon^{\rm fit}_{\rm ej}$.
        We show the fitting formula Eq.~\eqref{eq:fit_vej} and the polynomial fits.
        From top to bottom the models are arranged based on their $\chi_{\nu}^2$: from lowest to highest.
    }
    \label{fig:ejecta:dyn:v}
\end{figure}

%% ELECTRON FRACTION
\begin{figure}[t]
    \centering 
    \includegraphics[width=0.49\textwidth]{./figs/residuals/residuals_sets_yeej.pdf}
    \caption{
        Relative differences between data and fits for the 
        mass-averaged electron fraction of the dynamical ejecta.
        We show the polynomial fits, and Eq.~\eqref{eq:polyfit2} and Eq.~\eqref{eq:polyfit22}.
        %% \alp{there is no fitting formula here, correct?}.
        %% \vn{Yes, we decided to use only polynomials}
        Here $\Delta Y_{e\: \rm ej} = Y_{e\: \rm ej} - Y^{\rm fit}_{e\: \rm ej}$.
        From top to bottom the models are arranged based on their $\chi_{\nu}^2$: from lowest to highest.
    }
    \label{fig:ejecta:dyn:y}
\end{figure}