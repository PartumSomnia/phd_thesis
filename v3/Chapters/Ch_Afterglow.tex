% Chapter Template

\chapter{Non-thermal emission from compact object merger} % Main chapter title

\label{ch:afg} % Change X to a consecutive number; for referencing this chapter elsewhere, use \ref{ChapterX}


\section{$\gamma$-ray bursts}

\acp{GRB} are irregular pulses of gamma-ray radiation with broken power-law (non-thermal) spectrum, peaking at KeV-MeV \citep{Band:1993,Kouveliotou:1993,Meegan:1992xg}.

With respect to the duration, \ac{GRB} are split into two categories: \ac{SGRB}, that last $< \sim 2$~s and long \ac{GRB} that last $> \sim 2$~s. The latter are the result of the collapse of massive $\geq 15M_{\odot}$ stars, while the former, at least in part, is attributed to mergers of compact objects. Only very recently it was directly confirmed \citep{TheLIGOScientific:2017qsa}. However, the exact physical origin of different duration \ac{GRB} is not fully understood.

Indications that long \ac{GRB} are associated with core-collapse supernovae, \acp{SN}, are two fold. These \acp{GRB} are typically observed in star-forming regions of their host galaxies \citep[\eg][]{Bloom:2000pq,Bloom:2002hc,Fruchter:2006py,Christensen:2004yx,CastroCeron:2006jh} and several \acp{GRB} are spectroscopically associated with Type Ic \acp{SN}, albeit these \acp{GRB} were significantly less bright and might not be typical \acp{GRB} \citep[\eg][]{Liang:2006ci,Bromberg:2011fm}. Additionally, the late time behaviour of some \acp{GRB} includes a \acp{SN}-like "bump" in the optical and spectral changes that might imply that underlying \acp{SN} flux becomes dominant over \acp{GRB} \citep[\eg][]{Bloom:1999,Woosley:2006fn}.

The \acp{GRB} are distant events, most of which were localized to outside the local group \citep[\eg][]{Mao:1992,Piran:1992,Fenimore:1993}. Particularly useful for distance estimation were the observations of \ac{GRB} afterglow, fading X-ray \& optical emission, that allow to estimate the redshift
\citep[\eg][]{Costa:1997cg,Frontera:1997ae}.

Analysis of the multi-wavelength afterglow data for \acp{GRB} \cite[\eg][]{Panaitescu:2001bx} suggested the mechanism behind the afterglow emission is the synchrotron radiation from the forward, external forward-shock, which forms when \ac{GRB}-ejecta sweeps-up the circumburst medium
\footnote{The specific indications are the power law decay of the light curves, $F_{\nu}\propto^{-1}$ and power-law spectrum $F_{\nu}\propto\nu^{-0.9\pm 0.5}$.} 
\citep{Rees:1992ek,Paczynski:1993gz,Meszaros:1993ju,Meszaros:1996sv}.

The temporal behavior of many (but not all) \acp{GRB} shows a change, a steepening of the light-curve (to $F_{\nu}\propto t^{-2.2}$) at $\sim 1$~day after the burst. This is usually attributed to the 
%\gray{deceleration of the colimated GRB-outflow, jet, and decrease on the realtivisitc beaming. This in turn makes the edge of the jet visible to an observer.} 
finite angular extend of the \ac{GRB}-ejecta, jet \citep[\eg][]{Rhoads:1999wm,Sari:1999mr}. When jet decelerates and relativistic beaming decreases (and the jet edge becomes visible), the optical and X-ray lightcurves decay achromatically faster. This achromatic transition from slow to faster decay is called "jet-break".

%% PROBLEMO -- jet-break is not a universal feature.
Notably, this jet-break is not observed in all GRBs which presents a question why \citep[\eg][]{Fan:2006pj,Panaitescu:2006,Liang:2007ti,Sato:2006jg,Liang:2007rn,Curran:2007cp,Racusin:2008bx}

%% PROBLEMO -- GRB density seems uniform, but SSE models predict wind-like profile
Models of the broadband emission of \acp{GRB} with jet-break showed that the circomburst medium, \ac{CBM}, is uniform with number density \red{$\sim 10^{-3}$} \citep{Panaitescu:2001bx}. If \acp{GRB} produced in collapse of massive stars \citep{Woosley:1993,Paczynski:1997yg}, this contradicts the expected density profile from stellar winds, \eg, $\rho\propto r^{-2}$ \citep[\eg][]{Dai:1998iz, Chevalier:1999jy, Chevalier:1999mi,Ramirez-Ruiz:2001} \red{this might be very outdated.}

%% sGRB
Regarding the short duration \ac{GRB}. Their origin was first connected with the elliptical galaxes with older stellar population \citep[\eg][]{Gehrels:2005qk,Fox:2005kv,Barthelmy:2005bx,Berger:2005dr,Panaitescu:2005er,Bloom:2005qx,Guetta:2005bb,Nakar:2007yr} and thus with merger of neutrons stars. A more direct evidence came with the detection of \GRB{} \cite{Abbott:2018wiz}, a \ac{SGRB} that accombaned the \ac{GW} detection and \ac{kN}.

Continuous observations of \ac{SGRB} showed a complex time behavior of early afterglow X-ray emission, in particular a presence of a plateau ($F_{x}\propto t^{-1/2}$), after the initial sharp decrease ($F_{x}\propto t^{-3}$) which a standard forward shock model does not predict. This implied that early X-ray afterglow is shaped by a variety of physical processes \citep{Zhang:2005fa}.

Two main questions that stem from these observations: is the mechanism behind the prompt $\gamma$-ray emission and early afterglow emission is the same (or do they originate from the same outflow), and is the early X-ray radiation produced by the external shock (just a blast wave takes long time to become self-similar) or does it originate from an internal shock?
An indication that the long-lived central engine activity might affect the afterglow came from the observed sharp increase in X-ray flux (flares) on a scale of minutes to hours after the end of the \ac{GRB}
\citep{Burrows:2005ww,Chincarini:2007fp,Chincarini:2010,Margutti:2011}, which could not be attributed to the inhomogeneities in the \ac{CBM}.
%% PROBLEMO!
Thus, the early X-ray behaviour of \acp{GRB} $t < 10^{4}$~s post-burst is not well understood and seems to be in tension with standard afterglow forward shock emission model.

%\textcolor{red}{
%    One of the foremost unanswered questions about GRBs is the physical mechanism
%    by which prompt $\gamma$-rays the radiation that triggers detectors on board
%    GRB satellites are produced. Is the mechanism the popular internal shock
%    model 6 \cite{(Rees and Meszaros, 1994)}, the external shock model, or something
%    entirely different? Are $\gamma$-ray photons generated via the synchrotron process
%    or inverse-Compton process, or by a different mechanism? Answers to these
%    questions will help us address some of the most important unsolved problems
%    in GRBs  how is the explosion powered in these bursts? Does the relativistic
%    jet produced in these explosions consist of ordinary baryonic matter, electron positron
%    pairs, or is the energy primarily in magnetic fields?
%}

Once again, while it is suggested that the high energy emission, after the propmt phase is produced by the synchrotron process in the external forward shock, \citep{Kumar:2009,Ghisellini:2010}, the mechanism behind the high and low energy $\gamma$-ray emission in the prompt phase remains unknown. 
Possible mechanisms include: inverse Compton and synchrotron emission in internal and external shocks
\citep[\eg][]{Rees:1992ek,Dermer:1998py,Lyutikov:2003ih,Zhang:2011} and 
photospheric radiation with contribution from multiple \ac{IC} scatterings
\citep[\eg][]{Thompson:1994zh,Ghisellini:1998jy,Meszaros:1999gb,Peer:2005qoc,Peer:2008udu,Giannios:2006jb,Ioka:2007qk,Asano:2009gi,Lazzati:2010,Beloborodov:2010,Toma:2011}.

%% from the Afterglow paper
One of the \ac{EM} counterparts, that observed for \GW{} was the short \ac{GRB}, \GRB{} \citep{Savchenko:2017ffs,Alexander:2017aly,Troja:2017nqp,Monitor:2017mdv,Nynka:2018vup,Hajela:2019mjy}, detected by the space observatories Fermi \citep{TheFermi-LAT:2015kwa} and INTEGRAL \citep{Winkler:2011}.

This \ac{SGRB} was dimmer then any other events of its class. 
Different interpretations for its dimness and slow rising flux were proposed: off-axis jet, cocoon or structured jet. 
Now it is commonly accepted that \GRB{} was a structured jet observed off-axis 
\citep[\eg][]{Fong:2017ekk,Troja:2017nqp,Margutti:2018xqd,Lamb:2017ych,Lamb:2018ohw,Ryan:2019fhz}.
The \GRB{} late emission, the afterglow, provided further information on 
the energetics of the event and on the properties of the circumburst medium \citep[\eg][]{Hajela:2019mjy}.


\subsection{kilonova afterglow}

In addition to the \ac{GRB} beamed emission, the non-thermal, more isotropic emission is expected from electrons accelerated in shocks formed between the (mildly) relativistic ejecta and the \ac{ISM} \citep{Nakar:2011cw}. This emission is expected to peak in radio band and continue on a time scale of tens of years after merger. Notably, all ejecta components will contribute to the emission, but  depending on the ejecta velocities and kinetic energy, the brightness in different frequencies and timescales differs. 

Various ejecta components interact with each other and with \ac{ISM}. The latter, generates a long-lived blast wave. The shock, propagating upstream, amplifies (random) magnetic fields and accelerates electrons, that subsequently emit synchrotron radiation. This process in phenomenological similar to the for the \ac{GRB} afterglow and \ac{SN} early remnants. 

Numerical simulations of \ac{BNS} mergers showed the presence of mildly relativistic ejecta several 
studies on the possible non-thermal electromagnetic emission of this ejecta have been carried out 
\citet[\eg][]{Piran:2012wd,Hotokezaka:2015eja,Radice:2018pdn}. Additionally, the observed non-thermal emission from \GW{}, before the achromatic (jet) break, was interpreted as the non-thermal emission 
from the ejecta \citep{Mooley:2017enz}.

Specifically, a strong radio emission is expected from \ac{BNS} ejecta \citep{Piran:2012wd,Hotokezaka:2015eja}. Such radio remnant is expected to peak on a time-scale 
of years after the merger and be visible over a similar timescale. Notably, this is assuming that the 
ejecta is expanding into the unshocked \ac{ISM}, as it was shown that if the \ac{ISM} is 
pre-shocked by the jetted outflow, and the \ac{ISM} density is reduced, the kilonova afterglow would 
be delayed. 
In \citet{Piran:2012wd}, the synthetic \ac{kN} afterglow \acp{LC} are calculated for a set of \ac{NR} \ac{BNS} merger simulation with properties typical to the Galactic binary population and \ac{ISM} density usually found in the Galactic disk, $\nism\sim1\ccm$. Authors showed that the from a binary of two $1.4\Msun$ \acp{NS}, the kilonova afterglow would peak $\sim10$~years after the merger if $\nism = 0.1\ccm$ and $\sim3$~years if $\nism = 1\ccm$ in radio bands, $\nu=1.4$~GGz and $\nu=150$~MGz. Notably, both values of the \ac{ISM} density are larger than what is inferred for \GW{}. Indeed, jet fitting models and idependent analysis of the diffuse emission (\red{hajela+19}) suggest $\nism\in(10^{-4},10^{-2})\gcm$ \citep{Hajela:2019mjy}.
In \citet{Piran:2012wd} authors focus on the observational prospects of this afterglow and compare it to other \ac{EM} emission expected for \ac{BNS}. 





