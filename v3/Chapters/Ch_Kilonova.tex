% Chapter Template

\chapter{Chapter Title Here} % Main chapter title

\label{ch:kilonova} % Change X to a consecutive number; for referencing this chapter elsewhere, use \ref{ChapterX}

In this chapter we discuss the thermal \ac{EM} counterpart to \ac{BNS} mergers.


%% =====================================================================================
%%
%%               I N T R O D U C T I O N
%%
%% =====================================================================================

\section{Introduction}

%% Based on the Barns Thesis 
\red{this is based on the Barnes PhD thesis on Opacities for Kilonva (Rad.Transport)}
\red{Based on Metzger paper }

The decay of the \rproc{} elements releases energy, that thermalizes in the ejecta 
and lead to the observed \ac{EM} transient, the kilonova \citep[\eg][]{Metzger:2016pju}.

This is day-to-week long, thermal, supernovae-like transient, powered by the radioactive decay of heavy, neutron rich elements, synthesized in the expanding merger ejecta \citep{Li:1998bw}. They accompany BNS abd NSBH (where ejecta os present) mergers and surve as a probe of origin of meaviest elements in the universe \cite{Metzger:2010}.

During the \ac{BNS} mergers, the neutron-rich material is ejected (see Chapter \ref{ch:bns_sims} for details).
The conditions of the eject aare such that they allow for the \rproc{}, that produces heavy elements 
far from the valley of stability (see Chapter \ref{ch:nucleo} for details).
\citet{Li:1998bw} suggested that an electromagnetic transient can be powered by the radioactive decay of the material ejected in \ac{BNS} or \ac{NSBH} mergers. They also showed that contrary to the normal \acp{SN}, the ejecta would quickly become transparent to its own emission, peaking on a timescale of around a few days. 
The main difficulty in this pioneering work was the lack of a \ac{nuc} model to model to estimate the radioactive heating of the ejecta. 

Consider the following simplified approach. let the total energy release by the radioactive decay of isotopes '$i$' be $\dot{Q}_i \propto \exp(-t/\tau_i)$, where is $\tau_i$ is the half-life. If $\tau$ are distributed equally per logarithmic time (at any $t$ the dominant species have $\tau\sim t$), the heating rate of the ejecta at time $t$ is

\begin{equation}
\dot{Q}_{LP} = \frac{f M c^2}{t}
\end{equation}

where $f$ is free parameter and $M$ is the ejecta mass.

Notably, later models, that considered \ac{NRN} computed on thermodynamic histories of the expanding ejecta (from \ac{NR} simulations) indicated that heating ray depends on time, at late time, via a steep power law \citep{Metzger:2010,Roberts:2011,Korobkin:2012uy}. See also \citet{Hotokezaka:2017dbk}
for the discussion on physical principles behind this decay.

In addition, provided by \citet{Li:1998bw}, normalization $f$ resulted in overestimation of the peak luminosity of the \ac{kN}, that plagued the Kilonova searches for the upcoming decade \cite{Rosswog:2005su,Dong:2015oea,Bloom:2005qx,Kocevski:2009gv}. 

The first self-consistent estimation of the heating rates based on the \ac{NRN} calculations of the \rproc{} in the ejecta, carried out by the \citet{Metzger:2010}, showed that the based on the dynamical ejecta \ac{RM} transient is $\sim10^3$ times brighter then nove -- \ie, the kilonova term was coined. It was also shown the ejecta electron fraction does not affect the heating rates considerably, and performed the first estimations of the thermalization efficiency of decay products. \red{ can be linked to the TE section/appendix from PhD}

The term macronova was however coind by \citet{Kulkarni:2005jw} who considered a transient powered by the decay of radioactive $^{56}$Ni and free neutrons. However, as it was shown later, $^{56}$Ni can only be formed in small quantities on the outskirts of the ejecta

Besides the intricate radioactive heating rates, determining the ejecta opacity presents a formidable challenge. 
The general lack of experimental data and numerical models of the singly and doubly ionized heavy $r$-process elements opacity, complicates the issue. Initially, gray iron group opacities, used by supernovae community, were adopted \citep{Roberts:2011}. However, it was shown later that this is an underestimation of the \rproc{} elements opacities \citep{Kasen:2013xka}. The high line density and complex atomic structures of lanthanides and actinides can raise their opacities significantly. This was later confirmed by \citep{Tanaka:2013ana} . Higher opacities shift the peak time to later times ${\sim}1$~weak \citep{Barnes:2013wka} and shift the spectral peak from optical/UV to \ac{NIR}.


\subsection{Basic model of the \ac{kN}}

Consider a shell of an ejecta, expanding with constant velocity, $\upsilon$, such that $R\approx\upsilon t$ at any point in time $t$. Assume spherical symmetry, justified by ejecta previous lateral expansion \citep{Roberts:2011,Grossman:2013lqa,Rosswog:2013kqa}

Assume further that the ejecta is hot.
The thermal energy is not immediately radiated away due to high optical depth and long diffusion timescales:

\begin{align}
\tau &= \rho \kappa R = \frac{3}{4}\frac{M\kappa}{4\pi R^2}\\ 
t_{diff} &\approx \frac{R}{c}\tau = \frac{3}{4}\frac{M\kappa}{4\pi c R} = \frac{3}{4}\frac{M\kappa}{\pi c \upsilon t}
\end{align}

where $M$ is the ejecta mass, $\kappa$ is the opacity (cross section per unit mass), $\rho$ is the density,. \eg, $\rho=3M/(4\pi R^3)$ is the mean density.

The diffusion timescale decreases with ejecta expanding. 
When $t_{diff}$ reaches $t$, the radiation can escape the ejecta \citep{Arnett:1982}. Hence, the characteristic timescale of the peak of emitted radiation 

\begin{equation}
    t_{peak} = \Big(\frac{3}{4}\frac{1}{\beta}\frac{M\kappa}{\pi \upsilon c}\Big)
\end{equation}

where the constant $\beta$ depends on the exact density profile of the ejecta. 
The $t_{peak}$ is of order of days for lanthanides-free and weeks for lanthanides-rich ejecta.

Regading the ejecta temperature. If there is no additional heating, even ejected with $10^9$~K, the ejecta cools very quickly via adiabatic losses. Thus, when it reaches $R_{peak} = \upsilon t_{peak}$, when it is transparent, it is effectively cold and invisible. 

However, there is an additional source of energy withing the ejecta. Several of them, in fact. These might include the contribution from the central engine, $r$-process element radioactive heating). These contribute to the total heating rate $\dot{Q}(t)$, that relates to the peak luminocity of the transient via \textit{Arnett's Law} \citep{Arnett:1982}.

Thus, overall, three ingredients are required to understand the observations of \ac{kN}. These are 
\begin{itemize}
    \item The ejecta properties: $M$, $\upsilon$, $Y_e$,
    \item The composition of the expanding ejecta and its optical opacity.
    \item Dominant sources of energy within the ejecta, heating rate $\dot{Q}(t)$, and how efficient this energy thermalizes.
\end{itemize}

We discuss the properties of the ejecta from \ac{BNS} mergers in the Chapter \ref{ch:bns_sims}.
Here we focus on two other kilonova ingredients.
\red{Here I skip a large part of the paper with BNS and NSBH ejecta discussion}

\subsection{Opacity}

The ejecta is composed of the produces of \rproc{}, a complex mixture of heavy elements with 
rich atomic structure. 
%% free-free opacities
For the photons of the lowest energy (FIR), \eg, lowest frequency, the free-free absorption in the ionized gas is the dominant source of opacity. Expansion, the recombination removes free $e$, also decreases $\rho$, and thus $\kappa_{ff}$.
%% bound-bound opacities
For \ac{NIR}/optical photons, the bound-bound transitions are the main source of opacity. Here the dependency on the ejecta composition strongly affects the \textit{effective} continuum opacity. If the most complex atoms in the ejecta belong to the iron group, with valence $d$-shell electrons, then the opacity is moderate. However, presence of elements with valence, partially filled $f$-shell, (lanthinides \& actinides), then the opacity increases by up to two orders of magnitude \citep{Kasen:2013xka,Tanaka:2013ana,Fontes:2015,Fontes:2017zfb}. Bound-bound opacity rises with photon $\nu$ (as the number of lines).
%% ---
The \magenta{plank mean expansion opacity} can be approximated as $\kappa_r = 200 (T/4000K)^{5.5}$ cm$^2$g$^{-1}$ for $T\in(1-4)\times10^3$~K and just $\kappa_r=200$ cm$^2$g$^{-1}$ for $T\in(4-10)\times10^{3}$~k, motivated by Fig.10 of \citep{Kasen:2013xka}. More accurate opacity estimations are plagued by the complexity of the atomic structure. The quantum mechanics models of high-$Z$ atoms exists, but has to be calibrated to the so far absent experimental data.
%% Line opacity -> Effectve continoum opacity
Additional complexity arises in approximation the line opacities to the effective opacity.
One common way is to consider the line expansion opacity formalism \citep{Pinto:2000}, that is based on the Sobolev approximation. This method was applied to \acp{kN} modeling by \citet{Barnes:2013wka} and \citet{Tanaka:2013ana}. However, it is unreliable if line width is large \ie, if line spacing of strong lines become comparable to the intrinsic thermal line width \citep{Kasen:2013xka,Fontes:2015,Fontes:2017zfb}. 
%% clumping
In addition, clumping that might occur when $T\leq10^3$~K might have a strong effect \citep{Takami:2014oqa}. The formation of the ``\rproc{} dust'' may have a complex effect on optical/UV opacity. The process of dust formation is complex and not well understood in general \citep{Cherchneff:2009sj,Lazzati:2016}
%% Ejecta re-ionisation
For even higher energy photos, UV/X-ray, bound-free transitions dominate the opacity. For that ejecta ought to become mostly neutral, which occures natually as it cools, unless there is a source of ionizing radiation, \eg, the remnant. See for details \cite[\eg][]{Metzger:2013cha}. Even though very large luminocities are required from the engine initially, they decrease rapidly as ejecta expands. Thus, at late times it is possible that ejecta would be re-ionized, especially in case of a long-lived remnant \citep{Metzger:2013cha}. The re-ionisation can reduce the optical opacity, reducing the prominence of FIR peak, a generally regarded distingushed feature of a Kilonova.
%% X-ray,. Gamma-rays, thermalization
At even higher energies, hard X-ray photons, an imporatnt source of opacity is the electron scattering with Klein-Nishina corrections. Importantly, that if the wavelendgth of photones become smaller then the salae of an atom, the contribution from both, free and bound into nuclei electrons ought to be considered. At high energies, the scattering of photons is inelastic. However, these processes are important as ejecta opacity to very high energy photons, gamma rays with energy in order of MeVs, determins the thermalisation of the $r$-process decay products.
%% pair-creation
For an extremely high photon energies $h\nu \gg m_e c^2$a pair creation becomes important. In particular this is so if a remnant is magnetar with large spin-down luminosity. Then, the at peak of the Kilonova emission, the pair creation might prevent pair-creation photons from escaping the kilonova. 

\subsection{Unified Toy Model}
\red{commented}
%The way to model electormagnetic emission from the ejecta is to perform multi-dimensional, multi-group radiative transfer simulation coupled to hydrodynamica (or MHD) simulation of the ejecta itself.
%
%Here a simplified model is considered or a transient, powered by the radioactive decay within the ejecta only. Several other assumptions are made.
%In particular, the ejecta expansion is homologous (faster matter ahead of slow one) \cite{(Rosswog et al 2014)}
%The mass-velocity distribution is $M_{\upsilon} = M(\upsilon / \upsilon_0)^{-\beta}$, for $\upsilon \geq \upsilon_0$,
%where $M$ is the total mass, $\upsilon_0$ is the average, minimum velocity. $\beta$ can be assumed to te $3$, \cite{(Bauswein et al 2013a)}. However see \cite{Piran et al 2013} for a more complex velocity profiles.
%
%The diffusion timescale defines when the radiation escapes the ejecta. For a layer with $\upsilon$ and $M_{\upsilon}$ and opacity $\kappa_{\upsilon}$ it is 
%
%\begin{equation}
%t_{d,\upsilon} \approx \frac{3}{4\pi}\frac{M_{\upsilon}\kappa_{\upsilon}}{\beta Rc} = \frac{1}{4\pi}\frac{M_{\upsilon}^{4/3}\kappa_{\upsilon}}{M^{1/3}\upsilon_0 t c}
%\end{equation}
%
%where $\beta=3$ was assumed. 
%
%This equation implies that at time $t=t_{d,\upsilon}$ the radiation from the layer $M_{\upsilon}$ peaks.
%
%The $M_{\upsilon}(t)$ is related to the total mass of the ejecta and peak time (when radiation diffuses from the entire ejecta)
%
%\begin{equation}
%M_{\upsilon}(t) = 
%\begin{cases}
%M(t/t_{peak})^{3/2},& t<t_{peak}, \\
%M, &t>t_{peak}
%\end{cases}
%\end{equation}
%
%where $t_{peak} = (3M\kappa / (4\pi \beta \upsilon c))^{1/2}$ with $\upsilon = \upsilon_0$. \red{Did not understand}
%
%Outer layers with $M_{\upsilon} < M$ peaks first, while the deepest layers peak later but set the luminocity of the whole ejecta (assuming that the the opacity is constant in the ejecta. 
%
%The radial evolution of each layer $M_{\upsilon}$ of mass $dM_{\upsilon}$ is given by 
%
%\begin{equation}
%\frac{dR}{dt} = \upsilon,
%\end{equation}
%
%and the layer's thermal energy changes according to 
%
%\begin{equation}
%\label{eq:theory:mkn:energ}
%\frac{dE_{\upsilon}}{dt} = \underbracket{-\frac{E_{\upsilon}}{R_{\upsilon}} \frac{dR_{\upsilon}}{dt}}_{PdV\text{ losses}} - \underbracket{L_{\upsilon}}_{\text{rad. los.}} + \underbrace{\dot{Q}}_{\text{heating sources}},
%\end{equation}
%
%where the radiative losses take form
%
%\begin{equation}
%L_{\upsilon} = \frac{E_{\upsilon}}{t_{d,\upsilon} + t_{lc,\upsilon}},
%\end{equation}
%
%in which the $t_{lc,\upsilon} = R_{\upsilon}/c$ limits the energy loss to the light crossing time (important for when the layer is optically thin) \red{did not understand}.
%
%The heating sources $\dot{Q}$ include
%
%\begin{equation}
%\dot{Q}(t) = \underbrace{\dot{Q}_{r,\upsilon}}_{\text{radioactivity}} + \underbrace{\dot{Q}_{mag}}_{\text{magnetar}} + \underbrace{\dot{Q}_{fb}}_{\text{fall-bak accretion}}
%\end{equation}
%
%Next, even though in principle the effect of radiation pressure on ejecta ought be considered, in case where radioactive heating, total energy input $\int \dot{Q}_{r,\upsilon}dt < E_{kin,0}$ of the ejecta, \cite{(Metzger et al 2011; Rosswog et al 2013)}, this effect can be neglected.
%Meanwhile, central engine might provide enough energy to modify the free expansion model. Then the equation for the central shell velocity evolution reads 
%
%\begin{equation}
%\label{eq:theory:mkn:velcenteng}
%\frac{d}{dt}\Bigg(\frac{M\upsilon_0^2}{2}\Bigg) = M\upsilon_0\frac{d\upsilon_0}{dt} = \frac{E_{\upsilon_0}}{R_0}\frac{dR_0}{dt}
%\end{equation}
%
%Here, the term with $E_{\upsilon_0}$ balances the PdV \textit{loss} term in the thermal energy eequation (for $dE_{\upsilon}/dt$)
%
%To compute the mitted radtion, first assume the black-body emission, the thermal emission, wit heffective temperature 
%
%\begin{equation}
%T_{eff} = \Bigg(\frac{L_{tot}}{4 \pi \sigma R_{ph}^2}\Bigg)^{1/4}
%\end{equation}
%
%where $L_{tot} = \sum(L_{\upsilon dm_{\upsilon}})$. is the total luminocity (cumuklative for all mass shells). 
%At the point where optical depth $\sum(\kappa_{\upsilon}dm_{\upsilon})=1$ the photosphere is located wit hradius $R_{ph}(t)$. 
%The flux density of the source at photon frequency $\nu$ is given by 
%
%\begin{equation}
%F_{\nu}(t) = \frac{2\pi h \nu^3}{c^2} \frac{1}{\exp\Big(\frac{h\nu}{kT_{eff}}\Big) - 1} \frac{R_{ph}^2}{D^2}
%\end{equation}
%
%where, $D$ is the distnace to the source. (Cosmological effects are neglected here).
%
%Additionally, the opacity $\kappa_{\upsilon}$ depends on the temperature of the layer $T_{\upsilon}$, that itself cna be comptued as 
%
%\begin{equation}
%T_{\upsilon} = \Bigg(\frac{3E_{\upsilon}}{4\pi a R^{3}_{\upsilon}}\Bigg)^{1/4}
%\end{equation}
%
%assuming that the ejecta internal energy is dominated by the raditiona. 
%
%Finally, in order to compute the electromagnetic emission from the ejecta, the equation Eq.~\eqref{eq:theory:mkn:energ} ought be solved for $E_{\upsilon}$ (and $L_{\upsilon}$) for every shell with $dM_{\upsilon}$ and $\upsilon>\upsilon_0$. 
%The velocity distribution can be assumed fixed (\textit{e.g.,} $M_{\upsilon} = M(\upsilon/\upsilon_0)^{-\beta}$,  if only the internal heating are important. If however, the central engine energy input is important, the the velocity of the central layer evolves according to Eq.~\eqref{eq:theory:mkn:velcenteng}.
%Initial kinetic energy of the ejecta is quicly removed by the adiabatic expansion and the thermal energy of the ejecta, when its emission peaks, is domunated by the heating.

\subsection{\rproc{} heating}

Consider an ejecta mass layer $dM_{\upsilon}$ (with velocity $\upsilon$).
The layer has a fraction $X{r,\upsilon}$ of \rproc{} elements, specific heating, $\dot{e}_r(t)$,  for which is 

\begin{equation}
\dot{Q}_{r,\upsilon} = dM_{\upsilon}X_{r,\upsilon}\dot{e}_{r}(t).
\end{equation}

The heating occurs through a combination of $\beta$-decays, $\alpha$-decays and fission \citep{Metzger:2010,Barnes:2016umi,Hotokezaka:2017dbk}. These decay products them thermalize with an effciency $\varepsilon_{th,\upsilon}$, depending on interactions between them and thermal plasma.
Thus, neutrinos can freely escape the ejecta. Very high energy photos, gamma rays, are also free after about $\sim 1$~day as the Klein-Nishina opacity decreases \citep{Hotokezaka:2017dbk,Barnes:2016umi}.
The $\alpha$ and $\beta$ particles however interact efficiently with the matter via ionization \citep{Barnes:2016umi} and Coulumb scattering \cite{Metzger:2010}.
\red{Here it repeats the Barns et al findings on thermalization efficiency}
For a fixed energy $\alpha$-particles thermalzie more efficiently, then $\beta$-particles.
For charged particles, the process depends on the magnetic field strength and configuration \citep{Barnes:2016umi}.
Additionally, if actinides are produced in \rproc{}, their decay, involving $\alpha$-particles allows for high thermalization efficiency. \red{double check with before, Barnes}. Thus, nuclear phsyics input, that determins the amount of actinides, have an effect on the thermalization efficinecy of \rproc{} elements in the ejecta.

For a neutron-rich ejecta, $Y_e\leq0.2$, the heating rate is dominated by a large statistical ensemble of nuclei, and the following can be assumed \citep{Korobkin:2012uy},

\begin{equation}
\dot{e}_r = 4\times 10^{18} \varepsilon_{th,\upsilon}(0.5 - \pi^{-1} \arctan[(t-t_0)/\sigma])^{1.3} \text{ erg } \text{s}^{-1} \text{g}^{-1}
\end{equation}

Here $t_0=1.3$~s, $\sigma=0.11$~s constants. The $\varepsilon_{th,m}$ is the thermalisation efficiency.

The heating rate prescibed by this equation has first a constant segment, $\propto1$~s, (depletion of free neutrons by $r$-process) and a decrease segment $\propto t^{-1.3}$, (when heavy nuclei decay to stability) \citep{Metzger:2010,Roberts:2011}. 
Notably, at higher $Y_e$, the nuclear heating is doinated by specific nuclei and has a complex form.

It was shown however, that on a timescale relevant for \ac{kN}, and $Y_e$ present in \ac{BNS} ($Y_e\leq0.4$), the heating rate can be assumed constant within the accuracy of a few.
(see \eg, Fig.~7 in \citet{Lippuner:2015gwa}).

The dependency of $\dot{Q}_{r.\upsilon}$ on the nuclear physics models was shown to be week, unlike the $r$-process abundances themselves \citep{Eichler:2014kma,Wu:2016pnw,Mumpower:2015ova}

Regarding the thermalization efficiency of the energy released, \citep{Barnes:2016umi} provides a recepi that sets the $\varepsilon_{th,\upsilon}$ decreasing from $\sim0.5$ at around $1$~day to $\sim0.1$ at around $1$~week. 

\begin{equation}
\varepsilon_{th,\upsilon}(t) = 0.36 \Bigg[ \exp(-a_{\upsilon}, t_{day}) + \frac{\ln(1+2b_{\upsilon} t_{day}^{d_{\upsilon}})}{2b_{\upsilon}t_{day}^{d_{\upsilon}}} \Bigg]
\end{equation}

where $t_{day}$ is the time in days, ${a_{\upsilon}, b_{\upsilon}, d_{\upsilon}}$ are the constants that depend on the ejecta layer properties, mass and velocity. 


\subsection{\ac{kN} properties}

From a simple, toy model \citep{Metzger:2016pju}, the ejecta properties can be translated 
into the properties of light curves. 

Consider the case where the lanthanides-rich ejecta is present, \eg, in tidal outflows of \ac{BNS} and \ac{NSBH} mergers.
There a toy model predicts a light curve from such low-$Y_e$ ejecta that peaks in \ac{NIR} (in relatively good agreement with radiation transport model of \cite{Barnes:2016umi}) on a timescale of several days (week). This is so-called ``red \ac{kN}''.
The disagreement with radiation transfer models most noticeable in the post-peak period, where toy model predicts sharp decay, while the radiation transport models predict smooth decline. The reason for it is the toy model's assumption of optically thick black-body emission. As ejecta expands cools and become optically thin, this assumption breaks down.
It is important to note, that the toy model does not take into account other emergent sources of opacity at late times, such as clumping, dust formation, photo-ionization from central engine. These may smooth the post-peak light curves.

Next, consider the ejecta with high electron fraction, \eg, \nwind, or \ac{SWW}, reprocessed by neutrinos.
Such ejecta would have negligible amount of elements of lanthanides group \citep{Metzger:2014ila} and thus have a different EM signature.
The emission from such ejecta peaks in optical/R,I bands on a time scale of days. It is $2-3$ magnitudes brighter then red \ac{kN}. This component is called ``blue \ac{kN}''
This emission is assumed to be of polar origin and contribute to the total \ac{EM} signature of the BNS ejecta. 


\subsection{Other \ac{EM} counterparts}
\red{commented}
%\subsubsection{Free Neutron Precursor}
%
%The sufficiently high density and low veloicyt of the bulk of the ejecta assures that there is enough time for the $r$
%-process to remove free neutrons. However, a small fraction of the ejecta was shown to have hight enough velocity to retain its free neutrins and escape the dens slow part, \textit{e.g.,} \cite{(Bauswein et al 2013a)}. The origin of this component is the shock-heated intefrace between two neutron stars as they collide. 
%The outer layers of the ejecta then can be \red{superheated} by this \red{'neutron skin'}, modifying the Kiloniva signal. \cite{(Metzger et al 2015a; Lippuner and Roberts 2015)}
%
%Here we consider such ejecta, that contains free neutrons. 
%Consider layer $dM_{\upsilon}$, that contain a fraction $X_{n;\upsilon}$ of free neutons, specific heating rate of which $\dot{e}_n(t)$. Then, the heating rate in the layer reads 
%
%\begin{equation}
%\dot{Q}_{r;\upsilon} = dM_{\upsilon} X_{n,\upsilon}\dot{e}_n(t).
%\end{equation}
%
%The initial mass fraction of neutrons $X_{n,\upsilon}$ is defined as 
%
%\begin{equation}
%X_{n,\upsilon} = \frac{2}{\pi}(1 - Y_e)\arctan\Big(\frac{M_{n}}{M_{\upsilon}}\Big),
%\end{equation}
%
%is an arbitrary assumed interpolation between the neutron rich ($M\ll M_{n}$) inner layers with $X_{n} = 1-2Y_e$ and neutron-free ($M\gg M_n$) layers.
%
%Assuming the averabe neutron half-0ife of $900$~s, the specific heating rate $\dot{e}_{n}$ is 
%
%\begin{equation}
%\dot{e}_n = 3.2 \times 10^{14} \exp[-t/\tau_n] \text{ erg } \text{s}^{-1} \text{g}^{-1},
%\end{equation}
%
%Simultaneously, as fraction of free neutrons increases in outermost layrs, the fraction of $r$-process elements decreases as $X_{r,\upsilon} = 1 - X_{n.\upsilon}$, which has to be accounted for in Eq.~\eqref{eq:theory:mkn:energ}.
%
%The effect of free neutrons on Kilonova lightcurves is the following.
%Even for a very small mass, $\sim 10^{-4}M_{\odot}$ of freen-nutron ejecta, with $Y_e\sim0.1$, the UVR luminocities are increased considerably, in the first hours after merger.
%The reason for that is, $\dot{e}_{n} > \dot{e}_r$ by at loeast an order of magnitude on a tiemscales up to 1 hour postmerger. Also, this timescale is comaprable with the diffusion time scale for the neutron mass layer. 
%
%\begin{equation}
%t_{peak,\upsilon} \approx \Bigg(\frac{M_{\upsilon}^{4/3}\kappa_{\upsilon}}{4\pi M^{1/3}\upsilon_0 c}\Bigg)^{1/2} \sim 3.7 \text{ hours}.
%\end{equation}
%
%And 
%
%\begin{equation}
%L_{peak} \approx \frac{E_n \tau_n}{t_{peak;\upsilon}^2} \propto 3\times10^{41}
%\end{equation}
%
%which is insensitive to the mass of the layer itself. 
%This emission is expected to peak in optical/UV band due to the high ejecta temperature during the first hours after merger. 
%
%%%
%
%\subsection{Engine Power}
%
%Additional heating for kilonova might come from the object, left after the merger. In case of BNS it might be the MNS. In case of NSBH it is a BH with accretion disk. This is expected to make Kilonova more luminous than what $r$-porcess products decay might produce.
%
%A large fraction of Short GRB, $\sim(15-25)\%$ is followed by the prolonged ($10-100$~s) hump of $X$-ray emission, \cite{Norris and Bonnell 2006; Perley et al 2009 Kagawa et al 2015}
%
%It is however uncertain, how much energy does the central engine provides. Here some examples are considered.
%
%\subsubsection{Fall-Back Accretion}
%
%A merger leaves a finite amount of mass bound gravitationally to the central object, that falls back of a time timescale of seconds to days \cite{(Rosswog 2007; Rossi and Begelman 2009; Chawla et al 2010; Kyutoku et al 2015)}.
%
%\gray{This is a part of the outflow that was not energetic enough to leave the system. It eventually falls back on the central object. This is not the disk itself...}
%
%The rate of fall-back at late timnes $t\gg 1$~s can be approximated by a power law 
%
%\begin{equation}
%\dot{M}_{fb} \approx \Bigg( \frac{\dot{M}_{fb}(t=0.1~\text{s})}{10^{-3}M_{\odot}\text{s}^{-1}} \Bigg) \Bigg( \frac{t}{0.1 \text{s}} \Bigg)^{-5/3}
%\end{equation}
%
%The value $10^{-3}M_{\odot}\text{s}^{-1}$ is the normalization chosen for BNS. For NSBH it be different by an order of magnitude \cite{(Rosswog 2007)}. 
%
%Notably, the fall-back of the material removed on a dynamical timescales, can be stalled by the continous winds from the disk \cite{(Fernandez et al 2015b)}
%
%Additionally an onset of $r$-process heating in the disk might provide an additional source of outflow that would stall the fall-back on a seconds to minutes timescale \cite{Metzger et al 2010a)}.
%
%On a longer timescale, days to weeks, there seems to be no mechansm that can suppress the fall back completely. This it might still be a relevant source of energy for Kilonva.
%
%The matter that reaches the central objects accrets. This is super-Eddington accetion that releases energy, $L_{acc} \propto \dot{M}_{fb} c^2$ that can heat the ejecta and enhance  the Kilonova emission. Additionally, the accretion might result in the formation of a relativistc jet (similar to GRB) that might account for the extended $X$-ray emission that sometiems follow the GRB.
%As accretion flow subsides, the jet power decreases and it becomes unstable to the magnetic Kink instability \cite{(Bromberg and Tchekhovskoy 2016)}. Then the energy is dissipated pramarely via heating up the ejecta, by magnetic reconnections instead of non-thermal emission. 
%
%The fall-back acctretion can power a mildly relativist, wide-angle disk wind. As the wind collides with the (ejected prior) ejecta shells, its energy thermalizes. 
%
%Overall, the ejecta heating rate due to fall-back accretion can be described as 
%
%\begin{equation}
%\dot{Q}_{fb} = \varepsilon_{j}\dot{M}_{fb}c^2
%\end{equation}
%
%where $\varepsilon_{j}$ is the jet/disk wind efficiency factor. See \cite{Tchekhovskoy et al 2011). Kisaka and Ioka (2015)} for the discussion of efficiency.
%
%For instace the 130603B, was detected with an \ac{NIR} excess. It was initially attributed t othe radiactive heating \cite{Tanvir et al (2013)} \cite{Berger et al (2013)}. On the contrary, \cite{Kisaka et al (2016)} suggested that it might be attributed to the absorbed and re-emitted (reprocessed) $X$-ray emission. 
%
%%%
%
%\subsubsection{Magnetar Remantns}
%
%The outcome of the NSBH merger is always a black hole. Meanwhile an outcome of the BNS merger depends sensetively on the maximum allowed mass for a non-rotating NS ($M_{max}(\Omega=0)$). 
%This value is bounded, \textit{e.g.,} $\geq 2M_{\odot}$ \cite{(Demorest et al 2010; Antoniadis et al 2013)} and $< 3M_{\odot}$, where the upper limit is given by the casuality constrants on the EOS. 
%Withing this boundaries the fate of the remnant is uncertain. Incidently, the observations shows that NS has mass $\sim 1.4M_{\odot}$. Merger of two of this objects thus result in a remnant of mass $\sim2.5M_{\odot}$ ($\approx7.5\%$ of the mass was lost to GW and neutrinos \cite{(Timmes et al 1996)}). If the resulting mass is lower then $M_{max}(\Omega=0)$, it promptly collapses. Otherwise a stable (short- or long-lived) remnant can be formed.
%
%Consider a rotating remnant. An upper limit on a rotating object, is the object that is rotating close to the mass-shedding limit. 
%
%Given the remnant's moment of inertia $I$ and \red{angular velocity} $\Omega$, then rotational period $P = 2\pi / \Omega$. 
%
%Such object has energy 
%
%\begin{equation}
%E_{rot} = \frac{1}{2}I\Omega^2 \approx 10^{53} \Big(\frac{I}{I_{LS}}\Big)\Big(\frac{M_{ns}}{2.3M_{\odot}}\Big)^{3/2}\Big(\frac{P}{0.7\text{ms}}\Big)^{-2} \text{ ergs }
%\end{equation}
%
%Here the remnatn's moment of inertial, $I$ is normmalized to $I_{LS} \approx 1.3\times 10^{45}(M_{NS}/1.4M_{\odot})^{3/2}$ g cm$^{2}$ (motivated by Fig.1 \cite{Lattimer and Schutz (2005}).
%
%This energy exceeds by a factor of $10^3$ the ejecta kinetic energy of radioactive decay energy.
%If this energy can be extracted via channels other then GW (\textit{e.g.,} EM torques), then the EM signal accompanying the merger would be significantly enhanced, \cite{(Gao et al 2013; Metzger and Piro 2014; Gao et al 2015; Siegel and Ciolfi 2016a)}. For a remnant that is supported by the differential roatation, only a part of the ritational energy is availalbe (as the rmenant would eventually collapse loosing it). The Fig.8 shows the dependency of the \textit{extractable} rotational energy as a function of the remnants mass.
%
%The electromagnetic tourques allows to extract the totatinal energy from a remnant with strong magnetic fields. Such fies are expected for the merging NS, due to amplifications, reaching values found in galactic magnetars \cite{(Price and Rosswog 2006; Zrake and Mac-Fadyen 2013; Kiuchi et al 2014)}. However, this amplification occures at small scats and at early times post merger, producing a complex field topology, that evolves with time \cite{(Siegel et al 2014)}. The magnetic field strength at the end of the differential rotation phase is however uncertain. There are speculations that it might remain at $10^{15-16}$~G.
%
%Consider an aligned dipole rotator (different from a vacuum the vacuum dipole). Its spin-down luminocity is \cite{Spitkovsky (2006); Philippov et al (2015)} 
%
%\begin{equation}
%L_{sd} = 
%\begin{cases}
%\frac{\mu^2 \Omega^4}{c^3} = \frac{(B R_{ns}^3)^2 \Omega^4}{c^3} &\text{ if } t< t_{coll} \\
%0 &\text{ if } t> t_{coll}
%\end{cases}
%\end{equation}
%
%The charactersitic 'spin-down timescale' over which an order of unity fraction of the rotational energy is removed is 
%
%\begin{equation}
%t_{sd} = \frac{E_{rot}}{L_{sd}}\Bigg|_{t=0}
%\end{equation}
%
%which is of an order of $\sim 150$~ms for a remnant of the mass $M=2.3M_{\odot}$, $I=I_{LS}$, $B=10^{15}$~G and $P_0=0.7$~ms, 
%
%where $P_0$ is the initial spin-period.
%The mass-shedding limit of this remnant is $P=0.7$~ms. 
%
%The lifetime of the unstable remanant can be estimated as 
%
%\begin{equation}
%L_{extract} = \int_0^{t_{coll}} L_{sd} dt
%\end{equation}
%
%where $t_{coll}$ is the time of the collpase, that marks olse the end of the extraction of the rotational energy. $L_{extract}$ is the total amount of energy extracted from rotation.
%The $t_{coll}$ falls rapidly with the remnant mass, after it passes the stale NS upper limit.
%
%Long-lived magnetar can power the 'prompt-like' X-ray emission (found in sGRB \textit{e.g.,} \cite{(Gao and Fan 2006)Metzger et al (2008b); Bucciantini et al (2012)} ). Additionally, the sGRB with extended emission were explaiend by phenomenological models of magnetar spind-down \cite{(Gompertz et al 2013)}. The observed X-ray and optical plateus were discusssed in \cite{(Rowlinson et al 2010, 2013; Gompertz et al 2015)} and the late-time excess emission was adressed in \cite{(Fan et al 2013; Fong et al 2014a)}. Notably, all models requrie rather large magnetic fields of $\sim 10^{16}$~G.
%
%The formation of the jet and sGRB is subjected to uncertainties. 
%It was argues that magnetar model is not viable due to heavy baryonic pollution in the polar region above the surface \cite{(Murguia-Berthier et al 2014, 2016).}. This led to the develpment of the model, in which GRB is generated after the remnant collapse to a BH, which might happen minutes after the merger. And while this still allow to explain the extended X-ray emission (magnetar spin-down and radiation diffussion through the ejecta). 
%However, if in spin-down the remnant raches a solid-body rotation, a collapse of such a remnatn is not predicted to leave a massive disk, sufficient to power the GRB \cite{Margalit et al (2015)}. The disk that was formed after the merger is expected to be either accreted or spread out (via ) too much for short GRB to be generated. 
%
%There are observational evidences that BH is not mandatory for producing a jet. For instance, the (e.g., Circinus X-1; \cite{Fender et al 2004}), galactic acretring NS.
%While indeed the region above the neutron star is polluted by neutrino-driven wind on a time scale of seconds postmerger \cite{(Dessart et al 2009; Murguia-Berthier et al 2014, 2016)}, the expected strong magnetic field $B\gg 10^{15}$~G, small scale magnetic flux bundles (that dominate dynamically over the thermal or ram pressure of the wind) could confine the plasma \cite{(Thompson 2003)}. Then, originating from the disk open field lines, carrying the Poyntim flux of the GRB jet, would be relatively free of baryonic matter due to centrifugal barrier. 
%Note, that sheer rotation of the NS would result in periodicity in openning of the polar field lines. This, in turn, might lead to a variability in the transinet (without requiring baryon pollution at all). 
%The presence of the NS \textit{after} the GRB is suported by observations: extended X-ray emission that does not follow the model of the fall-back accreting BH. 
%\gray{the early X-ray varaiablility is sometimes attributed to the afterglow phase, \cite{(Holcomb et al 2014)}, but it is too rapid for a foward or reverse shocks}
%
%Magnetic spind-down power, injected into the merger ejecta (behind it) could enhance the Kilonova emission (\cite{Yu et al (2013)}). Similar mechanism was considered for the SLSN \cite{(Kasen and Bildsten 2010; Woosley 2010; Metzger et al 2014)}. This in essence, reminds one of a fall-black powered emission considered before.
%
%A pulsar injects a relativistc wind of $e^{\pm}$ pairs into the surrounding environment (\textit{e.g.,} Crab Nebula). Near the termination shock, wind undegoes the shock dissipation, forming the so-called 'magnetar wind nebulae' of relativistic particles \cite{(Kennel and Coroniti 1984)}. The high density of the BNS merger environment assures a rapid cooling of these pairs (via synchrotron or inverse-compton emission) \cite{(Metzger et al 2014; Siegel and Ciolfi2016a,b)} generating the broadband emission (akin the emission from pulsar wind nebulae \textit{e.g.,} \cite{Gaensler and Slane 2006)}). The inner walls of the expanding ejecta would absorb, UV and X-ray photons, reprocess and emit in optical/IR \cite{(Metzger et al 2014)} contributing and enahncing Kilonova.
%
%Notably, the magnetar wind-nebulae emission does not necessarly undergoes thermalization within the ejecta. If the spectral windows allow, \textit{e.g.,} for instance for hard X-ray
%\footnote{where the bound-free transitions lie at lower energies. Additionally this is possible for hight enerngy $\gg$ MeV photons, that fall into the gap between declining Klein-Nishina cross-section and before the rise of $\gamma-\gamma$ opacities}
%, the emission will escape the ejecta without being reprocessed.
%Additionally, low-mass ejecta can undergo complete ionsiation and allow even lowere energies photones to pass without thermalization. This 'leaking radiation' might be an important EM signal to mergers \cite{(Metzger and Piro 2014; Siegel and Ciolfi 2016a,b; Wang et al 2016).}. 
%
%consider the ejecta heating rate provided by the magnetar spin-down as 
%
%\begin{equation}
%\dot{Q}_{sd} = \varepsilon_{th}L_{sd}
%\end{equation}
%
%where $\varepsilon_{th}$ is the thermalization efficiency, that ranges between $1$ when the ejecta is very opaque (hearly times) to a low value, for low opacities.
%
%Notably, there is another sink for spin-down radiation that is of it utmost importance at early times, where high energy $\gamma$-rays are present in the nebula behind the ejecta \cite{Metzger and Piro (2014)}. These $\gamma$-rays create $e^{\pm}$ pairs (when compactness of the cloud is high). Coming in 'seed photons' can then be compton up-scattered on these particles, becoming energetic enought to produce a new $e^{\pm}$ pair. This initializes a 'pair cascade'. High fraction ($\leq 0.1$) of puslar spin-down power $L_{sd}$ falls into the rest pass of the $e^{\pm}$ pairs \cite{(Svensson 1987; Lightman et al 1987)}. Hence, for the spind-down radaition to reach (and theramlize within) the ejecta, it must diffuse through the 'pair cloud', experiencing $PdV$ adiabatic losses.
%To paramterize the effect, introduce the Thompson optical depth of the pair cloud $\tau_{es}^n$. If This optical depth exceeds the optical depth of the ejecta itslef, then only a fraction of the actual magnetar spin-down power can be thermalzied within the ejecta. 
%This effect of 'pair cloud' can be approximated by suppressing the observed luminosity. Floowing \cite{Metzger and Piro (2014) and Kasen et al (2015),} 
%
%\begin{equation}
%L_{obs} = \frac{L}{1 + (t_{life}/t)}
%\end{equation}
%
%where $L$ is the Kilonova luminocity, computed from the energy equation \eqref{eq:theory:mkn:energ} (with magnetar heat source).
%The $t_{life}/t$ is the caracteristic 'lifetime' of a non-thermal photon in the nebula, relative to the ejecta expansion timescale, written as
%
%\begin{equation}
%\frac{t_{life}}{t} = \frac{\tau_{es}^{n} \upsilon}{c(1 - A)}
%\end{equation}
%
%where $\tau_{es}^n\propto Y L_{sd}$ and $A$ is the frequency averaged albedo of the ejecta ($A\propto 0.5$).
%
%Overall, the pair trapping is able to reduce the effective luminocity of the magnetar powered kilonova by several orders of magnitude (due to reduce thermalization efficenty) at early times.
%
%Energy input from the magnetar spind down, can in itself raise the observed peak luminocities. Note however, that in case of the only temporarly stable remnant, the energy import would be terminated at collapse.
%
%%%
%
%\subsection{Implications}
%
%sGRB is a good smoking gun for Kilonova searches.
%However, it, and its afterglow should not outshine the Kilonova. For instance, in GRB 130603B \cite{(Berger et al 2013; Tanvir et al 2013)} the observed \ac{NIR} excess would require ejecta of $0.05-0.1M_{\odot}$ to be explaiend. This is generally too high for dynamical ejecta only \cite{(Hotokezaka et al 2013b; Tanaka et al 2014; Kawaguchi et al 2016).}, but might be achieved with winds from the disk and remnant \cite{(Metzger and Fernandez 2014)} see also \cite{Kasen et al 2015)}. However, high observed high luminocity might not be a result of radioactive heating alone, but hits towards the contribution from the central engine, fall-back accretion or spin-down luminocity.
%
%Discussion on how different properties of the Kilonova affect detection possibilities and different biasas might araise.
%
%\red{This might serve as a gread introduction to the thesis!}

\subsection{\AT{}}

The observation of the kilonova (kN) AT2017gfo \citep{Coulter:2017wya,Chornock:2017sdf,Nicholl:2017ahq,Cowperthwaite:2017dyu,Tanvir:2017pws,Tanaka:2017qxj}
associated to the binary neutron star (BNS) merger GW170817 \citep{TheLIGOScientific:2017qsa} provided
evidence 
that the ejection of neutron-rich matter from compact binary mergers is a primary site for
$r$-process
nucleosynthesis~\citep{Lattimer:1974a,Li:1998bw,Kulkarni:2005jw,Rosswog:2005su,Metzger:2010sy,Roberts:2011xz,Kasen:2013xka}.
%% ---
The \ac{kN} is the \ac{EM} \ac{UV}/optical/\ac{NIR} transient, powered by the 
radioactive decay of the elements synthesized via \rproc{} in expanding ejecta.
%% --- 
The peak in \ac{NIR} band of \AT{} occurred several days after the
merger \citep{Chornock:2017sdf}.
It is generally consistent with emission from the high opacity material, with with a sizable fraction of lanthanides \citep{Kasen:2013xka}
%% ---
The peak in \ac{UV}/optical band, however, occurred in less than one day after the
merger \citep{Nicholl:2017ahq}. 
This requires the opacity of the material to be sufficiently low, suggesting  that only partial \rproc{} \nuc{} took place in the corresponding ejecta  \citep{Martin:2015hxa}.
%% ---
Smianalytic two-components (red and blue) spherical model \ac{kN} models to the \AT{} observations provided estimates for the ejecta properties for these two components. 
Specifically, for the langhinide poor (rich) \ie, blue (red) components, the required mass is $2.5\times10^{-2}M_{\odot}$ ($5.0\times10^{-2}M_{\odot}$) and velocity $0.27$c ($0.15$c)
\citep{Cowperthwaite:2017dyu,Villar:2017wcc}
(See however \citep{Waxman:2017sqv} for an alternative interpretation.)
%% ---
Similar estimates are obtained with 1D radiation transport \ac{kN} models
\citep{Tanvir:2017pws,Tanaka:2017qxj}.



%% =====================================================================================
%%
%%               M E T H O D
%%
%% =====================================================================================



\section{Method}

We compute the \ac{kN} \acp{LC} using the multicomponent, anisotropic semi-analytical \texttt{MKN} model first introduced in \citet{Perego:2017wtu} and largely based on the \ac{kN} models presented in \citet{Grossman:2013lqa} and \citet{Martin:2015hxa}.
%% ---
The geometry of the ejecta is directly imported from the 
\ac{NR} simulations in a form of the angular profiles with 
respect to the polar angle and averaging over the azimuthal angle.
In a several cases we also use the analytical ejecta profiles, with either smooth of step-like dependency on the polar angle.
%% ---
The viewing angle $\theta_{\text{obs}}$ is measured as the angle between the 
polar axis and the line of sight of the observer.
%% ---
Each ejecta component (\eg, \ac{DE}, \ac{SWW}) is described through 
the angular distribution of its ejected mass, $M_{\text{ej}}(\theta)$, 
velocity $\upsilon_{\text{ej}}(\theta)$,
and opacity, $\kappa_{\text{ej}}(\theta)$.
%% ---
The polar angle $\theta$, measured from the rotational axis is discretized in $N_\theta=30$ angular bins evenly spaced in $\cos{\theta}$.
%% ---
Additionally, within each ray, the matter 
has a fixed velocity distribution, $\xi(\upsilon)$ such that $\xi(\upsilon) \propto (1 - \left(\upsilon/\upsilon_{\text{max}}\right)^{2})^{3}$, where $\xi(\upsilon) \dd \upsilon$ is the matter contained in an infinitesimal layer of speed $\left[\upsilon,\upsilon+\dd \upsilon\right]$, and $\upsilon_{\text{max}}=\upsilon_{\text{max}}(\upsilon_{\text{RMS}})$ is the maximum velocity at the outermost edge of the component.
%% ---
The characteristic quantities $\varrho$, $\upsilon$ and $\kappa$ are then evaluated for every bin according to the assumed input profiles.
For every bin, we estimate the emitted luminosity using the radial model described in Ref.~\cite{Perego:2017wtu}, and in \S{4} of \citet{Barbieri:2019kli}.
%% ---
The model assumes that the thermal radiation is emitted at the photosphere, located at $R_{\text{ph}}$, with 
effective emission temperature $T_{\text{eff}}$ via the 
Stefan-Boltzmann law, \ie, the black-body radiation. 
This assumption is justified to model the emission from the early time ejecta, when it is hot and opaque. However, at later times, \red{as ejecta falls out of \ac{LTE} with its radiation, the assumption breaks.}.

The time-dependent nuclear heating
rate $\epsilon_{\text{nuc}}$ \red{assert notations} entering these calculations is approximated by an analytic fitting formula, derived from detailed nucleosynthesis calculations~\cite{Korobkin:2012uy},

\begin{equation}
\label{eq:epsnuc}
\epsilon_{\text{nuc}}(t)= \epsilon_0 \, \frac{\epsilon_{\text{th}}(t)}{0.5} \, \epsilon_{\text{nr}}(t) \,\left[ \frac{1}{2} - \frac{1}{\pi} \arctan\left(\frac{t-t_0}{\sigma}\right)\right]^{\alpha}\,,
\end{equation}

where $\sigma = 0.11$~s, $t_0 = 1.3$~s, $\alpha=1.3$ and $\epsilon_{\text{th}}(t)$ is the thermalization 
efficiency tabulated according to \citet{Barnes:2016umi}.
\red{$\epsilon_0$ is not introduced. Is it a constant?}
%% --- 
The heating factor $\epsilon_{\text{nr}}(t) $ is introduced as in \citet{Perego:2017wtu} to roughly adjust the Eq.~\eqref{eq:epsnuc} in the regime of mildly neutron-rich matter (characterized by an initial electron fraction $Y_e \gtrsim 0.25$), \citep[see, \eg][]{Martin:2015hxa}:

\begin{equation}
\label{eq:epsnr}
\epsilon_{\text{nr}}(t,\kappa) = \left[1-w(\kappa)\right] + w(\kappa)\,\epsilon_{Y_e}(t)\,,  
\end{equation}

where $w(\kappa)$ is a logarithmic smooth clump function such that $w(\kappa < 1~\igscm) = 1$ and 
$w(\kappa > 10~\igscm)=0$ and the factor $\epsilon_{Y_e}(t)$ accounts for the dependency on $Y_e$:
if $Y_e < 0.25$, then $\epsilon_{Y_e}(t)=1$, otherwise, when $Y_e \ge 0.25$,

\begin{equation}
\label{eq:epsye}
\epsilon_{Y_e}(t) =\epsilon_{\text{min}}+{\epsilon_{\text{max}}}{\left[1+ e ^{4(t/t_\epsilon-1)}\right]}^{-1}\,,
\end{equation}

where $t_\epsilon = 1~{\text{day}}$, $\epsilon_{\text{min}}=0.5$ an $\epsilon_{\text{max}} = 2.5$.

The opacity of an atom depends on its ionization state \red{might reference Barns work}. When atoms become neutral, the photon opacity sharply decreases. This was shown to have a stron effect on \acp{LC} in high-frequency bands, \eg, $V$, $U$, $B$ and $g$ \citep{Villar:2017oya}.
In order to account for the drop in opacity when the temperature falls below the certain value that corresponds to the full recombination, the $T_{\text{floor}}$ \cite{Kasen:2017sxr,Kasen:2018drm}, 
this is set as a minimum value $T_{\text{eff}}$.
Due to the presence of lanthanides in the ejcta, two floor 
temperatures are introduce depending on the composition,
the $T_{\text{floot}}^{\text{Ni}}$ and $T_{\text{floot}}^{\text{La}}$ for the lanthanides free and richi ejecta respectively.


The emission coming from the different bins is combined to obtain the spectral flux at the observer location:

\begin{equation}
\label{eq:spectral_flux}
F_{\nu}(\mathbf{n},t) = \int_{\mathbf{n}_{\Omega} \cdot \mathbf{n}> 0} \left( \frac{R_{\text{ph}}(\Omega,t)}{D_L} \right)^2  B_{\nu}(T_{\text{eff}}(\Omega,t))~\mathbf{n} \cdot  \dd\boldsymbol{\Omega} 
\end{equation}

where $\mathbf{n}$ is the unitary vector along the line of sight, $\mathbf{n}_{\Omega}$ is the unitary vector spanning the solid angle $\Omega$, $D_L$ is the luminosity distance, $R_{\text{ph}}$ is the local radial coordinate of the photospheric surface, and $B_{\nu}(T_{\rm eff})$ is the spectral radiance at frequency $\nu$ for a surface of temperature $T_{\text{eff}}$

We also make use of the apparent AB magnitude mag$_b$ in a given photometric band $b$, defined as:

\begin{equation}
\label{eq:mag}
\text{mag}_b(\mathbf{n},t) = -2.5 \log_{10}\left( F_{\nu_b}(\mathbf{n},t) \right)-48.6\,,
\end{equation}
where $\nu_b$ is the effective central frequency of band $b$.


%% \section{Results} %% \subsection{Multi-Component Model}

\begin{figure}
    \centering 
    \includegraphics[width=0.48\textwidth]{profiles_op.pdf}
    \caption{Graphic representation of the analyzed
        ejecta profiles for isotropic and anisotropic cases
        from an azimuthal perspective and for a fixed moment of time.
        The black dot represents the remnant and the dashed line is the projected orbital
        plane of the binary. The shadowed areas describe the ejecta profiles: the shape
        characterizes the mass distribution, while the colors refer to 
        the prior assumptions on the opacity parameter.
        In particular, blue regions denote opacities lower than $5~\igscm$,
        red regions refer to opacities greater than $5~\igscm$,
        and oranges areas indicate a broadly distributed opacity.
        All shells are isotropically expanding with a constant velocity.
        (Adapted from \citet{Breschi:2021wzr})
    }
    \label{fig:cartoon}
\end{figure}


%% =====================================================================================
%%
%%               R E S U L T S
%%
%% =====================================================================================

In the following we perform two types of the analysis (i) the fully \ac{NR}-informed \ac{kN} models, where the the only ejecta components considered, are those found in simulations and (ii) the parametric \ac{kN} models, where we use the analytic ejecta profiles, with the parameters obtained from fitting formulae (see Ch.~\ref{ch:stat_anal}).


\section{\ac{SWW} for the blue component of \AT{}}

The \ac{BNS} merger ejecta consists of several anisotropic components.
Hence, the \ac{kN} model ought to account for the anisotropy of the ejecta composition. Additionally, the interaction between components needs to be included, but we leave this to future works.
%% ---
Indeed, outflow properties inferred for \AT{} using multi-components and 2D \ac{kN} models including the ejecta anisotropy and cross-component irradiation are broadly compatible with the results from simulations, \eg, \citep{Perego:2017wtu,Kawaguchi:2018ptg}.
%% ---
The particular challenge however, is to reproduce the early blue emission. 
Both semi-analytical and radiation transport
models require ejecta properties different from those found in
simulations. In particular, 
this component requires fast low opacity, massive ejecta \citep{Fahlman:2018llv} that is generally 
not found in \ac{BNS} simulations.
%% ---
Notably, the early blue component can be explained by the emission arising 
in the interaction between a relativistic jet and the ejecta
\citep{Lazzati:2016yxl,Bromberg:2017crh,Piro:2017ayh}.
However, simulations show that that successful jets do not deposit a sufficient amount of thermal energy in the ejecta for this mechanism to work \citep{Duffell:2018iig}. 
Other possibilities include the presence of highly magnetized winds \citep{Metzger:2018uni,Fernandez:2018kax},
or the presence of the so-called viscous-\ac{DE} \citep{Radice:2018ghv}.
These two explanations require the development of large-scale strong magnetic fields.
%% ---
In this section we show that the presence of the massive \ac{SWW} can help explain the early blue emission, relaxing the need for strong ordered magnetic fields. 

\begin{figure}[t]
    \centering
    \includegraphics[width=0.49\textwidth]{kilonova/mkn_dd2_band.pdf}
    \caption{Bolometric kN light curves in three representative bands from blue to
        infrared for the two simulations with turbulence viscosity compared to
        \AT{} data from~\citep{Villar:2017wcc}.
        The color gradient is the effect related to different
        \ac{SWW} masses, that suggests possible variations of the light
        curves for different \ac{BNS}. The band is computed by extracting the
        \ac{SWW} mass from DD2 every $10$~ms until the end of the simulation, and
        then by linearly extrapolating the data to $250$~ms.
        (Adapted from \citet{Nedora:2019jhl})
    }
    \label{fig:knlc}
\end{figure}

To this end we consider two equal mass \ac{NR} \ac{BNS} models with LS220 and DD2 \acp{EOS}, that produce short and long-lived \ac{MNS} remnant respectively. 
%% ---
For both models we produce \ac{NR}-informed \acp{LC}, using both the \ac{DE} and \ac{SWW} for the model with the long-lived \ac{MNS} remnant, and \ac{DE} only for the model with the short-lived one.
%% ---
The result is shown in Fig.~\ref{fig:knlc}.
We observe, that the \ac{DE} only is insufficient to explain the \AT{} both high and low frequency bands irrespective of the model. However, the informed by both \ac{DE} and \ac{SWW} \ac{LC} of the model with DD2 \ac{EOS} is sufficiently bright to explain the emission in 'z' band. 
%% However, it appears dimmer than the early blue \AT{} emission in 'g' band. 
%% ---
The emission in low frequency bands requires a more massive 
\ac{SWW} with mass ${\gtrsim} 2\times10^{-2}M_{\odot}$, implying a remnant lifetime of ${\gtrsim}200$~ms, as the \ac{SWW} mass flux is present as long 
as \ac{MNS} remnant is present (see Sec.\ref{sec:bns:dynamics:sww}).
However, a more massive \ac{SWW} is incompatible with
the early emission for the low-frequency bands of \AT{}.
%% --- 
In order to explain the late emission in the low frequncy bands and 
early emission in high frequency bands, a combination of the \ac{SWW} and 
viscous ejecta from the disintegration of the disk are required.
%% --- 
Our results, however have uncertainties related to our simplified calculation of
the \ac{kN} \acp{LC} which is expected to be less accurate at
late times when absorption features and deviations from \ac{LTE} become more relevant \citep[see \eg][]{Smartt:2017fuw}.
To model the \ac{kN} emission more robustly, the time- and energy-dependent 
photon radiation transport models are required 
\citep{Kasen:2017sxr,Tanaka:2017qxj,Miller:2019dpt,Bulla:2019muo}.
Additionally, the systematic uncertainties in
nuclear (\eg, mass models, fission fragments and $\beta$-decay
rates) and atomic (\eg, detailed wavelength dependent opacities for
\rproc{} element) physics enter all the current \ac{kN} models 
\citep{Eichler:2014kma,Rosswog:2016dhy,Gaigalas:2019ptx}.

\subsection{Conclusion}

\red{Copied}
Standard kN models applied to the early AT2017gfo light curve are in
tension with ab-initio simulations conducted so far.
While alternative interpretations have been proposed, they are either
disfavored by current simulations and observations (e.g. jets) \citep{Bromberg:2017crh,Duffell:2018iig},
or require the presence of large-scale strong magnetic 
fields which might not be formed in the postmerger
\citep{Metzger:2018uni,Fernandez:2018kax,Radice:2018ghv,Ciolfi:2019fie}. 
We identified a robust dynamical mechanism for mass ejection that
explains early-time observations without requiring any fine-tuning.
The resulting nucleosynthesis is complete and produces all
$r$-process elements in proportions similar to solar system abundances.
Methodologically, our work underlines the importance of employing
NR-informed ejecta for the fitting of light-curves.
Further work in this direction should 
include better neutrino-radiation transport and magnetohydrodynamic effects
\citep{Siegel:2017nub,Fujibayashi:2017puw,Radice:2018xqa,Radice:2018pdn,Miller:2019dpt}. 



%% \section{Results}


\section{Fit-informed kilonova}

%% KILONVA PLOTS
\begin{figure*}[t]
    \centering 
    \includegraphics[width=0.49\textwidth]{kilonova/mkn_multiband_dyn_NR.pdf}
    \includegraphics[width=0.49\textwidth]{kilonova/mkn_multiband_dynsec_NR.pdf}
    \caption{
        Comparison between one component light curves (\textit{left panel}) and
        two components light curves (\textit{right panel}) in $g$, $z$ and $K_s$
        bands using direct NR input or the fitting formulae for the
        dynamical ejecta and disk mass. 
        The $y-$axis displays the difference between the peak time (\textit{top panel}), $\Delta t_{\rm peak} = t_{\rm peak; NR} - t_{\rm peak; fit}$, and peak magnitude, $\Delta m_{\rm peak} = m_{\rm peak; NR} - m_{\rm peak; fit}$, (\textit{bottom panel});
        the $x-$axis shows selected BNS models of \DSrefset{}.
        The fits employed here are the polynomials in $(q,\tilde{\Lambda})$ used with the 
        best fitting coefficients, calibrated to \DSheatcool{} (that includes \DSrefset{}).
        The plot shows that 
        the light curves generated with the dynamical ejecta fits (one
        component) tend to underestimate the peak times and magnitudes
        of NR-informed light curves, especially in the $K_s$ band. In case of dynamical ejecta and disk wind (two
        components) light curves, the peak
        time is less constrained ($\pm 2$~days) in the $K_s$ band, but the
        peak magnitudes is predicted more accurately $\pm0.5$~mag. }
    %% \vn{Note! That for NR-informed lightcurves \textbf{full} NR ejecta profile (for dyn. ej.)
    %% is fed into MKN. Thus here the geometric effects and poor fit performance contribute to the
    %% qdeviation.}
    \label{fig:mkn_example}
\end{figure*}




