% Chapter 1

\chapter{Introduction} % Main chapter title

\label{Chapter1} % For referencing the chapter elsewhere, use \ref{Chapter1} 

%----------------------------------------------------------------------------------------


\red{biased for Shibata papers [from his 2019 paper with hotokezaka]}

Mergers if neutron star binaries (binary neutron stars (BNS)) and neutron star-black hole 
(NSBH) are one of the most promising sources of graviational waves (GWs) for ground-based
detectors (\eg, Advanced LIGO and Virgo, and KAGRA \citep{Abadie:2010hv,Accadia:2010aa,Akutsu:2017kpk}).
On August 17, 2017 advanced LIGO and Aadvanced Virgo made the first observation of GWs from a binary 
neutron star merger, the event \GW{}. It is expected that advanced GW observatories will detect more
events such as \GW{} in the near future.

It is expected that a significant amount of matter is ejected during the neutron star merger. 
This neutron-rich matter, the ejecta, is a promising site for the rapid neutron racture 
nucleosynthesis, ($r$-process, see sec.\ref{sef:nucleo}), responsible for the creation of the 
heaviest elements in the Universe \citep{Lattimer:1974slx,Eichler:1989ve,Thielemann:2017acv}.
Related to the neutron-rich heavy element synthesis in the ejecta, an electromagnetic (EM) transient
(Kilonova/mactronova) is expected to be powered by the radioactive decay of the \rproc{} 
elements \citep{Li:1998bw,Metzger:2010,Roberts:2011,Goriely:2011vg,Korobkin:2012uy,Barnes:2013wka,Tanaka:2013ana}.
This is the electromagnetic counterpart to the gravitational waves from BNS that, if detected, 
could constrain aid with the source sky localization and allow further constrain models 
of the chemical evolution and origin of the heavy \rproc{} elements 
This is supported by the recent observation of ultra-violet, optical, and infrared signals of
\GW{} \citep{TheLIGOScientific:2017qsa,Tanaka:2017qxj,Arcavi:2017xiz,Coulter:2017wya,Cowperthwaite:2017dyu,Drout:2017ijr,Evans:2017mmy,Kasliwal:2018fwk,Pian:2017gtc,Smartt:2017fuw,Tanvir:2017pws}. 
In addition to the nuclear decay powered EM counterpart, a long-lasting synchrotron emission 
is multi-wavelengths is expected from the merger ejecta propagating through the interstellar
medium (ISM) \citep{Nakar:2011cw}. 
Such signal, if detected, would provide an additional information on the merger ejecta velocity profile,
unafeccted by the ill-constrained details of the \rproc{} nucleosynthesis. 

In order to perform a quantitative studies of the aforementioned topics, first the main aspects 
of the BNS merger have to clarified. These aspects include the merger process, the mass ejection, 
the nucleosynthesis and subsequent decay of the heavy elements in the ejecta, and EM emission
arising from the ejecta. Numerical relativity simulations that include accurate microphysical processes, 
neutrino radiation transport and magnetohydrodynamics (MHD) are the best tool in this regard. 

Owing to the recent rapid advancement in the field of modeling the BNS in numerical-relativity (NR) 
the detailed simulations of the mergers are now achievable.
This allowed to investigate the effects of the neutron star matter equation of state, (EOS) that 
take into account the finite-temperature effects \citep{Duez:2009yz,Sekiguchi:2011zd},
the effects of the neutrino cooling \citep{Sekiguchi:2011zd,Deaton:2013sla,Foucart:2014nda,Palenzuela:2015dqa} and neutrino heating \citep{Sekiguchi:2015dma,Foucart:2016rxm}, and
MHD instability \citep{Kiuchi:2014hja,Kiuchi:2015sga,Kiuchi:2015sga}.
Numerical relativity simulations are currently the best way to study the mergers and predict the features, 
that can be tested with later observations.

In BNS mergers the mass ejection processes have been studied with NR simulations have been first 
investigated by \citet{Hotokezaka:2013b} while in the NSBH mergers by \citet{Foucart:2012vn}. A large number of numerical relativity 
simulations have been performed to study the nature of the first material that is being ejected 
on the dynamical timescale (the dynamical ejecta) \cite{(Sekiguchi:2015dma,Palenzuela:2015dqa,Lovelace:2013vma,Kyutoku:2013wxa,Foucart:2015vpa,Foucart:2015gaa,Foucart:2016rxm,Sekiguchi:2016bjd,Lehner:2016lxy,Radice:2016dwd,Foucart:2016vxd,Kyutoku:2017voj,Dietrich:2018uni,Dietrich:2016lyp,Bovard:2017mvn,Radice:2018pdn}. This works showed that ejecta mass depends on the binary parameters, such as mass ration, and on 
the EOS. The ejecta was found to host material with broad range of compositions, where the latter 
is usually quantified with electron fraction $Y_e$, which is the electron number density per baryon 
number density. This wide range of $Y_e$, is however consistent with the observed abundance patterns 
of \rproc{} elements (with mass number larger $A\sim90$) in the Solar System and metal poor-stars \citep{Wanajo:2014,Radice:2016dwd}.

After a BNS merger, a remnant is born. It can be a black hole (BH) or a massive, stable or unstable, neutron star (MNS). Both could be surrounded by a gravitationally bound matter, a disk (torus). 
The \pmerg{} evolution have been investigated by many groups \citep{Fernandez:2013tya,Metzger:2014ila,Perego:2014fma,Fernandez:2014cna,Just:2014,Fernandez:2016sbf,Siegel:2017nub,Fujibayashi:2017xsz,Fernandez:2018kax}. 
Simulations showed that a certain fraction of the disk can become unbound, and be ejected from the system, 
by viscous, nuclear recombination or MHD effects. 
This long-term ejecta was found to be more massive then the dynamical ejecta in certain cases and thus,
even more important for EM counterparts and nucleosynthetic yields.

In this thesis we discuss the numerical relativity simulations performed with the code \texttt{WhiskyTHC}



In this thesis we perform and analyze numerical relativity simulations of merging neutron stars. 
These simulations are performed via solving the equations of general relativity, hydrodynamics and radiation, neutrino, transport via special numerical schemes. 

In this chapter we provide a brief description of the main equations and methods used to produce simulations analyzed in this thesis. 
For the sace of bravity we limit the discussion to the main results and implication important for our work.
For the underlying principles of the Eintein's theory of General Relativity, for which we here the reado to \red{[GR refs]}.
For the discussion and derivation of general relativistic hydrodynamics and refer the interested reader to \red{[GRHD refs]}.
For the Discussion on the radiation transport we refer to \red{GR-Rad refs}

%% from GRLES Raduce paper
Multimessenger observations of BNS mergers are starting to constrain the poorly known properties of
matter at extreme densities [11,12,20–36] and the physical processes powering short g-ray bursts (SGRBs)
[37–42]. They are also beginning to reveal the role played by compact binary mergers in the chemical
enrichment of the galaxy with r-process elements [8,13,43–62]. The key to the solution of some of the most
pressing open problems in nuclear and high-energy astrophysics – such as the origin of heavy elements,
the nature of neutron stars (NSs), and the origin of SGRBs – is encoded in these and future observations.
However, theory is essential to turn observations into answers.

%% ============================================================ Chapter :: Theory/Methods 

