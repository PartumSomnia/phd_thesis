\chapter{Introduction} % Main chapter title

\label{ch:intro} % Change X to a consecutive number; for referencing this chapter elsewhere, use \ref{ChapterX}

%% =====================================================================================
%%
%%              G E N E R A L  I N T R O D U C T I O N
%%
%% =====================================================================================


%% =====================================================================================
%%
%%              O B S E R V A T O N A L  C O N T E X T
%%
%% =====================================================================================

\section{Observational context}


%% --------------------------------------------------------------------------
%%               N U C L E O S Y N T H E S I S
%% --------------------------------------------------------------------------

\subsection{\rproc{} \nuc{}}\label{sec:nuc:rproc}


%% --------------------------------------------------------------------------
%%               K I L O N O V A
%% --------------------------------------------------------------------------

\subsection{Kilonova}


%% --------------------------------------------------------------------------
%%               A F T E R G L O W S
%% --------------------------------------------------------------------------

\subsection{$\gamma$-ray bursts and kilonova afterglows}


%% =====================================================================================
%%
%%              T H E O R E T I C A L  P I C 
%%
%% =====================================================================================

\section{Theoretical picture}

In This section we briefly overview the current understanding of the \ac{BNS} 
merger, the dynamics of the inspiral, effects of tides, \pmerg{} hydrodynamics 
and open questions.

%% --------------------------------------------------------------------------
%%               I N S P I R A L
%% --------------------------------------------------------------------------

\subsection{Inspiral}

If \acp{NS} formed through a classical stellar evolution channel of the massive binary, 
their orbit is mostly circular with little to non eccentricity \cite{4}. 
%
The binary system looses energy to the emission of \acp{GW}, and the orbit of \acp{NS} 
shrinks. Stars inspiral. The last ${\sim}10^3$ orbits, for instance, can be completed in 
a matter of minutes. During the last part of the inspiral, the \ac{GW} signal rises in 
frequency and amplitude (so-called chirp), reaching the peak at \textit{moment of merger}.


\subsubsection{Two Body Dynamics}

The dynamics of two well separated stars that have relatively small angular velocity 
(\ie, quasi-adiabatic inspiral of point-masses) the \ac{PN} approximation to \ac{GR}, 
(the expansion in $\upsilon/c$, with $\upsilon/c\ll 1$) is applicable.
%In other words, this is the stage of evolution when the orbital timescale is 
%significantly larger than the orbital one, \ie, $\dot{\Omega}/\omega^2\ll 1$.
%
The amplitude and the phase evolution of the \ac{GW} signal, at the leading order, 
is given by the quadruple formula 
%
\begin{equation}
\label{eq:intro:gw_wave}
h(t) \sim \frac{1}{d}\mathcal{M}_c^{5/3} f_{GW}^{2/3} = \nu\frac{M}{d}(Mf_{GW}(t))^{2/3}, \hspace{5mm} \phi(t) \sim 2\mathcal{M}_c^{-5/8}t^{5/8} = 2\nu^{-3/8}(t/M)^{5/8},
\end{equation}
%
where $\mathcal{M}_c = M\nu^{3/5}$ is the chirp mass, $M = M_A + M_B$ is the binary mass, 
$\nu = M_A M_B/M^2$ is the symmetric \mr{}, $f_{GW} = \dot{\phi}$, $d$ is the distance of the source.
%Here, the $G=c=1$ are the units, and geometric coefficients are neglected for brevity.
%
Obviously, \acp{NS} are not point-masses. The \ac{NS} finite size affects (tidal effects) the 
inspiral and, consequently, emitted \acp{GW}. In the \ac{PN} formalism these effects enter 
and $5$th \ac{PN} order.
%
%The \ac{PN} expansion, being the asymptomatic expansion, looses its accuracy at high 
%frequencies $f_{GW}\gtrsim 50\,$Hz. 
%
Another approximation to the two-body dynamics in \ac{GR} is the \ac{EOB} formalism,
which is a Hamiltonian formalism, applicable to all stages of the binary evolution.
%It is has an advantage of being applicable both at low and high frequencies and can be 
%applied throughout the inspiral, merger na \pmerg{} \cite{28}.
It spiritually similar to that, commonly used in Newtonian dynamics to 
describe the motion of two bodies via the motion of a single particle with effective mass 
$\mu=\nu M$ in the effective potential. Approximating the dynamics in \ac{GR} requires 
employing effective particle, $\mu$ and defective metric. 
%
Damour \& Nagar \cite{29} extended the \ac{EOB} formalism to \ac{BNS} mergers by 
introducing finite size effects. For the up-to-date review we refer tp \cite{30}.


\subsubsection{Effects of tides}

A method to treat the finite-size effects in self-gravitating objects in \ac{PN} 
dynamics was, proposed by Damour, Soffel and Xu \cite{31}, is based on ``skeletonizing'' 
objects into worldlines with global properties -- the \textit{outer problem}.
%
It was matched to the \textit{inner problem}, that considered how the worldtube 
around one body is influenced by another. In the case of \ac{BNS}, the latter is 
referred to the tidal effects induced by the external gravitional field of the companion. 
%Matching the outer problem with the inner translates into the inclusion of the tidal 
%deformation effects into the orbital dynamics (and consequently, \ac{GW} emission). 
The fully relativistic formulation of the \textit{inner problem} was derived in 
\cite{33,32,34}. 
%The key parameters of the formulation are the 
%external tidal moments,
%introduced as symmetric-trace-free projections of the derivatives of the externally 
%generated parts of the local gravitoelectric, $\bar{E}_{\alpha}$ and gravitomagnetic, 
%$\bar{B}_{\alpha}$, fields that read 
%$G_L^A = \partial_{\langle L-1}\bar{E}^A_{\alpha_l\rangle}|_{X^{\alpha}}\rightarrow 0$, 
%(where $X^{\alpha}$ are local coordinates \cite{32}).
%In the local frame of the first body, the \red{internally generated mass} $M_L ^A$ and spin 
%$S_L ^A$ multipole moments, (here $L=i_1,i_2,...,i_l$, is the multi-index) depend on the external
%tidal moments via 
%\textit{tidal polarizability coefficients} $\mu_l$ and $\sigma_l$, expressed as 
%
%\begin{equation}
%    M_{L}^{A} = \mu_l G_{L}^A, \hspace{5mm} S_{L}^A = \sigma_l H_L^A,
%\end{equation}
%
%where $M_L ^A$ and $S_L ^A$ are the internally generated mass and spin miltipole moments.
%(here $L=i_1,i_2,...,i_l$, is the multi-index), and 
%$G_{L}^A$ and $ H_L^A$ are the external tidal moments, related to gravitoelectric and 
%gravitomagnetic fields respectively. 
The key components of the formulation are the external gravitoelectric 
(gravitomagnetic) fields, $l$-th order of which induces the mass (spin) multipolar moments 
of the same order in \ac{NS}. These moments are characterized by %$G\mu_l$ ($G\sigma_l$), 
tidal polarizability coefficients, that can be written as dimensionless relativistic 
Love numbers \cite{32,34}.
%
%The $l$-th order (external) gravitoelectric (gravitomagnetic) fields induce $l$-th order 
%mass (spin) multipolar moments in a \ac{NS}, characterized by respective coefficients, 
%$G\mu_l$ ($G\sigma_l$), that have dimensions of [length]$^{2l+1}$.
%%
%The dimensionless relativistic Love numbers are defined as
%%
%\begin{equation}
%k_l = \frac{(2l - 1)!!}{2}\frac{G\mu_l}{R^{2l + 1}}, \hspace{5mm} j_l = \frac{(2l-1)!!}{2}\frac{G\sigma_l}{R^{2l+1}},
%\end{equation}
%
%where $R$ is the radius of a \ac{NS}. 
%In many studies, only the dominant $l=2$ mode is considered, \eg, \cite{33}. 
%For \acp{BH}, $\mu_l=\sigma_l=0$ \cite{32,34}.
%Sometimes, only dominant quadrupole $l=2$ gravitoeletric coeffcient is considered \cite{33}.
%
If the external field can be viewed as quasi-static (``adiabatic tides'') the 
Love numbers can be computed by considering the stationary perturbations of the spherical 
relativistic star, \ie, solving the \ac{TOV} equations, which are considered in full \ac{GR}, 
due to strong dependency of the tidal coefficients on the star compactness, defined as 
$C_{i} = GM_i/R_i^2c^2$. 
%
Thus, Love numbers carry the imprint of the \ac{EOS} on the \ac{BNS} dynamics.
%
%On the other hand, if the external field is dynamic, than the effects induced by a \ac{NS}, 
%at linear order in the deformation, can be formulated as a superposition of the proper modes 
%of the \ac{NS}. The excitation of modes occur when the resonant frequency of modes is matched 
%by the star's orbital frequency. 
%This approach has been studied in Newtonian gravity, in \ac{GR} for a test-mass circling the 
%\ac{NS}, and for \acp{NS} of similar mass in \ac{PN} theory \cite{35,36,37}.
%Dynamical external field induces ``dynamical tides'', among which the most important are the 
%pressure modes ($f$-modes) in the non-resonant way\footnote{
%    The resonance for $f$-modes occur in kHz
%    regime, that is achieved only at merger, when two \acp{NN} are no longer isolated objects. 
%    Other mode,s such as $g$- and $r$-modes do not contribute significantly, due to their 
%    lower energies (albeit also lower frequencies).
%}
%
%The tidal effects can be included into the \ac{PN} two-body dynamics by modifying the 
%\ac{GR} effective action as 
%%
%\begin{equation}
%S = S_{GR} + S_{pointmass} = \frac{1}{16\pi G}\int R\sqrt{|g|}dx - \sum_{A}\int M_A \dd s_A
%\end{equation}
%%
%where the last term of the \ac{RHS} is the ``skeletonized'' representation \red{as} a point mass, 
%with non-minimal coupling of worldlines, that read 
%%
%\begin{equation}
%S_{nonminimal} = \sum_{A}\frac{\mu_l^{A}}{2l!}\int (G_L^A)^2 \dd s_A + \frac{l \sigma_l^A}{l! 2(l+1)} \int(H_{L}^A)^2 \dd s_A.
%\end{equation}
%%
%The added term changes the dyanmics at $5$th \ac{PN} order. The change is linear in tidal deformations.
%%
%At the leading \ac{PN} (Newtonian) order, only $l=2$ gravitoelectric terms appear when tidal 
%contributions are included. The term reads 
%%
%\begin{equation}
%\label{eq:intro:tidal_largangian}
%L^{LO}_{tidal} = k_2^A G M_B^2\frac{R_A^5}{r^6} + (A\leftrightarrow B),
%\end{equation}
%%
%where $r$ is the distance between \acp{NS}. 
%
Writing the modified \ac{GR} action with the inclusion of ``skeletonized'' representation
of a \ac{NS}, the tidal effects can be added into the \ac{PN} description of the two-body dynamics.
The resulted Lagrangian, at the leading (Newtonian), order shows that at small distances 
between stars, $r$, the introduced corrections are attractive. 
%
Further, the Kepler law given by the quadrupolar gravitoelectric term, 
%
%\begin{eqnarray}
%\Omega^2 r^3 = G M \Big[ 1 + 12 \frac{M_A}{M_B} \frac{R_A^5}{r^5} k_2^A + (A\leftrightarrow B) \Big].
%\end{eqnarray}
%
shows that the finite-size effects manifests as an increased orbital frequency at a given radius.
In other words, the \acp{NS} spin faster, if tidal effects are present and merger sooner 
producing signal with higher frequency \cite{29}. 
%When $r=R_A+R_B$, the frequency at merger can be estimated. 
%There, $2GM\Omega\approxeq2(M_B/(MC_B) + M_B/(MC_B))^{-3/2}$ \cite{29}. 
%For \acp{NS} of the same mass one obtains 
%%
%\begin{equation}
%f_{GW}^{contact} \approxeq 1.327 \Big( \frac{C}{0.15} \Big)^{3/2} \Big( \frac{M}{2.8M_{\odot}} \Big) \text{kHz}.
%\end{equation}
%
%Simulations show that the contact between the two \acp{NS} happens approximately 2-4 \ac{GW} cycles 
%prior to merger at an even lower frequency \cite{38}

Within the \ac{EOB} framework, the dynamics of the system can be expressed in terms of the 
effective Hamiltonian, $H_{EOB}$
%that for zero-spin case can be written 
%%
%\begin{equation}
%H_{EOB} = M\sqrt{1 + 2\nu (\hat{H}_{eff} - 1)}, 
%\hat{H}_{eff} = \frac{H_{eff}}{\mu} = \sqrt{A(u;\nu)(1 + p_{\phi}^2u^2 + 2\nu(4-3\nu)u^2p_{r^*}^4) + p_{r^*}^2}
%\end{equation}
%%
%where $u=GM/rc^2$ is the Newtonian potential. 
%Consder the Schwarszchild spacetime and a particle in it with $\nu\rightarrow0$ and 
%$A(u;0) = 1-2u$. Then, the $H_{eff}$ becomes the particle Hamiltonian.
%
.
The finite-size effects can be included as tidal component to the potential, $A$, as 
$A = A_0 + A_{tidal}$ \cite{40}, 
%that can be written as 
%%
%\begin{equation}
%A_{tidal} = \sum_{l\geq 2}\Big[ k_l^{A+}u^{2l+2}(1+\alpha_1^{(l+)}u ... ) + k_{l}^{A-}u^{2l+3}(1+\alpha_1^{(l-1)}u + ...) + (A\leftrightarrow B) \Big]
%\end{equation}
%%
%where $\alpha_i^{(l)}(\nu)$ are coefficients and 
%%
%\begin{equation}
%k_{l}^{A+} = 2k_l^A\big( \frac{M_A}{MC_A} \big)^{2l+1}\frac{M_B}{M_A}, \hspace{5mm} k_l^{A-} = 2j_l^A\Big( \frac{M_A}{M C_A} \Big)^{2l + 1} \frac{M_B}{M_A}
%\end{equation}
%%
%are the multipolar tidal polarizability coupling constants. 
that depends on the multipolar tidal polarizability coupling constants $k_{l}^{A\pm}$.
%
In the Newtonian limit the \ac{EOB} Hamiltonian reads 
%
\begin{equation}
H_{EOB} \approxeq Mc^2 + \frac{\mu}{2}p^2 + \frac{\mu}{2}(A-1) = Mc^2 + \frac{\mu}{2}p^2 + \frac{\mu}{2}\Big( -\frac{2 G M}{c^2 r^2} + \cdots - \frac{\kappa_2^T}{r^5} \Big).
\end{equation}
%
where $\kappa_2^T = \kappa_2^A + \kappa_2^B$ is the constant accounting for the tidal 
interactions at leading order.
%
For a physically motivated range of masses (1-2)$\Msun$ and \mr{}s $q\in[1,2]$ the 
$\kappa_2^T\sim[50,500]$. 
%
Similarly, one can defile the $\Lambda_2^i = 2/3 k_2^i (c^2 R_i/GM_i)^5$ with $i\in\{A,B\}$.
Then, instead of $\kappa_2^T$, the \textit{reduced tidal deformability} is used
%
\begin{equation}
\label{eq:theory:Lambda}
\tilde{\Lambda} = \frac{16}{13}\frac{(M_A + 12M_B)M_A^4}{M^5}\Lambda_A + (A\leftrightarrow B).
\end{equation}
%

Consequently, the effects of tides appear in the waveform computation, as radiation reaction 
compliments the conservative dynamics discussed above \cite{41} and discussed in \cite{6,42}.
%
%At leading order the stationary phase approximation of the waveform reads
%%
%\begin{equation}
%h(f) = Af^{-7/6}e^{-i(\Psi_0(x) + \Psi_{tidal}(x))} = Af^{-7/6}e^{-i(\Psi_0(x)-39/4\kappa_2^Tx^{5/2})}
%\end{equation}
%%
%where $x(f) = (\pi G M f / c^3)^{2/3}$ and $\Psi_0(x)$ is point-mass phase.
%
Notably, the $k_2^T$ fully determines the tidal contribution to the waveforms at leading order. 
Hence, from observed \ac{GW} signal, the $k_2^T$ (or $\tilde{\Lambda}$) can be well estimated. 
%
%There have been extensive tests of the \ac{EOB} formalism for \ac{BNS} mergers agains \ac{NR} 
%simulations \cite{43,38,44,45,46}. 
%The accuracy of the \ac{EOB} approximation of tides, the $A_{tidal}$, decreases during 
%the last orbits and for very large $k_2^T$. Modified versions of \ac{EOB} were developed 
%with gravitational self-force calculations of tides computed at high-order TEOBResumS 
%\cite{44,47,46,48} and dynamical tides (SEOBNRT) \cite{49}.
%When spin cannot be neglected, the tidal interactions become more complex \cite{50,51}.
%The oblateness of a spinning \ac{NS} leads to the deformed gravitational field that is 
%characterized by the quadrupole tensor, and produces an attractive contribution to the 
%potential, modifying the inspiral at the $2$nd \ac{PN} order $\mathcal{O}(\upsilon/c)^4$
%\cite{50}. 
%There are also hybrid models of \ac{EOB} and \ac{NR} \cite{52,53}

\subsubsection{Gravitational Waves}

The most accurate way to compute waveforms is the conduct \ac{NR} simulations with microphysical 
\ac{EOS}. However, these simulations are computationally expensive and not avaiilable or feasible 
in certain renages of the parameter space.
The \ac{EOB} approach allows to compute the waveforms at broad range of frequencies and 
covers all stages of the binary evolution. 
The \ac{EOB} models can be augmented with the formulism descibing the 
high frequency emission of the remnant (kiloHertz), that itself can be build from the 
available \ac{NR} \ac{HD} simulations \cite{56,57,58}.

From the \ac{GW} signal observed with ground based facilities, contains the information 
about the chirp mass, 
%(related to $I_{-10}$),
%where $I_p = \int \dd\ln f(\gamma(f))f x^{2p}(f)$ with $\gamma(f)df$ being the 
%measure of the detector noise \cite{59,6},
primarily at low frequencies. Additionally, the low frequency part of the signal, 
$\leq50$~Hz and $\leq100$~Hz contains information on the symmetric \mr{} and \ac{SNR}.
On the other hand, the information about tidal effects (tidal parameters) is related to 
%$I_{+10}$,
higher frequency signal, evaluation of which requires accurate high-frequency waveforms.
%
From the Newtonian limit discussed above, it follows that the system dynamics at merger 
is determined mainly by $\kappa_2^T$ \cite{44}. \ac{NR} simulations verified this 
prediction \cite{60,55}. 

With respect to the \GW{} most of the information was obtained in $30-600$~Hz frequency 
range ($\sim 1300$ orbits).
The signal was interpreted as coming from a \ac{BNS} merger with total mass of $\simeq2.7\Msun$,
chirp mass $\mathcal{M}=1.186(1)\Msun$, \mr{} $q\in[1,1.34]$ and $\tilde{\Lambda}\simeq 300$ 
(with an upper bound of $\sim800$) \cite{1,2,61}.
Inclusion of the \ac{EM} counterparts into the analysis leads to higher $\tilde{\Lambda}$ being 
more favoured \cite{62,63}.
%
The estimation of individual masses and \mr{} was more subjected to uncertanties and 
systematic, \ie, spin prior \cite{2}. Due to the partial degeneracy between the tidal 
parameters and the \mr{}, the \ac{EOS} constraints are also subjected to uncertainties. 
Specifically, assuming that stars had small spin $\leq 0.05$, the stars radii can be 
constrained to $11-12$~km \cite{64,65}. 
Assuming also that \ac{EOS} can support a non-rotating \ac{NS} with a mass $1.97\Msun$, 
results in  $R\sim 11.9\pm 1.4$~km and $90\%$ credibility \cite{65}.
%
%\ac{GW} signal during the inspiral (including tidal phasing) also provides constraints 
%on the \ac{EOS} viewed in the pressure-density diagram \cite{65}. 
%For an example case of a \ac{BNS} with $\rho_{\rm max} \approxeq 2 \rho_0$, 
%the estimated value of pressure is $P(2\rho_0)=3.5_{-1.7}^{2.7}\times10^{34}$ dyn cm$^{-2}$ 
%at $90\%$ confidence level.
%
%For \GW{} the peak of the \ac{GW} signal, the merger, was not detected. However, from the 
%probability distribution of $\tilde{\Lambda}$, adopting \ac{NR}-motivated fitting 
%formula, it is possible to asses the peak frequency that falls in $1.2 - 2$kHz \cite{55}.
%The \ac{GW} peak luminocity is estimated to be $\geq 0.1\times 10^{56}$ erg/s \cite{61}. 


%% --------------------------------------------------------------------------
%%               P O S T - M E R G E R
%% --------------------------------------------------------------------------

\subsection{Merger and \pmerg{}}