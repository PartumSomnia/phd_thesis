%% =====================================================================================
%%
%%               P O S T - P R O C E S S I N G
%%
%% =====================================================================================
\chapter{Postprocessing methods} \label{ch:ppr}

%% =====================================================================================
%%
%%               P O S T - P R O C E S S I N G
%%
%% =====================================================================================


%\section{Postprocessing tools and methods}

In this appendix we discuss the methods used to analyze the simulation data,
focusing on the overall average quentities that can be used for comparisons 
of the remnant, disk and ejecta. 
%
For this we mainly use the mass-averaged quantities defined as 
%
\begin{equation}
\label{eq:ppr:average}
\langle f \rangle = \frac{\sum_i f(m_i)m_i}{\sum_i m_i}\, ,
\end{equation}
%
where $m_i$ is the mass contained in the $i$-th bin.



\section{Disk \& Remnant}\label{sec:bns_sims:method:disk}

It is common to discuss the \pmerg{} state of the \ac{BNS} systems in terms of the remnant, 
a \ac{NS} or a \ac{BH}, and a disk or torus surrounding it. 
%However there is not unified convention in how to define the disk and separate it from the former. 

%A disk around a \ac{BH} is the matter outside the apparent horizon 
%\citep[\eg][]{Dietrich:2015iva,Dietrich:2016hky}. 
%However, owing to the disk accretion onto a \ac{BH}, the evaluation time is a 
%crucial parameter, and unfortunately is not consistent in the literature, 
%(\eg $\sim1$~ms in \citet{Dietrich:2015iva,Dietrich:2016hky} and $\sim30$~ms in \citet{Sekiguchi:2016bjd}).
While a disk around a \ac{BH} can be defined as matter outside the apparent horizon, a disk around a \ac{NS} remnant is more difficult to define. 
Commonly, the 
threshold $\rho\sim 10^{13}\,$\gcm{} is assumed. The choice is motivated by studies that 
showed that the angular velocity profiles become approximately Keplerian at this point 
\citep[\eg][]{Shibata:2005ss,Shibata:2006nm,Hanauske:2016gia,Kastaun:2016elu}.
This convention was adopted in several recent works 
\citep{Radice:2018pdn,Kiuchi:2019lls,Vincent:2019kor}.
The extraction time here is also important, but less so, as we find that the accretion on the 
\ac{NS} is considerably slower.
%In the case where the remnant is a neutron star, the disk definition usually includes 
%the density cut. For instance, in \cite{Radice:2018pdn,Kiuchi:2019lls,Vincent:2019kor} 
%the disk is assumed to encompass the matter with $\rho < 10^{13}$~\gcm. 
%The threshold $\rho\sim 10^{13}$~\gcm corresponds to the point in the remnant where
%the angular velocity profiles becomes approximately Keplerian, 
%\citep[\eg][]{Shibata:2005ss,Shibata:2006nm,Hanauske:2016gia,Kastaun:2016elu}.
%The extraction time here is also important, but less so, as we find that the accretion 
%on the NS is considerably slower.
%
%Overall, we estimate that these differences can amount to a systematic factor of a few,
%which we employ for the statistical analysis in section \ref{sec:stat:anal}

% In this thesis we define the disk as a matter that satisfies two criteria 
%$\alpha > 0.15$ and $\tho < 10^{13}\gcm$, where $\alpha$ is the lapse function 
%(see section \ref{sec:theory:gr3p1}).

We compute the disk baryonic mass evaluating
the volume integral of the conserved rest-mass density, 
$D=\sqrt{\gamma}~W\rho$,
%
\begin{equation}
\label{eq:method:mdisk}
M_{\text{disk}} = \int D \dd^3 x\, 
\end{equation}
%
from $3$D snapshots of our simulations.% in postprocessing.

%% In \cite{Sekiguchi:2016bjd}, the disk mass is extracted at
%% ${\approx} 30$~ms outside the AH. In \cite{Radice:2018pdn}, the disk mass is computed
%% as the baryonic mass outside the AH at BH formation, while for NS
%% remnants the criterion $\rho < 10^{13}$ g cm$^{-3}$ is used. 
%% In \cite{Kiuchi:2019lls} for both BH and NS outcome the $\rho < 10^{13}$ g cm$^{-3}$ 
%% criterion is used and time of the extraction is not specified. 
%% In \cite{Vincent:2019kor} the density criterion is the same, however the simulations 
%% are significantly shorter (${~\sim 7.5}$~ms) than in other
%% works. Overall, we estimate that these differences can amount to a
%% systematic factor of a few.



\section{Density modes} \label{sec:bns_sims:method:modes}

%% FROM THE LETTER 

The hydrodynamic instability in newly formed \acp{MNS} 
is monitored by a decomposition in Fourier modes
$e^{- i m \phi}$ of the Eulerian rest-mass density on the equatorial plane 
\citep[see Eq.~(1) in][]{Radice:2016gym} and characterized by the
development of a $m=2$ followed by a $m=1$ mode 
\citep{East:2015vix,Paschalidis:2015mla,Radice:2016gym,Lehner:2016wjg,Bernuzzi:2013rza,Kastaun:2014fna}.
%In the short-lived remnant (LS220) the $m=1$ mode
%is subdominant with respect to the $m=2$, and it reaches a maximum close to the collapse
%\citep{Bernuzzi:2013rza}. Instead, in the long-lived remnant (DD2) the $m=1$
%becomes the dominant mode at $\sim$20~ms and persists throughout the
%remnant's lifetime, while the $m=2$ efficiently dissipates via
%gravitational-wave emission \citep{Bernuzzi:2015opx,Radice:2016gym}.
%
%% FROM THE PAPER 
%
%During the post-merger evolution the neutron star oscillates the most prominant modes 
%are quasiradial mode $F$ ($m=0$), the $m=2$ $f$-model 
%and non-linear combinations of them \citep[\eg][]{Shibata:2000jt,Stergioulas:2011gd}.
%
%It has also been shown that the $m=1$, so-called one-armed spiral instability, 
%is present in the remnant of \ac{BNS} mergers  
%\citep{Paschalidis:2015mla,Radice:2016gym,East:2016zvv}.
%
%In order to investigate the dynamical instabilities in our simulations we 
%project the rest-mass density onto spherical harmonics,
%or, in other words, we perform the complex azimuthal mode decomposition of,
%the conserved rest-mass density.
For the data availability reasons we consider only $\rho(x,y,z=0,t)$, 
\ie{} restricting our analysis to the orbital plane $z=0$,

%\red{DOUBLE check the presence of Gamma and W}
%% \begin{equation}
%% \label{eq:modes}
%% C_m = \int \rho(x,y,z=0,t) W e^{-i m \phi} \sqrt{\gamma} %% \text{d}x \text{d} y \, ,
%% \end{equation}
\begin{equation}
\label{eq:modes}
C_m(t) = \int \rho(x,y,z=0,t) e^{-i m \phi(x,y)} \text{d}x \text{d} y \, .
\end{equation}
%(see \eg~\citet{Baiotti:2009gk}).

% where %% $\rho$ is the rest mass density, 
% $\gamma$ is the determinant of the three-metric and $W$ is the
% Lorentz factor between the fluid and the Eulerian observers. 
%
We note that the above quantities are gauge dependent.
%
%\red{Dietrich in his thesis thinks that the growing m=1 mode is "We can not
%    exclude the possibility that the growing m = 1 mode is triggered by numerical effects,
%    but we think that it is a physical hydrodynamical effect due to mode couplings."
%    Page 53 of the thesis
%}
%

Notably, frequencies of the modes can be computed with the Fourier analysis 
of the $\rho_{\text{max}}$ and projections $C_{m}$. 
We leave this for the future work.
%as in this work we are interested only in their magnitude. 





\section{Angular momentum} \label{sec:bns_sims:method:ang_mom}

%\red{First, white that in GR the angular momentum is not clearly defined}

%% FROM LETTER 

%From the fluid's stress energy tensor,
%we compute the angular momentum density flux $J_r = T_{ra}(\partial_\phi)^a$,
%where $\phi$ is the cylindrical angular coordinate;
%angular momentum is conserved if $(\partial_\phi)^a$ is a Killing vector.
%
%% FROM PAPER 

The fluid's angular momentum analysis in the remnant and disk is performed
assuming axisymmetry,
%(see Appendix~\ref{app:ang} for derivation)
that is, we assume $\phi^{\mu} = (\partial_{\phi})^{\mu}$ is a Killing
vector. Accordingly, the conservation law, Eq.~\eqref{eq:theory:tmunu_eq_0} 
reads
%
\begin{equation}
\partial_t(T^{\mu\nu}\phi_{\nu}n_{\nu}\sqrt{\gamma}) -
\partial_i(\alpha T^{i \nu}\phi_{\nu}\sqrt{\gamma}) = 0 \ ,
\end{equation}
%
where $n^\mu$ is the normal vector to the spacelike hypersurfaces of
the spacetime's $3+1$ decomposition.
%
The equation implies the conservation of the angular momentum, 
\begin{equation}
J = % \int j dV = 
%- \int \, T_{\mu\nu}n^{\mu}\phi^{\nu}\,\dd ^3x = 
-\int \,
T_{\mu\nu}n^{\mu}\phi^{\nu}\,\sqrt{\gamma}\, \dd^3 x\ .
\end{equation}
%
In the cylindrical coordinates $x^i=(r,\phi,z)$ adapted to the symmetry, 
the angular momentum density is  
%
\begin{equation}
j = %-
\rho h W^2 v_{\phi} \ ,
\label{eq:method:ang_mom}
\end{equation}
%
and the angular momentum flux is 
%
\begin{equation}
\alpha\sqrt{\gamma}T^r _{\nu}\phi^{\nu} =
\alpha\sqrt{\gamma}\rho h W^2 (v^{r}v_{\phi}) .
\end{equation}
%
We evaluate these quantities from the 3D snapshots of our simulations.




\section{Ejecta} \label{sec:bns_sims:method:ejecta}

%To model and study the electromagnetic counterparts to mergers, the amount and properties 
%of the material leaving the system are needed.
The matter expelled at high velocity during \ac{BNS} mergers 
may ultimately become unbound from the 
central gravitational potential. 
There are two indicators commonly adopted to mark the unbound matter.

\subsection{The Geodesic criterion}

Assuming that the spacetime is stationary, ($\partial_t$ is the Killing vector) % \red{confirm}, 
the four-velocity, $u_t$, (along the time-like killing vector), 
is a constant of motion for geodesics. 
Additionally, if the space is asymptotically flat, at infinity the $u_t = -W$, 
where $W$ is the fluid element Lorentz factor. 
Then, if a fluid element has $u_t < -1$, it may be considered unbound 
as it will retain the non-zero positive velocity at infinity. 
%
The fluid asymptotic velocity reads 
\begin{equation}
\upsilon_{\infty} \simeq \sqrt{2E_{\infty}} = \sqrt{(1-u_t ^2)}.
\end{equation}
%
This criterion can be thought of as assuming that the fluid to be made of isolated 
particles that follow geodesics in the static spacetime. 
Indeed, the effects of \acp{EOS} 
(fluids pressure gradient, internal energy and heating) %(\eg, due to an $r$-process (\red{see section XXX})) 
are neglected which might lead to 
underestimation in the ejecta mass. 
%
%The space time is also assumed to be static.
%
Strictly speaking, none of these assumptions is 
fulfilled in the \ac{BNS} \pmerg{} environment. However, this criterion 
is widely used in the literature 
\citep[\eg][]{Radice:2018pdn,Vincent:2019kor}.



\subsection{The Bernoulli criterion}

From the relativistic Bernoulli equation \citep{Rezzolla:2013}, it follows
that for a stationary relativistic flow, the $hu_t$ is constant along the 
fluid worldliens. Here $h$ is the (relativistic) enthalpy, which is
defined up to a constant factor. 
%
If at the spatial infinity the enthalpy is set so $h\rightarrow-1$, 
%\red{in Vincent it is $h\leftarrow 1$}
the condition $hu_t < -1$ would mark the unbound matter 
(as in the assymptotically flat space-time the 
$u_t = -W$ for the flow particles following geodesics).
%
The associated asymptotic velocity is calculated as 
%
\begin{equation}
\upsilon_{\infty} \simeq \sqrt{2 (h (E_{\infty}+1)-1)}. 
\end{equation}
%
The criterion can be regarded as assuming all the internal energy of the fluid 
gets added to the fluid kinetic energy, as the fluid decompresses.
% \red{(pressure drops?)}.
%
%The $r$-process nucleosynthesis that occurs in the outflow deposits the energy.
%\red{In Vincent it is assumed that the difference in binding energy between the 
%    particles in NSE at a given $\rho$, $T$ and $Y_e$, and their binding energy at
%    the same $Y_e$ but low $\rho$ and $T$ is added/substracted from the fluid's kinetic energy.
%    Out-of-NSE evolution and effects this neglected \citep[][see]{Foucart:2016vxd}
%    [BUT I AM NOT SURE IF THIS IS THE CASE FOR OUR SIMULATIONS! MAYBE NOT!]}
%
This criterion has been reported to estimate the amount of unbound material more accurately
\citep{Foucart:2015gaa}. 
Moreover, it was reported that the Bernoulli criterion leads to up to twice 
the amount of ejecta detected in comparison with the geodesic criterion, 
if the estimation is done within a given volume \citep{Kastaun:2014fna}.



We adopt the geodesic criterion to study the "burst-like", short outflows,
such as \ac{DE}, where the pressure gradient is not expected to make a significant contribution.
For the steady-state outflows, like \pmerg{} winds we adopt the Bernoulli criterion.
%
The term ejecta would refer to the material gravitationaly 
unbound according to either of the criteria.
%
All ejecta properties are evaluated at $R \simeq 294$ from the center of the 
simulation domain unless stated otherwise. 