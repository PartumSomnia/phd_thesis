%% ============================
%%
%% Appendix A
%%
%% ============================

\chapter{Details on the \rproc{} \nuc{}}
\label{app:nuc}
%% \externaldocument{intro}
%% --------------------------------------

\section{Parameterized \rproc{}}

Here we discuss the results of the \citet{Lippuner:2015gwa}

The parameterized calculations (see sec\ref{sec:nucleo_method}) shown that 
the \rproc{} \nuc{} has the following properties depending on the entropy and electron fraction.

\begin{itemize}
    \item High Entropy -- high neutron-to-seed ratio, even though seed nuclei are light, -- full \rproc{}
    \item Low entropy, \rproc{} is jugged as there are not many free neutrons available. However, it still preceeds and the seed nuclei are very heavy.
    \item At high $Y_e$ the, where ejecta becomes less neutron rich, the full \rproc{} is no longer produced as there are not enough neutrons per seed nucleus to rich the third peak.
    \item High entropy and low $Y_e$ allows for $3$rd peak reproduction still as there are few seed nuclei and neutron to seed ration is high.
\end{itemize}


\subsection{Lanthanides turnoff and heating rate as a function of $Y_e$}

A turn off is a point in parameter space, when the mass fraction of synthesized lanthanides (actinides) $X_{\text{La}}$ ($X_{\text{Ac}}$) drops below $10^{-3}$. For example, a maximum $Y_e$ (and minimum $\bar{A}_{\text{fin}}$) for which the $X_i \geq 10^{-3}$ is a turn off point. 

The turnoff prevents a large amount of lanthanides (actinides) from being produced and suppresses the fission cycling.

Here $\bar{A}_{\text{fin}}$ is the final average mass number,

\begin{equation}
\bar{A}_{\text{fin}} = \frac{1}{Y_{\text{seed}}(0) + Y_{\alpha}/18},
\end{equation}

where $Y_{\alpha}(0)$ is the initial abundance of alpha particles, $Y_{\text{seed}}(0)$ is the initial abundance of the seed nuclides ($A\geq12$). The $Y_{\text{seed}}(0) + Y_{\alpha}/18$ thus denotes the number abundance of heavy nuclei at the end of the $r$-process, assuming that it takes $\sim 18$ $\alpha$-particles to create a seed nuclei \cite{Woosley:1992}. Further, assuming that the number of seed nuclides by the end of the \rproc{} equals the initial number of seeds plus those produced in $\alpha$-process, \ie, neglecting the effects of fission cycling, the total mass-fraction of heavy nuclei at the end of the \nuc{} is $1$. Note that $\bar{A}_{\text{fin}}$ is a function of initial abundances only and a good indicator of the direction that \nuc{} would take. 

Considering a range in $s\in\{1, 10, 30, 100 \}$, $\tau\in\{0.1, 1, 10\}$~ms and $Y_e\in(0,0.5)$,
the \nuc{} calculations state, nuclear heating rates, $M\varepsilon$, ($1$day, $10^{-2}M_{\odot}$) are rather constant up to $Y_e\approx 0.4$, when a particular nuclides start dominate the heating. The number of fission cycles is increased for high $s$ models, due to increased production of seed nuclides by triple-$\alpha$ process. 

Neutron-rich freeze-out occurs at high $s$ and small $\tau$, where ejecta high temperature and rapid expansion allows for a fewer capture events on seed nuclei and free neutrons escape. An ejecta component with such conditions was suggested to produce a ultraviolet precoursour to the kilonova on a timescale of hours, powered by the decay of frozen-out neutrons \citep{Metzger:2014yda}.

\gray{See if that can be modelled easily for our BNS models}

The heating rate, $M\varepsilon$, are roughly constant with $Y_e$, at $1$ day, as long as the fraction $X_{\text{La+Ac}}$ produced is constant. 
After lanthanides production ceases and their fraction falls, the instantaneous heating rate (at $1$ day) decreases only slightly (up to $1$ order of magnitude). 
Considering the integrated nuclear heating amount as a function of $Y_e$ ($s=\text{const}$, $\tau=\text{const}$), it does decreases by $1-2$ orders of magnitude, when the \rproc{} stops producing unstable heavy lanthanides, and stable nuclides are produced directly.


\subsection{Fission cycling}


Fission cycling is a process where freshly synthesized via strong \rproc{} heavy nuclides with $A\sim 300$ undergo fission just to become seed nuclides for a similar \rproc{} leading to an $A\sim 300$ nuclide. This process eliminates the dependency of the final abundances on the initial conditions and it also limits the maximum nuclide mass that ca be achieved. 

The number of fission cycles can be estimated via a ration of the seed nuclides at time zero and the number of seeds at the time when there are no more free neutrons available. This is motivated by the fact that neutron capture itself does not create new seeds, only increases the mass of them, while fission, splitting heavy nuclide in two, generates additional seeds. 

The number of fission cycles is tight to how much lanthanides and actinides are produced. In particular, as fission cycling limits the maximum mass of the nuclide that can be created, the fraction of lanthanides, actinides as well as heating $\varepsilon$ are insensitive to the initial $Y_e$. The number of cycles is thus tight to the initial $Y_e$. The lower it is, the more free neutrons available and thus more cycles would occur. After the fission cycling stops the $r$-process maximum mass drops to $A\sim 250$. Thus the amount of actinides drops as only the litest of them can be produced that do not fission immediately. 


\subsection{Lanthanide production and heating rate in the full parameter space}
\red{Can be used in the main text}

The properties of the \ac{EM} signal, kilonova, are determined by the energy available, \ie, nuclear heating rate, and how easy this energy can escape the medium, \ie, photon opacity. Both, in turn, depend on the amount of lanthanides (and actinides) available. 

Consider a parameter space in $Y_e$, $s$ and $\tau$. The mass fraction $X_{\text{La}+\text{Ac}}$ have the following general dependency on the parameters. 

For a very low ($Y_e \sim 0.01$) the lanthanides production is mostly assured with exception of two cases: high $s$ - low $\tau$ and very large $\tau$, low $s$. In case of the former, the reason is the increased timescale for the neutron capture and lowered mass of the seed nuclei, which might lead to the neutron-rich freeze-out. In case of the latter, the long expansion timescale results in the newly-synthesized heavy elements decaying under the high density and temperature conditions sufficient for photodissociation, binning the composition back to NSE. When the material does cool down enough for the $r$-process to produce heavy elements agian, the electron fraction has been already increased through $\beta$-decays, allowing for the weak $r$-process only. 

For an intermediate $Y_e \sim 0.25$, where the boundary between lanthanides-right and lanthanides-free composition lies. At the $Y_e=0.25$, lanthanides are produced if the $s \sim 1$ is very low or very high $s \sim 100$. In case of the former this is due to seed nuclei being initially very heavy and few captures are needed to produces lanthanides, while in case of the latter, a large number of free neutrons is available as triple-$\alpha$ process is suppressed \citep{Hoffman:1996aj}, Meanwhile, high-$\tau$ low-$s$ and low-$\tau$ high-$s$ (symmetric corners of distribution) are lanthanides-free. 

Considering the heating rate, $M\varepsilon$, at $1$~day for $M=10^{-2}M_{\odot}$ the following dependencies observed.
For $Y_e\in[0.04,0.35]$ the heating rates (at $1$~day) remain almost independent of $s$ and $\tau$. The dependency on $Y_e$ becomes prominent at $Y_e>0.35$, however the dependency on $s$ and $\tau$ are rather weak. In particular, there heating rates peak at $Y_e=0.425$ due to the decay of $^{66}$Cu that has a rather large $Q$-value.
Notably, no significant correlation found between $X_{\text{La}+\text{Ac}}$ and heating rate at $1$~day. This independence from composition results from the fact that at $1$~day the main contributions to the heating rate would come from nuclides with corresponding half-life (decay energy depends on half-life). And thus, even though the composition of the material does depend on the $Y_e$, $s$ and $\tau$, the same nuclides are considered when heating rates are estimated. At very high $Y_e$, however, the heating rate can be dominated by a selected nuclides and thus it depends on the composition.


\subsection{Fitted nuclear heating rates}


Decay of newly synthesized heavy elements powers kilonova. There are however, particular nuclei decay of which has the largest contribution. To determine these, the fractional heating contributions of all nuclides are integrated. The time window is between $0.1$ and $100$ days, $s\in[1,100]$ $k_B/\text{baryon}$ and $\tau\in[0.1,500]$~ms. 

The single most important nuclide in the aforementioned window is $^{132}$I. The origin of this nuclide is the decay of double magic $^{132}$Sn, that is produced in large quantities and then decays within minutes to $^{132}$Te (through $^{132}$Sb). The decay of $^{132}$Te is not very energetic, but it results in $^{132}$I, which decays with $Q=3.6$~MeV. This determines its large contribution to the heating. 

Overall, at very low $Y_e < 0.25$, nuclei from the second and third peaks ($A\sim 130$ and $A\sim 200$) domiante the heating (and few $A\sim 250$ ones), while at higher $Y_e$, most of the heating comes from the second peak. This is in agreement with dominant $\beta$-decay nuclei found in \citep{Metzger:2010}. In addition, heating comes from the spontaneous fission of heavy elements, produced alongside actinides.

At higher $Y_e\in(0.25,0.375)$, the first $r$-process peak, $A\sim 88$ makes a significant contribution as well as neutron-rich side of the iron-peak. 

At even higher $Y_e>0.375$, the proton-rich side of stability around the iron peak exerts a contribution.




\section{The influence of neutrinos on r-process nucleosynthesis in the ejecta of black hole–neutron star mergers}
%% \section{NEUTRINOS AND BHNS R-PROCESS NUCLEOSYNTHESIS}
%% \cite{Roberts:2016igt}




The influence of neutrinos on r-process nucleosynthesis
in the ejecta of black hole–neutron star mergers

Considering the output of the general relativistic simulation of a \ac{NSBH} merger, mapping it onto the Newtonian smoothed particle hydrodynamics \ac{SPH} code and investigate the evolution of ejecta before the outflow is homologous. The \nuc{} is added in postprocessing, considering different levels of neutrino irradiation from the disk. The neutrino irradiation can affect the final abundances as neutrino absorption on free neutrinos turn them into protons, changing the $Y_e$, adding to the seed nuclei budget.


\subsection{Introduction}

NSBH were expected to be detected by LIGO/VIRGO \citep{TheLIGOScientific:2014jea,TheVirgo:2014hva}, (\gray{and they were but with no counterpart}), possibly hinting on the origin of short Gamma ray Bursts \cite[\eg][]{Lee:2007js} and \rproc{} nucleosynthesis \citep[\eg][]{Lattimer:1976,Korobkin:2012uy,Bauswein:2014qla}.

While galactic chemical evololution hinted that \ac{SN} might be largely responsible for \rproc{} elements enrichment \citep[\eg][]{Qian:2000bh,Argast:2003he}, the conditions within \acp{SN} are not perfect for this process \cite[\eg][]{Arcones:2012wj}. These conditions however, are present when neutron stars or NS and BH merger \citep{Freiburghaus:1999}. Modeling galactic chemical evolution is difficult, as these are high yield rare events that require a significant time between progenitor formation and merger. In particular, discrepancies between such models and observations of \rproc{} elements abundances in low metallicity hallow \citep{Qian:2000bh,Argast:2003he}. More complex models of galactic chemical evolution show however a reconciliation of the disagreement \citep{Matteucci:2014,Shen:2015,VanDeVoort:2015,Ishimaru:2015} \gray{And recent observations, GW170817, indicate the mergers are one of the main sources of $r$-process material.}

Weak interactions were shown to be important in mergers of two neutron stars, affecting the properties of the outflow \citep{Wanajo:2014wha,Goriely:2015fqa,Sekiguchi:2015dma,Foucart:2015gaa,Palenzuela:2015dqa,Radice:2016dwd} \gray{Can Add long paper in the list}. The outcome of the merger has a complex dependency on \ac{EOS}, electromagnetic field and neutrino effects \citep[\eg][]{Neilsen:2014hha,Palenzuela:2015dqa}.

In case of NS and BH merger, weak interactions are expected to play much lesser role. Low entropy in the tidal part of the dynamical ejecta supresses the $e^{-}$ and $e^{+}$ captures. Neutrino interactions are supressed by the high outflow speeds and low neutrino luminosity \citep[\eg][]{Foucart:2014nda,Foucart:2015vpa}. Thus, strong \rproc{} is likely to occure in the \ac{NSBH} outflows \citep{Lattimer:1976,Lattimer:1977,Korobkin:2012uy,Bauswein:2014qla}.

The merger rates for \ac{NSBH} is quite uncertain (\gray{Check the LIGO papers}). The ejecta mass from the mergers depends on the binary parameters, BH spin and mass in particular \citep{Foucart:2012vn,Hotokezaka:2013kza,Bauswein:2014qla,Kyutoku:2015gda}. Increasing the former (in the direction of the orbital angular momentum), increase the amount of ejecta, while increasing the latte leads to a complex effect on the ejecta mass \citep{Kyutoku:2015gda}.


\subsection{Methods}

Code: Spectral Einstein Code (SpEC) with neutrino leackage scheme. EOS: LS220

SPH evolution of ejecta

\subsubsection{Nuclear reaction network and weak interactions}

The nucleosynthesis calculation requires the density history of the ejecta, its initial composition and entropy. This can be done performing a hydrodynamics simulation of the expanding ejecta (which allows to account for the feedback the nucleosynthesis has on ejecta, \eg, generated entropy) or assuming an expansion mode, \eg, homologous expansion. Change in electron fraction during the expansion can be included via weak interactions. Otherwise, it can remain constant.

Notably, a single \ac{EOS} may not extend to the temperature-density region where \rproc{} is still possible in the expanding ejecta. As the nuclear reactions affect the \ac{EOS} only marginally below $\rho\sim 10^{12}$g cm$^{-3}$, a switch to another, multi-species non-degenerate ideal gas \ac{EOS}, \citep[\eg][]{Timmes:1999}. However, caution must be exercised in this procedure, as a single nucleus approximation of the LS220 eos predicts nuclei different from thos from full \ac{NSE} calculation and thus different internal energy. 

For a given density history (of a Lagrangian particles), the \nuc{} can be performed via \ac{NRN} \texttt{SkyNet} \citep{Lippuner:2015gwa}. If it is required to track the evolution of the electron fraction, (which can be the case for high densities and temperatures where weak interactions are not in equilibrium), the NSE evolution mode can be used. Below $T\sim 7$GK, however, a full \ac{NRN} evolution is required.
The inclusion of weak reactions, -- absorption of neutrinos and anti-neutrinos by electrons and positrons, and electrons and positions by free nucleons. The expression for neutrino capture rate is complex, which requires different distribution functions for electrons and positrons. In its applied form it does not include momentum transfer to the nucleons and neglects weak magnetism corrections (note however, that might be of importance for the neutrino-driven with \citep{Horowitz:2001xf}). Neutrino-distribution is usually assumed to the be of the Fermi-Dirac sahpe in energy sapce. While the simple spherical emitting surface is a common approximation, the geometry might be of importance as the \ac{NSBH} (\ac{BNS}) merges form a disk-shape emitting region.


\subsection{Results and discussion}

The prime quantity defining the nucleosynthesis in the ejecta is its electron fraction. 

Due to absence of neutrino emitting central object, shocks at collision and due to short timescales, the electron fraction of the ejecta from \ac{NSBH} mergers is generally expected by the that of the initial beta-equilibrium.
Such low  $Y_e < 0.1$, the full \rproc{} with a large number of fission cycles is expected
On the other hand, in case of the \ac{BNS}, the significant neutrino irradiation is expected raising the ejecta electron fraction through weak interactions
If the $Y_e$ is raised to $0.25$, the nucleosynthesis becomes dependent on other ejecta properties, such as expansion timescale and entropy. 

In case where neutrino absorption is not 'ab-initio' included into the simulation, the effect of it can be estimated from the total neutrino luminosity and neutrino processing timescale. 

\begin{equation}
\tau_{\nu} \approx 67.8 \text{ms} \: \Big(\frac{r}{250\text{km}}\Big)^2 L_{\nu_e, 53}^{-1} T_{\nu_e, 5}^{-1}
\end{equation}

where $r$ is the radius of the fluid element, $L_{\nu_e, 53}$ is the electron neutrino luminosity in units of $10^{53}$erg s$^{-1}$ and $T_{\nu_e,5}$ is the electron neutrino spectral temperature in units of $5$MeV. The spectral temperature can be estimated for a gray scheme by considering the temperature of the emission region, and adopting that the average neutrino energy is $\varepsilon_{\nu} \approx 3.15T$ 

\red{[Quote]Electron antineutrinos are unlikely to contribute significantly to the neutrino interaction timescale[Why?]} \red{[Answer]This is because in the low entropy outflows of BHNS mergers almost all protons are locked in heavy nuclei and thus have very low neutrino capture cross-sections.[EndQuote]}.

Next, assuming that all protons are incorporated into the heavy nuclei, neutrino luminocity and ejecta velocity remain constant, and that neutrino irradiation from the disk start at after merger, also neglecting $e^{-}$, $e^{+}$ captures, 

\begin{equation}
\frac{d Y_e}{dr} = \frac{\theta(r - \nu t_{\nu, on})}{\nu \tau_{\nu} Y_{e; eq}}(Y_{e; eq} - Y_e)
\end{equation}

where $Y_{e; eq} = \langle Z \rangle_{\text{nuclei}} / \langle A \rangle_{\Omega_j}\text{nuclei}$ and $t_{on}$ is the time when neutrino luminosities reach their saturation value.
The result of this analysis states taht the the neutrino irradiation is unlikely to play a significant role in changing the composition of the ejecta from \ac{NSBH} mergers.
Running with the full \ac{NRN} results in very little changes in ejecta composition during the first $\sim10$~ms after while the changes at $\sim20$~ms are driven by beta-decays in $r$-process.


\subsection{Nucleosynthesis and neutrino induced production of the first r-process peak}

Consider in more details, the effect neutrinos have on the isotropic abundances in ejecta. 
Note that luminosity of the electron anti-neutrino is higher then the one of the electron neutrino due to \magenta{re-leptonization} of the neutrino emitting disk \cite{Foucart:2015gaa}. However, nucleosynthesis yilds are insensitive to the chosen electron antineutrino luminosity because of the \magenta{$\alpha$-effect}.


The result is, again, a confirmation, that the \ac{NSBH} ejects, dynamically, a large amount of \rproc{} rich material, with both second and third \rproc{} peaks reproduced robustly at all considered neutrino luminosities which is to be expected for the outflow with initially very low $Y_e$, that passes several fission cycles. 

Notably, the abundance of the first \rproc{} peak at mass number $A=78$ depends on the neutrino luminosity but in all cases it is under-produced relative to the solar abundance when normalizing to the second and third peaks.

The origin of seed nuclei for the first peak production is the following. After falling from \ac{NSE}, and the strong eqilibrium ceases to hold, the material is composed of heavi nuclei and free neutrons, which, while still being close to the torus, turns into proton. Then it caputres another neutron forming $D$. Then, two deutrons form an $\alpha$-particles, two of which then undergo a neutron-catalyzed triple-alpha reaction, forming a low-mass seed nuclei. A similar process proceeds in neutron rich neutrino driven winds.
These low-mass seed nuclei capture neutrons forming heavier elements, but cannot generally ass the $N=50$ ($Z=28$) in the \rproc{} path due to exhaustion of free neutrons. In the end these seeds end up as a first-peak elements. 


\subsection{Details of the first peak production mechanism}

Consider the indirect production of the first peak nuclei. This relies on the neutrino induced seed production, \eg, neutrino catalyzed triple alpha process, that converts $6$ protons into one seed nucleus. For a $L_{\nu_e}\sim 10^{53}$~erg s$^{-1}$, about $\sim10\%$ of the $r$-process material stops at the first peak. 

A time needed for an indirectly produced seed nuclei to be processed to its final $N=50$, the $\tau$ timescale, is $70-600$~ms postmerger for $L_{\nu_e;52}\in[1,20]$. The time it takes to go from chrage $Z_1$ to $Z_2$ can be estimated considering the waiting point apporxiamtion (c.f., \cite{Kratz et al., 1993}) and beta-decay timescales of intermediate nuclei. Notably at high temperatures, where  $(n,\gamma)$ reaction is in equilibrium with its inverse reaction, the $\tau$ timescale depend only on hydrodynamics $\rho$, $T$, $Y_n$. 
This explains the dependency of the first $r$-process abundances on the ejecta parameters. 

The combined timescale that is required for a seed nucleus to go to the start of the first \rproc{} peak and get processed through the first peak is $\sim500$~ms and $\sim100$~ms respectively for the aforementioned range of neutrino luminosities. The final first peak abundances then scales linearly with neutrino luminosity. After the $\sim600$ms postmerger the free neutrinos are exhausted.
Thus, overall, the neutrino flux in a time-frame of the charge conversion, $[70, 70+100]$~ms, determines the amount of neutrino-induced first peak production. 

Notably, while up to $L_{\nu_e}5\times10^{52}$erg s$^{-1}$, the first peak element production scales linearly with $L_{\nu_e}$, at this value, the saturation occurs. In addition, for a certain slow moving fluid elements with low initial electron fraction and entropy, the neutrino luminosity can have a non-monotonic behavior of the first peak abundance with the neutrino luminosity.

Regarding the effects of thermodynamic conditions and their effect on the first pear elements production the following can be stated. Increased neutrino luminosity raises the fluid electron fraction and temperature. Hence, the total number of seed nuclei is increases. Higher entropy also reduces the ability of the synthesized material to pass the $A=8$ stability gap. Notably at very low temperatures (lower $L_{\nu_e}$) the induced seed production changes considerably and is able to pass the $N=50$ maxiumum. 
The first peak elements are produced even when there no neutrinos through fission of heavy nuclei.


\subsection{Isotopic and elemental abundances, galactic chemical evolution, and low metallicity halo stars}

It was shown that for a set of \ac{BHNS} models, the nucleosynthesis in the ejecta differs only a little. The reason for this is twofold. On one side, abundances at the second and third peak are insensitive to the binary parameters due to fission cycling. On the other, for a given neutrino luminosity, the neutrino produced first peak is insensitive to binary parameters. Notably, larger difference is expected if the ejecta velocity varies more, which in turn changes the time when the neutrino exhaustion the ejected material.

Comparison between solar \rproc{} abundance pattern and models yields show an overall qualitative agreement but considerable quantitative disagreement. In particular, an overproduction of the third peak elements. This disagreement can be attribution to uncertainties in ejecta properties and uncertainties in nuclear data. The latter, in particular can amount to a factor of ten\footnote{In addition, the neutrino-induced fission, that can affect the abundances of the first peak elements, especially in ejecta that undergoes strong neutrino irradiation, is not included in the \ac{NRN}} 

Overall, it is unlikely that \ac{NSBH} is a dominant source of \rproc{} elements. 

\textbf{Elemental abundance pattern} $Y_Z = \sum_i \delta_{Z_i,Z}Y_i$ is commonly used in studyng the abundance patterns in \ac{MP} stars. The comparison is meaningfull for high $Z$ elements, $(51,81)$.
Elements below Mo can be synthesized in \ac{MP} stars via multiple other channels.

As in stars with high enough metallicity the \sproc{} can contribute to the abundance pattern, present in \eg, sun, the comparison can also be done with low metallicity halo stars. Data is available in

\subsubsection{Conclusion}
Considered are the \ac{BHNS} mergers with neutrino irradiation, ejecta follow-up via \ac{SPH} code and nucleosynthesis in post-processing. 
Findings: 
\begin{itemize}
    \item Second and thirds $r$-process peaks are robustly reproduced. (This is not always the case for \ac{BNS} )
    \item Neutrinos do not change the pre-$r$-$Y_e$ considerably, and full $r$-process occures in the entire ejecta. 
    \item Weak dependency of $r$-process final abundances and binary parameters.
    \item Neutrino irradiation affect nucleosynthesis in a way besides changing $Y_e$, -- inducing low mass seed nuclei production via triple-$\alpha$ process. This enhances the synthesis of first peak elements. This is relevant for both \ac{BHNS} and \ac{BNS}. 
    \item Final abundnaces qualitatively agree with solar for second and third peak. The third peak is however overproduced. 
    \item Uncertanties in ejecta parameters and nuclear data input for \ac{NRN} can affects the result significantly.
\end{itemize}



\section{Signatures of hypermassive neutron star lifetimes on rprocess nucleosynthesis in the disk ejecta from neutron star mergers}

\red{We investigate the nucleosynthesis of heavy elements in the winds ejected by accretion disks formed in neutron star mergers.}


\subsection{Introduction}

%% Motivation
Origin of \rproc{} material in the universe is uncertain. Observations of \ac{MP} stars in the galactic halo suggests that a mechanism that produces robust abuncnace patters, that of the Solar system, for $A>130$ \cite{(e.g., Sneden et al., 2008).}, while observations of old very \ac{MP} stars suggest that this mechanism has operated since the early universe \cite{(e.g., Cowan et al., 1999; Ji et al., 2016)}. Less robust abundance patters is found for light $r$-process elements, as diviations observed between the one of the Solar System and \ac{MP} stars \cite{(e.g., Montes et al., 2007).}. Similarly, meteoritic abundances show difference in the timescales on which light and heavy \rproc{} elements formed. This suggests more than one main source of \rproc{} nucleosynthesis \cite{Wasserburg et al., 1996}.

%% Prodocution cites 
While the \ac{CCSN} can contribute to the budget of light $r$-process elements ($A\leq 130$), e.g., \cite{Roberts et al., 2010; Fischer et al., 2010, Hrdepohl et al., 2010}, \cite{Martinez-Pinedo et al., 2012; Wanajo, 2013}, most heavy elements are produced in compact object mergers, \ac{BNS}, \ac{NSBH}, as conditions in their ejecta is highly neutron-rich \cite{(Lattimer and Schramm, 1974)}. These mergers are candidates for gravitational wave telescopes \cite{(e.g., LIGO Scientific Collaboration, 2010; IGO Scientific Collaboration et al., 2015)}, are a focus of numerical relativity modelling \cite{(e.g., Lehner and Pretorius, 2014; Paschalidis, 2017)} and search for electromagnetic counterparts \cite{(e.g., Rosswog, 2015; Fernandez and Metzger, 2016; Tanaka, 2016).}. 
\red{Collapsars are missing}.

%% Dyn. Ej. as a production cite
The dynamical ejecta of \ac{BNS}/\ac{NSBH} mergers is favorable site for the \rproc{} nucleosythesis. Its low initial electron fraction assures the robust \rproc{} abundance pattern for $A>130$ due to \magenta{fission cycling} \cite{(e.g., Goriely et al., 2005),}, that depends weakly on the binary parameters \cite{(e.g., Goriely et al., 2011; Korobkin et al., 2012; Bauswein et al., 2013)}. Additionally, neutrino absorption within the ejecta allows for the production of lighter $A<130$ elements \cite{(e.g., Wanajo et al., 2014; Goriely et al., 2015; Sekiguchi et al., 2015; Radice et al., 2016; Foucart et al., 2016a,b; Roberts et al., 2017)}

%% Disk outflow
A disk formed in \ac{BNS} or \ac{NSBH} mergers can ejecta a fraction of its mass of a longer timescales \cite{(e.g., Ruffert et al., 1997; Lee et al., 2009; Metzger et al., 2009b}. Such ejecta has large $Y_e$, raised by weak interactions \cite{(e.g., Dessart et al., 2009; Fernandez and Metzger, 2013; Perego et al., 2014)}. The mechanisms contributing to mass ejection are weak interactions with a thermal timescale $\sim30$~ms, angular momentum transport processes and nuclear recombination with timescales $\sim1$~s. Mass of the ejecta can exceed that of the dynamical ejecta \cite{e.g., Fernandez and Metzger, 2016} with the neutrino-dorven compoennt accounting for only a small fraction, present if massive neutron star was formed behore the collapse \cite{(Dessart et al., 2009; Fernandez and Metzger, 2013; Metzger and Fernandez, 2014; Just et al., 2015}. 

%% Previous works 
The disk outflows has been shown be the cite of light and heavy $r$-process elements production in early works, that made use of parametric treatment to obtain thermodynamic trajectories for composition analysis \cite{(e.g., Surman et al., 2008; Wanajo and Janka, 2012)}. A robust pattern in produced light elelements with variability in heavy elements was found in ourflowed for disks surrounding the BH \cite{Just et al., 2015; Wu et al., 2016)}. The neutrino driven (polar) outflows from the disks surrounding a massive neutron star, however, were found to the the cite of primaraly light elemnt production $A<130$ \cite{Martin et al., 2015, Perego et al., 2014}

%% This work 
Here the nucleosynthesis in the ejecta from disks surrounding massive NS is considered, with varying lifetimes of the remnant. The parametrized approach mimics the effect that different EOS, angular momentum traspt mechanisms and neutrino cooling would have on the remnant \cite{(e.g., Paschalidis et al., 2012; Kaplan et al., 2014).}. Nucleosytheissi is computed in postprocessing on tracer particles. 


\subsection{Method}

Use \texttt{FLASH} code, that solves Newtonian hydro (with $\alpha$ viscosity) and lepton number (including neutrino absorption) conservation in axisymmetric. 

Neutrinos are treated via leakage scheme for cooling and 'light-bulb' approximation for self-absorption, considering only charged-particle interactions. The surface of the \ac{NS} is treated as a reflecting surface, that also has an outward neutrino flux.

No feedback from \ac{NRN} on the hydro evolution. 

\ac{NRN} starts at $T\leq 10^9$~K, or at maxiumim, if lower. Thus, it starts at \ac{NSE}. \texttt{SkyNet} includes viscouse heating and neutrino heating effects from therodynamic trajectories. 

Important timescales:
The orbital time at the initial density peak $t_{orb}=f(R_d, Mc)\sim3$~ms, with $R_d$ being the radius of the diensity peak, $M_c$ -- the mass of the central object. 
The initial thermal time of the disk $t_{th}=f(M_d, e_{i,d}, L_{\nu,52})\sim30$~ms, where $e_{i,d}$ is the initial specific internal energy of the disk, $L_{\nu,52}$ is a typical neutrino luminosity from the disk.
The initial viscous timescale of the disk $t_{vis}=f(\alpha, H/R, R_d, M_c)\sim200$~ms, where
$\alpha$ is the viscosity parameter, $H/R$ is the height-to-radius ration of the disk.

Neutrino driven outflow are launched on the thermal timescale. Long-term outflows are launched on a viscous timescale. 

\magenta{dimensionless spin parameter} $\chi = J_c / (G M_c ^2)$, where $J$ is the BH angular momentum, $c$ is the speed of light, $G$ is the graviational constant. 


\subsection{Results and discussion}
\red{very brief}

When a \ac{MNS} collapses to a \ac{BH} a \magenta{rarefaction wave} moves outward from the inner boundary, quenching the thermal outflow \citep{Metzger:2014ila}. After this disks re-adjusts and settles in on the path of a viscous evolution of a disk around a BH. 

Densities of $\sim 10^11$g cm$^{-3}$ are sufficient to trap neutrinos locally, creating a local emission hotspot. 
High density in the disk's midplane shadows the outer disk from irradiation by the remnant. This is also supported by monte-carlo simulations \citep{Richers:2015lma}. Hence, the large weak interatcion timescale leads to the $Y_e$ on the inner disk, near the plane, being low $Y_e\sim0.1$. Away from the midplane, the timescale for the weak interactions is shorted and the electron fraction is higher. In particular, material at $\sim45^{\circ}$ displays high electron fraction.
With time, the density of the inner disk decreases, the $Y_e$ rises (as the degree of matter degeneracy decreases), the intensity of weak interaction decreases. 


\subsubsection{Neutrino emission}

Due to disk re-absorption of the $\nu_e$ and anti-$\nu_e$, the corresponding luminocities between the remnant and the disk remain very simialr. 
After the collapse and disk re-adjustment, the $L_{\bar{\nu}_e}$ dominates over $L_{\nu_e}$, as weak interactions \magenta{leptonize} the initially low $Y_e$ disk (positron capture rates are higher then the electron ones), and the continuous decrease in $\rho$, leads to increase in equilibrium $Y_e$.


\subsubsection{Ejecta properties}

To connect the disk and ejecta properties, consider trace particles conditions when they last pass the point of $Y_{e,\:T=5\text{GK}}$ and $s_{T=5\text{GK}}$. At lower temperatures, the composition moves out of \ac{NSE}, requiring the full \ac{NRN} -- hence conditions at $T=5$GK are the initial conditions for the nucleosynthesis.

The correlation between the remnant lifetime and both the amount of the ejecta and its electron fraction is noted \citep{Metzger:2014ila}. The matter is ejected doe to the strong neutrino heating, viscouse heating and nuclear recombination. Matter ejected from the innder accretion disk reaches the $\beta$-equilibrium and ejects with higher electron fraction (?). 

The mas of low-$Y_e$ ejecta depends weakly on the \ac{MNS} lifetime as long as it is longer then $30$~ms. This is because the longer \ac{MNS} leaves the higher the average $Y_e$ of the ejecta, more ejecta is ejected. 

Ejecta components identified. \nwind{}, it shows a tight correlation between $Y_{e,\:T=5\text{GK}}$ and $s_{T=5\text{GK}}$. with $Y_e$ reaching $0.5$. 
Second ejecta, shows weak correlation between electron fraction and entropy, and associated with the disk reaching the \magenta{advective state} with little to none neutrino heating or cooling and is dreiven by the energy injection due to angualr momentum transport and nuclear recombination \citep{Metzger:2014ila}. Associated with the convection within the disk.

To qualitatibely asses the amount of the \nwind{}, consider the contribution of neutrinos to changing the $s_{T=5\text{GK}}$ of a tracer, \textit{i.e.,} the increase in entropy deu to neutrino absorption, neglecting cooling. A tight 'boomegang' shape between $Y_{e,\:T=5\text{GK}}$ and $s_{T=5\text{GK}}$ is an indicative of such wind. \red{See that this can be added to the long paper investigation of the \nwind{}}.
Note taht this component is present if $t>30$~ms, lifetime. 


\subsubsection{Nucleosynthesis}

Standard techniques: consider mass-averaged composition, abundances, multiplied by the ejecta mass to highlight their relative contributions to the different \rproc{} regions. 
Usually, solar abundances are scaled to match the second peak of the models.
\magenta{Rare-Earth peak} located at $A\sim165$.
\magenta{First peak} $A\sim80$
\magenta{Second peak} $A\sim130$
Regarding other sources or \rproc{} material. The expected abundance pattern is weighted toward the $3$rd peak for the dynamical ejecta \cite{(e.g., Goriely et al., 2011;  Wanajo et al., 2014; Roberts et al., 2017)} and towards $1$st peak for \ac{CCSN} \cite{(e.g., Wanajo, 2013; Shibagaki et al., 2016; Vlasov et al., 2017)}. The solar \rproc{} abundances then, is a combination of all the sources, weighted by their rate and yield per event (taking into account production delays).
Shit to the higher $A$ for the third peak, for models, might be die to well-known shortcoming of the FRDM mass model \cite{(e.g., Mendoza-Temis et al.,2015; Mumpower et al., 2016)}
Peak at $A\sim132$ might be attributed to "late-time heating that photodissociates neutrons from synthesized heavy elements" which leads to "additional neutron capture and a pile up of material at the doubly magic nucleus" (see \cite{e.g., Wu et al. (2016) })
For the long-lived models, if the lifetime $\geq 100$~ms, the first peak $A\in[70,90]$ is overproduced with respect to the solar values, then the abundances are normalzied to the second peak. 

A way to quanify the \textbf{relative contribution} of model's abundances to the different regions of the \rproc{} distribution, the average abundances around those peaks can be considerd, normalized to the solar values. 
For instance, assume that $A_1 \in [70,90]$, $A_2 \in [125,135]$, $A\in[186,203]$ and $A\in[160,166]$ for the $1$st, $2$nd $3$rd and rare-Earth peaks respectively. Let $Y_{i}$ be the summed final abundances within each peak. Then, the realtive contribution can be evaluated as 

\begin{equation}
\frac{Y_{1st}}{Y_{2nd}} = \log_{10}\frac{Y_{1st}}{Y_{2nd}} - \log_{10}\frac{Y_{1st,\odot}}{Y_{2nd,\odot}},
\end{equation}

with those denoted with $\odot$ correspodong to the solar system abundances peaks. 
These values would generally go between $-2$ and $2$ 


\subsubsection{BH spin mimicking HMNS lifetime}

high-Ye material also correlates with BH spin (\cite{Fernandez et al., 2015a}).
While a larger BH spin has a similar overall effect on the disk ejecta composition as a longer \ac{HMNS} lifetime.
At best, a rapidly spinning \ac{BH} can mimic a \ac{HMNS} of modest lifetime.


\subsubsection{Lanthanides, actinides, and heating rates}

The light curve and spectrum of the radioactive decay powered \ac{EM} transient are determined by two propetiesL the opacity (of up to $\times10$ of iron-group elements) of the material and the radioactive heating rate. largest opacities are found in lanthanides $Z\in[58,71]$ and actinides $z\in[90,103]$, that have open $f$-shells, and thus complex atomic line structures \cite{(Kasen et al., 2013; Tanaka and Hotokezaka, 2013; Fontes et al., 2015)}

One considers the trajectory-averaged lanthanide $\langle X_{La} \rangle$ and actinide $\langle X_{Ac} \rangle$ mass fraction of the ejecta. Respectively, in dynamical ejecta $\langle X_{La} \rangle$ and $\langle X_{Ac} \rangle$ are usually $10^{-2}$ and $10^{-3}$. For along lived remnants, the $\langle X_{La} \rangle$ drops to $10^{-3}$. This for the dynamical ejecta the increase in opacity might reach $10$, in disk ejecta it is more of a factor of a few. 

In addition to the disk outflow, dynamical ejecta, its spatial distribution and compostion defines the color of a Kilonova. 

There is a correlation between $Y_{e, 5GK}$ and $s_{5GK}$, is increase in former requires either strong neutrino heating, that increases $Y_e$ or long timescales, thermal timescales, which are sufficient for weak interactions to raise $Y_e$ to its equilibrium value.

A material with high entropy $S\geq30$ $k_B$baryon$^{-1}$ and electron fraction $Y_{e,5GK}\in(0.25,030)$ also produces a significant fraction of lanthanides and actinides. This is due the dependency of the $La$, $Ac$ abundances on the neutron-to-seed ratio, $Y_n/Y_{seed}$. If it is $\sim40$ the $La$ and $Ac$ are produced. The $Y_n/Y_{seed}=35$ represents the boundary. In the ejecta, at high entropies and $Y_e>0.25$, such $Y_n/Y_{seed}$ can be reached and exceeded. Hence, the production of $La$ and $Ac$. Note that extremely high $s_{5GR}$, \eg, $\geq 200$, can allow $La$ and $Ac$ production even at $Y_{e,5GK}>0.4$. 

For the Kilonova modeling, the quantity of importance is the total total radioactive heating rate $\varepsilon_{tot}$ at a given time after merger, which depends on the ejecta mass. Note, that it is not addicitve between different ejecta components. note, that in case of the \ac{BNS}, the high opacity, quasi-spherical ejecta \cite{(e.g., Hotokezaka et al.,2013b)} can obstruct the disk ekecta. However, during the high-temperature phase, the lanthanides-rich material allows phitons to escape \cite{(Barnes and Kasen, 2013; Fernandez et al., 2017)}, and thus the short-lived blue component must be seen.

To properly account to the emergy deposition by the radiactive heating, the energy converstion from the decay products into thermal energy, that has a limited efficiently \cite{(e.g., Metzger et al., 2010; Hotokezaka et al., 2016; Barnes et al., 2016)}, ought to be considered. Taking the total heating rates woould provide the upper limits to the bolometric luminosity estimation. 

The heating rate falls with increassing $Y_e$, until it starts to oscillate. The oscillations are given gy the NSE being domianted by nuclides with matching $Y_e$ and the half-life matching the time of the heating rate computation. If there is no such nuclei, the heating is drastically reduced. 


\subsubsection{Comparison with r-process abundances in metal-poor halo stars}
\red{very very brief}

It is believed that \ac{MP} stars in the galactic halo have been enriched with $r$-process elements, which they show in the specra, by a few events \cite{(Cowan et al., 1999)}

The abundances (with respect to $Z$) are scaled to match solar values in $Z\in[56,75]$, as models predict similar abundances in this range and match solar ans \ac{MP} abundances well. 
Lower $Z$ elements are oversporduced by models, however the qualitative agreemnt at lower $Z$ found in MP stars.
Overall, the disk outflowes for $Z=44$ and $A=47$ display abundances incosistent with pure disk outflows. Thus suggests that an additioanl source contributed to the enrichment of the MP stars.
Notably, that when only the dynamical ejecta is considered, producing elements with $Z>55$, the combined final abundance pattern is consistent with MP halo stars \cite{(cf. Just et al., 2015).}. For $Z<42$, it might me that SN contributed. 


\subsubsection{Impact of angular momentum transport}

While energy deposition by neutrinos can unbind a significant amount of the material, \cite{Ruffert et al., 1997}, so can the transport of angular momentum. The latter modifies disk hydrostatic balance, changing the centrifugal force and thus inducing a disk spread, while it is being accreted. In addition, it increases the local entropy of the fluid. The quntatice description of the mechanisms depends on the exact way the process is modelled. Similarly, MHD turbulence heats up the fluid \cite{Hirose et al., 2006}. 

Notably, here the inclusion of the viscosity increases the disk mass by a factor of $6$, allowing almost all disk mass to be ejected. 

In late-time ejecta, large dispersion in $s_{5GK}$ for agiven $Y_{e,5GK}$ might be attributed to the convective motions in the advective phase of the disk. 

Notably, an inclusion of the viscosity can enhance the $\nu$-driven wind, as the former produces disk spreading, and contributing to the energy gain by the ejecta from neutrinos. 


\subsection{Conclusion}

A note on nucleosynthesis. Since dynamical ejecta tends to be neutron-rich, producing heavy elements between $2$nd and $3$rd peak, if the abundances of the latter are consistent with the solar abundances, it implies, that the short-lived remnats are dominant source or $r$-process material. However, if the third peak is over-produced with respect to the second one, then a source of lighter elements is required, \textit{i.e.,} long-lived remnants and their outflows. 