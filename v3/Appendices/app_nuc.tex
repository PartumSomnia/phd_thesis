%% ============================
%%
%% Appendix A
%%
%% ============================

\chapter{Details on the \rproc{} \nuc{}}
\label{app:nuc}
%% \externaldocument{intro}
%% --------------------------------------

\section{Parameterized \rproc{}}

Here we discuss the results of the \citet{Lippuner:2015gwa}

The parameterized calculations (see sec\ref{sec:nucleo_method}) shown that 
the \rproc{} \nuc{} has the following properties depending on the entropy and electron fraction.

\begin{itemize}
    \item High Entropy -- high neutron-to-seed ratio, even though seed nuclei are light, -- full \rproc{}
    \item Low entropy, \rproc{} is jugged as there are not many free neutrons available. However, it still preceeds and the seed nuclei are very heavy.
    \item At high $Y_e$ the, where ejecta becomes less neutron rich, the full \rproc{} is no longer produced as there are not enough neutrons per seed nucleus to rich the third peak.
    \item High entropy and low $Y_e$ allows for $3$rd peak reproduction still as there are few seed nuclei and neutron to seed ration is high.
\end{itemize}


\subsection{Lanthanides turnoff and heating rate as a function of $Y_e$}

A turn off is a point in parameter space, when the mass fraction of synthesized lanthanides (actinides) $X_{\text{La}}$ ($X_{\text{Ac}}$) drops below $10^{-3}$. For example, a maximum $Y_e$ (and minimum $\bar{A}_{\text{fin}}$) for which the $X_i \geq 10^{-3}$ is a turn off point. 

The turnoff prevents a large amount of lanthanides (actinides) from being produced and suppresses the fission cycling.

Here $\bar{A}_{\text{fin}}$ is the final average mass number,

\begin{equation}
\bar{A}_{\text{fin}} = \frac{1}{Y_{\text{seed}}(0) + Y_{\alpha}/18},
\end{equation}

where $Y_{\alpha}(0)$ is the initial abundance of alpha particles, $Y_{\text{seed}}(0)$ is the initial abundance of the seed nuclides ($A\geq12$). The $Y_{\text{seed}}(0) + Y_{\alpha}/18$ thus denotes the number abundance of heavy nuclei at the end of the $r$-process, assuming that it takes $\sim 18$ $\alpha$-particles to create a seed nuclei \cite{Woosley:1992}. Further, assuming that the number of seed nuclides by the end of the \rproc{} equals the initial number of seeds plus those produced in $\alpha$-process, \ie, neglecting the effects of fission cycling, the total mass-fraction of heavy nuclei at the end of the \nuc{} is $1$. Note that $\bar{A}_{\text{fin}}$ is a function of initial abundances only and a good indicator of the direction that \nuc{} would take. 

Considering a range in $s\in\{1, 10, 30, 100 \}$, $\tau\in\{0.1, 1, 10\}$~ms and $Y_e\in(0,0.5)$,
the \nuc{} calculations state, nuclear heating rates, $M\varepsilon$, ($1$day, $10^{-2}M_{\odot}$) are rather constant up to $Y_e\approx 0.4$, when a particular nuclides start dominate the heating. The number of fission cycles is increased for high $s$ models, due to increased production of seed nuclides by triple-$\alpha$ process. 

Neutron-rich freeze-out occurs at high $s$ and small $\tau$, where ejecta high temperature and rapid expansion allows for a fewer capture events on seed nuclei and free neutrons escape. An ejecta component with such conditions was suggested to produce a ultraviolet precoursour to the kilonova on a timescale of hours, powered by the decay of frozen-out neutrons \citep{Metzger:2014yda}.

\gray{See if that can be modelled easily for our BNS models}

The heating rate, $M\varepsilon$, are roughly constant with $Y_e$, at $1$ day, as long as the fraction $X_{\text{La+Ac}}$ produced is constant. 
After lanthanides production ceases and their fraction falls, the instantaneous heating rate (at $1$ day) decreases only slightly (up to $1$ order of magnitude). 
Considering the integrated nuclear heating amount as a function of $Y_e$ ($s=\text{const}$, $\tau=\text{const}$), it does decreases by $1-2$ orders of magnitude, when the \rproc{} stops producing unstable heavy lanthanides, and stable nuclides are produced directly.


\subsection{Fission cycling}


Fission cycling is a process where freshly synthesized via strong \rproc{} heavy nuclides with $A\sim 300$ undergo fission just to become seed nuclides for a similar \rproc{} leading to an $A\sim 300$ nuclide. This process eliminates the dependency of the final abundances on the initial conditions and it also limits the maximum nuclide mass that ca be achieved. 

The number of fission cycles can be estimated via a ration of the seed nuclides at time zero and the number of seeds at the time when there are no more free neutrons available. This is motivated by the fact that neutron capture itself does not create new seeds, only increases the mass of them, while fission, splitting heavy nuclide in two, generates additional seeds. 

The number of fission cycles is tight to how much lanthanides and actinides are produced. In particular, as fission cycling limits the maximum mass of the nuclide that can be created, the fraction of lanthanides, actinides as well as heating $\varepsilon$ are insensitive to the initial $Y_e$. The number of cycles is thus tight to the initial $Y_e$. The lower it is, the more free neutrons available and thus more cycles would occur. After the fission cycling stops the $r$-process maximum mass drops to $A\sim 250$. Thus the amount of actinides drops as only the litest of them can be produced that do not fission immediately. 


\subsection{Lanthanide production and heating rate in the full parameter space}
\red{Can be used in the main text}

The properties of the \ac{EM} signal, kilonova, are determined by the energy available, \ie, nuclear heating rate, and how easy this energy can escape the medium, \ie, photon opacity. Both, in turn, depend on the amount of lanthanides (and actinides) available. 

Consider a parameter space in $Y_e$, $s$ and $\tau$. The mass fraction $X_{\text{La}+\text{Ac}}$ have the following general dependency on the parameters. 

For a very low ($Y_e \sim 0.01$) the lanthanides production is mostly assured with exception of two cases: high $s$ - low $\tau$ and very large $\tau$, low $s$. In case of the former, the reason is the increased timescale for the neutron capture and lowered mass of the seed nuclei, which might lead to the neutron-rich freeze-out. In case of the latter, the long expansion timescale results in the newly-synthesized heavy elements decaying under the high density and temperature conditions sufficient for photodissociation, binning the composition back to NSE. When the material does cool down enough for the $r$-process to produce heavy elements agian, the electron fraction has been already increased through $\beta$-decays, allowing for the weak $r$-process only. 

For an intermediate $Y_e \sim 0.25$, where the boundary between lanthanides-right and lanthanides-free composition lies. At the $Y_e=0.25$, lanthanides are produced if the $s \sim 1$ is very low or very high $s \sim 100$. In case of the former this is due to seed nuclei being initially very heavy and few captures are needed to produces lanthanides, while in case of the latter, a large number of free neutrons is available as triple-$\alpha$ process is suppressed \citep{Hoffman:1996aj}, Meanwhile, high-$\tau$ low-$s$ and low-$\tau$ high-$s$ (symmetric corners of distribution) are lanthanides-free. 

Considering the heating rate, $M\varepsilon$, at $1$~day for $M=10^{-2}M_{\odot}$ the following dependencies observed.
For $Y_e\in[0.04,0.35]$ the heating rates (at $1$~day) remain almost independent of $s$ and $\tau$. The dependency on $Y_e$ becomes prominent at $Y_e>0.35$, however the dependency on $s$ and $\tau$ are rather weak. In particular, there heating rates peak at $Y_e=0.425$ due to the decay of $^{66}$Cu that has a rather large $Q$-value.
Notably, no significant correlation found between $X_{\text{La}+\text{Ac}}$ and heating rate at $1$~day. This independence from composition results from the fact that at $1$~day the main contributions to the heating rate would come from nuclides with corresponding half-life (decay energy depends on half-life). And thus, even though the composition of the material does depend on the $Y_e$, $s$ and $\tau$, the same nuclides are considered when heating rates are estimated. At very high $Y_e$, however, the heating rate can be dominated by a selected nuclides and thus it depends on the composition.


\subsection{Fitted nuclear heating rates}


Decay of newly synthesized heavy elements powers kilonova. There are however, particular nuclei decay of which has the largest contribution. To determine these, the fractional heating contributions of all nuclides are integrated. The time window is between $0.1$ and $100$ days, $s\in[1,100]$ $k_B/\text{baryon}$ and $\tau\in[0.1,500]$~ms. 

The single most important nuclide in the aforementioned window is $^{132}$I. The origin of this nuclide is the decay of double magic $^{132}$Sn, that is produced in large quantities and then decays within minutes to $^{132}$Te (through $^{132}$Sb). The decay of $^{132}$Te is not very energetic, but it results in $^{132}$I, which decays with $Q=3.6$~MeV. This determines its large contribution to the heating. 

Overall, at very low $Y_e < 0.25$, nuclei from the second and third peaks ($A\sim 130$ and $A\sim 200$) domiante the heating (and few $A\sim 250$ ones), while at higher $Y_e$, most of the heating comes from the second peak. This is in agreement with dominant $\beta$-decay nuclei found in \citep{Metzger:2010}. In addition, heating comes from the spontaneous fission of heavy elements, produced alongside actinides.

At higher $Y_e\in(0.25,0.375)$, the first $r$-process peak, $A\sim 88$ makes a significant contribution as well as neutron-rich side of the iron-peak. 

At even higher $Y_e>0.375$, the proton-rich side of stability around the iron peak exerts a contribution.




\section{The influence of neutrinos on r-process nucleosynthesis in the ejecta of black hole–neutron star mergers}
%% \section{NEUTRINOS AND BHNS R-PROCESS NUCLEOSYNTHESIS}
%% \cite{Roberts:2016igt}




The influence of neutrinos on r-process nucleosynthesis
in the ejecta of black hole–neutron star mergers

Considering the output of the general relativistic simulation of a \ac{NSBH} merger, mapping it onto the Newtonian smoothed particle hydrodynamics \ac{SPH} code and investigate the evolution of ejecta before the outflow is homologous. The \nuc{} is added in postprocessing, considering different levels of neutrino irradiation from the disk. The neutrino irradiation can affect the final abundances as neutrino absorption on free neutrinos turn them into protons, changing the $Y_e$, adding to the seed nuclei budget.


\subsection{Introduction}

NSBH were expected to be detected by LIGO/VIRGO \citep{TheLIGOScientific:2014jea,TheVirgo:2014hva}, (\gray{and they were but with no counterpart}), possibly hinting on the origin of short Gamma ray Bursts \cite[\eg][]{Lee:2007js} and \rproc{} nucleosynthesis \citep[\eg][]{Lattimer:1976,Korobkin:2012uy,Bauswein:2014qla}.

While galactic chemical evololution hinted that \ac{SN} might be largely responsible for \rproc{} elements enrichment \citep[\eg][]{Qian:2000bh,Argast:2003he}, the conditions within \acp{SN} are not perfect for this process \cite[\eg][]{Arcones:2012wj}. These conditions however, are present when neutron stars or NS and BH merger \citep{Freiburghaus:1999}. Modeling galactic chemical evolution is difficult, as these are high yield rare events that require a significant time between progenitor formation and merger. In particular, discrepancies between such models and observations of \rproc{} elements abundances in low metallicity hallow \citep{Qian:2000bh,Argast:2003he}. More complex models of galactic chemical evolution show however a reconciliation of the disagreement \citep{Matteucci:2014,Shen:2015,VanDeVoort:2015,Ishimaru:2015} \gray{And recent observations, GW170817, indicate the mergers are one of the main sources of $r$-process material.}

Weak interactions were shown to be important in mergers of two neutron stars, affecting the properties of the outflow \citep{Wanajo:2014wha,Goriely:2015fqa,Sekiguchi:2015dma,Foucart:2015gaa,Palenzuela:2015dqa,Radice:2016dwd} \gray{Can Add long paper in the list}. The outcome of the merger has a complex dependency on \ac{EOS}, electromagnetic field and neutrino effects \citep[\eg][]{Neilsen:2014hha,Palenzuela:2015dqa}.

In case of NS and BH merger, weak interactions are expected to play much lesser role. Low entropy in the tidal part of the dynamical ejecta supresses the $e^{-}$ and $e^{+}$ captures. Neutrino interactions are supressed by the high outflow speeds and low neutrino luminosity \citep[\eg][]{Foucart:2014nda,Foucart:2015vpa}. Thus, strong \rproc{} is likely to occure in the \ac{NSBH} outflows \citep{Lattimer:1976,Lattimer:1977,Korobkin:2012uy,Bauswein:2014qla}.

The merger rates for \ac{NSBH} is quite uncertain (\gray{Check the LIGO papers}). The ejecta mass from the mergers depends on the binary parameters, BH spin and mass in particular \citep{Foucart:2012vn,Hotokezaka:2013kza,Bauswein:2014qla,Kyutoku:2015gda}. Increasing the former (in the direction of the orbital angular momentum), increase the amount of ejecta, while increasing the latte leads to a complex effect on the ejecta mass \citep{Kyutoku:2015gda}.


\subsection{Methods}

Code: Spectral Einstein Code (SpEC) with neutrino leackage scheme. EOS: LS220

SPH evolution of ejecta

\subsubsection{Nuclear reaction network and weak interactions}

The nucleosynthesis calculation requires the density history of the ejecta, its initial composition and entropy. This can be done performing a hydrodynamics simulation of the expanding ejecta (which allows to account for the feedback the nucleosynthesis has on ejecta, \eg, generated entropy) or assuming an expansion mode, \eg, homologous expansion. Change in electron fraction during the expansion can be included via weak interactions. Otherwise, it can remain constant.

Notably, a single \ac{EOS} may not extend to the temperature-density region where \rproc{} is still possible in the expanding ejecta. As the nuclear reactions affect the \ac{EOS} only marginally below $\rho\sim 10^{12}$g cm$^{-3}$, a switch to another, multi-species non-degenerate ideal gas \ac{EOS}, \citep[\eg][]{Timmes:1999}. However, caution must be exercised in this procedure, as a single nucleus approximation of the LS220 eos predicts nuclei different from thos from full \ac{NSE} calculation and thus different internal energy. 

For a given density history (of a Lagrangian particles), the \nuc{} can be performed via \ac{NRN} \texttt{SkyNet} \citep{Lippuner:2015gwa}. If it is required to track the evolution of the electron fraction, (which can be the case for high densities and temperatures where weak interactions are not in equilibrium), the NSE evolution mode can be used. Below $T\sim 7$GK, however, a full \ac{NRN} evolution is required.
The inclusion of weak reactions, -- absorption of neutrinos and anti-neutrinos by electrons and positrons, and electrons and positions by free nucleons. The expression for neutrino capture rate is complex, which requires different distribution functions for electrons and positrons. In its applied form it does not include momentum transfer to the nucleons and neglects weak magnetism corrections (note however, that might be of importance for the neutrino-driven with \citep{Horowitz:2001xf}). Neutrino-distribution is usually assumed to the be of the Fermi-Dirac sahpe in energy sapce. While the simple spherical emitting surface is a common approximation, the geometry might be of importance as the \ac{NSBH} (\ac{BNS}) merges form a disk-shape emitting region.


\subsection{Results and discussion}











\section{Lightcurves}
\red{To be moved to Kilonova chapter}
\red{For AB magnitudes look the Lippuner Paper 
Computing AB magnitude from light curve
Page 143
}


Consider a rather simplified way of modeling a kilonova. 
The expansion is assumed to be homologous, \ie, $r = \upsilon t$ with the density structure being

\begin{equation}
\rho(t, r) = \rho_0(r/t)\Big(\frac{t}{t_0}\Big),
\end{equation}

instead of uniform density of the expanding ball, as the latter might show superluminal expansion velocities.

The \ac{NEN} gives the heating as a fucntion of time, $\varepsilon(t)$, but only a fraction of released energy (in form of neutrinos and gamma-rays) will be passed into the material, thermolized in the material. This fraction is difficult to calculate. It requries gamma ray transport schemes. One can assume it to be constant, \textit{i.e.,} $f=0.3$ \cite{Barnes:2013wka}. Thus, only $f\varepsilon$ of the released in nucleosynthesis energy is deposited into the material.

Assuming the homologous outflow, the velocity can be considered as a Lagrangian coordinate and the Lagrangian radiative transport equations can be written to the first order in $\upsilon/c$ \citep{Mihalas:1984}
The equation of state can be assumed to be the one of an ideal gas, $u=3T/(2\mu)$ and the Eddington factor entering the Lagrange equations can be obtained from the static Boltzmann transport equation. 
This method is similar to \citep{Ensman:1994}. 

Calculating the exact opacities of a mixture of atoms with complex line structure, such as iron-group elements, and even more so, lanthanides and actinides is a difficult task. \red{it was adressed though by Tanaka et al}.
However, if the detailed opacity calculations are beyond the scope, one can assume an approximate opacities given by 

\begin{equation}
\kappa = \kappa_{\text{Fe}}(T) + \sum_i \max[\kappa_{\text{Nd}}(T,\: X_i) - \kappa_{\text{Fe}}(T),\: 0],
\end{equation}

where $\kappa_{\text{Fe}}(T)$ and $\kappa_{\text{Nd}}(T,\: X_i)$ are the iron and neodymium opacities given in \cite{Kasen:2013xka}, $X_i$ is the mass fraction lanthanides or actinides species. 
For the gray opacity calculations, the Plank mean opacity can be assumed (it is a good approximation when wavelength dependent opacity is calculated in the Sobolev approximation)


\subsection{Dependence of kilonova light curves on the outflow properties}


In case of a lanthanides-rich ejecta, $Y_e=0.01$ $(Y_e=0.19)$ the different in heating rates is small. However, the difference in opacity results in the peak time difference of about a weak and magnitude difference of about $1$ magnitude. In addition, the effective temperature at peak is also much lower for the $Y_e=0.01$. 
In case of a lanthanides-poor ejecta, $Y_e=0.25$ $(Y_e=0.5)$, the difference in heating rates is more prominent as in the latter, mostly stable nuclei are produced.The difference in peak time of the latter 1-2 days later and its peak magnitude about $1$ order dimmer. This is the effect of the reduced heating.
Thus difference in opacity dominates the light curve properties at low $Y_e$ while effect of different heating rates on the lightcurve is prominent at high $Y_e$. 
The interplay between opacity in heating can be seen from the fact that more heating leads to the brighter late part, because hotter ejecta maintins its high opacity (more populated excited levels) for longer \citep{Kasen:2013xka}. 

The effect of different entropy of the ejecta is the most prominent at $Y_e=0.25$, where the $s=1k_B/$baryon and $s=10k_B$/baryon give lanthanides-righ and -poor light curves.
Thus, if lanthanides are produced the transient is expected to be longer, dimmer and redder, while of the lanthanides are not produced, it is brighter, bluer and shorter.
The peak times and magnitudes also assessed by \citet{Roberts:2011,Barnes:2013wka,Tanaka:2013ana}. 


\subsection{Mass estimates of potential kilonova observations}


A particular example for the investigation of the depenedency of a lightcurve on ejecta mass can be done for the GRB130603B, for which observations is infrared and optical \citep{Berger:2013wna,Tanvir:2013pia}

The lower limit on the ejected mass can be computed. The method is the following. Consider a set of parameters for ejects, velocity and mass, compute the AB magnitude for the transients produced bu this ejecta. Account for the cosmology and instrument filter responds for the observed data. Compare the observed magnitudes with model ones and interpolated the minimum that reproduces observations.


\subsection{Conclusion}


The final amount of lanthanides and actinides depends mostly on $Y_e$. Ejecta is free of these elements for $Y_e > 0.25$. Notably, lanthanides-free ejecta can be achieved also for low $Y_e$ under special conditions in terms of entropy and expansion timescale. High $s$ low $\tau$ ang low $Y_e$, for instance, give a neutron rich freeze-our. Contrary, at low $s$, high $\tau$, there is a late time heating that resets the composition to the \ac{NSE} corresponding to a higher $Y_e$. 

High opacity of lanthinides (a factor of a 100) results in the peak of a Kilonova occurring a weak later and at $1$ magnitude magnitude lower. Similar results were found in \citet{Roberts:2011;Kasen:2013xka,Tanaka:2013ana,Grossman:2013lqa}. Heating rates however, at $1$ day depend weakly on the lanthinides abundances, and thus on $Y_e$. When ejecta is initially low $Y_e$ the dependency is small, as the same ensemble of nuclides dominating the heating is produced. If the $Y_e$ is high, however, the specific nuclides dominate the heating and its rate changes strongly depending on composition. 