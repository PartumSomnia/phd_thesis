% Appendix Template

\chapter{Appendix Title Here} % Main appendix title

\label{app:fit} % Change X to a consecutive letter; for referencing this appendix elsewhere, use \ref{AppendixX}

This appendix summarizes all fit coefficients.
Dynamical ejecta coefficients can be found in 
Tab.~\ref{tab:dynfit:poly} and 
Tab.~\ref{tab:dynfit:fit_form} for the polynomials and fitting
formulae respectively.
Disk coefficients can be found in 
Tab~\ref{tab:diskfit:poly} and 
Tab.~\ref{tab:diskfit:form} for the polynomials and fitting
formulae respectively.
The coefficients of the recommended fitting formulae, as discussed in
the conclusion, are highlighted in the tables.

%% --------------------------------
%% Best fit parameters for dyn ejecta and disk


%\begin{table*}
\begin{sidewaystable}
    \caption{
        \label{tab:dynfit:poly}
        Dynamical ejecta properties:
        coefficients for polynomial regression of various
        quantities. Results for both
        first order and second order polynomials are reported $P_2^1(\tilde{\Lambda})$ and $P_2^2(q, \tilde{\Lambda})$
        The recommended calibration for $P_2^2(q,\Lambda)$ is highlighted.
    }
    \scalebox{0.90}{
    \begin{tabular}{l|l|ccccccccc}
        \hline\hline
        Quantity &Datasets & $b_0$ & $b_1$ & $b_2$ & $b_3$ & $b_4$ & $b_5$ &  $\chi^2_{\nu}$ & $R^2$  \\ \hline
        $\log_{10}(\md)$ & \DSrefset{} & $-3.69$ & $4.34\times10^{-3}$ & $-3.66\times10^{-6}$ & & & & 3.0 & 0.035 \\ 
        & \& \DSheatcool{} & $-2.10$ & $-5.84\times10^{-4}$ & $8.86\times10^{-8}$ & & & & 37.3 & 0.056 \\ 
        & \& \DScool{} & $-2.85$ & $8.99\times10^{-4}$ & $-7.42\times10^{-7}$ & & & & 45.6 & 0.017 \\ 
        & \& \DSnone{} & $-2.35$ & $-1.29\times10^{-5}$ & $1.82\times10^{-8}$ & & & & 123.6 & -0.020 \\ 
        \hline
        $\vd$ [c] &  \DSrefset{} & $4.63\times10^{-1}$ & $-9.58\times10^{-4}$ & $7.30\times10^{-7}$ & & & & 3.2 & 0.213 \\ 
        & \& \DSheatcool{} & $3.43\times10^{-1}$ & $-4.84\times10^{-4}$ & $3.25\times10^{-7}$ & & & & 3.3 & 0.211 \\ 
        & \& \DScool{} & $2.77\times10^{-1}$ & $-2.38\times10^{-4}$ & $1.39\times10^{-7}$ & & & & 6.3 & 0.133 \\ 
        & \& \DSnone{} & $2.50\times10^{-1}$ & $-6.71\times10^{-5}$ & $2.16\times10^{-8}$ & & & & 7.6 & 0.051 \\ 
        \hline
        $\yd$ & \DSrefset{} & $3.17\times10^{-1}$ & $-5.82\times10^{-4}$ & $5.41\times10^{-7}$ & & & & 43.7 & 0.062 \\ 
        & \& \DSheatcool{} & $1.99\times10^{-1}$ & $-3.08\times10^{-5}$ & $4.62\times10^{-8}$ & & & & 38.6 & 0.026 \\ 
        & \& \DScool{} & $1.45\times10^{-1}$ & $1.09\times10^{-4}$ & $-6.91\times10^{-8}$ & & & & 36.3 & 0.017 \\ 
        \hline
        $\athetarms$ [deg] & \DSrefset{} & $4.09\times10^{+1}$ & $-5.32\times10^{-2}$ & $5.20\times10^{-5}$ & & & & 21.7 & 0.045 \\ 
        & \& \DSheatcool{} & $2.55\times10^{1}$ & $3.76\times10^{-3}$ & $4.33\times10^{-6}$ & & & & 18.7 & 0.047 \\ 
        & \& \DScool{} & $1.47\times10^{1}$ & $3.37\times10^{-2}$ & $-1.79\times10^{-5}$ & & & & 14.3 & 0.115 \\ 
        \hline\hline
        $\log_{10}(\md)$ & \DSrefset{} & $1.04$ & $-3.31$ & $-6.89\times10^{-3}$ & $4.19\times10^{-1}$ & $5.09\times10^{-3}$ & $5.83\times10^{-7}$ & 1.6 & 0.748 \\ 
\rowcolor{lightgray} & \& \DSheatcool{} & $3.35\times10^{-2}$ & $-1.44$ & $-6.24\times10^{-3}$ & $-1.36\times10^{-1}$ & $3.99\times10^{-3}$ & $9.03\times10^{-7}$ & 56.4 & 0.192 \\ 
        & \& \DScool{} & $-5.75$ & $3.84$ & $1.19\times10^{-3}$ & $-1.02$ & $-5.72\times10^{-4}$ & $-4.98\times10^{-7}$ & 24.4 & 0.122 \\ 
        & \& \DSnone{} & $-4.87$ & $3.63$ & $-1.53\times10^{-3}$ & $-1.15$ & $1.09\times10^{-3}$ & $7.34\times10^{-8}$ & 44.4 & 0.702 \\
        \hline
        $\vd$ [c] & \DSrefset{} & $7.20\times10^{-1}$ & $-2.04\times10^{-1}$ & $-1.20\times10^{-3}$ & $-4.05\times10^{-2}$ & $3.92\times10^{-4}$ & $5.20\times10^{-7}$ & 1.1 & 0.769 \\ 
\rowcolor{lightgray} & \& \DSheatcool{} & $6.39\times10^{-1}$ & $-1.98\times10^{-1}$ & $-8.98\times10^{-4}$ & $-3.63\times10^{-2}$ & $3.42\times10^{-4}$ & $3.26\times10^{-7}$ & 1.7 & 0.626 \\ 
        & \& \DScool{} & $5.67\times10^{-1}$ & $-3.26\times10^{-1}$ & $-3.58\times10^{-4}$ & $5.35\times10^{-2}$ & $1.27\times10^{-4}$ & $1.25\times10^{-7}$ & 5.1 & 0.324 \\ 
        & \& \DSnone{} & $4.31\times10^{-1}$ & $-2.13\times10^{-1}$ & $-6.80\times10^{-5}$ & $4.61\times10^{-2}$ & $3.06\times10^{-6}$ & $2.04\times10^{-8}$ & 6.8 & 0.162 \\ 
        \hline
        $\yd$ & \DSrefset{} & $-3.13\times10^{-2}$ & $2.84\times10^{-1}$ & $5.89\times10^{-4}$ & $-1.48\times10^{-1}$ & $-2.02\times10^{-4}$ & $-2.78\times10^{-7}$ & 9.1 & 0.824 \\ 
\rowcolor{lightgray} & \& \DSheatcool{} & $2.65\times10^{-1}$ & $2.52\times10^{-2}$ & $2.31\times10^{-4}$ & $-6.28\times10^{-2}$ & $-1.88\times10^{-4}$ & $-1.86\times10^{-8}$ & 9.7 & 0.768 \\ 
        & \& \DScool{} & $-2.53\times10^{-1}$ & $6.26\times10^{-1}$ & $5.02\times10^{-4}$ & $-2.39\times10^{-1}$ & $-3.04\times10^{-4}$ & $-1.25\times10^{-7}$ & 25.0 & 0.345 \\
        \hline
        $\athetarms$ [deg] & \DSrefset{} & $-6.85\times10^{1}$ & $1.29\times10^{2}$ & $1.18\times10^{-1}$ & $-5.31\times10^{+1}$ & $-2.78\times10^{-2}$ & $-6.97\times10^{-5}$ & 4.5 & 0.819 \\ 
\rowcolor{lightgray} & \& \DSheatcool{} & $-4.80\times10^{1}$ & $1.21\times10^{2}$ & $5.92\times10^{-2}$ & $-5.10\times10^{+1}$ & $-2.26\times10^{-2}$ & $-2.52\times10^{-5}$ & 4.2 & 0.804 \\ 
        & \& \DScool{} & $-1.04\times10^{2}$ & $1.77\times10^{2}$ & $1.10\times10^{-1}$ & $-6.04\times10^{+1}$ & $-6.50\times10^{-2}$ & $-2.47\times10^{-5}$ & 8.7 & 0.483 \\
        \hline\hline
    \end{tabular}
}
\end{sidewaystable}
%\end{table*}

%\begin{table*}
\begin{sidewaystable}
    \caption{
        \label{tab:dynfit:fit_form}
        Dynamical ejecta properties:
        coefficients for the fitting formulae discussed in the text for various datasets.}
    \scalebox{0.90}{
    \begin{tabular}{l|l|l|ccccccc}
        \hline\hline
        Quantity&Fit & Datasets & $\alpha$ & $\beta$ & $\gamma$ & $\delta$ & $n$ &  $\chi^2 _{\nu}$ & $R^2$ \\
        \hline
        $\log_{10}(\md)$ & Eq.~\eqref{eq:fit_Mej}  & \DSrefset{} & $1.089\times10^{-1}$ & $4.900\times10^{-1}$ & $6.487$ & $-7.187$ & $3.110\times10^{-1}$ & 2.2 & 0.401 \\ 
        &      & \& \DSheatcool{} & $-1.172\times10^{-1}$ & $-4.157\times10^{-1}$ & $2.434\times10^{-1}$ & $2.363\times10^{-1}$ & $3.175\times10^{-1}$ & 16.9 & 0.079 \\ 
        &      & \& \DScool{} & $-1.448\times10^{-1}$ & $-1.433$ & $2.487$ & $2.827$ & $3.004\times10^{-1}$ & 30.1 & 0.112 \\ 
        &      & \& \DSnone{} & $1.370\times10^{-2}$ & $-6.171\times10^{-1}$ & $2.202$ & $-1.279$ & $5.503\times10^{-1}$ & 84.8 & 0.016 \\ 
        \hline
        $\log_{10}(\md)$ & Eq.~\eqref{eq:fit_Mej_Kruger}  & \DSrefset{} & $-1.914\times10^{-3}$ & $2.204\times10^{-2}$ & & $-6.912\times10^{-2}$ & $1.288$ & 1.6 & 0.527 \\ 
        &      & \& \DSheatcool{} & $-1.051\times10^{-3}$ & $1.160\times10^{-2}$ & & $-3.717\times10^{-2}$ & $1.299$ & 10.6 & 0.158 \\ 
        &      & \& \DScool{} & $-1.212\times10^{-3}$ & $1.351\times10^{-2}$ & & $-4.319\times10^{-2}$ & $1.318$ & 11.9 & 0.241 \\ 
        &      & \& \DSnone{} & $-3.667\times10^{-4}$ & $3.100\times10^{-3}$ & & $-1.068\times10^{-2}$ & $1.628$ & 39.9 & -0.214 \\ 
        \hline
        $\vd$ [c] & Eq.~\eqref{eq:fit_vej}& \DSrefset{} & $-7.591\times10^{-1}$ & $1.333$ & $-1.541$ & & & 1.5 & 0.635 \\ 
        &      & \& \DSheatcool{} & $-5.867\times10^{-01}$ & $1.145$ & $-1.207$ & & & 2.4 & 0.428 \\ 
        &      & \& \DScool{} & $-4.089\times10^{-1}$ & $9.296\times10^{-1}$ & $-7.041\times10^{-1}$ & & & 6.1 & 0.170 \\ 
        &      & \& \DSnone{} & $-3.650\times10^{-1}$ & $8.229\times10^{-1}$ & $-1.130$ & & & 6.8 & 0.157 \\ 
        \hline\hline
    \end{tabular}
    }
\end{sidewaystable}
%\end{table*}

%\begin{table*}
\begin{sidewaystable}
    \caption{
        \label{tab:diskfit:poly}
        Disk mass:
        coefficients for polynomial regression of various
        quantities. Results for both
        first order and second order polynomials are reported $P_2^1(\tilde{\Lambda})$ and $P_2^2(q, \tilde{\Lambda})$
        The recommended calibration for $P_2^2(q,\Lambda)$ is highlighted.
    }
    \scalebox{0.90}{
    \begin{tabular}{l|ccccccccc}
        \hline\hline
        Datasets & $b_0$ & $b_1$ & $b_2$ & $b_3$ & $b_4$ & $b_5$ &  $\chi^2_{\nu}$ & $R^2$  \\
        \hline
        \DSrefset{} & $-2.40\times10^{-2}$ & $5.55\times10^{-4}$ & $-3.94\times10^{-7}$ & & & & 2574.1 & 0.027 \\ 
        \& \DSheatcool{} & $-1.03\times10^{-2}$ & $4.07\times10^{-4}$ & $-2.23\times10^{-7}$ & & & & 1074.1 & 0.092 \\ 
        \& \DScool{} & $-7.46\times10^{-2}$ & $4.99\times10^{-4}$ & $-2.41\times10^{-7}$ & & & & 757.4 & 0.299 \\ 
        \& \DSnone{} & $-6.86\times10^{-2}$ & $4.80\times10^{-4}$ & $-2.12\times10^{-7}$ & & & & 603.9 & 0.408 \\ 
        \hline
        \DSrefset{} & $-1.57$ & $2.07$ & $9.83\times10^{-4}$ & $-6.67\times10^{-1}$ & $-2.55\times10^{-4}$ & $-4.61\times10^{-7}$ & 425.4 & 0.415 \\ 
\rowcolor{lightgray} \& \DSheatcool{} & $-1.51$ & $2.04$ & $7.71\times10^{-4}$ & $-6.45\times10^{-1}$ & $-2.74\times10^{-4}$ & $-2.52\times10^{-7}$ & 174.8 & 0.542 \\ 
        \& \DScool{} & $-1.47$ & $2.02$ & $6.85\times10^{-4}$ & $-6.28\times10^{-1}$ & $-3.17\times10^{-4}$ & $-1.44\times10^{-7}$ & 202.2 & 0.671 \\ 
        \& \DSnone{} & $-8.57\times10^{-1}$ & $1.13$ & $4.22\times10^{-4}$ & $-3.74\times10^{-1}$ & $3.46\times10^{-5}$ & $-2.13\times10^{-7}$ & 197.6 & 0.659 \\ 
        \hline\hline
    \end{tabular}
    }
\end{sidewaystable}
%\end{table*}

\begin{table*}
%\begin{sidewaystable}
    \caption{
        \label{tab:diskfit:form}
        Disk mass: coefficients for the fitting formulae discussed in the text for various datasets. }
    \scalebox{0.80}{
    \begin{tabular}{l|l|ccccccc}
        \hline\hline
        Fit & Datasets & $\alpha$ & $\beta$ & $\gamma$ & $\delta$  & $\chi^2 _{\text{dof}}$ & $R^2$ \\
        \hline
        Eq.~\eqref{eq:fit_Mdisk} & \DSrefset{} & $1.457\times10^{-1}$ & $2.833\times10^{-2}$ & $4.755\times10^{+2}$ & $4.632$ & 1927.3 & 0.103 \\ 
        & \& \DSheatcool{} & $1.349\times10^{-1}$ & $3.322\times10^{-2}$ & $4.578\times10^{+2}$ & $1.945\times10^{-1}$ & 784.7 & 0.173 \\ 
        & \& \DScool{} & $-9.829\times10^{1}$ & $9.845\times10^{1}$ & $-3.158\times10^{+2}$ & $1.790\times10^{+2}$ & 543.2 & 0.342 \\ 
        & \& \DSnone{} & $-3.737\times10^{1}$ & $3.756\times10^{1}$ & $-9.683\times10^{+2}$ & $4.028\times10^{+2}$ & 574.9 & 0.436 \\ 
        \hline
        Eq.~\eqref{eq:fit_Mdisk_Kruger} & \DSrefset{} & $-1.017$ & $1.006$ & $1.307\times10^{1}$ & & 2198.9 & 0.152 \\ 
        & \& \DSheatcool{} & $-1.789$ & $1.045$ & $8.457$ & & 894.8 & 0.233 \\ 
        & \& \DScool{} & $-4.309$ & $8.633\times10^{-1}$ & $1.439$ & & 629.9 & 0.400 \\ 
        & \& \DSnone{} & $-4.247$ & $8.384\times10^{-1}$ & $1.349$ & & 442.8 & 0.506 \\ 
        \hline
        \hline\hline
    \end{tabular}
    }
%\end{sidewaystable}
\end{table*}

  
%% Coefficients
%% --------------------------------



\red{NOT YET APPROVED}
\red{NOT rephrased }
\section{$\chi_\nu^2$ statistics for individual datsets}
\label{app:datasets}

In this appendix we discuss the $\chi_\nu^2$ statistics of all fitting formulae 
dataset-vise instead of adding them up, as was done in the main text.
%% --- 
In Tab.~\ref{tbl:fit:ejecta:chi2dofsallInd} we compare the different fits 
for the dynamical ejecta properties and disk mass in terms of the reduced 
$\chi$-squared, $\chi^2 _{\nu}$.
%% --- ejecta mass
Regarding the ejecta mass we find that the datasets that are 
more uniform in their physics and method, \eg, \DSrefset{} and 
\DScool{} display the lowest $\chi^2 _{\nu}$.
The largest $\chi^2 _{\nu}$ is found for the \DSheatcool{}, that 
incorporates both, models with M1 and leakage+M0 neutrino schemes. 
This further highlights the systematic uncertainties due 
to different neutrino treatment. 
%% --- ejecta velocity 
For the ejecta velocity, the largest $\chi^2 _{\nu}$ is found 
for the \DScool{} across all fitting formulae. 
This might be attributed to the systematic uncertainties that pure 
leakage neutrino scheme introduces for models with different outcomes, \eg,
prompt collapse and stable remnants. 
%% --- ejecta electron fraction and RMS angle
With respect to ejecta electron fraction and RMS half-opening angle, 
the difference in $\chi^2 _{\nu}$ are small. Notably, for the 
$\ayd$, the $\chi^2 _{\nu}$ is similar between the \DSrefset{}
and \DSheatcool{}. This indicates the consistency in neutrino reabsorption 
effects on the ejecta composition in these datasets.
%% --- disk mass
For the disk mass we find the largest $\chi^2 _{\nu}$ for \DSrefset{} 
and \DScool{} across all fitting formulae. 
This can be explained by the presence of prompt collapse simulations 
in both datasets.
However, a larger set of models cover the binary parameter space more 
uniformly is required to investigate this further.
%% --- conclusion
%% Overall we find that for all quantities the $P_2^2{q,\tilde{\Lambda}}$ is 
%% a reasonably good fitting model. 





%% --------------------------------
%% Ejecta 

\begin{table}[t]
    \caption{
        Dataset-vise 
        reduced $\chi$-squared $\chi^2 _{\nu}$ for different
        fitting models for the dynamical ejecta properties
        and disk mass. $N$ is the number of models in the dataset.
        Mean is the simulation
        average, $P_n(x,y)$ is a polynomial of order $n$ in the variables $x,y$. 
        %% The best fitting model is characterized by the lowest value of $\chi_{\nu}^2$.
    }
    \label{tbl:fit:ejecta:chi2dofsallInd}
    \scalebox{0.86}{
        \begin{tabular}{l|l|cccccc}
            \hline\hline
            $\log_{10}(\md)$ & Dataset & $N$ & Mean & Eq.~\eqref{eq:fit_Mej} & Eq.~\eqref{eq:fit_Mej_Kruger} & $P_2^1(\tilde{\Lambda})$ & $P_2^2(q,\tilde{\Lambda})$ \\ \hline
            & \DSrefset{} & 34 & 3.84 & 2.23 & 1.58 & 3.03 & 1.55 \\ 
            %% & \DSrefset{} & 34 & 3.38 & 2.04 & 1.72 & 2.70 & 1.44 \\ 
            &  \DSheatcool{}  & 30 & 70.46 & 15.82 & 121.76 & 232.66 & 156.87 \\ 
            &  \DScool{}  & 42 & 23.37 & 9.89 & 6.35 & 8.55 & 7.63 \\ 
            &  \DSnone{}  & 165 & 118.11 & 31.38 & 24.64 & 76.82 & 26.39 \\ 
            \hline\hline
            $\langle v_{\rm ej}\rangle$ & Dataset & $N$ & Mean & Eq.~\eqref{eq:fit_vej} & & $P_2^1(\tilde{\Lambda})$ & $P_2^2(q,\tilde{\Lambda})$ \\ \hline
            %% & \DSrefset{} & 34 & 3.30 & 1.23 & & 2.92 & 0.92 \\ 
            & \DSrefset{}  & 34 & 3.76 & 1.51 & & 3.24 & 1.05 \\
            &  \DSheatcool{}  & 27 & 3.68 & 3.41 & & 2.93 & 2.63 \\
            &  \DScool{}  & 42 & 10.33 & 10.97 & & 10.13 & 9.99 \\ 
            &  \DSnone{}  & 143 & 6.98 & 5.39 & & 5.98 & 5.34 \\
            \hline\hline
            $\langle Y_{\rm e}\rangle$ & Dataset & $N$ & Mean &  & & $P_2^1(\tilde{\Lambda})$ & $P_2^2(q,\tilde{\Lambda})$ \\ \hline
            %% & \DSrefset{} & 34 & 41.74  & &  & 42.66 & 8.73 \\ 
            & \DSrefset{} & 34 & 42.49 &  & & 43.69 & 9.07 \\ 
            &  \DSheatcool{}  & 30 & 16.81  & &  & 17.83 & 8.73 \\  
            &  \DScool{}  & 35 & 12.74  & &  & 13.64 & 12.69 \\
            \hline\hline
            $\langle \theta_{\rm RMS}\rangle$ & Dataset & $N$ & Mean & & & $P_2^1(\tilde{\Lambda})$ & $P_2^2(q,\tilde{\Lambda})$ \\ \hline
            %% & \DSrefset{} & 34 & 20.41 & & & 21.22 & 4.45 \\ 
            & \DSrefset{} & 31 & 20.68 & & & 21.66 & 4.55 \\ 
            &  \DSheatcool{}  & 7 & 1.96 & & & 3.11 & 7.54 \\ 
            &  \DScool{} & 35 & 9.68 & & & 8.19 & 7.27 \\ 
            \hline\hline
            $M_{\text{disk}}$ & Dataset & $N$ & Mean & Eq.~\eqref{eq:fit_Mdisk} & Eq.~\eqref{eq:fit_Mdisk_Kruger} & $P_2^1(\tilde{\Lambda})$ & $P_2^2(q,\tilde{\Lambda})$ \\ \hline
            %% & \DSrefset{} & 31 & 2896.33 & 1927.20 & 363.14 & 2541.76 & 354.31 \\ 
            & \DSrefset{} & 31 & 2956.22 & 1927.27 & 2198.85 & 2574.14 & 425.41 \\ 
            &  \DSheatcool{} & 23 & 16.74 & 8.53 & 4.43 & 7.79 & 1.51 \\ 
            &  \DScool{}  & 26 & 2920.32 & 42.18 & 3.24 & 296.48 & 260.77 \\
            &  \DSnone{}  & 39 & 1671.69 & 101.97 & 23.62 & 166.21 & 42.33 \\
            \hline\hline
        \end{tabular}
    }%scalebox
\end{table}



  %% tbl:fit:ejecta:chi2dofsallInd
%% Coefficients
%% --------------------------------
