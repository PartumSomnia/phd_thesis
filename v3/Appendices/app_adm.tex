%% ============================
%%
%% Appendix A
%%
%% ============================

\chapter{Derivation of the ADM system of equations}
\label{app:adm}
%% \externaldocument{intro}
%% --------------------------------------

\section{Hamiltonian Field Theory}

First we recall the generalized coordinates $\boldsymbol{q}$ 
and their covariant derivatives $\nabla\boldsymbol{q}$. 
In light of the spacetime decomposition discussed above, 
we divide the $\boldsymbol{\alpha}$ into the time $dt$ and spatial parts 
represented by the antisymmetric symbol ${^{(3)}\boldsymbol{\alpha}}$ as 

\begin{equation}
\boldsymbol{\alpha} = dx^0 \wedge dx^1 \wedge dx^2 \wedge dx^3 = dt \wedge {^{(3)}\boldsymbol{\alpha}}.
\end{equation}

Next, we introduce the "time derivative" as a Lie derivative along the vector field $\vec{t}$ as 

\begin{equation}
\dot{\boldsymbol{q}} := \mathcal{L}_{\vec{t}}\boldsymbol{q}.
\end{equation}

As the $\Lambda(\boldsymbol{q}, \nabla\boldsymbol{q})$ is the 
Lagrangian density, a conjugate momentum can be defined as 

\begin{equation}
\boldsymbol{p} := \frac{\partial\Lambda}{\partial\dot{\boldsymbol{q}}},
\end{equation}

Assuming that $\boldsymbol{p}$ and $\nabla\boldsymbol{q}$ can be expressed as a function of $\boldsymbol{q}$ and $\boldsymbol{p}$, inspired by the Legendre transformation, we define the Hamiltonian and its density as

\begin{align}
\mathcal{H} &= \boldsymbol{p}\cdot\dot{\boldsymbol{q}} - \mathcal{L}(\boldsymbol{q}, \nabla\boldsymbol{q}) \\
H &= \int_{\Sigma}\mathcal{H}{^{(3)}\boldsymbol{\alpha}}
\end{align}

Additionally we define the quantity 

\begin{equation}
J = \int_{0}^{t}H(\boldsymbol{q},\boldsymbol{p})dt = \int_{0}^{t}dt\int_{\Sigma}\mathcal{H}(\boldsymbol{q},\boldsymbol{p}){^{(3)}\boldsymbol{\alpha}} = \int_{0}^{t}dt\int_{\Sigma}{^{(3)}\boldsymbol{\alpha}}\Big(\boldsymbol{p}\cdot\dot{\boldsymbol{q}} - \mathcal{L}(\boldsymbol{q},\nabla\boldsymbol{q})\Big).
\end{equation}

Consider the variation of the $J$ with respect 
to the $\delta\boldsymbol{p}$ and $\delta\boldsymbol{q}$ as

\begin{equation}
\delta J = \int_{0}^{t}\delta H(\boldsymbol{q},\boldsymbol{p})dt = \int_{0}^{t}dt (\dot{\boldsymbol{q}}\delta\boldsymbol{p}+\boldsymbol{p}\delta\dot{\boldsymbol{q}}) - \int_{0}^{t}dt\delta\Lambda(\boldsymbol{q}, \nabla\boldsymbol{q}),
\end{equation}

and consider the last term, the variation of the Lagrangian 

\begin{equation}
\delta\Lambda = \int_{\Sigma}{^{(3)}\boldsymbol{\alpha}}\Bigg[\frac{\delta\Lambda}{\delta\dot{\boldsymbol{q}}}\delta\dot{\boldsymbol{q}}+\frac{\delta\Lambda}{\delta\boldsymbol{q}}\delta\boldsymbol{q}\Bigg].
\end{equation}

The first term in the square brackets can be reduced to $\boldsymbol{p}\delta\dot{\boldsymbol{q}}$, suingthe definition of the conjugate momentum. The second term can be treated, applying the Euler-Lagrange equations Eq.~\eqref{eq:theory:eulerlagrange}. These manipulations result in

\begin{equation}
\delta\Lambda = \int_{0}^{t}dt\int_{\Sigma}{^{(3)}\boldsymbol{\alpha}}(\boldsymbol{p}\delta\dot{\boldsymbol{q}} + \dot{\boldsymbol{p}}\delta\boldsymbol{q}).
\end{equation}

Thus we obtain that 

\begin{equation}
\int_{0}^{t} \delta H(\boldsymbol{q},\boldsymbol{p})dt =   \int_{0}^{t}dt\int_{\Sigma}{^{(3)}\boldsymbol{\alpha}}(\dot{\boldsymbol{q}}\cdot\delta\boldsymbol{p}-\dot{\boldsymbol{p}}\cdot\delta\boldsymbol{q}),
\end{equation}

and as $\delta\boldsymbol{p}$ and $\delta\boldsymbol{p}$ are arbitrary, 
the Hamilton equations read

\begin{equation}
\dot{\boldsymbol{q}}=\frac{\delta H}{\delta\boldsymbol{p}}, \hspace{5mm} \dot{\boldsymbol{p}} = -\frac{\delta H}{\delta\boldsymbol{q}}.
\label{eq:theory:hamiltoneqs}
\end{equation}

The Hamiltonian formalism can be used to re-derive the field-equations in a from 
that once the initial data is specified on a hypersurface $\Sigma_0$ for $\boldsymbol{q}$ 
and $\boldsymbol{p}$, the equations Eq.~\eqref{eq:theory:hamiltoneqs}) 
would govern whole the evolution.


%\section{Three-metric}
%
%There are exist coordinates that are adapted to the $3+1$ foliation, namely $\{t, x^i\}$ with $\vec{\partial}_i\cdot \vec{n} = 0$. In these coordinates the $\nabla t = dt$ and $\vec{t} = \vec{\partial}_t$. 
%
%The connection between $\boldsymbol{g}$ and $\boldsymbol{\gamma}$ is $g_{\mu\nu}=\vec{\partial}_{\mu}\cdot\vec{\partial}_{\nu} $ and can be expressed in terms of $\alpha$ and $\vec{\beta}$ as
%
%\begin{align}
%\text{spatial components: } g_{ik}&=\vec{\partial}_{i}\cdot\vec{\partial}_{j} =\gamma_{ik}, \\
%\text{time component: } g_{tt} &= \vec{\partial}_{t}\cdot\vec{\partial}_{t} = \vec{t}\cdot\vec{t} = - (\alpha^2-\vec{\beta}\cdot\vec{\beta}), \\
%\text{mixed components: } g_{ti} &= \vec{\partial}_{t}\cdot\vec{\partial}_{i} = \vec{t}\cdot\vec{\partial}_i = (\alpha\vec{n}+\vec{\beta})\cdot\vec{\partial}_i=\beta_i,
%\end{align}
%we we made use of $\vec{\beta}$ being the spatial vector, \textit{i.e} $\vec{\beta}\cdot\vec{\beta}=\gamma_{ik}\beta^i\beta^k$.
%
%The line-element can be thus written as
%\begin{equation}
%ds^2 = -(\alpha^2-\beta_i\beta^i)dt^2 +2\beta_i dx^i dt + \gamma_{ik} dx^i dx^k.
%\end{equation}




%\section{Extrinsic Curvature and Constraint equations}
%
%We define the \textit{extrinsic curvature} of a $D-1$-suface $\Sigma_t\subset\mathcal{M}$ at a point $\mathcal{P}\in\Sigma_t$ as mapping $\boldsymbol{K}$ such that $\boldsymbol{K}(\boldsymbol{\upsilon})=-\nabla_{\boldsymbol{\upsilon}}\boldsymbol{n}$. Note, that the $\boldsymbol{K}$ thus does not depend on $\alpha$ and $\vec{\beta}$, it is a purely spatial tensor. The components of the extrinsic curvature are \\
%
%\begin{equation}
%K_{\mu\nu} = -{\gamma^{\alpha}}_{\mu}\nabla_{\boldsymbol{u}}^{\alpha} n_{\nu} = -\frac{1}{2}\mathcal{L}_{\vec{n}}\gamma_{\mu\nu},
%\label{eq:theory:extrcurvdef}
%\end{equation}
%where $\mathcal{L}_{\vec{n}}$ is the Lie derivative along the vector field $\vec{n}$. \\
%From the (\ref{eq:theory:extrcurvdef}) the extrinsic curvature can be interprated as a "speed of the $\vec{n}$ during the parallel transport along the hypersurface $\Sigma_t$".
%
%Codazzi equations relate the $4D$ Ricci tensor to the extrinsic curvature as
%
%\begin{equation}
%D_{\beta}K-D_{\alpha}{K^{\alpha}}_{\beta}=R_{\gamma\delta}n^{\delta}{\gamma^{\gamma}}_{\beta},
%\label{eq:theory:formomentum}
%\end{equation}
%
%here $K$ is a trace of the tensor $\boldsymbol{K}$. \\
%
%Gauss equation realtes the $3D$ Riemann tensor $^3{R_{\alpha\beta\gamma}}^{\delta}$ to the $4D$ one and the $\boldsymbol{K}$ as
%
%\begin{equation}
%^3{R_{\alpha\beta\gamma}}^{\delta} = {\gamma^{\mu}}_{\alpha}{\gamma^{\nu}}_{\beta}{\gamma^{\lambda}}_{\gamma}{\gamma^{\delta}}_{\sigma}{R_{\mu\nu\lambda}}^{\delta}-K_{\alpha\gamma}{K_{\beta}}^{\delta}+K_{\beta\gamma}{K^{\delta}}_{\alpha}.
%\label{eq:theory:forhamiltconst}
%\end{equation}
%
%The \textit{momentum constraint} thus cab be obtained by substituting the (\ref{eq:theory:EFE}) into  (\ref{eq:theory:formomentum}) which yields
%
%\begin{equation}
%D_{\beta}K-D_{\alpha}{K^{\alpha}}_{\beta} = -8\pi{\gamma^{\alpha}}_{\beta} n^{\gamma}T_{\alpha\gamma}=:8\pi j_{\beta},
%\label{eq:theory:momconstraint}
%\end{equation}
%where $j^{\alpha}$ is the ADM momentum density. \\
%
%The \textit{Hamiltonian constrant} can be obtained by substituting EFE (\ref{eq:theory:EFE}) into the (\ref{eq:theory:forhamiltconst}), yielding 
%
%\begin{equation}
%^3 R+ K^2 - K_{\alpha\beta}K^{\alpha\beta} = 2G^{\alpha\beta}n_{\alpha}n_{\beta} = 16\pi n_{\alpha}n_{\beta} T^{\alpha\beta} =: 16\pi E,
%\label{eq:theory:hamilconstraint}
%\end{equation}
%where $E$ is the ADM energy density. 
%
%The obtained constraint equations represent a set of elliptic equations that must be satisfied on every hyprsurface $\Sigma_i$ of the foliation. It is however, possible to show that Eistein equations preserve the constraints, meaning that if they are satisfied at the initial slice $\Sigma_0$ they will be satisfied at any time in the future. 





\section{The Hamiltonian Formulation of the Einstein Equations}

Here we briefly sketch to path of derivation of the \ac{EFE} in the Hamiltonian framework.
We will elude most of the intimidate and computationally extensive steps, 
as well as derivation of the boundary terms. 
For this we refer to \cite{Poisson:2004}.

First it is useful to note that determinant of the three-metric $\sqrt{\gamma}$ 
can be expressed as $\sqrt{\gamma}=\sqrt{-g}/\alpha$. 
The $p$ is the trace of the canonical momentum $\boldsymbol{p}$.

Now, consider the scalar curvature, $R$

\begin{align}
G_{\mu\nu} &= R_{\mu\nu} - \frac{1}{2}Rg_{\mu\nu} \\
-Rg_{\mu\nu}n^{\nu}n^{\mu} &= 2(G_{\mu\nu} n^{\nu}n^{\mu}-R_{\mu\nu}n^{\mu}n^{\mu})\\
-Rn_{\mu}n^{\mu}& = 2(G_{\mu\nu}n^{\nu}n^{\mu} - R_{\mu\nu}n^{\mu}n^{\mu}) \\
R &= 2(G_{\mu\nu}n^{\mu}n^{\nu} - R_{\mu\nu}n^{\mu}n^{\nu}).
\end{align}

From the Gauss-Codacci equation Eq.~\ref{eq:theory:momconstraint}, 
which relates the spatial curvature $^{(3)}R$ to the spacetime 
curvature $R$, we have the following constraint relation

\begin{equation}
2G_{\mu\nu}n^{\mu}n^{\nu} = {^{(3)}R} + K^2 - K_{\mu\nu}K^{\mu\nu}.
\end{equation}

The $R_{\mu\nu}n^{\mu}n^{\nu})$ can be expressed as a combination of 
extrinsic curvature and total divergences as 

From the definition of the Ricci tensor $R_{\mu\nu}$, we have:

\begin{align}
R_{\mu\nu} &= {R_{\mu\gamma\nu}}^{\gamma} \\
R_{\mu\nu}n^{\mu}n^{\nu} &= {R_{\mu\gamma\nu}}^{\gamma} \\
&= -(\nabla_{\mu}\nabla_{\gamma} - \nabla_{\gamma}\nabla_{\mu})n^{\gamma}n^{\nu} \\
&= n^{\mu}(\nabla_{\mu}\nabla_{\gamma} - \nabla_{\gamma}\nabla_{\nu})n^{\gamma} \\
&= (\nabla_{\mu}n^{\mu})(\nabla_{\gamma}n^{\gamma}) - \nabla_{\mu}(n^{\mu}\nabla_{\gamma}n^{\gamma}) - (\nabla_{\gamma}n^{\mu})(\nabla_{\mu}n^{\gamma}) + \nabla_{\gamma}(n^{\mu}\nabla_{\mu}n^{\gamma}) \\
&= K^2 - K_{\mu\gamma}K^{\mu\gamma} - \nabla_{\mu}(n^{\mu}\nabla_{\gamma}n^{\gamma}) + \nabla_{\gamma}(n^{\mu}\nabla_{\mu}n^{\gamma})
\end{align}

In case of variations with compact support, that we are interested in, 
the total divergences, last two terms, can be neglected. Then the result is

\begin{equation}
R_{\mu\nu}n^{\mu}n^{\nu}= K^2 - K_{\mu\nu}K^{\mu\nu}.
\label{eq:theory:rmunu_as_func_k}
\end{equation}

Using the fact that $\sqrt{\gamma}=\sqrt{-g}/\alpha$ and the 
Eq.~\eqref{eq:theory:rmunu_as_func_k}) we obtain the Lagrangian density 
in terms of the variables of the hypersurface:

\begin{align}
\Lambda &= \sqrt{-g}R \\
&= \alpha\sqrt{\gamma}R \\
&= 2\alpha\sqrt{\gamma}(G_{\mu\nu}n^{\mu}n^{\nu} - R_{\mu\nu}n^{\mu}n^{\nu})\\ 
&= 2\alpha\sqrt{\gamma}\Big(\frac{1}{2}[{^{(3)}R} - K_{\mu\nu}K^{\mu\nu} + K^2] - K^2 - K_{\mu\nu}K^{\mu\nu}\Big)
\end{align}

Together with the contribution from matter fields, we obtain

\begin{equation}
\Lambda = \Lambda_g+\Lambda_m= \frac{1}{16\pi}\alpha({^{(3)}R} + K_{\mu\nu}K^{\mu\nu} - K^2)\sqrt{\gamma}+\Lambda_m
\end{equation}

Next we note that the extrinsic curvature of a
surface $\Sigma$ is defined as $K_{\mu\nu} = \nabla_{\mu}n_{\nu}$. 

To relate $K_{\mu\nu}$ to the metric, we make use of the following property of Lie derivatives:

\begin{align}
\mathcal{L}_{\vec{n}}g_{\mu\nu} &= n^{\gamma}\nabla_{\gamma}g_{\mu\nu} + g_{\gamma\nu}\nabla_{\mu}\upsilon^{\gamma} + g_{\mu\gamma}\nabla_{\nu}\upsilon^{\gamma} \\
&= \nabla_{\mu}n_{\nu}+\nabla_{\nu}\upsilon_{\nu} \\
&=2\nabla_{\mu}n_{\nu}
\end{align}

where the second line holds when $\nabla_{\gamma}\mu$ is the
natural derivative operator corresponding to the metric $g_{\mu\nu}$ 
and the third line holds because $K_{\mu\nu}$ is symmetric.

Substituting this into our definition of $K_{\mu\nu}$,

\begin{align}
K_{\mu\nu} &= -\frac{1}{2}\mathcal{L}_{\vec{\vec{n}}}g_{\mu\nu} \\
&= -\frac{1}{2}\mathcal{L}_{\vec{\vec{n}}}(\gamma_{\mu\nu}-n_{\mu}n_{\nu}) \\
&= -\frac{1}{2}\mathcal{L}_{\vec{\vec{n}}}\gamma_{\mu\nu} \\
&= -\frac{1}{2}[n^{\gamma}\nabla_{\gamma}\gamma_{\mu\nu} + \gamma_{\gamma\nu}\nabla_{\mu}\upsilon^{\nu} + h_{\mu\gamma}\nabla_{\nu}\upsilon^{\gamma}] \\
&= -\frac{1}{2\alpha}[\alpha n^{\gamma}\nabla_{\gamma}\gamma_{\mu\nu} + \gamma_{\gamma\nu}\nabla_{\mu}\alpha\upsilon^{\nu} + h_{\mu\gamma}\nabla_{\nu}\alpha\upsilon^{\gamma}] \\
&= -\frac{1}{2\alpha}{\gamma_{\mu}}^{\gamma}{\gamma_{\nu}}^{\delta}[\mathcal{L}_{\vec{t}}\gamma_{\gamma\delta}-\mathcal{L}_{\vec{\beta}}\gamma_{\gamma\delta}] \\
&= -\frac{1}{2\alpha}{\gamma_{\mu}}^{\gamma}{\gamma_{\nu}}^{\delta}[\partial_t\gamma_{\mu\nu}-D_{\mu}\beta_{\nu}-D_{\nu}\beta_{\mu}]
\end{align}

and on the hypersurface $\Sigma$ the projection operators are not needed. So we obtain

\begin{equation}
K_{\mu\nu} = -\frac{1}{2}\mathcal{L}_{\vec{n}}\gamma_{\mu\nu}=-\frac{1}{2\alpha}(\partial_t\gamma_{\mu\nu}-D_{\mu}\beta_{\nu}-D_{\nu}\beta_{\mu})
\end{equation}

which us to express the canonical momentum $p^{\mu\nu}$ as

\begin{align}
p^{\mu\nu} &= \frac{\partial\Lambda}{\partial\dot{\gamma}_{\mu\nu}} \\
&= -\frac{\sqrt{\gamma}}{16\pi}\alpha\Bigg[\frac{\partial {^{(3)}R}}{\partial\dot{\gamma}_{\mu\nu}} + \frac{\partial(K_{\mu\nu}K^{\mu\nu})}{\partial\dot{\gamma}_{\mu\nu}} - \frac{\partial K^2}{\partial\dot{\gamma}_{\mu\nu}}\Bigg] \\
&= \frac{\sqrt{\gamma}}{16\pi}(K\gamma^{\mu\nu} - K^{\mu\nu}),
\end{align}
where 
\begin{equation}
\frac{\partial K_{\mu\nu}}{\partial \dot{\gamma}_{\mu\nu}} = \frac{1}{2\alpha}, \hspace{5mm} \frac{\partial {^{(3)}R}}{\partial \dot{\gamma}_{\mu\nu}} = 0, \hspace{5mm}\frac{\partial K^2}{\partial \dot{\gamma}_{\mu\nu}} = \frac{\gamma^{\mu\nu}K}{\alpha}
\end{equation}

assuming that there is no explicit dependency of the $\Lambda$ on $dot{\gamma}_{\mu\nu}$.

Since, $\alpha$ and $\vec{\beta}$ are related to the the gauge freedom, as there are many ways manifold $\mathcal{M}$ can be split into hypersurfaces, the momenta associated with these function and vector is zero. 

Thus, the Hamiltonian density is

\begin{align}
\mathcal{H} &= p^{\mu\nu}\dot{\gamma}_{\mu\nu} - \Lambda \\
&= -\sqrt{\gamma}\alpha{^{(3)}R} + \frac{\alpha}{\sqrt{\gamma}}\Big[p^{\mu\nu}p_{\mu\nu}-\frac{1}{2}p^2\Big] + 2p^{\mu\nu} D_{\mu}\beta_{\mu} -\Lambda_m \\
%    &=  \frac{\sqrt{\gamma}}{16\pi}\Bigg\{\alpha\Big[-{^{(3)}R}+h^{-1}p^{\mu\nu}p_{\mu\nu}-\frac{1}{2}h^{-1}p^2\Big] - 2\beta_{\nu}\big[D_{\mu}(h^{-1/2}p^{\mu\nu})\big] + D_{\mu}(h^{-1/2}\beta_{\nu}p^{\mu\nu})\Bigg\} \\
&= \frac{\sqrt{\gamma}}{16\pi}\Bigg\{\alpha\Big[ -{^{(3)}R} + \gamma^{-1}p^{\mu\nu}p_{\mu\nu}-\frac{1}{2}\gamma^{-1}p^2\Big] +  2\beta_{\nu}\Big[D_{\mu}(\gamma^{-1/2}p^{\mu\nu})\Big] - 2D_{\mu}(\gamma^{-1/2}\beta_{\nu}p^{\mu\nu}) \Bigg\} - \Lambda_m,
\end{align}

where we restored the correct $16\pi$ factor in the last line.

As the we consider variations with compact suppot, the last boundary term, can be neglected. 

Now we consider the variation of the matter action $S_m$ with respect to the $\alpha$ and $\vec{\beta}$

\begin{align}
\frac{\delta S_m}{\delta \alpha} &=-\alpha\frac{\delta S_m}{\delta g_{00}} = -\alpha\sqrt{-g}T^{00} = -\alpha^2\sqrt{\gamma}T^{00} = -\sqrt{\gamma}T^{\mu\nu}n_{\mu}n_{\nu} \\
\frac{\delta S_m}{\delta \beta_{\mu}} &= \frac{\delta S_m}{\delta g_{\mu 0}} =\frac{1}{2}\sqrt{-g}T^{\mu 0} = -\frac{1}{2} \sqrt{\gamma}T^{\mu\nu}n_{\nu}.
\end{align}

As the variation of the Hamiltonian $H$ with respect to a quantity with 
vanishing canonical momentum is zero, we obtain two equations 

\begin{align}
\frac{\delta H}{\delta \alpha} &= 0 = -{^{(3)}R} + \gamma^{-1}p^{\mu\nu}p_{\mu\nu}-\frac{1}{2}\gamma^{-1}p^2 + 16\pi T^{\mu\nu}n_{\mu}n_{\nu} \\
\frac{\delta H}{\delta \beta_{\mu}} &= 0 = - D_{\mu}(\gamma^{-1/2}p^{\mu\nu}) + 8\pi{\gamma^{\mu}}_{\nu}n_{\gamma}T^{\nu\gamma}.
\label{eq:theory:hamiltonianvariation}
\end{align}


Note, that the $\delta H / \delta\beta_{\mu}$ is actually a Frech\'et differential $dH$, $\delta \beta_{\mu}$, which is writes as

\begin{equation}
\langle dH,\delta\beta \rangle = \delta\beta_{\mu}\big[-D_{\nu}(\gamma^{-1/2}p^{\mu\nu})+8\pi n_{\gamma}T^{\mu\nu}\big], 
\end{equation}

containing $\delta\beta_{\mu}$ which is spatial. Thus only the spatial part is being constrained in the equation above. To account for that the projector ${\gamma^{\mu}}_{\nu}$ is added to the $\delta H/\delta \beta_{\mu}$. 

The pair of equations Eq.~\eqref{eq:theory:hamiltonianvariation} is in fact the constraint 
equations derived before, namely the Eq.~\eqref{eq:theory:momconstraint} and Eq.~\eqref{eq:theory:hamilconstraint}, and as we now see,
they are related to the coordinate freedom of $\mathcal{M}$ decomposition and a 
coordinate freedom on hypersurfaces. 

Proceeding with the Hamiltinan formalism we note that equation Eq.~\eqref{eq:theory:hamiltoneqs} 
leads to the evolution equations for the three-metric, 
assuming that $\Lambda$ explicitly does not depend on the momentum

\begin{equation}
\dot{\gamma}_{\mu\nu} =\frac{\delta H}{\delta p^{\mu\nu}} = 2\gamma^{-1/2}\alpha\big(p_{\mu\nu}-\frac{1}{2}\gamma_{\mu\nu}p\big) - D_{\nu}\beta_{\mu}-D_{\mu}\beta_{\nu}
%    -2D_{(\mu}\beta_{\nu)},
\label{eq:theory:_adm_metric_evo}
\end{equation}

The evolution equations for the canonical momentum can read

\begin{align}
\dot{p}^{\mu\nu} = -\frac{\delta H}{\delta \gamma_{\mu\nu}} = &+ \alpha\gamma^{1/2}\big({^{(3)}R}^{\mu\nu}-\frac{1}{2}{^{(3)}R\gamma^{\mu\nu}}\big) \\
& - \frac{1}{2}\alpha\gamma^{-1/2}\gamma^{\mu\nu}\big(p_{\gamma\delta}p^{\gamma\delta}-\frac{1}{2}p^2\big) \\
& + 2\alpha\gamma^{-1/2}\big(p^{\mu\gamma}{p^{\nu}}_{\gamma}-\frac{1}{2}pp^{\mu\nu}\big) \\
& - \gamma^{1/2}\big(D^{\mu}D^{\nu}\alpha-\gamma^{\mu\nu}D^{\gamma}D_{\gamma}\alpha\big) \\
& - \gamma^{1/2}D_{\gamma}\big(\gamma^{-1/2}\beta^{\gamma}p^{\mu\nu}\big) \\
&+ 2p^{\gamma(\mu}D_{\gamma}\beta^{\nu)} + 8\pi \alpha \gamma^{1/2}S^{\mu\nu},
\label{eq:theory:_adm_mom_evo}
\end{align}

where $A_{(\mu\nu)} = 0.5(A_{\mu\nu}+A_{\nu\mu})$ the convention was used. 

where $S^{\mu\nu}={\gamma^{\mu}}_{\alpha}{\gamma^{\nu}}_{\beta}T^{\alpha\beta}$. 

Taking the variation of the matter field we noted that 

\begin{equation}
\frac{\delta S}{\delta \gamma_{ik}} = \frac{\delta S_m}{\delta g_{ik}} = \frac{1}{2}\sqrt{-g}T^{ik}
\end{equation}

The set of equations 
Eq.~\eqref{eq:theory:hamiltonianvariation}, 
Eq.~\eqref{eq:theory:_adm_metric_evo} and 
Eq.~\eqref{eq:theory:_adm_mom_evo}) comprise the \ac{ADM} system. 
A more widely used from of these equations is in turms of $\gamma_{ij}$ and $K_{ij}$ reads

\begin{align}
(\partial_t - \mathcal{L}_{\vec{\beta}})\gamma_{ik} &= -2\alpha K_{ik}; \\
(\partial_t - \mathcal{L}_{\vec{\beta}})K_{ik} &= -D_{i}D_{k}\alpha + \alpha\big(R_{ik} - 2K_{ij}{K^j}_k+KK_{ik}\big) - 8\pi\alpha\big(S_{ik} - \frac{1}{2}\gamma_{ik}(S-E)\big); \\
{^{(3)}R} + K^2 - K_{ik}K^{ik} &= 16\pi E; \\
D_{i}K-D_{k}{K^k}_i &= 8\pi j_i,
%% \label{eq:theory:adm}
\end{align}

where $S = \gamma^{ij}S_{ij}$.
These equations constitute the \ac{IVP} for \ac{EFE} and are known as \ac{ADM} equations. 
The last two equations are the constraint equations. They determine how to set the initial 
data on the hypersurface $\Sigma_0$, via prescribing the three-metric and extrinsic curvature. 
The first two equations then govern the evolution.

\todo{make sure that the coefficients in formuals are consistent, $16\pi$ might me missing or $-$}
\todo{Makse sure that $\Lambda$ stands for largangian density and $\mathcal{L}$ for lie derivative}

