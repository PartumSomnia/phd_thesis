%% =====================================================================================
%%
%%               C O D E -- W H I S K Y - T H C
%%
%% =====================================================================================
\chapter{Simulation Setup} \label{app:whisky}%\label{sec:bns_sims:setup}

%We perform the \ac{NR} \ac{BNS} merger simulations using the code \ac{WhiskyTHC}

\section{Initial Data}


We compute the irrotational \ac{BNS} configurations in quasi-circular orbit for 
each \ac{EOS} considered, \eg, SLy4, SFHo, LS220, BLh and DD2 \acp{EOS}.
The initial configuration is obtained via the pseudo-spectral code \texttt{Lorene} \citep{Gourgoulhon:2000nn}, 
that solves the general relativistic initial data problem. %\red{in what approximation? Thin sandwitch?}.
The initial separation (of the qusi-circular orbit) is chosen ${\sim}40\,$km and that
 corresponds to $~2-3$ orbits before merger.
%
The \acp{EOS} tables used for the initial data computation are the minimum temperature slices
$(T\sim 0.5 - 0.1)$~MeV of the finite temperature \acp{EOS} tables used for the evolution.
The assumption of the neutrino-less beta-equilibrium is made.
At constant temperature, at lowest densities, the photon energy (radiation) is a dominant 
contribution to pressure, and we substruct that contribution from the tables.
%
%In the evolution code, passing the initial data, the mapping is done from the zero temerature
In the evolution code, the electron fraction is set by the beta equilibrium condition. 
The specific internal energy is reset in accordance with minimum 
temperature slice of the \ac{EOS} table used for evolution.

%Errors present in the initial data in introduced during the mapping 
%result in a small oscillations of \acp{NS}.
%In terms of relative changes in central density these amounts to 
%$\sim2-3\%$ \citep{Radice:2018pdn}





\section{Evolution}

%In order to perform numerical simulations of fluid flow, accurate numerical codes are essencial. 
%Codes that flux-conservative finite-difference HRSC schemes offer a ciertain degree of simplicity, 
%while high-order finite volume schemes are more computationally expensive (as they require 
%solution of multiple Riemann problems at the interface between regions) \cite{Reisswig:2009us,Shu:2001rep} 
%as well as complex averaging and de-averaging procedures \cite{Tchekhovskoy:2007zn}.

The evolution of \ac{BNS} systems is performed with \wisky{} \ac{NR} code
 \citep{Radice:2013apa,Radice:2012cu,Radice:2013xpa,Radice:2013hxh,
    Radice:2015nva,Radice:2016dwd,Radice:2018pdn,Radice:2020ids}.
\wisky{} is a result of combination of the \texttt{Whisky} code \citep{Baiotti:2004wn} and \texttt{THC} code.
The \texttt{THC} part stands for the Templated-Hydrodynamics Code developed using the \texttt{Cactus} framework 
\citep{Goodale:2003}. 
In \texttt{THC}, the state-of-the-art flux-vector splitting scheme are employed. 
%The "templated" in the code name stands for a modern paradigm in C++ programming, the templated 
%programming, where a part of the code can be generated from the prescribed templates at 
%compiling time (\textit{e.g.,} \cite{Yang:2001})
%
%The \texttt{THC} has several primitive variable reconstruction schemes implemented, 
%such as MP5, classical monotonicity preserving \cite{Suresh:1997,Mignone:2010} the weighted 
%essentially non oscillatory (WENO) schemes WENO5 and WENO7 \cite{Liu:1994,Jiang:1996,Shu:1997} 
%and two bandwidth-optimized WENO schemes WENO3B and WENO4B \cite{Martin:2006,Taylor:2007}, 
%contracted for modeling the compressible turbulence. Note, that the number 
%in scheme name stands for a formal order of accuracy.
%
%High-order flux-vector splitting finite-differencing techniques came from 
%the former, while the module for the recovery of the primitive quantities as 
%well as the equation of state framework from the latter \cite{Galeazzi:2013mia}. 
%Tabulated temperature and composition dependent equation of states can be used.
%
\wisky{} solves the equations of \ac{GRHD} in conservation form using a \ac{FD} scheme. 

%% Spacetime evolution
The spacetime is evolved using the Z4c formulation of Einstein's equations
(see Sec.~\ref{sec:theory:z4c}) as implemented in the \texttt{CTGamma} code
\citep{Pollney:2009yz,Reisswig:2013sqa}
which is a part of the \texttt{Einstein Toolkit} \citep{Loffler:2011ay}.
%
%\cite{Bernuzzi:2009ex,Hilditch:2012fp} as implemented in the \texttt{CTGamma} code
%\cite{Pollney:2009yz,Reisswig:2013sqa} which is part of the \texttt{Einstein Toolkit} 
%\cite{Loffler:2011ay}.
%
The non-linear stability of evolution is assured via Kreiss-Oliger dissipation. 
The spacial discritisation is done via fourth-order finite-differencing implemented in \texttt{CTGamma}.
%
%\red{tripple check the FD is used for space time evol and central FV method is for hydro}
The flux reconstruction is done in local-characteristic variables using the \ac{MP5} scheme.
% see \textit{e.g.,} \cite{Rezzolla:2013} \red{CHECK}.
%The space-time is evolved using the Z4c formulation (see Sec.~\ref{sec:theory:nr}), 
%solved via \ac{FD} code publicly available through \texttt{Einstein Toolkit}, 
%\citep{McLachlan,Loffler:2011ay}.%\red{CHECK}.

%There, the central stencil is used throughout, and only terms associated with the advection along the 
%shift vector are treated using the upwinded by one grid point stencil. 
%The accuracy of the scheme is available at 6th and 8th order, while $4$th is commonly employed. 
%In addition, the fifth order Kreiss-Oliger style artificial dissipation \cite{Kreiss:1973} is 
%added to aid with non-linear stability. 
%

%\paragraph{Hydrodynamics}
The code solves continuity equations Eq.~\eqref{eq:theory:contineq} separately for 
protons and neutrinos, $n_n$ and $n_p$, with the \ac{RHS} prescribed by the 
lepton number deposition rate due to the absorption and emission of neutrinos 
and antineutrinos, $R_p^\mu$ and $R_n^\mu$, (with $R_p = -R_n$) 
from leakage plus M0 scheme (See~\ref{sec:nr_methods:neut}).
%The code evolves the proton and neutron number densities, $n_n$ and $n_p$
%respectively, as 
%
%\begin{equation}
%\label{eq:wthc:pndens}
%\nabla_\nu (n_p u^\mu) = R_p^\mu \ \ , \ \ 
%\nabla_\nu (n_n u^\mu) = R_n^\mu \ .
%\end{equation}
%
%\gray{in Radice2016dwd it is $\nabla_{\alpha}(n_e u^{\alpha}) = R$}
%
%\gray{in Galezzi2013, for Whisky, the equations are separate for baryon and leptons: 
%$\nabla_{\alpha}(n_bu^{\alpha})=0$ and $\nabla_{\alpha}(n_eu^{\alpha})=N$, 
%where the $n_b$ and $n_e$ are the baryon and electron number densities respectively.}
%
%Here $u^{\mu}$ is the fluid four-velocity, $R_p = -R_n$ is the net
%lepton number deposition rate due to the absorption and emission of neutrinos 
%and antineutrinos (see Sec.~\ref{sec:theory:neut})
%
%The $R_{p,n}$ is computed according to the neutrino M0 scheme (see Sec.~\ref{sec:theory:neut})
%
The number densities are related as $n_p=Y_e n$ where $n = n_p + n_e$ is the baryon 
number density and $Y_e$ is electron fraction.
%
The matter of a neutron star is approximated with ideal fluid 
with stress-energy tensor Eq.~\eqref{eq:theory:tmunu_perf},
%
%\begin{equation}
%T_{\mu\nu} = \rho h u_{\mu} u_{\nu} + Pg_{\mu\nu}
%\end{equation}

%where $\rho=m_{\text{b}} n$ is the baryon rest-mass density, 
%$n$ the baryon number density, $m_{\text{b}} \simeq 10^{-24}\,$g 
%the neutron mass, 
%\gray{if Galezzi13 it is nucleon mass which is actrually related to the EOS.}
%$h=1+\epsilon + P/\rho$ the specific enthalpy, 
%$\epsilon$ the specific internal energy (energy density),
%and $P$ is \gray{total isotropic} pressure.

%Written in a covariant form, the Euler equation for balance of energy and momentum reads

The code solves the Euler equations (Eq.~\eqref{eq:theory:tmunu_eq_0}) 
with the \ac{RHS} given being $ Q u^{\mu}$
%
%\begin{equation}
%\label{eq:wthc:euler}
%\nabla_\nu T^{\mu\nu} = Q u^{\mu} \ ,
%\end{equation}
%
%\gray{in Radice2016dwd it is $\nabla_{\beta}T^{\alpha\beta}=\Psi^{\alpha}$
%    with $\Psi^{\alpha} = Q u^{\alpha}$.
%}
%\gray{In the Galezzi:2013 it is $\nabla_{\alpha}T^{\alpha\beta}=\Psi^{\beta}$.
%    There the $T^{\alpha\beta}$ accounts for the ordinary matter and for trappend 
%neutrinos and photons, but it does not include free-streaming neutrinos. Assumed to be 
%similar to the 'test-fluid' they are neglected in constracting RHS of the Eistein equations.
%}
%
where $Q$ is the net energy deposition rate due to absorption
and emission of neutrinos also treated with the M0 scheme  
(see Sec.~\ref{sec:nr_methods:neut}).
%\red{JUST PUT REFS TO THE EQs THAT ARE IN THE 'nuetrino' AND 'viscosity' sections}




%\subsection{Numerical methods}
The continuity and Euler equations are discretized via \ac{HRSC} schemes
(see Sec.~\ref{sec:theory:num_meth}).
%
%\eqref{eq:wthc:euler} and \eqref{eq:wthc:pndens}.
Specifically, the code employs central \ac{KT} scheme \citep{Kurganov:2000} with 
\ac{HLLE} flux formula \citep{Einfeldt:1988} and non-oscillatory reconstruction 
of the primitive variables with the \ac{MP5} scheme of \citep{Suresh:1997}.
%
Shock capturing schemes require the presence of a low density atmosphere around neutron stars.
We assume the constant value of $\rho_0 = m_p n \approx 6\times 10^4$~\gcm.
%
The rest-mass consirvation in the presence of artificial atmosphere is assured via 
\ac{PPL} approach.
%
The local number densities of neutrons and protons separately, are assured via 
multi-fluid advection method of \citet{Plewa:1998nma}
%
The outflow properties are extracted when the density exceeds the atmosphere density
by several orders of magnitude.
%
The coupling between the spacetime and hydrodynamic evolution is achieved via 
the \ac{MOL}.
%
Time integrator of choice is the \ac{SSP} third-order \ac{RK} scheme \citep{Gottlieb:2009}.
The timestep is regulated by the \ac{CFL} condition, that required \ac{CFL} factor 
to be $<0.25$ for numerical stability. 
To assure that the \ac{PPL} implemented in \texttt{WhiskyTHC} maintains 
the density positive, the \ac{CFL} factor is set to $0.15$.


%\subsection{AMR}
The code is build on the \texttt{Carpet} \ac{AMR} driver \citep{Schnetter:2003rb} from the 
\texttt{Cactus} computational toolkit \citep{Goodale:2003}, that incorporates
\texttt{Carpet} Berger-Oliger-style mesh refinement \citep{Berger:1989,Berger:1984} 
with sub-cycling in time and refluxing. %\textcolor{red}{in Thesis it is said, --
% no refluxing was done yet} 
%The code uses the Berger-Oliger conservative adaptive mesh renement (AMR) \cite{Berger:1984} 
%with sub-cycling in time and \red{refluxing (Davids thesis does not have refluxing)} 
%\cite{Berger:1989,Reisswig:2012nc} as provided by the \texttt{Carpet module} of the \texttt{Einstein Toolkit} 
%\cite{Schnetter:2003rb}. 


%\subsection{Neutrino scheme}
%
%
%\begin{table}
%    \caption{
%        Weak reactions employed in our simulations and references for their implementation.
%        In the left column, $\nu \in \{\nu_e, \bar{\nu}_e, \nu_{x}\}$ denotes any neutrino species, 
%        $\nu_{x}$ any heavy-lepton neutrinos, $N \in\{n, p\}$ a nucleon, and $A$ any nucleus.
%        In the central column the role of each reaction is highlighted, with "P" standing for 
%        production, "A" for absorption opacity and "S" for scattering opacity. When two roles are
%        indicated, the second refers to the inverse ($\leftarrow$) reaction.
%        Table is taken from \cite{Radice:2018pdn}.
%    }
%    \label{tab:leakage}
%    \begin{center}
%        \begin{tabular}{lll}
%            \hline\hline
%            Reaction & Role &  Ref. \\ 
%            \hline
%            $p + e^- \leftrightarrow \nu_e + n $          & P,A & \cite{Bruenn:1985}  \\
%            $n + e^+ \leftrightarrow \bar{\nu}_{e} + p $  & P,A & \cite{Bruenn:1985}  \\
%            $e^+ + e^- \rightarrow \nu + \bar{\nu}$       & P & \cite{Ruffert:1995fs} \\
%            $\gamma + \gamma \rightarrow \nu + \bar{\nu}$ & P & \cite{Ruffert:1995fs} \\
%            $N + N \rightarrow \nu + \bar{\nu} + N  + N$  & P & \cite{Burrows:2004vq} \\
%            $\nu + N \rightarrow \nu + N$                 & S & \cite{Ruffert:1995fs} \\
%            $\nu + A \rightarrow \nu + A$                 & S & \cite{Shapiro:1983du} \\
%            \hline\hline
%        \end{tabular}
%    \end{center}
%\end{table}
%
%\red{JUST Name the chemes and reference the equations}
%\red{JUST PUT REFS TO THE EQs THAT ARE IN THE 'nuetrino' AND 'viscosity' sections}
%
%\subsection{Code General Setup}

\section{Geometry and resolution}

The simulation domain is a cube of $3.024$~km each side, 
whose center is at the center of mass of the binary.
The \ac{AMR} structure has $7$ refinement levels, with the finest 
covering both compact objects during the inspiral and the remnant \pmerg{}.
%
We consider several resolution setups: Low (\texttt{LR}), 
standard (\texttt{SR}) and high (\texttt{HR}) with the finest 
refinement level being 
$h=246$~m, $h=185$~m and $h=123$~m respectively.
%
In simulations where the neutirno M0 scheme is included,
it is switched on shortly before the merger. 
%The equations \eqref{eq:method:whisky:eq7} and \eqref{eq:method:whisky:eq9} 
The Eqs.~\eqref{eq:theory:neut:balanseq1} and \eqref{eq:theory:neut:balanseq2}
are then solved on the uniform spherical grid with radius ${\approx}756$~km, 
and resolution $n_r\times n_{\theta}\times n_{\phi} = 3096 \times 32 \times 64$
grid points.
%
A subset of models discussed in this thesis include the effective treatment of viscosity
(see Sec.~\ref{sec:nr_methds:visc}). 

%We consider $33$ distinct binary with total masses 
%$\red{[None,None]}$ and mass-ratio $q\in[1.00,1.82]$.
%In all models the neutrino leakage plus M0 scheme. 
%Most models were computed at at least two resolutions. 
%and include the effect of subgrid turbulence, viscosity.


%\gray{Summary of all results in given in the table...}
%
%\gray{Each run is nameed as}
%
%\gray{We simulate each model for at least $\red{None}$~ms after 
%the merger or a few milliseconds after BH formation}