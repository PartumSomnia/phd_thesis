%% ============================
%%
%% Appendix A
%%
%% ============================

\chapter{Derivation of Euler equations}
\label{app:eul}
%% \externaldocument{intro}
%% --------------------------------------


\section{The General-Relativistic Boltzmann Equation}

In special relativity the Boltzmann equation was expressed by Synge \cite{Synge:1957}. Later Chernikov \cite{Chernikov:1962} and Tauber and Weinberg \cite{Tauber:1961} proposed its extension to the general relativity. \\
The list of applications of the Boltzmann equation was limited to the relativistic gas at first \cite{Israel:1963}. Later the list was supplemented by transient relativistic thermodynamics \cite{Israel:1979wp}, radiative transfer \cite{Lindquist:1966}, core-collapse supernovae \cite{Bruenn:1985} and others (see \textit{e.g.}, \cite{Cercignani:2002} and references therein). \\

Different formulations of the general relativistic Boltzmann equation exists in the literature. Lindquist \cite{Lindquist:1966} and Ehlers \cite{Ehlers:1971} proposed a geometrical interpretation. Later, a formulation based on Riemannian structure of tangent bundles was proposed by Sasaki \cite{Sasaki:1958,Sasaki:1962}. In addition, Debbasch and van Leuuwen \cite{Debbasch:2009a,Debbasch:2009b} recently provided a detailed derivation, albeit strongly focused on the algebraic aspects while eluding simple geometrical interpretation. \\
Here we recall the detailed derivation of the general relativistic Boltzmann equation, using modern differential geometry notation by Radice. 

\textcolor{red}{This.Is.Tough. Pure math. Copied from David + his sources.}

\subsection{The geometry of the tangent bundle}

Let the $\mathcal{M}$ be $4$ dimensional differential manifold such that $(\mathcal{M},\: g_{\alpha\beta})$ form the $C^2$ spacetime. The set of tangent vectors of $\mathcal{M}$ constitutes \textit{tangent bundle} of $\mathcal{M}$, the we denote as $T\mathcal{M}$. The set of all unit vectors of $\mathcal{M}$ constitute the \textit{subbundle} of $T\mathcal{M}$. \\
\textcolor{gray}{incompressible vector field}\\
\textit{Every Killing vector field of $\mathcal{M}$ is in incompressible vector field}

\subsubsection{Extended transformation and extended tensors}

Let the $T\mathcal{M}$ be the set of all the tangent vectors of $\mathcal{M}$. The $T\mathcal{M}$ has a natural topology, bundle structure with $\mathcal{M}$ and the base - linear vector space $E^i$. We call $T\mathcal{M}$ the \textit{tangent bundle} of $\mathcal{M}$. Natural projection, or a projection map $\pi:\: T\mathcal{M}\rightarrow\mathcal{M}$.  \\

Let $U$ be a coordinate neighborhood, or a coordinate patch of $\mathcal{M}$ with $n$ variables $x^{\alpha}$ as coordinates. Then, every tangent vector of $\mathcal{M}$ at a point $p\in U$ with $2n$ variables $(x^i,\upsilon^{\alpha})$. Here $x^{\alpha}$ are coordinates of $p$ with respect to the coordinate patch ${x^{\alpha}}$ and $\upsilon^{\alpha}$ are components of a tangent vector in the natural frame that constitutes by the vectors $\partial/\partial x^4$ at $q$. Thus, the vector $\vec{p}$ at $q$ can be written as:

\begin{equation}
\vec{p} = p^{\alpha}\frac{\partial}{\partial^{\alpha}}
\end{equation}

and its dual as 

\begin{equation}
\underline{p} = p_{\alpha}dx^{\alpha}:=g_{\alpha\beta}p^{\beta}dx^{\alpha}
\end{equation}

In addition we introduce a coordiante patch $TU$, $\{z^A\}$, where $A$ runs from $0$ to $7$ of $T\mathcal{M}$ as 

\begin{equation}
z^{\alpha} = z^{\alpha}, \hspace{10mm} z^{\alpha+4} = p^{\alpha}.
\end{equation}

Now, let the $U(x^{\alpha})$ and $\hat{U}(\hat{x}^{\alpha})$ be the two coordinate patches of $\mathcal{M}$ such that $U\cap\hat{U}$ is not empty. Then the intersection of the coordinate patches is also not empty. 
for every coordinate transformation of $\mathcal{M}$, there is a corresponding matrix $\frac{\partial \hat{x}^{\alpha}}{\partial x^{\beta}}$.
The coordinate transformation is then

\begin{equation}
\hat{x}^{\mu} = \hat{x}^{\mu}(x), \hspace{5mm} \hat{p}^{\mu} = \frac{\partial\hat{x}^{\mu}}{\partial x^{\nu}}p^{\nu}
\end{equation}

which denotes the extended transforation of the $\hat{x}^{\mu} = \hat{x}^{\mu}(x)$. \\


The corresponding Jacobian matrix is 
\renewcommand\arraystretch{1.6} %% it stretches the matrix
\begin{equation}
\frac{\partial\hat{z}^A}{\partial z^B} = 
\begin{pmatrix}
\frac{\partial\hat{x}^{\alpha}}{\partial x^{\beta}} & 0 \\
\frac{\partial^2\hat{x}^{\alpha}}{\partial x^{\beta} \partial x^{\gamma}}p^{\gamma} & \frac{\partial\hat{x}^{\alpha}}{\partial x^{\beta}} 
\end{pmatrix}
\end{equation}
\renewcommand\arraystretch{1.0}




\subsubsection{Vectors on $T\mathcal{M}$}

As we will need to introduce connections on a tangent bundle, here we discuss the double tangent bundle, ot a second tangent bundle. Since $T\mathcal{M}$ is a vector bundle on its own right, its tangent bundle has the secondary vector bundle structure $TT\mathcal{M}$. Let point $b\in TU$ and $T_b T\mathcal{M}$ be the tangent space to $T\mathcal{M}$ at $b$. \\
Given a vector $\partial/\partial x^{\alpha}$ at a point $b$, it can be "pushed forward" to the point on the $TT\mathcal{M}$ by means of so called \textit{differential of} $\pi$, whitten as $\pi_*$ \cite{Frankel:2002}.
On a natural basis the push-forward acts as 

\begin{equation}
\pi_*\Big[\frac{\partial}{\partial x^{\alpha}}\Big] = \frac{\partial}{\partial x^{\alpha}}, \hspace{5mm} \pi_* \Big[\frac{\partial}{\partial p^{\alpha}}\Big] = 0,
\end{equation}

and the pull back 

\begin{equation}
\pi^* {\text d} x^{\alpha} = {\text d} x^{\alpha}.
\end{equation}

Consider a vector field $\vec{X} \ in TT\mathcal{M}$  in a vicinity of the point $b$, which is associated with the point $q$ of $\mathcal{M}$ and vector $\vec{x}\in T_{q}\mathcal{M}$. Let $b{\lambda}$ be the flow of $b$ generated by $\vec{X}$. The $b(\lambda)$ is associate with $q(\lambda)$, the one parameter family of points of $\mathcal{M}$. The $b(\lambda)$ is also associated with $\vec{x}(\lambda)$ the one parameter family of vectors on $T\mathcal{M}$.  \\

The vector field $\vec{X}$ is called \textit{vertical} if the $q(\lambda)\in\mathcal{M}$ are constant along the flow. Similarly, the vector field $\vec{X}$ is called \textit{horizontal} if $\vec{x}(\lambda)\in T_p \mathcal{M}$ is "constant" along the flow, meaning that $\vec{x}(\lambda)$ is just $\vec{x}$ that is \textit{parallel transported} to $q(\lambda)$. \\
As there is no unique way to perform a parallel transport, the linear connection $\nabla$ on $\mathcal{M}$ has to be chosen. This choice is akin choosing two vector spaces $\mathcal{O}_b$ and $\mathcal{V}_b$ of the horizontal and vectical vectors respectively at each point $b$ that the direct sum of these spaces yields

\begin{equation}
\mathcal{O}_b\oplus \mathcal{V}_p = T_b T\mathcal{M}.
\end{equation}

Having the connection allows to prescribe a manner of lifting curves from the base manifold $T\mathcal{M}$ into the $T_b T\mathcal{M}$ \textcolor{red}{I need to fix this and understand}. A lift is the unique horizontal vector $\vec{X}\in T_bT\mathcal{M}$ whose projection is a vector $\vec{x}\in T_q\mathcal{M}$.\\ 
\textcolor{red}{fill it}
Let us now define a \textit{connection vector basis} adopted to the aforementioned split of $T_b T\mathcal{M}$ $\{\text{D}/\partial x^A \}:=\{\text{D}/\partial x^{\alpha}, \partial/\partial p^{\alpha} \}$ where 

\textcolor{red}{I did not find where this is derived from... difficult}

\begin{equation}
\frac{\text{D}}{\partial x^{\alpha}}{\partial x^{\alpha}} := \frac{\partial}{\partial x^{\alpha}} - {\Gamma^{\beta}}_{\alpha\gamma}p^{\gamma}\frac{\partial}{\partial p^{\beta}}.
\end{equation}

Similarly a connection can be build for differential forms. Using the pull-back $\pi^*$ the dual basis $\{ \text{D}z^{A} \}:=\{\text{d}x^{\alpha}, \text{D}p^{\alpha}\}$ that satisfies 

\begin{equation}
\text{D} = \text{d} p ^{\alpha} + {Gamma^{\alpha}}_{\beta\gamma}p^{\gamma}\text{d}x^{\beta}.
\end{equation}

\subsubsection{Metric on $T\mathcal{M}$}

Note that 

\begin{equation}
\frac{\partial^2 \hat{x}^{\mu}}{\partial x^{\nu}\partial x^{\lambda}}p^{\lambda} = {\hat{\Gamma}^{\mu}}_{\delta\gamma}p^{\lambda}\frac{\partial\hat{x}^{\delta}}{\partial x^{\nu}}.
\end{equation}

Let us assume that for any point $b\in T\mathcal{M}$ there exist an open set $TU$, such that $b\in TU$ with a coordinate system on $TU$ that satisfies

\begin{equation}
G_{AB} = (\boldsymbol{\eta}\otimes\boldsymbol{\eta})_{AB},
\end{equation}

where $\boldsymbol{\eta} = \text{diag}(-1, 1, 1, 1)$. \\
Let the $\hat{x}^A$ denote the generic coordinate system on $TU$. Then the metric in this coordinate system can be expressed as

\begin{align}
\hat{G}_{\mu\nu} &= \frac{\partial \hat{x}^{\alpha}}{\partial x^{\mu}}\frac{\partial \hat{x}^{\beta}}{\partial x^{\nu}}\eta_{\alpha\beta} + \frac{\partial \hat{x}^{\alpha}}{\partial x^{\mu}}{\hat{\Gamma}^{\gamma}}_{\:\:\:\alpha\lambda}p^{\lambda}\frac{\partial \hat{x}^{\beta}}{\partial x^{\nu}}{\hat{\Gamma}^{\delta}}_{\:\:\:\beta\xi}p^{\xi}\eta_{\gamma\delta}; \\
\hat{G}_{\mu\: \nu+4} &= \frac{\partial \hat{x}^{\alpha}}{\partial x^{\mu}}\frac{\partial \hat{x}^{\gamma}}{\partial x^{\nu}}{\hat{\Gamma}^{\beta}}_{\:\:\:\gamma\lambda}p^{\lambda}\eta_{\alpha\beta}; \\
\hat{G}_{\mu+4 \: \nu+4} &= \frac{\partial \hat{x}^{\alpha}}{\partial x^{\mu}}\frac{\partial \hat{x}^{\beta}}{\partial x^{\nu}} \eta_{\alpha\beta}
\end{align}

and the line element 

\begin{align}
dS^2 &= \hat{G}_{AB}d\hat{z}^A d\hat{z}^B = \hat{g}_{\mu\nu}\text{d}\hat{x}^{\mu}\text{d}\hat{x}^{\nu} + \hat{g}_{\mu\nu}[\text{d}p^{\mu} + {\hat{\Gamma}^{\mu}}_{\:\:\:\alpha\beta}p^{\beta}\text{d}x^{\alpha}] [\text{d}p^{\nu} + {\hat{\Gamma}^{\nu}}_{\:\:\:\alpha\beta}p^{\beta}\text{d}x^{\alpha}] \\
&= \hat{g}_{\mu\nu}\text{d}\hat{x}^{\mu}\text{d}\hat{x}^{\nu} + \hat{g}_{\mu\nu}\text{D}\hat{x}^{\mu}\text{D}\hat{x}^{\nu}
\end{align}

It is possible to show that the determinant $|\text{det}\boldsymbol{G}| = g^{2}$ as the transformation from the natural frame to the connection frame is unimodular \cite{Lindquist:1966}. Thus the volume pseudo-form on $T\mathcal{M}$ is in the coordiante patch $TU$

\begin{align}
\text{Vol}^8 &:= -g \text{d}x^{0} \wedge \text{d}x^{1} \wedge ... \wedge \text{d}p^{3} := - g\text{d}^{4}x \text{d}^{4}p, \\
&:= -g \text{d}x^{0} \wedge \text{d}x^{0} \wedge ... \wedge \text{D}p^{3} :=-g \text{d}^{4}x\text{D}^4 p
\end{align}

\textcolor{red}{I kinda gave up here and just copied.}


\section{The Liuville theorem}

Let us start by introducing a \textit{cotangent bundle}. Let $\mathcal{M}$ be a differentiable manifold. Similarly to the construction of the tangent bundle, we can make a set of covectors on a given manifold into a vector bundle over $\mathcal{M}$, denoted $T^*\mathcal{M}$ and called \textit{cotangent bundle} of $\mathcal{M}$.  Similarly we can define a contangent bundle of a tangent one $T^*T\mathcal{M}$. The contangent bundle $T^*\mathcal{M}$ is the vector bundle dual to the tangent bundle $T\mathcal{M}$. 

Let us start by defining \textit{Poincar\'e} 1-form on $T\mathcal{M}$, $\underline{\lambda}\in T^* T\mathcal{M}$. Consider point $q$ on a manifold $\mathcal{M}$ and a point $A$ associated with $q$ on a tangent bundle $T\mathcal{M}$. Let there be a 1-form $\underline{\alpha}\in T^* _q\mathcal{M}$. The $\underline{\lambda}$ and $\underline{\alpha}$ are uniquely connected $\underline{\lambda} = \pi^* \alpha$, and the former is called the \textit{Poincar\'e} 1-form. In local coordinate patch, $TU$ it is expressed as

\begin{equation}
\underline{\lambda} = p_{\alpha} \text{d}x^{\alpha}.
\end{equation}

the associated vector is 

\begin{equation}
\vec{\lambda} = p^{\alpha} \frac{\text{D}}{\partial x^{\alpha}} = p^{\alpha}\frac{\partial}{\partial x^{\alpha}} - p^{\alpha}{\Gamma^{\beta}}_{\alpha\gamma}p^{\gamma}\frac{\partial}{\partial p^{\beta}},
\end{equation}

is called the $\textit{geodesic flow field}$. \\
This flow represents a phase-space flow of particles moving along geodesics. \\

Consider a mass shell, that at a point $q\in U$can be defined as a set:\\
\textcolor{red}{remider: I have no idea how is this possible...}

\begin{equation}
\mathcal{S}_m = \big\{ p^{\alpha}\in T_q\mathcal{M}: p_{\mu}p^{\mu}+m^2 =:f(p) = 0 \big\}.
\end{equation}

The normal to the mass-shell is 

\begin{align}
\text{if } m &\neq 0 \hspace{5mm} \underline{\pi}:=\frac{q}{2m}\text{d}f, \hspace{5mm} \text{d}f = \frac{\partial f}{\partial x^{\mu}} + \frac{\partial f}{\partial p^{\mu}}\text{d}p^{\mu} = 2p_{\mu}\text{d}p^{\mu}, \\
\text{if } m &= 0 \hspace{5mm} \underline{\pi}:=\frac{1}{2}\text{d}f
\end{align}

Next, we introduce a unique form $\underline{\nu}$ whose restriction on $T_q\mathcal{M}$ is equal to $\underline{\pi}$. 

\begin{align}
\text{if } m &\neq 0 \hspace{5mm} \underline{\nu} = \frac{1}{m}p_{\alpha}\text{D}p^{\alpha} \\
\text{if } m &= 0 \hspace{5mm} \underline{\nu} = p_{\alpha}\text{D}p^{\alpha}
\end{align} 

Note that $\underline{\nu} = 0$, meaning that the $\underline{\nu}$ is irrotational. It becomes clear if we re-express it as 

\begin{equation}
\underline{\nu} = \frac{1}{2m}\frac{\text{D}f}{\partial p^{\alpha}}\text{D}p^{\alpha}
\end{equation} 

for massive particle case. For the mass-less the procedure is analogous. \\

It can be shown that $\underline{\lambda}$ is ncompressible \textit{i.e.,} $\text{d}^{\star}\underline{\lambda} =\star \text{d}\star\underline{\lambda} =0$. \\

In addition, both $\underline{\lambda}$ and $\underline{\nu}$ are harmonic forms as 

\begin{equation}
\nabla\underline{\nu} = 0 , \hspace{5mm}
\nabla\underline{\lambda} = [\text{dd}^{\star} + \text{d}^{\star}\text{d}]\underline{\lambda} = 0.
\end{equation}

Let us now consider the density of states in the phase space, of particles moving along tgeodeiscs with velocities on the mass shell. \\ 
In the previous section we introduced a flux of the vector field $\vec{X}$ across $\Sigma$ in \ref{eq:theory:flux_of_flow}, we define the following six-form

\begin{align}
\boldsymbol{\omega} &= \star\big(\underline{\nu}\wedge\underline{\lambda}\big) = i_{\vec{\lambda}} i_{\vec{\nu}}\text{Vol}^8 \\
&= i_{\vec{\lambda}}\Big[i_{\vec{\nu}}\big(\text{Vol}^{4}_{x}\wedge\text{Vol}^{4}_{p}\big)\Big] = i_{\vec{\lambda}} \big[\text{Vol}^{4}_{x}\wedge\text{Vol}^{3}_{p}\big],
\end{align}

where we used the definition of $\text{Vol}^8$. \\
The introduced four forms read,

\begin{align}
\text{Vol}_x ^4 &:= \sqrt{-g} \text{d}x^{0} \wedge \text{d}x^{1} \wedge \text{d}x^{2} \wedge \text{d}x^{3}, \\
\text{Vol}_p ^4 &:= \sqrt{-g} \text{D}p^{0} \wedge \text{D}p^{1} \wedge \text{D}p^{2} \wedge \text{D}p^{3}, \\
\text{Vol}_p ^3 &:= i_{\vec{\nu}}\text{Vol}_p ^4,
\end{align}

where the four-forms are on the $TU$ and the latter three-form is on the mass shell $S_m$. \\

Consider coordinates adopted to the mass-shell, where $\underline{\nu} = (p_0/m)\text{D}p^0$ and $\underline{\nu} = p_0\text{D}p^0$ in the massive nad massless cases respectively, the three-form becomes

\begin{equation}
\text{Vol}^3 _p =\frac{\sqrt{-g}}{-p_0}\text{D}p^1\wedge\text{D}p^2\wedge\text{D}p^3
\end{equation}

Now we have a three-form $\text{Vol}^3 _p$ and a four-form $\text{Vol}_x ^4$. In the context of the ADM foliation, we split spacetime manifold as $\mathcal{M}=\mathcal{R}\times\Sigma$, with $x^0 = \text{const}$ being constant hypersurfaces with normal $\underline{n} = - \alpha\text{d}x^0$ and $\alpha$ -- the lapse function. We can now simplify the $\boldsymbol{\omega}$, splitting $\text{Vol}_x ^3$ as 

\begin{align}
\text{Vol}_x ^4 &= -\underline{n}\wedge\text{Vol}_x ^3 \hspace{5mm} \text{where,} \\
\text{Vol}_x ^3 &= i_{\vec{n}}\text{Vol}_x ^4 = \sqrt{\gamma}\text{d}x^1\wedge\text{d}x^2\wedge\text{d}x^3
\end{align}

and the $\boldsymbol{\gamma}$ is the three-metric induced on the slices.  \\
The resulted coordinates, adapted to the mass shell and the spacetime foliation read

\begin{align}
\boldsymbol{\omega} &=-(\vec{p}\cdot\vec{n})\frac{1}{-p_0}\sqrt{\gamma}\sqrt{-g}\text{d}x^1\wedge\text{d}x^2\wedge\text{d}x^3 \wedge\text{D}p^2\wedge\text{D}p^2\wedge\text{D}p^3 \\
&= \frac{p^0}{-p_0}|g|\text{d}x^1 \wedge\text{d}x^2\wedge\text{d}x^3 \wedge\text{D}p^2\wedge\text{D}p^2\wedge\text{D}p^3
\end{align}

Now, consider a six-vector, $\delta_i x \delta_i p$ with $i\in\{1,2,3\}$. The $\delta_i x$ are tangent vectors to the slice $\Sigma$ and the $\delta_i p$ are tangent to mass shell $S_m$. \\
The action of $\boldsymbol{\omega}$ on the six-vectors $\delta_1 x$, $\delta_2 x$, $\delta_3 x$, $\delta_1 p$, $\delta_2 p$, $\delta_3 p$ yields

\begin{align}
\boldsymbol{\omega}(\delta_1 x,...,\delta_3 p) =& \frac{p^0}{-p_0}|g|\big[\text{d}x^1\wedge\text{d}x^2\wedge\text{d}x^3\big](\delta_{1}x,\delta_{2}x,\delta_{3}x)\times \\
& \hspace{10mm} \Big[\text{D}p^1\wedge\text{D}p^2\wedge\text{D}p^3\Big](\delta_1 p, \delta_2 p, \delta_3 p) \\
& \hspace{2mm} -\frac{p^0}{-p_0}|g|\big[\text{d}x^1\wedge\text{d}x^2\wedge\text{d}x^3\big](\delta_{1}p,\delta_{2}p,\delta_{3}p)\times \\
& \hspace{10mm} \Big[\text{D}p^1\wedge\text{D}p^2\wedge\text{D}p^3\Big](\delta_1 x, \delta_2 x, \delta_3 x) = \\
=& \frac{p^0}{-p_0}|g|\big[\text{d}x^1\wedge\text{d}x^2\wedge\text{d}x^3\big](\delta_{1}x,\delta_{2}x,\delta_{3}x)\times \\
& \hspace{10mm} \Big[\text{D}p^1\wedge\text{D}p^2\wedge\text{D}p^3\Big](\delta_1 p, \delta_2 p, \delta_3 p) = \\
=& \frac{p^0}{-p_0}|g|\big[\text{d}x^1\wedge\text{d}x^2\wedge\text{d}x^3\big](\delta_{1}x,\delta_{2}x,\delta_{3}x)\times \\
& \hspace{10mm} \Big[\text{d}p^1\wedge\text{d}p^2\wedge\text{d}p^3\Big](\delta_1 p, \delta_2 p, \delta_3 p), \\
\end{align}

where we used that $\text{d}x^i(\delta_j p)=0$ and the relation

\begin{equation}
\text{D}p^{i}(\delta_j p) = \text{d}p^{i}(\delta_j p) - {\Gamma^i}_{\alpha\beta}p^{\alpha}\text{d}x^{\beta}(\delta_j p) = \text{d}p^i(\delta_j p)
\end{equation}

Thus, on the space-like hypersurface $\Sigma$ and on the mass shell we have

\begin{equation}
\boldsymbol{\omega} = \frac{p^0}{-p_0}|g|\text{d}x^1\wedge\text{d}x^2\wedge\text{d}x^3\wedge\text{d}p^1\wedge\text{d}p^2\wedge\text{d}p^3 =: \boldsymbol{\Omega}
\end{equation}

The $\boldsymbol{Omega}$ can be split as 

\begin{align}
\boldsymbol{\Omega} &= \boldsymbol{\Lambda} \wedge \boldsymbol{\Pi}, \hspace{5mm} \text{where} \\
\boldsymbol{\Lambda} &= p^0 \sqrt{-g}\text{d}x^1\wedge\text{d}x^2\wedge\text{d}x^3 \\
\boldsymbol{\Pi} &=  \frac{1}{-p_0}\text{d}p^1\wedge\text{d}p^2\wedge\text{d}p^3
\end{align}

The defined forms $\boldsymbol{\Lambda}$ and $\boldsymbol{\Pi}$ can be written in a coordinate-independent way at any point $q\in\mathcal{M}$ as 

\begin{equation}
\boldsymbol{\Lambda} = \star_{\mathcal{M}}\underline{\lambda}, \hspace{5mm} \boldsymbol{\Pi} = \star_{T_q\mathcal{M}}\underline{\pi}
\end{equation}

and this are intrinsic forms in $T\mathcal{M}$. In addition, the $\boldsymbol{\Lambda}$ and $\boldsymbol{\Pi}$ are the proper geodesics flux
volume form on $\Sigma\in\mathcal{M}$ and mass shell $S_m\in T_q\mathcal{M}$ at a point $q\in U$ respectively. \\

Let us now consider the geodesic flow $\vec{\lambda}$. It generates a "tube" in a phase space, that we limit with $S_1$ and $S_2$ sections. Then the flux of points in phase space associated with geodesic flow is $\int_{S}\boldsymbol{\omega}$. It is possible to show that the flux satisfies

\begin{equation}
\int_{S_1}\boldsymbol{\omega} = \int_{S_2}\boldsymbol{\omega}
\label{eq:theory:liuville}
\end{equation}

which is the \textit{Liouville’s Theorem} in the relativistic case. \\

To see taht this is indeed the case, consider the exterior differential of $\boldsymbol{\omega}$

\begin{equation}
\star\text{d}\omega = \text{d}^{\star}(\underline{\nu}\wedge\underline{\lambda}) = d^{\star}\underline{\nu}\wedge\underline{\lambda} + \underline{\nu}\wedge\text{d}^{\star}\underline{\lambda}.
\end{equation}

The $\text{d}^{\star}=\text{const}=k$ as $\text{dd}^{\star}\underline{\nu}=0$. In addition, the $\text{d}^{\star}\underline{\lambda}=0$. This allow us to write 

\begin{equation}
\text{d}\omega = -k(\star\lambda).
\end{equation}

We note the $\star\underline{\lambda}$ is the volume form of the hypersurfaces orthogonal to $\vec{\lambda}$. Hence, the $\star\underline{\lambda}[\vec{\lambda},...]=0$ along the "tube" in phase space. 

\begin{equation}
\int_S\text{d}\boldsymbol{\omega} = 0.
\end{equation}

the \ref{eq:theory:liuville} is recovered, if we use the Stoke’s Theorem, and using the fact that the $\boldsymbol{\omega}$ vanishes along the part of the boundary tangent to $\vec{\lambda}$.

\textcolor{red}{Note that I still have no Idea what I have written. I need to go through the original materal, which I could not find... at least I could not find what I could read and understand. }


\section{The Boltzmann equation}


Let us introduce the phase space version of the mass flux, the 6-form representing the number of phase-space trajectories crossing $S$ of the phase tube between $S_1$ and $S_2$ cross sections. \\
In the absence of collisions we obtain 

\begin{equation}
\int_{S_1}\boldsymbol{\mu} = \int_{S_2}\boldsymbol{\mu}.
\end{equation}

Remembering that $\boldsymbol{\omega}$ represents the density of states in phase space of particles moving along geodesics, we obtain 

\begin{equation}
\boldsymbol{\mu} = F\boldsymbol{\omega},
\end{equation}

where $F$ is \textit{invariant distribution function}, \textcolor{gray}{i.e. F is the Radon-Nikodym derivative of $\boldsymbol{\mu}$ with re $\boldsymbol{\omega}$}. \\

Consider now that collisions change the number of phase trajectories as 

\begin{equation}
\delta N = \int_{S_2} \boldsymbol{\mu} - \int_{S_1}\boldsymbol{\mu} = \int_S \text{d}\boldsymbol{\mu} = \int_S \text{d}F\wedge\boldsymbol{\omega}.
\end{equation}

where 

\begin{equation}
\text{d}F\wedge\boldsymbol{\omega} = \text{d}F\wedge\star (\underline{\nu}\wedge\underline{\lambda}) = \langle\text{d}F,\underline{\lambda}\rangle\star\underline{\nu} - \langle\text{d}F,\underline{\nu}\rangle\star\underline{\lambda},
\end{equation}

where $\langle\cdot,\cdot\rangle$ is a scalar product between forms and can be written as

\begin{equation}
\langle\boldsymbol{\alpha},\boldsymbol{\beta}\rangle\text{Vol}^8 := \boldsymbol{\alpha}\wedge\star\boldsymbol{\beta},
\end{equation}

and with the $\star\underline{\lambda}=0$ on $S$ we obtain 

\begin{equation}
\delta N = \int_S\langle\text{d}F\underline{\lambda}\rangle\star\underline{\nu} = \int_S\mathcal{C}[F]\star\underline{\nu},
\end{equation}

where we denoted the effect of collisions as

\begin{equation}
\langle\text{d}F\underline{\lambda}\rangle = \mathcal{C}[F].
\end{equation}

This is the Boltzmann equation. In component from it reads as 

\begin{equation}
p^{\alpha}\frac{\partial F}{\partial x^{\alpha}} - {\Game^{\gamma}}_{\alpha\beta}p^{\alpha}p^{\alpha}\frac{\partial F}{\partial p^{\gamma}} =\mathcal{C}[F].
\end{equation}

In the coordinate system adapted to the equation, when $P^0 = p^0(p^i)$, the equation reads \cite{Cercignani:2002}:

\begin{equation}
p^{\alpha}\frac{\partial F}{\partial x^{\alpha}} - {\Game^{i}}_{\alpha\beta}p^{\alpha}p^{\alpha}\frac{\partial F}{\partial p^{i}} =\mathcal{C}[F].
\end{equation}

Remembering that that the $\underline{\lambda}$ is incompressible, we can write

\begin{equation}
\langle\text{d}F,\underline{\lambda}\rangle = \text{d}^{\star}[F,\underline{\lambda}],
\end{equation}

This allows to obtain a conservative formulation of the Boltzmann equation, that indicates the conservation of the number of particles \cite{Cardall:2002bp}

\begin{equation}
\text{d}^{\star}[F\underline{\lambda}] = \mathcal{C}[F].
\end{equation}

Next, we re-introduce the Levi Civita connection $\nabla$ in phase space. For incompressible $\underline{\lambda}$ it gives

\begin{equation}
\nabla_A\lambda^A=0,
\end{equation}

while the Boltzmann equation becomes 

\begin{equation}
\lambda^A\partial_A F=\mathcal{C}[F]
\end{equation}

and its conservative form 

\begin{equation}
\nabla_A[Fp^{A}] = \mathcal{C}[F],
\label{eq:theory:liouvilletheorem}
\end{equation}

or in component form

\begin{equation}
\frac{1}{|g|}\frac{\partial}{\partial x^{\mu}}\Bigg[|g|Fp^{\mu}\Bigg] + \frac{p_0}{|g|}\frac{\partial}{\partial p^{k}}\Bigg[\frac{|g|}{-p_0}{\Gamma^k}_{\alpha\beta}p^{\alpha}p^{\beta}F\Bigg] = \mathcal{C}[F].
\end{equation}


\section{From the Boltzmann Equation to the Euler Equation}


In order to derive from the Boltzmann equation the equations of hydrodynamics, we need to first define the needed varaibles of the kinetic description, such as mass and energy fluxes. We not that the density flux can easly be obtained from $\boldsymbol{\mu}$. We write the mass flow then as 

\begin{equation}
\boldsymbol{\rho} = \int_{S_m} \boldsymbol{\mu}.
\end{equation}

Recalling the definition of the rest-mass denisty four-vector $J$, we note that

\begin{equation}
\int_{\Sigma}(-J^{\mu}n_{\mu})\text{Vol}_x ^3 = \int_{\Sigma\times S_{m}} Fp^0\boldsymbol{\Pi}\text{Vol}_x ^3 = \int_{\Sigma\times S_{m}} F\boldsymbol{\Omega} = \int_{\Sigma}\boldsymbol{\rho}
\end{equation}

Thus, the $J$ can be written as 
\textcolor{red}{this is not clear how it was obtain. Must go through again.}

\begin{equation}
J^{\mu} = \int_{S_m}Fp^{\mu}\boldsymbol{\Pi}
\end{equation}

The second moment of the distribution function $F$ in a similar way gives the stress energy tensor 

\begin{equation}
T^{\mu\nu} = \int_{S_m} F p^{\mu}p^{\nu}\boldsymbol{\Pi}
\end{equation}

the components of which in the frame comoving with the fluid are

\begin{equation}
T^{\mu\nu} = 
\begin{pmatrix}
E & \vec{F} \\
\vec{F} & \boldsymbol{P} \\
\end{pmatrix}
\end{equation}

where $E$ and $\vec{F}$ are the energy density and flux respectively, and $\boldsymbol{P}$ is the stress tensor. 

The nature of the collisional operation ultimately defines the equilibrium configuration distribution function $F$ \cite{Cercignani:2002}, and thus the form of the stress-energy tenor. 

Now we use the Liouville Theorem to obtain the equations of hydrodynamics. 
To accomplish that we insert the $\boldsymbol{\Psi}$, a tensorial funcition of $p^i$, into the theorem, eqiation \ref{eq:theory:liouvilletheorem} on both sides and integrate with respect to $\boldsymbol{\Pi}$ as

\begin{equation}
\int_{\text{I\!R}}\nabla_{A}[F\lambda^A\boldsymbol{\Psi}]\boldsymbol{\Pi}=\int_{\text{I\!R}}\mathbb{C}[F]\boldsymbol{\Psi}\boldsymbol{\Pi}
\end{equation}

where we used that $\lambda^A\nabla_{A}\boldsymbol{\Psi}=0$ as $\vec{\lambda}$ is the geodesic flow. \\

Letting the $F$ decay for large momenta we obtain the transfer equation \cite{Israel:1963,Cercignani:2002}:

\begin{equation}
\nabla_{\mu}\int_{\text{I\!R}} F\boldsymbol{\Psi}p^{\mu}\boldsymbol{\Pi} =\int_{\text{I\!R}} \mathcal{C}[F]\boldsymbol{\Psi}\boldsymbol{\Pi},
\label{eq:theory:transferequation}
\end{equation}

\textcolor{red}{check the sources. White intermediate steps}

In a particular case of a simple gas $\Psi$ can be shown to be one of the $\{1,p^0,p^1,p^2,p^3\}$ \cite{Cercignani:2002}. Then the right hand side of the transfer equation becomes 

\begin{equation}
\int_{\text{I\!R}} \mathcal{C}[F]\boldsymbol{\Psi}\boldsymbol{\Pi} = 0.
\end{equation}

These $\boldsymbol{\Psi}$ are related to the quantities conserved by the collisional operator and are called \textit{collisional invariants}. Choice of $1$ would yield the mass conservation, while $p^{\mu}$ -- the energy and momentum conservation. \\

Now, having cancelled the R.H.S of the eq. \ref{eq:theory:transferequation}, we obtain the conservation laws in a following form

\begin{equation}
\nabla_{\mu}J^{\mu} =0 \hspace{10mm}\nabla_{\nu}T^{\mu\nu} =0
\end{equation}