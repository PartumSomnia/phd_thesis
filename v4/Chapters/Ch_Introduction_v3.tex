\chapter{Introduction} \label{ch:intro}

%% =====================================================================================
%%
%%              G E N E R A L  I N T R O D U C T I O N
%%
%% =====================================================================================

%% From Just
%A pair of massive stars at the end of their evolution, undergo \ac{SN} explosion, forming, 
%in certain cases, a pair of compact objects orbiting each other. A particular interesting 
%example is a pair of \acp{NS}, compact, but heavy objects sustained against gravitational 
%collapse by the neutron degeneracy pressure. The theory of \ac{GR} predicts that the orbit 
%of the system shrinks, as \acp{NS} loose energy and angular momentum to \acp{GW}. The 
%loss continues until \acp{NS} collide at their last orbit a form an axisymmetric object.
%In this thesis we investigate such a merger, focusing on the aftermath evolution of the 
%remnant. 
%
%The high compactness of \acp{NS} lead to an energetic, explosive merger, where a certain
%fraction of the \ac{NS} matter is ejected from the system at mildly relativistic 
%velocities. In addition to the complex dynamics of the system after merger that might 
%induce additional matter outflows, this makes the \ac{BNS} mergers a strong contributor 
%to the cosmic chemical evolution. The matter ejected at/after mergers, \ie, ejecta, has 
%unique properties, rarely found in other astrophysical cites. Specifically, the abundance 
%of free neutrons allow for the so-called rapid neutron capture process,
%the \rproc{}, that is responsible for the production of the heaviest elements in the 
%Universe, lanthanides and actinides. 
%
%Wide range of possible types and properties of ejecta lead to a similarly broad 
%range in \ac{EM} counterparts to \ac{BNS} mergers. Perhaps, two of the most 
%well studied ones are the \ac{kN}, a thermal counterpart powered by the decay of 
%newly synthesized heavy elements in the ejecta, and \acp{SGRB}, generally non-thermal 
%emission from the ultrarelativistic collimated outflow, formed after the merger. 
%Study of these \ac{EM} counterparts in conjuncture with \acp{GW} emission allows to 
%gain unique insigts into the inner workings of the \ac{BNS} merger and previously 
%unobtainable constraints on the theory of gravity, the properties of matter at 
%supranuclear densities, origin of the \ac{SGRB}, cosmic chemical evolution. 
%
%The complexity, non-linearly, non-stationarity and multidimensionality of physical 
%processes operating at \ac{BNS} mergers on a broad range of scales of length and time 
%implies that self-consistent, quantitative studies are only possible with numerical 
%simulations. These simulations, performed with numerical codes that took years of 
%develop and test, are very computationally expensive, rare and require detailed 
%postprocessing and analysis. Moreover, the self-consistent modeling of the merger and 
%\ac{EM} counterparts is still beyond the reach of modern methods. Generally, the 
%the short-term (hundred of milliseconds) evolution of the merger itself 
%is handled with \ac{NR} codes while the \nuc{} and \ac{EM} emission are evaluated 
%after, in postprocessing. Strengthening the connection between these methods is one of 
%the goals of this thesis. 
%
%In the following sections we sketch the astrophysical background to clarify the 
%context of our study and we conclude the chapter by summarizing the main points of 
%motivation for this thesis and its structural arrangement.


%% <<< From Radice Review >>>
Mergers of \acp{BNS} are at the center of a variety of physical processes in astrophysics.
The first ever detection of such event by \ac{LIGO}/Virgo and 
numerous \ac{EM} observatories, \GW{}, interpreted as merger of two 
\acp{NS}, have significantly advanced our understanding 
of gravity, physics of dense matter, \acp{SGRB} and origins of \rproc{} elements 
\citep{TheLIGOScientific:2017qsa,Abbott:2018wiz,GBM:2017lvd}. 

Emitted during the inspiral, \acp{GW} delivered a plethora of information about  
\ac{NS} \ac{EOS} at supernuclear densities 
\citep{Hinderer:2009ca,Damour:2012yf,DelPozzo:2013ala}. 
%
The properties of matter at densities several times that of the 
nuclear matter, however, were not well constrained, as the \pmerg{} \ac{GW} 
was not detected. Future observations of the high frequency \pmerg{} signal 
will shed more light on these properties 
\citep{Sekiguchi:2011mc,Radice:2017lry,Most:2018eaw,Bauswein:2018bma}, 
constraining one of the main quantities, the tidal deformability.

The conditions within matter ejected at mergers, \ie, ejecta, are such that the  rapid neutron capture (\rproc{}) 
\nuc{}, responsible for the production of the heaviest elements in the Universe
% such as gold 
can take place \citep{Cowan:2019pkx}.
Whether \acp{BNS} mergers are the prime source of this material is still unknown.
% This was confirmed by \ac{MM} observations of \GW{} \cite{12}. 
% however, it is unclear whether \ac{BNS} mergers is the dominant source of 
% \rproc{} elements in the universe, or if other \rproc{} cites are required to 
% explain the observed abundances in the oldest stars, \ac{UFG} and our solar system. 

Single \acp{NS} are very compact but massive objects, 
%where compactness $C_{i} = GM_i/R_i^2c^2\propto0.15$ and 
for description of which the effects of \ac{GR} cannot be neglected.
A pair of \acp{NS} orbiting each other slowly loses its 
angular momentum to \acp{GW}. The timescale for the radiation reaction, however, 
is much longer than the orbital period for most of the inspiral and the 
system evolution can be considered adiabatic. 
% For instance, the inspiral can be considered as a sequence of circular orbits. 
% However, during last orbits before merger, the finite size (tides) and \ac{HD}
% effects starts to become important. 
The inspiral ends at the onset of the Roche lobe overflow, when the binary 
reaches the mass-shedding limit \citep{Bejger:2004zx}.

In order to study the dynamical phase of \ac{BNS} mergers and \pmerg{} evolution, 
sophisticated \ac{NR} simulations are required. Modern, state-of-the-art methods 
include full \ac{GR}; composition-dependent nuclear \ac{EOS} with finite-temperature 
effects, \ac{GRMHD} and advanced neutrino transport with varying degree of approximation,
\citep{Sekiguchi:2011zd,Wanajo:2014wha,Foucart:2015gaa,Palenzuela:2015dqa,Sekiguchi:2016bjd,Kiuchi:2017zzg,Radice:2017zta,Fujibayashi:2017puw}.

%In this thesis we perform \ac{NR} simulations of \ac{BNS} mergers, report on their 
%qualitative and quantitative picture and its implication for the \ac{EM} signatures.
%We focus on the nuclear astrophsyics aspect of the mergers, and on the comparison 
%between theoretical predictions and observations of \GW{}, discussing the 
%\rproc{} \nuc{}, thermal \ac{EM} transient, and non-thermal \ac{EM} afterglow. 

In this chapter we provide a brief summary of the current understanding of the 
\ac{BNS} mergers, their impact on the galactic chemical evolution, and their 
\ac{EM} counterparts. For the sake of brevity we allude most of the technical details, 
and we refer the interested reader to the following reviews and references therein.
%
For the general overview on the topic we refer to \citet{Shibata:2016},
For a more recent reviews done by different leading groups we refer to 
\citet{Radice:2020ddv,Bernuzzi:2020tgt,Shibata:2019wef}.
%
For the discussion on \ac{EM} counterparts to mergers we refer to 
%For the most recent reviews on the topic we recommend the following reviews:
%Radice, for the discussing of the \pmerg{} dynamics and ejecta,
%Bernuzzis, for the \ac{GW} aspect of the mergers \cite{Bernuzzi:2020tgt}
%Shibata \& Hotokezaka for the discussing of the ejecta \cite{Shibata:2019wef}.
\citet{Kumar:2014upa,Fernandez:2015use,Metzger:2019zeh}.
%
We conclude the chapter stating the aims and structure of this thesis.

%This chapter is organized as follows...
%First we discuss the observational context of the \ac{BNS} mergers, focusing on the 
%ejecta \nuc{} and \ac{EM} counterparts.
%Then we overview the current picture of \ac{BNS} mergers with emphasis on the 
%\pmerg{} dynamics and mass ejection mechanisms. 
%Finally, we state the goals of this thesis.




%% =====================================================================================
%%
%%              T H E O R E T I C A L  P I C 
%%
%% =====================================================================================

%\section{Theoretical picture of \ac{BNS} mergers}

%In This section we briefly overview the current understanding of the \ac{BNS} 
%merger, the dynamics of the inspiral, effects of tides, \pmerg{} hydrodynamics 
%and open questions.

%% --------------------------------------------------------------------------
%%               I N S P I R A L
%% --------------------------------------------------------------------------

\section{Inspiral}

If \acp{NS} are formed through a classical stellar evolution channel of a massive binary, 
their orbit is mostly circular with little to non eccentricity \citep{Aasi:2013wya}. 
%
The binary system looses energy emitting \acp{GW}, and the orbit of \acp{NS} 
shrinks. Stars inspiral increasingly fast, and the last ${\sim}10^3$ orbits can be completed in 
a matter of minutes. As stars approach each other, the \ac{GW} signal rises in 
frequency and amplitude (so-called chirp), reaching the peak at the \textit{moment of merger}.

%% ---------------------------
\subsection{Two Body Dynamics}

The dynamics of two stars, that are sufficiently separated to have relatively small 
angular velocity (\ie, quasi-adiabatic inspiral of point-masses) 
can be described via the \ac{PN} approximation to \ac{GR} 
(the expansion in $\upsilon/c$, with $\upsilon/c\ll 1$).
%In other words, this is the stage of evolution when the orbital timescale is 
%significantly larger than the orbital one, \ie, $\dot{\Omega}/\omega^2\ll 1$.
%
The amplitude, $h(t)$, and the phase, $\phi(t)$, evolution of the \ac{GW} signal, at the leading order, 
is given by the quadruple formula \citep{Radice:2020ddv} 
%
\begin{equation}
\label{eq:intro:gw_wave}
h(t) \sim \frac{1}{d}\mathcal{M}_c^{5/3} f_{GW}^{2/3} = \nu\frac{M}{d}(Mf_{GW}(t))^{2/3}, \hspace{5mm} 
\phi(t) \sim 2\mathcal{M}_c^{-5/8}t^{5/8} = 2\nu^{-3/8}(t/M)^{5/8},
\end{equation}
%
where $A$ and $B$ denote two \acp{NS}, $\mathcal{M}_c = M\nu^{3/5}$ is the chirp mass, $M = M_A + M_B$ is the total binary mass, 
$\nu = M_A M_B/M^2$ is the symmetric \mr{}, $f_{GW} = \dot{\phi}$, and $d$ is the distance of the source.
%Here, the $G=c=1$ are the units, and geometric coefficients are neglected for brevity.
%
In reality, however, \acp{NS} are not point-masses. 
The finite size (tidal) effects modify the 
inspiral and, consequently, emitted \acp{GW}. 
%In the \ac{PN} formalism these effects enter and $5$th \ac{PN} order.
%
%The \ac{PN} expansion, being the asymptomatic expansion, looses its accuracy at high 
%frequencies $f_{GW}\gtrsim 50\,$Hz. 

Another approximation to the two-body dynamics in \ac{GR} is the \ac{EOB} formalism,
which is a Hamiltonian formalism, applicable to all stages of the binary evolution.
%It is has an advantage of being applicable both at low and high frequencies and can be 
%applied throughout the inspiral, merger na \pmerg{} \cite{28}.
It is spiritually similar to the approach, commonly used in Newtonian dynamics to 
describe the motion of two bodies via the motion of a single body with an effective mass, 
$\mu=\nu M$, in the effective potential. Approximating the dynamics in \ac{GR} requires 
employing effective particle, $\mu$, and effective metric. 
%
\citet{Damour:2009wj} extended the \ac{EOB} formalism to \ac{BNS} mergers by 
introducing the finite size effects. For the review we refer to \citet{Damour:2012mv}.


%% ------------------------------------
\subsection{Effects of tides}

A method to treat the finite-size effects in self-gravitating objects in \ac{PN} 
dynamics was proposed by \citet{Damour:1983a} and is based on ``skeletonizing'' 
objects into worldlines with global properties. This constitute the \textit{outer problem}, that was matched to the \textit{inner problem}, that considers how the worldtube 
around one body is influenced by another. In the case of \ac{BNS}, the latter is 
referred to the tidal effects induced by the external gravitational field of a companion. 
%Matching the outer problem with the inner translates into the inclusion of the tidal 
%deformation effects into the orbital dynamics (and consequently, \ac{GW} emission). 
The fully relativistic formulation of the inner problem was derived in 
\citet{Hinderer:2007mb,Damour:2009vw,Binnington:2009bb}. 
%The key parameters of the formulation are the 
%external tidal moments,
%introduced as symmetric-trace-free projections of the derivatives of the externally 
%generated parts of the local gravitoelectric, $\bar{E}_{\alpha}$ and gravitomagnetic, 
%$\bar{B}_{\alpha}$, fields that read 
%$G_L^A = \partial_{\langle L-1}\bar{E}^A_{\alpha_l\rangle}|_{X^{\alpha}}\rightarrow 0$, 
%(where $X^{\alpha}$ are local coordinates \cite{32}).
%In the local frame of the first body, the \red{internally generated mass} $M_L ^A$ and spin 
%$S_L ^A$ multipole moments, (here $L=i_1,i_2,...,i_l$, is the multi-index) depend on the external
%tidal moments via 
%\textit{tidal polarizability coefficients} $\mu_l$ and $\sigma_l$, expressed as 
%
%\begin{equation}
%    M_{L}^{A} = \mu_l G_{L}^A, \hspace{5mm} S_{L}^A = \sigma_l H_L^A,
%\end{equation}
%
%where $M_L ^A$ and $S_L ^A$ are the internally generated mass and spin miltipole moments.
%(here $L=i_1,i_2,...,i_l$, is the multi-index), and 
%$G_{L}^A$ and $ H_L^A$ are the external tidal moments, related to gravitoelectric and 
%gravitomagnetic fields respectively. 
The key components of the formulation are the external gravitoelectric 
(gravitomagnetic) fields, $l$-th order of which induces the mass (spin) multipolar moments 
of the same order, $l$, in \acp{NS}. These moments are characterized by %$G\mu_l$ ($G\sigma_l$), 
tidal polarizability coefficients, that can be written as dimensionless relativistic 
Love numbers \citep{Damour:2009vw,Binnington:2009bb}.
%
%The $l$-th order (external) gravitoelectric (gravitomagnetic) fields induce $l$-th order 
%mass (spin) multipolar moments in a \ac{NS}, characterized by respective coefficients, 
%$G\mu_l$ ($G\sigma_l$), that have dimensions of [length]$^{2l+1}$.
%%
%The dimensionless relativistic Love numbers are defined as
%%
%\begin{equation}
%k_l = \frac{(2l - 1)!!}{2}\frac{G\mu_l}{R^{2l + 1}}, \hspace{5mm} j_l = \frac{(2l-1)!!}{2}\frac{G\sigma_l}{R^{2l+1}},
%\end{equation}
%
%where $R$ is the radius of a \ac{NS}. 
%In many studies, only the dominant $l=2$ mode is considered, \eg, \cite{33}. 
%For \acp{BH}, $\mu_l=\sigma_l=0$ \cite{32,34}.
%Sometimes, only dominant quadrupole $l=2$ gravitoeletric coeffcient is considered \cite{33}.
%
If the external field can be viewed as quasi-static (``adiabatic tides'') the 
Love numbers can be computed by considering the stationary perturbations of the spherical 
relativistic star, \ie, solving the stellar structure, \ac{TOV} equations in full \ac{GR}, 
taking into account the strong dependency of the tidal coefficients on the star compactness, defined as 
$C_{i} = GM_i/R_i^2c^2$. 
%
Thus, Love numbers carry the imprint of the \ac{EOS} on the \ac{BNS} dynamics.
%
%On the other hand, if the external field is dynamic, than the effects induced by a \ac{NS}, 
%at linear order in the deformation, can be formulated as a superposition of the proper modes 
%of the \ac{NS}. The excitation of modes occur when the resonant frequency of modes is matched 
%by the star's orbital frequency. 
%This approach has been studied in Newtonian gravity, in \ac{GR} for a test-mass circling the 
%\ac{NS}, and for \acp{NS} of similar mass in \ac{PN} theory \cite{35,36,37}.
%Dynamical external field induces ``dynamical tides'', among which the most important are the 
%pressure modes ($f$-modes) in the non-resonant way\footnote{
%    The resonance for $f$-modes occur in kHz
%    regime, that is achieved only at merger, when two \acp{NN} are no longer isolated objects. 
%    Other mode,s such as $g$- and $r$-modes do not contribute significantly, due to their 
%    lower energies (albeit also lower frequencies).
%}
%
%The tidal effects can be included into the \ac{PN} two-body dynamics by modifying the 
%\ac{GR} effective action as 
%%
%\begin{equation}
%S = S_{GR} + S_{pointmass} = \frac{1}{16\pi G}\int R\sqrt{|g|}dx - \sum_{A}\int M_A \dd s_A
%\end{equation}
%%
%where the last term of the \ac{RHS} is the ``skeletonized'' representation \red{as} a point mass, 
%with non-minimal coupling of worldlines, that read 
%%
%\begin{equation}
%S_{nonminimal} = \sum_{A}\frac{\mu_l^{A}}{2l!}\int (G_L^A)^2 \dd s_A + \frac{l \sigma_l^A}{l! 2(l+1)} \int(H_{L}^A)^2 \dd s_A.
%\end{equation}
%%
%The added term changes the dyanmics at $5$th \ac{PN} order. The change is linear in tidal deformations.
%%
%At the leading \ac{PN} (Newtonian) order, only $l=2$ gravitoelectric terms appear when tidal 
%contributions are included. The term reads 
%%
%\begin{equation}
%\label{eq:intro:tidal_largangian}
%L^{LO}_{tidal} = k_2^A G M_B^2\frac{R_A^5}{r^6} + (A\leftrightarrow B),
%\end{equation}
%%
%where $r$ is the distance between \acp{NS}. 


Writing the modified \ac{GR} action with the inclusion of ``skeletonized'' representation
of \acp{NS}, the tidal effects can be added into the \ac{PN} description of the two-body dynamics.
The resulted Lagrangian, at the leading, Newtonian, order shows that at small distances 
between stars, $r$, the introduced corrections are attractive. 
%
Further, the Kepler law, given by the quadrupolar gravitoelectric term, 
%
%\begin{eqnarray}
%\Omega^2 r^3 = G M \Big[ 1 + 12 \frac{M_A}{M_B} \frac{R_A^5}{r^5} k_2^A + (A\leftrightarrow B) \Big].
%\end{eqnarray}
%
shows that the finite-size effects manifest as an increased orbital frequency at a given radius.
In other words, the \acp{NS} spin faster if tidal effects are present and merge sooner 
producing signal with higher frequency \citep{Damour:2009wj}. 
%When $r=R_A+R_B$, the frequency at merger can be estimated. 
%There, $2GM\Omega\approxeq2(M_B/(MC_B) + M_B/(MC_B))^{-3/2}$ \cite{29}. 
%For \acp{NS} of the same mass one obtains 
%%
%\begin{equation}
%f_{GW}^{contact} \approxeq 1.327 \Big( \frac{C}{0.15} \Big)^{3/2} \Big( \frac{M}{2.8M_{\odot}} \Big) \text{kHz}.
%\end{equation}
%
%Simulations show that the contact between the two \acp{NS} happens approximately 2-4 \ac{GW} cycles 
%prior to merger at an even lower frequency \cite{38}

Within the \ac{EOB} framework, the dynamics of the system can be expressed in terms of the 
effective Hamiltonian, $H_{EOB}$
%that for zero-spin case can be written 
%%
%\begin{equation}
%H_{EOB} = M\sqrt{1 + 2\nu (\hat{H}_{eff} - 1)}, 
%\hat{H}_{eff} = \frac{H_{eff}}{\mu} = \sqrt{A(u;\nu)(1 + p_{\phi}^2u^2 + 2\nu(4-3\nu)u^2p_{r^*}^4) + p_{r^*}^2}
%\end{equation}
%%
%where $u=GM/rc^2$ is the Newtonian potential. 
%Consder the Schwarszchild spacetime and a particle in it with $\nu\rightarrow0$ and 
%$A(u;0) = 1-2u$. Then, the $H_{eff}$ becomes the particle Hamiltonian.
%
.
The finite-size effects can be included as tidal component of the field potential, $A$, as 
$A = A_0 + A_{tidal}$ \citep{Bini:2012gu}, 
%that can be written as 
%%
%\begin{equation}
%A_{tidal} = \sum_{l\geq 2}\Big[ k_l^{A+}u^{2l+2}(1+\alpha_1^{(l+)}u ... ) + k_{l}^{A-}u^{2l+3}(1+\alpha_1^{(l-1)}u + ...) + (A\leftrightarrow B) \Big]
%\end{equation}
%%
%where $\alpha_i^{(l)}(\nu)$ are coefficients and 
%%
%\begin{equation}
%k_{l}^{A+} = 2k_l^A\big( \frac{M_A}{MC_A} \big)^{2l+1}\frac{M_B}{M_A}, \hspace{5mm} k_l^{A-} = 2j_l^A\Big( \frac{M_A}{M C_A} \Big)^{2l + 1} \frac{M_B}{M_A}
%\end{equation}
%%
%are the multipolar tidal polarizability coupling constants. 
that in turn depends on the multipolar tidal polarizability coupling constants $k_{l}^{A\pm}$.
%
In the Newtonian limit the \ac{EOB} Hamiltonian then reads 
%
\begin{equation}
H_{EOB} \approxeq Mc^2 + \frac{\mu}{2}p^2 + \frac{\mu}{2}(A-1) = Mc^2 + \frac{\mu}{2}p^2 + 
\frac{\mu}{2}\Big( -\frac{2 G M}{c^2 r^2} + \cdots - \frac{\kappa_2^T}{r^5} \Big)\, ,
\end{equation}
%
where $\kappa_2^T = \kappa_2^A + \kappa_2^B$ is the constant accounting for the tidal 
interactions at leading order.
%
For a physically motivated range of masses $(1-2)\,\Msun$ and \mr{}s $q\in[1,2]$ the 
$\kappa_2^T\sim[50,500]$. 
%
Similarly, one can defile the $\Lambda_2^i = 2/3 k_2^i (c^2 R_i/GM_i)^5$ with $i\in\{A,B\}$.
Then, instead of $\kappa_2^T$, the \textit{reduced tidal deformability} is used
%
\begin{equation}
\label{eq:intro:Lambda}
\tilde{\Lambda} = \frac{16}{13}\frac{(M_A + 12M_B)M_A^4}{M^5}\Lambda_A + (A\leftrightarrow B).
\end{equation}
%
Consequently, the effects of tides appear in waveform calculations, as radiation reaction 
compliments the conservative dynamics of the binary \citep{Damour:2008gu} 
(see also \citet{Damour:2012yf,Banihashemi:2018xfb}).

%At leading order the stationary phase approximation of the waveform reads
%%
%\begin{equation}
%h(f) = Af^{-7/6}e^{-i(\Psi_0(x) + \Psi_{tidal}(x))} = Af^{-7/6}e^{-i(\Psi_0(x)-39/4\kappa_2^Tx^{5/2})}
%\end{equation}
%%
%where $x(f) = (\pi G M f / c^3)^{2/3}$ and $\Psi_0(x)$ is point-mass phase.
%
The $k_2^T$ fully determines the tidal contribution to the waveforms at leading order. 
Hence, from observed \ac{GW} signal, the $k_2^T$ (or $\tilde{\Lambda}$), and 
consequently on the \ac{EOS}, can be estimated. 
%
%There have been extensive tests of the \ac{EOB} formalism for \ac{BNS} mergers agains \ac{NR} 
%simulations \cite{43,38,44,45,46}. 
%The accuracy of the \ac{EOB} approximation of tides, the $A_{tidal}$, decreases during 
%the last orbits and for very large $k_2^T$. Modified versions of \ac{EOB} were developed 
%with gravitational self-force calculations of tides computed at high-order TEOBResumS 
%\cite{44,47,46,48} and dynamical tides (SEOBNRT) \cite{49}.
%When spin cannot be neglected, the tidal interactions become more complex \cite{50,51}.
%The oblateness of a spinning \ac{NS} leads to the deformed gravitational field that is 
%characterized by the quadrupole tensor, and produces an attractive contribution to the 
%potential, modifying the inspiral at the $2$nd \ac{PN} order $\mathcal{O}(\upsilon/c)^4$
%\cite{50}. 
%There are also hybrid models of \ac{EOB} and \ac{NR} \cite{52,53}

%% -------------------------------------
\subsection{Gravitational Waves}

The most accurate way to compute waveforms is to conduct \ac{NR} simulations with microphysical 
\acp{EOS}. However, these simulations are computationally expensive and not available or feasible 
for certain areas of the parameter space. The \ac{EOB} approach allows to compute the waveforms 
for the broad range of frequencies and covers all stages of the binary evolution. 
%The \ac{EOB} models can be augmented with the formalism describing the high frequency emission 
%of the remnant (kiloHertz), that itself can be build from the 
%available \ac{NR} \ac{HD} simulations \citep{Bernuzzi:2015rla,Chatziioannou:2017ixj,Easter:2018pqy}.

The \ac{GW} signal observed with ground-based facilities, contains information 
about the chirp mass, 
%(related to $I_{-10}$),
%where $I_p = \int \dd\ln f(\gamma(f))f x^{2p}(f)$ with $\gamma(f)df$ being the 
%measure of the detector noise \cite{59,6},
primarily at low frequencies. Additionally, the low frequency part of the signal, 
${\leq}50\,$Hz and ${\leq}100\,$Hz, contains information on the symmetric \mr{} and \ac{SNR}.
On the other hand, the information about tidal effects (tidal parameters) is related to 
%$I_{+10}$,
higher frequency signal, evaluation of which requires accurate high-frequency waveforms.


From the Newtonian limit discussed above, it follows that the system dynamics at merger 
is determined mainly by $\kappa_2^T$ \citep{Bernuzzi:2014owa}. \ac{NR} simulations verified this 
prediction \citep{Zappa:2017xba,Breschi:2019srl}. 


With respect to the \GW{} most of the information was obtained in $30-600$~Hz frequency 
range (${\sim}1300$ orbits).
The signal was interpreted as coming from the \ac{BNS} merger with total mass of ${\simeq}2.7\,\Msun$,
chirp mass $\mathcal{M}=1.186(1)\,\Msun$, \mr{} $q\in[1,1.34]$ and $\tilde{\Lambda}\simeq300$ 
(with an upper bound of ${\sim}800$) \citep{TheLIGOScientific:2017qsa,Abbott:2018wiz,LIGOScientific:2018mvr}.
Inclusion of the \ac{EM} counterparts into the analysis resulted in higher $\tilde{\Lambda}$ being 
more favored \citep{Radice:2017lry,Radice:2018ozg,Breschi:2021tbm}.
%
The estimation of individual masses and \mr{} was more uncertain and depended on the 
prior chose for the spin \citep{Abbott:2018wiz}. Due to the partial degeneracy between tidal 
parameters and the \mr{}, the \ac{EOS} constraints are also subjected to uncertainties. 
%Specifically, assuming that stars had small spin ${\leq}0.05$, the stars radii can be 
%constrained to $11-12$~km \cite{De:2018uhw,Abbott:2018exr}. 
%Assuming also that \ac{EOS} can support a non-rotating \ac{NS} with a mass $1.97\Msun$, 
%results in  $R\sim 11.9\pm 1.4$~km and $90\%$ credibility \cite{Abbott:2018exr}.
%
%\ac{GW} signal during the inspiral (including tidal phasing) also provides constraints 
%on the \ac{EOS} viewed in the pressure-density diagram \cite{65}. 
%For an example case of a \ac{BNS} with $\rho_{\rm max} \approxeq 2 \rho_0$, 
%the estimated value of pressure is $P(2\rho_0)=3.5_{-1.7}^{2.7}\times10^{34}$ dyn cm$^{-2}$ 
%at $90\%$ confidence level.
%
%For \GW{} the peak of the \ac{GW} signal, the merger, was not detected. However, from the 
%probability distribution of $\tilde{\Lambda}$, adopting \ac{NR}-motivated fitting 
%formula, it is possible to asses the peak frequency that falls in $1.2 - 2$kHz \cite{55}.
%The \ac{GW} peak luminocity is estimated to be $\geq 0.1\times 10^{56}$ erg/s \cite{61}. 


%% --------------------------------------------------------------------------
%%               P O S T - M E R G E R
%% --------------------------------------------------------------------------

\begin{figure}[t]
    \centering
    \includegraphics[width=0.70\textwidth]{Fig_3_Rad.pdf}
    \caption{
        Overview of the different phases in an \ac{NS} merger and the relative timescales. 
        The inspiral ends with the merger, when the two stars start to fuse together. 
        The early \pmerg{} evolution is entirely driven by hydrodynamics and by \ac{GW} emission. 
        If the remnant does not collapse within ${\sim}10-20\,$ms, \ac{GW} losses
        subside and other physical processes become more important: 
        Angular momentum redistribution (which is due to turbulent viscosity) 
        and neutrino losses operate over a timescale of a tenth of a second to a few
        seconds. This is also the characteristic timescale for the evolution of the remnant disk. 
        If the remnant does not collapse over a timescale of a few seconds, then it will 
        spin down because of \ac{MHD} effects over a possibly much longer timescale 
        of several seconds to a few hours. 
        (Adapted from \citet{Radice:2020ddv}).
    }
    \label{fig:intro:RadFig1}
\end{figure}


%% -----------------------------------
\section{Merger and \pmerg{}}\label{sec:intro:merg_pmerg} %ef in GRLESS
%% ------------------------------------

After \acp{NS} inspiral and merge, the dynamics of the system becomes significantly 
more complex, as temperature and density rise by orders of magnitude and new 
physical effects, \eg, magnetic fields and weak interaction, start to influence the evolution. 
The \pmerg{} phase is not well understood and mainly explored with miltiphsyics \ac{NR} 
simulations with various degrees of sophistication and resolution. 

The summary of the \ac{BNS} \pmerg{} evolution is shown in Fig.~\ref{fig:intro:RadFig1}. 
There are several trajectories that a system can take, depending when/if the formed remnant 
collapses to a \ac{BH}. The early \pmerg{} phase is charaterized by strong \ac{GW} 
emission and hydrodynamic effects. After it, several interlinked processes govern the 
evolution, \eg, \ac{MHD} stresses, that contribute to the angular momentum  redistribution, 
and neutrino emission, that alters the matter composition and cools i.
If \ac{BH} does not form, the \ac{MHD} torques and residual \ac{GW} emission spin down 
the remnant.

%% ------------------------------------------------
\subsection{Dynamics and Thermodynamics Conditions} \label{sec:intro:remnant}

Prior to the merger, \acp{NS} can be considered as being in the cold, neutrino-less, 
weak equilibrium with only marginal heating due to tidal deformation at the last orbits.
The dynamics at merger is dominated by the \acp{NS} orbital motion
as $\upsilon_{rad}\ll\upsilon_{orb}$, 
%\footnote{
%    The orbital speed can be written as $\upsilon_{orb}\eqsim\Omega r\eqsim\sqrt{GM/(R_A + R_B)}$
%    that for equal mass binary is $\upsilon_{orb}/c\eqsim\sqrt{C}\eqsim0.39(C/0.15)^{1/2}$.
%    The radial velocity is beven by the evolution of the orbital frequency, as 
%    $\omega_r \eqsim 2\Omega r \dot{\Omega}/(3\Omega^2)$. The $\dot{\Omega}$ can be 
%    estimated from the fact, that orbital frequency satisfies 
%    $\dot{\Omega}_{GW}\sim(3456/125)(G\mathcal{M}/c^3)^5\Omega_{GW}^{11}$ during the 
%    inspiral. Then, the radial velocity reads 
%    \begin{equation}
%    \upsilon_r/c\eqsim\frac{192\pi}{15}\frac{G^3 M^3}{c^5(R_A + R_B)^3}\frac{q}{(1+q)^2}
%    \end{equation}
%    that for equal mass gives $\upsilon_r/c \eqsim 0.0034 (C/0.15)^3$.
%}.
%Merger time, $t_{merg}$, estimated from the \ac{GW} frequency at \acp{NS} collisition 
%for comparable \ac{NS} masses is given by 
%\begin{equation}
%    t_{merg}\eqsim\frac{1}{2f_{GW}^{contact}}\eqsim1.50\text{ms}\Big(\frac{M}{2.8\,\Msun}\Big)^{1}
%\end{equation}
%where the frequency of \acp{GW} when \acp{NS} come into contact, \ie, when the distance 
%between them, $r = R_A + R_B$ can be evaluated from the Kepler law, given by 
%the quadrupolar gravitoelectric term, % Eq.~7
%$2GM\Omega \eqsim 2(M_B/(MC_B) + M_B/(MC_B))^{-3/2}$ \cite{29},
%as 
%\begin{equation}
%    f_{GW}^{contact} \eqsim 1.327 \Big(\frac{C}{0.15}\Big)^{3/2}\Big( \frac{M}{2.8\,\Msun} \Big) \text{ kHz}
%\end{equation}
that is $\upsilon_{orb}\eqsim\Omega r\eqsim\sqrt{GM/(R_A + R_B)}$.
This indicates that 
more compact binaries experience more rapid, more violent mergers.

%%%% Remnant
%When \acp{NS} collide, their deformed cores squeeze past each other, triggering \ac{KHI},
%in the first bounce. After several of these bounces, the cores fuse into a single object.
At collision, \ac{KHI} is triggered by the \acp{NS} cores plunging into the lower density matter of the companion, 
squeezing past each other and inducing the first wave of gravity-driven compression. 
The maximum values of temperatures and densities are reached than \citep{Perego:2019adq}. 
As nuclear and centrifugal forces start to dominate, the cores bounce back until gravity 
takes over again. %This is referred to as core \bnc{}. 
%
Notably, formed in the violent, fast collision, the remnant core, while being far from 
hydrodynamic equilibrium, does not exhibit shocks. This is due to high speed of sound 
of nuclear matter at supra-nuclear densities.
%($c_s\gtrsim0.2\, c$, at $\rho\gtrsim\rho_0$)
%
Shocks, however, do form at the remnant \ac{NS} surface, accelerating matter to mildly-relativistic 
velocities. %(\ref{Sec:intro:bns:ejecta}).
The fluid inside the cores remains cold 
%$T\lesssim10\,$MeV, $s\lesssim1\,k_B$/baryon 
throughout the merger, while at the interface between cores, the compression and shear 
dissipation raises the temperature to $T\sim70-110\,$MeV.
This is accompanied by the formation of the generic structure, described
by a pair of rotating hot regions, offset by $\sim\pi/2$ with 
respect to dense cold regions \citep{Kastaun:2016yaf}.
%
The subsequent evolution proceeds towards more axisymmetric, stable remnant, but can be 
interrupted by the \ac{BH} formation. 

%%%% Disk Foramtion
The matter outside the bouncing cores, lifted by tidal torques and squeezed out at the 
collisional interface, forms a disk (or a torus).
Due to various contributions with different properties, the disk is highly non-uniform.
The overall properties of the disk, such as mass, have complex dependency on 
binary parameters, that can be expressed, at a first approximation, via fitting 
formulae to \ac{NR} simulations \citep{Radice:2017lry,Radice:2018xqa,Radice:2018ozg}. 
The generic disk evolution around the remnant consists of quasi-adiabatic expansion
of its outer layers 
%with $T^3/\rho^3\sim\text{const}$ as the \ac{EOS} is dominated by non-relativisitc baryons
and cooling of the inner regions. 
%
However, as was mentioned above, the newly born \ac{NS} remnant is not hydrodynamically stable. 
Its dynamics is characterized by the pronounced $m=2$ bar- and $m=1$ one-armed-
deformations \citep[\eg][]{Radice:2016gym}, inducing spiral waves, propagating through the 
disk. 
%(see the Fig.~\ref{fig:ang_mom_flux} and related discussion in Ch.~\ref{ch:bns_sims}). % shocking and heating up the fluid. 
Additionally, the former leads to the strong \ac{GW} emission 
in ${\sim}10-20$~ms \pmerg{}. The backreaction from the energy and angular momentum 
loss dumps the $m=2$ mode efficiently and \ac{GW} emission subsides. 
%We refer to this evolutionary phase as \ac{GW}-dominated \pmerg{} phase.
%Notably, the $m=1$ mode, however, can persist due to mode coupling 
%\cite{Dietrich:2016phd}. 
That marks the end of the ``\ac{GW}-driven phase'' of the \pmerg{} evolution. 

%%%% Disk Settling down 
Weak processes and spiral density waves cool and shock periodically the 
fluid, bringing the disk to the configuration with an overall smooth 
temperature profile 
%from $\sim10\,$MeV at $\rho\eqsim10^{13}$\gcm to $\sim0.1\,$MeV at $\rho\eqsim10^4$\gcm
%with entropy $\in(3,10)$ $k_B$/baryon
and quasi-Keplerian orbit.
%
%%%% IF BH forms
If the remnant collapses to a \ac{BH}, the densest part of the disk is 
accreted on the dynamical timescale, reducing the total mass by half 
%and the disk maxiumum density to $\sim10^{12}\,$\gcm,
and disk shrinks \citep{Perego:2019adq}.

%%%% Magnetic fields
While \acp{MF} are not expected to affect the \ac{BNS} inspiral, their influence 
on the \pmerg{} evolution can be strong \citep{Duez:2006qe,Kiuchi:2017zzg}, as they get amplified 
to the values exceeding that of a magnetar 
%$10^{16}\,$Gauss, 
by a variety of processes, 
\eg, flux freezing and compression, \ac{KHI} at the collisional interface \citep{Kiuchi:2015sga},
\ac{MRI}, \citep{Duez:2006qe,Kiuchi:2017zzg} and \ac{MF} winding \citep{Duez:2006qe},

Whether the ordered, large-scale \acp{MF} can form in \pmerg{} environment 
via the dynamo process is presently unknown. They are important in producing 
polar collimated outflows, jets \citep{Bucciantini:2011kx,Ruiz:2016rai} and mildly relativistic 
outflows \citep{Metzger:2018qfl,Fernandez:2018kax}. Random magnetic fields are also 
relevant for the \pmerg{} evolution, as they generate stresses, enhancing angular momentum transport. 
Presently, these processes are not well understood, as seed \ac{MHD} instabilities operate at 
small scales (centimeters) and cannot be resolved in global \ac{MHD} \ac{BNS} merger 
simulations.
% with reslistic initial condiitons
To be able to resolve the instabilities (to increase their scale), the seed \acp{MF} are 
artificially enhanced to the magnetar-strength \citep{Kiuchi:2015sga,Kiuchi:2017zzg}.
%
%%%% Alpha-viscosity model
The effect of \acp{MF} on the angular momentum transport can be approximated via 
the $\alpha$-viscosity model \citep{Shakura:1972te}, calibrated 
with very high resolution \ac{MHD} simulations \citep{Radice:2017zta,Radice:2020ids}. 
%\red{More on it? For the theiry and GRLESS model?}
These effects are important 
in determining the remnant structure, lifetime and hence, the \pmerg{} \acp{GW} 
\citep{Radice:2017zta,Shibata:2017xht} 
%The timescale for the angular momentum redistribution in the remnant \cite{80} 
%\begin{equation}
%    t_{rem} \eqsim \alpha^{-1}R_{rem}^2\Omega_{rem}c_s^{-2}\eqsim 0.56\,s\Big(\frac{\alpha}{0.001}\Big)^{-1}\Big(\frac{R_{rem}}{15\,\text{km}}\Big)^2 \Big( \frac{\Omega_{rem}}{10^4\,\text{kHz}} \Big) \Big(\frac{c_s}{0.2\,c}\Big)^{-2}
%\end{equation}
%where $\Omega_{rem}$ and $c_s$ are the angular momentum and sound speed respectively.
as the loss of angular momentum brings the remnant closer to either stable, 
rigidly rotating configuration or collapse \citep{Hotokezaka:2013iia}. 
%
The \acp{MF} effects within the Keplerian disk facilitate accretion 
\citep{Fernandez:2015use,Fujibayashi:2017puw,Fernandez:2018kax,Miller:2019dpt}.
%on a timescale
%\begin{equation}
%    t_{disk} = \alpha^{-1}\Big(\frac{H}{R}\Big)^{-2}\Omega^{-1}_K \eqsim 0.78 \Big(\frac{\alpha}{0.02}\Big)^{-1}\Big(\frac{H/R}{1/3}\Big)^{-2}\Big(\frac{M_{rem}}{2.5\,\Msun}\Big)^{-1/2}\Big(\frac{R_{disk}}{100\,\text{km}}\Big)^{3/2}
%\end{equation}
%where $M_{rem}$ is the mass of the central remnant and $R_{disk}$ is the radial 
%scale of the disk.

%%%% Neutrinos
The prime cooling mechanism in the post-\ac{GW}-dominated phase is the emission of 
neutrinos, produced in hot, dense areas of the disk and remnant, and that are 
able to escape \citep{Eichler:1989ve,Rosswog:2003rv,Sekiguchi:2011zd}. 
%The typical neutrino mean free path is 
%\begin{equation}
%    \lambda_{\nu} = \Big(n_B\sigma_0(E_{\nu}/m_e c^2)^2\Big)^{-1}\simeq 24.6\,\text{m}\,(\rho/10^{14}\,\text{g}\,\text{cm}^{-3})^{-1}(E_{\nu}/10\,\text{MeV})^{-2},
%\end{equation}
%where $n_B$ is the density of baryons, and 
%\begin{equation}
%    \sigma_0 \simeq 4G_{F}^2 (m_e c^2)^2 / (\pi(\hbar c)^4)\simeq 1.76\times 10^{-44} \, \text{cm}^2
%\end{equation}
%is the typical neutrino cross section scale, and $E_{\nu}$ is the neutrino energy.
%Considering the charactersitic remnant temperature as  $T_{rem}\simeq20\,$MeV 
%energy of the thermal neutrinos then $E_{\nu}\simeq 3.15T_{rem}$ and optical depth 
%$\tau_{\nu}\simeq R_{rem}/\lambda_{\nu}=\mathcal{O}(10^{4})$.
The neutrinos are radiated on a diffusion timescale \citep{Perego:2014fma}.
%\begin{equation}
%    t_{diff} \simeq \frac{\tau_{\nu}R_{rem}}{c}\simeq 4.28\,s\Big( \frac{R_{rem}}{15\,\text{km}} \Big)^{-1}\Big(\frac{M_{rem}}{2.5\,\Msun}\Big)\Big(\frac{T_{rem}}{20\,\text{MeV}}\Big)^2.
%\end{equation}

%%%% neutrinos in the remnant and disk
Within the remnant, neutrinos are in a weak and thermal equilibrium with matter 
due to charged current reactions. There, the production of electron neutrinos, $\nu_e$,
is suppressed by degeneracy and as chemical potential $\mu_n-\mu_p+\mu_e<0$ at high temperatures, 
the electron anti-neutrinos, $\bar{\nu}_{e}$, dominate. The effect of these ``trapped'' 
neutrinos on the remnant evolution is not very strong \citep{Foucart:2015gaa,Perego:2019adq}.
%
Within the disk, however, the optical depth for neutrinos is ${\simeq}1$ so 
they can diffuse out on a timescale of milliseconds, lowering the disk 
temperature. The cooling rate is controlled by the degeneracy state of neutrinos 
%that is kept at mild values by the feedback negative effect, higher 
%values have on the cooling rate.
\citep{Beloborodov:2008nx}.

During the early \pmerg{}, the luminosity of the electron antineutrinos 
$L_{\bar{\nu}_e} \gtrsim L_{\nu_e}$, as the 
free neutrinos are abundant in the disk and the absorption opacity for $\nu_e$ 
exceeds that of $\bar{\nu}_e$.
Thus, alongside heating and decompression, the initially cold matter in weak 
equilibrium, undergoes \textit{leptonization} 
%The heavy neutrinos, $\nu_{\mu,\tau}$, are balanced by pair processes and 
%decouple from matter at higher densities and temperatures within the remnant
%namely $\rho\gtrsim10^{13}$\gcm and $T\gtrsim8\,$MeV 
\citep{Perego:2014fma,Endrizzi:2019trv},
%The, BNS simulations including
%neutrino transport predict the mean neutrino energies at infinity
%$E_{\nu_{e}}(\sim 10\,\text{MeV})\lesssim E_{\bar{\nu}_e}(\sim 15\,\text{MeV}) \lesssim E_{\nu_{\mu,\tau}}(\sim 20\,\text{MeV})$. Notably, binaries with higher mass 
%show higher neutrino energies \cite{14,86}.
\ie, $n+\nu_e\rightarrow p + e^-$. %\red{check that, from Wiki}
%
Neutrinos with different energies decouple from matter in different regions.
%due to the stron dependency of the cross section on the incoming neutrino energy.
Mildly energetic $\nu_{e}$ and $\bar{\nu}_e$ decouple at ${\sim}10^{11}\,$\gcm, found 
in the disk, while low energy neutrinos decouple at higher ${\sim}10^{13}\,$\gcm.
The geometric surface along which neutrinos decouple is usually called ``neutrino 
surface'' \citep{Perego:2014fma,Endrizzi:2019trv}.
%The dependency of the location and the geometry of this surface on the thermodynamics 
%conditions and neutrino energy facilitates the need to coherent treatment of the 
%strong and weak interactions over a broad span of densities and temepratures.
%The energy dependent (spectral) neutrino radiation trasport is required in 
%\ac{BNS} merger simulations

%%%% Neutrino oscillations
While it has been shown that neutrino flavor conversions may occur in the \ac{BNS} 
\pmerg{} environment, \eg, the matter-neutrino oscillations, induced by the 
the fact that $\bar{\nu}_e$ decouple at smaller radii then $\nu_{e}$ 
\citep[\eg][]{Zhu:2016mwa,Tian:2017xbr}, and fast-flavor conversions above the 
neutrino surface \citep{Wu:2017drk}, their impact on the properties of the ejected 
material remains largely unexplored.
%More works of neutrino quantum kinetics equations \cite{90,91}, that include 
%collisional integral and angular and energy distributions of neutrinos are required.

%%%% EOS 
One of the main unknowns with respect to \ac{BNS} mergers, is the high density part, 
$\rho \gtrsim \rho_0$ of the \ac{NS} \ac{EOS} \citep[\eg][]{Hebeler:2013nza,Oertel:2016bki}, that corresponds to the 
relevant thermodynamic degrees of freedom and nucleonic interactions. For instance, 
the emergence of the new species, \eg, hyperons, pions, muons, and nuclear 
resonances, would lower the matter degeneracy, and thus, soften the \ac{EOS} \citep[\eg][]{Fore:2019wib,Vidana:2010ip}.
Additionally, the emergence of the deconfined quark matter, due to \ac{QCD} phase transition,
would alter \ac{EOS} at very high temperatures and densities,
the effect that is not well understood \citep{Busza:2018rrf}.

%% ---------------------------------------
\subsection{Fate of the Remnant}

The end product of \ac{BNS} merger depends primarily on the binary parameters, (masses, 
\mr{}) and \ac{EOS}, and in particular, on the maximum supported mass of a non-rotating 
\ac{NS}, $M_{max}^{TOV}$ \citep{Shibata:2016}, as well as on the finite temperature and 
non-beta-equilibrium composition effects. 
%
The \ac{BH} formation directly at merger is usually referred to as \ac{PC}. 
The conditions for it  are not well understood. 
Simulations show that in equal mass binaries, \ac{PC} occurs if the total mass 
$M\gtrsim M_{thr} = k_{thr}M_{max}^{TOV}$, where $k_{thr}\in(1.3,\,1.7)$ that depends on 
\ac{EOS} \citep{Shibata:2005ss,Shibata:2006nm,Hotokezaka:2011dh,Bauswein:2013jpa}.
%(or $\kappa_2^T\lesssim43-73$, or $\tilde{\Lambda}\lesssim338-386$.
Simulations of unequal mass binaries show that the threshold is lower \citep{Bauswein:2017vtn}, 
and $k_{thr}\propto C_{1.6}$, with $C_{1.6}$ being the compactness of the $1.6\,\Msun$ 
\ac{NS} \citep{Hotokezaka:2011dh,Bauswein:2013jpa,Bauswein:2017vtn}. 
%As \ac{PC} cases are 
%not expected to ejecta large amount of material and be \ac{EM}-loud (in case of 
%comparable mass binary), the \GW{} is believed to be not a \ac{PC} case, 
%\cite{Margalit:2017dij,Bauswein:2017vtn}. 

A remnant that does not undergo \ac{PC} is a massive \ac{NS}, temporarily supported 
by fast rotation \citep{Baumgarte:1999cq,Rosswog:2001fh,Shibata:2005ss,Shibata:2006nm,
    Sekiguchi:2011zd,Hotokezaka:2013iia,Bernuzzi:2015opx}, whose lifetime depends 
intricately on the \ac{EOS}, finite temperature effects, and viscosity. A commonly 
adopted classification, based on the properties of equilibrium models (neglecting the 
dynamical, finite temperature, and magnetic effects), \ie, $M_{max}^{TOV}$ and
$M_{max}^{RNS}$\footnote{
    $M_{max}^{RNS}$ is the maximum mass of a rigidly rotating \ac{NS} (no differential 
    rotation) supported by zero-temperature (cold) \ac{EOS}. Also sometimes referred as 
    mass-shedding limit.
    % $M_{max}^{TOV}$ and $M_{max}^{RNS}$ are agnostic to thermal or magnetic eects 
    %which can impact the stability of the remnant in nontrivial ways (108, 72)
}
distinguishes between 
(i) \ac{HMNS} if $M>M_{max}^{RNS}$, 
(ii) \ac{SMNS} if $M_{max}^{TOV} \leq M_{max}^{RNS}$,
and (iii) stable \ac{MNS} if $M < M_{max}^{TOV}$ \citep[\eg][]{Baumgarte:1999cq}.
A \ac{HMNS} is supported by differential rotation that viscosity reduces with time 
and it is expected to collapse to a \ac{BH}. A \ac{SMNS} can avoid the collapse 
even after reaching rigidly rotating configuration. The lifetime of a \ac{HMNS} and 
a \ac{SMNS} depends on the efficiency of mass and angular momentum loss 
(to, \eg, \acp{GW} and massive winds) and finite temperature effects \citep{Radice:2018xqa}. 
%
Discussing the \ac{NR} simulations we shall distinguish between \textit{short-lived} remnants, 
that collapse to \ac{BH} during \ac{GW}-dominated phase, 
% taht correspods to ${\sim}10-20\,$ms after merger \cite{107,60}
and \textit{long-lived} remnants otherwise.

If a remnant can achieve rigid rotation, its evolution then is characterized by the emission 
of \acp{GW} (due to residual asphericity) and magnetic braking, reducing further its 
angular momentum, until either a \ac{BH} forms, or a stable non-rotating equilibrium is reached.
The observations of \acp{SGRB} with X-ray plateu 
\citep{Zhang:2000wx,Lasky:2015lej,Fan:2013cra}, which if indeed is a consequence of 
a magnetar activity\footnote{
    There are other possible explanations behind the X-ray plateu emission of \acp{SGRB} 
    \citep{Oganesyan:2019jij}
}, suggest that the remnant may survive from seconds to hours \citep{Fan:2013cra,Ravi:2014gxa}.
Most of the energy, ${\sim}10^{52}\,$erg, that a remnant needs to lose before collapsing, if indeed \ac{SGRB} is powered by the accretion onto a \ac{BH}, 
is believed to be carried away in form of \acp{GW}, as \ac{EM} observations of \GW{} suggest.
%This emission can be observed for the nearby event \cite{111}, priding direct information 
%on the remnant fate. However this emission was not observed for \GW{} \cite{67,68}

From \ac{EM} observations of the merger the fate of the remnant can be inferred, 
albeit in the model-dependent way. 
In case of \GW{}, the presence of such counterpart 
strongly disfavors \ac{PC}, as in that case the ejection of any significant amount 
of matter is not expected \citep{Margalit:2017dij,Bauswein:2017vtn,Radice:2017lry}, while leaving 
the question whether it was \ac{HMNS} or \ac{SMNS} open. \citet{Margalit:2017dij} suggested 
that the remnant was short-lived, on account of the non-detection of the expected ${\sim}10^{52}\,$erg, 
from the spin-down of the long-lived one. Assuming thus that \GW{} produced a \ac{HMNS}, 
allows to put a constraint $M_{max}^{TOV} \lesssim 2.2\,\Msun$ \citep{Margalit:2017dij}.
%
On the other hand, if the magnetar model of \acp{SGRB} is applied to \GW{} 
\citep{Ai:2018jtv,Li:2018hzy,Piro:2018bpl}, the remnant is required to be long-lived
(${\sim}$months) and exhibit weak dipol \acp{MF} \citep{Ai:2018jtv}, leading to a different 
constraint on the maximum mass of the non-rotating star, $M_{max}^{TOV} \gtrsim 2.2\,\Msun$. 
%and up to an order of magnitude lower ejecta mass ${\sim}10^{-3}\,\Msun$, that is attributed 
%to reducied amount of emergy from radioactive decay is required to explain the observed 
%UV/optical/infrared observations \cite{16}. 
%There have been indications of the X-ray flaring activity in \GRB{} \cite{117}, but 
%follow-up observations did not confirm it \cite{118}
 



%% --------------------------------------------------------------------------
%%               N U C L E O S Y N T H E S I S
%% --------------------------------------------------------------------------

\section{Ejecta, \nuc{}, electromagnetic counterparts}

\subsection{$R$-process \nuc{}} \label{sec:intro:nucleo}

Nuclides with atomic number  $A\geq 56$ cannot be synthesized via nuclear burning 
due to their large Coulomb barriers and are produced via 
neutron capture processes \citep{Burbidge:1957}.
% Nucleosynthesis Cites
The maximum $A$ of a nuclide is limited by its binding energy, $Q_n$. 
At $Q_n \simeq1 \,$MeV photodisintegration starts breaking nuclides apart. 
The place in the parameter space of temperature and density where this occurs 
is called \textit{neutron drip line} \citep{Rolfs:1988}.
% Fast and Slow
Nuclides produced via neutron capture are generally unstable to $\beta$-decay,
and depending on whether the timescale for the decay is slower (faster) than 
the timescale of neutron capture, one distinguish rapid (slow) process called,
\rproc{} (\sproc{}).
%
The \sproc{} moves along the valey of stability, while \rproc{} moves along the 
neutron drip line.
%
When a nuclide reaches a closed neutron shell configuration, the cross-section 
for the subsequent neutron capture shrinks, and capture processes suspends until 
several $\beta$-decays take place. This results in the overproduction of nuclides 
that are located at the intersection between the neutron drip line and closed 
neutron shell for the \rproc{}. This manifests as ``peaks'' in the final abundance 
pattern at $A$ corresponding to these configurations.
Closed shell nuclides are located at $N=50,\: 82, \: 126$ and 
corresponding abundance peaks $A=80,\:130,\:194$ for \rproc{} 
(see \eg, \citet{Arnould:2007gh}).

%%%% R-proces cites
Conditions for the \rproc{} can be achieved in different astrophysical cites, \eg, 
certain types of \acp{SN} and \ac{BNS} mergers where very neutron rich (low 
electron fraction, $Y_e = 1/(1 + N_n/N_p)$, with $N_n$ and $N_p$ being the 
total proper number density of neutrons and protons respectively) 
conditions can be reached
\citep{Mathews:1990,Thielemann:2011,Lippuner:2015gwa,Siegel:2019mlp}. 
Regarding the \acp{SN}, winds driven by strong neutrino fluxes from the hot, deleptonized
core \citep{Qian:1996xt} were suggested as a promising cite \citep{Woosley:2002,Wanajo:2006mq}.
However, high $Y_e$ found in such winds do not allow for the full \rproc{} and only ``light'' heavy 
nuclide, up to $A\sim130$, can be synthesized 
\citep{Qian:1996xt,Thompson:2001ys,Fischer:2010,Roberts:2010,MartinezPinedo:2012rb,Wanajo:2013} 
%
A full \rproc{} can be achieved in so-called magnetorotationally driven \ac{CCSN},
(a rare type of \ac{CCSN} with a rapidly spinning strongly magnetized core), 
initiated by the \ac{MRI} and accompanied by a % magnetorotational processes \eg, 
formation of a 
collimated bipolar jet 
\citep{Wheeler:2000,Akiyama:2003,Burrows:2007yx,Mosta:2014jaa,Mosta:2015,Siegel:2019mlp}.
Conditions within such a jet were found to be suffient for full \rproc{} \nuc{} 
\citep{Winteler:2012,Nishimura:2015nca}.
%The rarity of this type of \acp{SN}, 
%however, might not allow for them to be the dominant source or \rproc{} material 
%\citep{Nishimura:2015nca} 

%Neutron-rich ejecta from \ac{BNS} and \ac{NSBH} mergers are regarded as one of 
%the main cites or \rproc{} \nuc{}.
%
%Mergers of two \acp{NS} or a \ac{NS} and a \ac{BH} are regarded as one of the main cites 
%of \rproc{} material.% (See Sec.~\ref{sec:intro:bns_merg}). 

%%%% Fission cycling
Under the very neutron-rich conditions, 
the nuclides beyond $A=300$ can be produced. Being unstable to fission, they 
decay into seed nuclides shortly after formation.
However, before they reach the valley of stability, neutron capture occurs again, and the cycle repeats.
This is so-called \textit{fission cycling}.
It is maintained as long as there are free neutrons, after which nuclides decay 
for the last time, forming a remarkably robust abundance pattern independent of the 
number of cycles (and thus on the exact conditions) 
(see figure 4 in \citet{Korobkin:2012uy}).

%%%% Cites
There is no consensus yet on what is the main source of \rproc{} material in the 
Universe. 
Observed \rproc{} abundances in \ac{MP} stars, formed early in the Galactic history, 
point towards a sources that was active in the early Universe, which is in tension with the long, 
$10^{6} - 10^{9}$ years, delay time required for compact object inspiral 
\citep{DeDonder:2004cx,Dominik:2012kk}. This estimate, however, depends strongly on the 
uncertain common envelop evolution phase of the massive binary (progenitors)
\citep[\eg][]{Dominik:2012kk}.
%
Observations of certain \acp{UFG} suggest that the stars there have been enrighed by rare, 
high-yield events (\eg, \ac{UFG} Reticulum II showed abundances similar to solar, 
while other \ac{UFG} galaxies show $2-3$ times lower abundances) \citep{Ji:2016}.
%
Earth crust and meteorites $^{244}$Pu studies also point towards rare, high-yield 
events \citep{Wallner:2015,Tsujimoto:2017}. This scenario has been confirmed with 
models of galactic mixing \citep{Hotokezaka:2015zea}.
%
It is however difficult to explain the observed uniform distribution of \rproc{} 
elements in the Galaxy with such events \citep{Argast:2003he}.
%
Population synthesis models have indicated that with a contribution from 
magnetorotationallydriven \acp{CCSN} the compact object mergers can account for the 
observed distribution \citep{Ishimaru:2015,Cescutti:2015,Wehmeyer:2015,VanDeVoort:2015}.



%% ----------------------------------
\subsection{Kilonova} \label{sec:intro:kilonova}

%The \rproc{} \nuc{} in ejecta is primarily determined by the electron fraction 
%\citep{Lippuner:2015gwa}, $Y_e$, producing $3$rd ($1$st) elements of the \rproc{} peaks 
%%(See Sec.~\ref{sec:intro:nucleo}) 
%if the $Y_e\gtrsim 0.3$ ($Y_e \lesssim 0.2$) 
%with respective abundances in the former remarkably close to solar. 
%%The transition at $Y_e{\simeq}0.25$ is rather sharp.

\citet{Li:1998bw} suggested that 
radioactive decay of the material enriched with \rproc{} elements,
ejected in \ac{BNS} or \ac{NSBH} mergers,
can power an \ac{EM} transient.
%
Authors showed that contrary to the normal \acp{SN}, 
the ejecta would quickly become transparent to its own emission, peaking on a timescale 
of around a few days. 
%
The main difficulty in this pioneering work was the lack of a \nuc{}
models to estimate the radioactive heating in the ejecta. 
%
The understanding of \ac{kN} has significantly improved since then
\citep[\eg][]{Kulkarni:2005jw,Metzger:2010,Roberts:2011,Metzger:2016pju,Wollaeger:2017ahm} 
and advanced even further with the detection of \AT{} \citep[\eg][]{Metzger:2019zeh}.


%% --------------------------
%% M O D E L L I N G  K I L O N O V A
%% --------------------------

The way to compute \ac{EM} emission from stratified, asymmetric ejecta is to perform 
multi-dimensional, multi-group radiative transfer simulation coupled to \ac{HD} (or \ac{MHD}) 
simulation of the ejecta itself \citep[\eg][]{Bulla:2019muo}.
%
It is however possible to compute the bolometric\footnote{
    Related to the total emitted radiation at all wavelengths. 
} properties considering the total amount 
of energy released and emitted by radioactive decay within ejecta. 

%
%Here a simplified model is considered of a transient, powered by the radioactive decay 
%within the ejecta only. Several other assumptions are made. In particular, the ejecta 
%expansion is homologous (faster matter ahead of slow one) \citep{Rosswog:2013kqa}. %%{(Rosswog et al 2014)}
%
%\red{this is based on the Barnes PhD thesis on Opacities for Kilonva (Rad.Transport)}
%\red{Based on Metzger paper }

%% --------------------------
%% SIMPLE MODEL
%% --------------------------

%\subsubsection{Basic ingredients}

%Consider the following simplified approach to compute the \ac{EM} signal from \rproc{}
%elements enriched material. 
Let the radioactive decay of a newly synthesized heavy isotope, $i$, release 
$\dot{Q}_i \propto \exp(-t/\tau_i)$ energy, with $\tau_i$ being its half-life.
Then, assuming equal distribution of $\tau_i$ per logarithmic time, 
%(at any $t$ the dominant species have $\tau\sim t$), 
the heating rate of the ejecta at time $t$ is 
$\dot{Q}_{LP} = f M c^2 / t$,
where $f$ is the free parameter and $M$ is the ejecta mass\footnote{
    In general heating is time-dependent as thermodynamic histories of the expanding 
    ejecta (from \ac{NR} simulations) showed \citep{Metzger:2010,Roberts:2011,Korobkin:2012uy}.
    See also \citet{Hotokezaka:2017dbk} for the discussion on physical principles behind the decay.
}.
%
%In addition, provided by \citet{Li:1998bw}, normalization $f$ resulted in overestimation of the peak 
%luminosity of the \ac{kN}, that plagued the \ac{kN} searches for a decade 
%\cite{Rosswog:2005su,Dong:2015oea,Bloom:2005qx,Kocevski:2009gv}. 

The first self-consistent estimation of the heating rates based on the \ac{NRN} 
calculations of the \rproc{} in the ejecta, were carried out by the \citet{Metzger:2010}. 
The authors showed, that based only on dynamical ejecta, the \ac{EM} transient is ${\sim}10^3$ 
times brighter then Novae -- hence, the term, kilonova was coined. 
%
%It was also shown the ejecta electron fraction does not affect the heating rates
%considerably, and performed the first estimations of the thermalization 
%efficiency of decay products.
%
%The term macronova was however coind by \citet{Kulkarni:2005jw} who considered a 
%transient powered by the decay of radioactive $^{56}$Ni and free neutrons. 
%However, as it was shown later, $^{56}$Ni can only be formed in small quantities on 
%the outskirts of the ejecta.

The calculation of observed \ac{EM} emission is complicated by the %ejecta opacity, 
%as there is a 
general lack of experimental data and numerical models of the optical opacity 
of matter enriched with singly and doubly ionized heavy \rproc{} elements. 
Iron-group gray opacities were initially considered \citep{Roberts:2011}, 
but were later found to severely underestimate those of 
lanthanides and actinides with their 
complex atomic structures (given by open $f$-shells) 
\citep{Kasen:2013xka,Tanaka:2013ana}. 
%
Higher opacities shift light curve peak time by ${\sim}1\,$weak \citep{Barnes:2013wka} 
and shift the spectral peak from optical/\ac{UV} to \ac{NIR}.

%% ----------------------------------------
%\subsection{Basic model of the \ac{kN}}
%% ----------------------------------------

For a single shell of hot and optically thick matter 
(in which the thermal energy is not immediately radiated away), 
expanding with constant velocity, $\upsilon$, 
%such that $R\approx\upsilon t$ at any point in time $t$, 
the radiation diffusion timescale is given by 
%
\begin{equation}
\tau = \rho \kappa R = \frac{3}{4}\frac{M\kappa}{4\pi R^2}, \hspace{5mm} 
t_{diff} \approx \frac{R}{c}\tau = \frac{3}{4}\frac{M\kappa}{4\pi c R} = \frac{3}{4}\frac{M\kappa}{\pi c \upsilon t}\, ,
\end{equation}
%
where $M$ is the ejecta mass, $\kappa$ is the opacity (cross section per unit mass), 
$\rho$ is the density, \eg, $\rho=3M/(4\pi R^3)$ is the mean density.
%
As ejecta expands and cools (via adiabatic losses) its opacities decreases, 
and so does the diffusion timescale. 
When $t_{diff}$ reaches $t$, the radiation can escape the ejecta \citep{Arnett:1982}. 
Hence, the characteristic timescale of the peak of emitted radiation 
%
%\red{check the eq.} -- Eq.5 in Metzger
\begin{equation}
t_{peak} = \Big(\frac{3}{4\pi}\frac{1}{\beta}\frac{M\kappa}{\upsilon c}\Big)^{1/2}\, ,
\end{equation}
%
where the constant $\beta$ depends on the exact density profile of the ejecta. 
The $t_{peak}$ is of order of days for lanthanides-free and weeks for lanthanides-rich ejecta.
The peak luminosity is set by the total heating rate $\dot{Q}(t)$ within the ejecta, 
as described by the \textit{Arnett's Law} \citep{Arnett:1982}. 

%% ----------------------------------------
%\subsubsection{Emission from stratified ejecta}
%% ----------------------------------------

%Consider the ejecta with a given mass-velocity distribution, that can be approximated 
%as $M_{\upsilon} = M(\upsilon / \upsilon_0)^{-\beta}$, for $\upsilon \geq \upsilon_0$,
%where $M$ is the total mass, $\upsilon_0$ is the average, minimum velocity. 
%The parameter $\beta$ can be set to $3$ \citep{Bauswein:2013yna}. %%{(Bauswein et al 2013a)}. 
%However see \citet{Piran:2012wd}%%{Piran et al 2013} 
%for a more complex velocity profiles.
%%
%%
%The diffusion timescale defines when the radiation escapes the ejecta. For a layer with 
%$\upsilon$ and $M_{\upsilon}$ and opacity $\kappa_{\upsilon}$ it is 
%%
%\begin{equation}
%t_{d,\upsilon} \approx \frac{3}{4\pi}\frac{M_{\upsilon}\kappa_{\upsilon}}{\beta Rc} = 
%\frac{1}{4\pi}\frac{M_{\upsilon}^{4/3}\kappa_{\upsilon}}{M^{1/3}\upsilon_0 t c}
%\end{equation}
%%
%where $\beta=3$ was assumed. 
%%
%This equation implies that at time $t=t_{d,\upsilon}$ the radiation from the layer 
%$M_{\upsilon}$ peaks.
%%
%The $M_{\upsilon}(t)$ is related to the total mass of the ejecta and 
%peak time (when radiation diffuses from the entire ejecta)
%%
%\begin{equation}
%M_{\upsilon}(t) = 
%\begin{cases}
%M(t/t_{peak})^{3/2},& t<t_{peak}, \\
%M, &t>t_{peak}
%\end{cases}
%\end{equation}
%%
%where $t_{peak} = (3M\kappa / (4\pi \beta \upsilon c))^{1/2}$ with 
%$\upsilon = \upsilon_0$. \red{Did not understand}
%%
%Outer layers with $M_{\upsilon} < M$ peak first, while the deepest layers peak later but 
%set the luminosity of the whole ejecta (assuming that the the opacity is constant in the 
%ejecta. 
%%
%The radial evolution of each layer $M_{\upsilon}$ of mass $dM_{\upsilon}$ is given by 
%%
%\begin{equation}
%\frac{dR}{dt} = \upsilon,
%\end{equation}
%%
%and the layer's thermal energy changes according to 
%%
%\begin{equation}
%\label{eq:theory:mkn:energ}
%\frac{dE_{\upsilon}}{dt} = \underbracket{-\frac{E_{\upsilon}}{R_{\upsilon}} 
%    \frac{dR_{\upsilon}}{dt}}_{PdV\text{ losses}} - 
%\underbracket{L_{\upsilon}}_{\text{rad. los.}} + \underbrace{\dot{Q}}_{\text{heating sources}},
%\end{equation}
%%
%where the radiative losses take form
%%
%\begin{equation}
%L_{\upsilon} = \frac{E_{\upsilon}}{t_{d,\upsilon} + t_{lc,\upsilon}},
%\end{equation}
%%
%in which the $t_{lc,\upsilon} = R_{\upsilon}/c$ limits the energy loss to the 
%light crossing time (important for when the layer is optically thin) \red{did not understand}.
%%
%The heating sources $\dot{Q}$ include
%%
%\begin{equation}
%\dot{Q}(t) = \underbrace{\dot{Q}_{r,\upsilon}}_{\text{radioactivity}} + \underbrace{\dot{Q}_{mag}}_{\text{magnetar}} + 
%\underbrace{\dot{Q}_{fb}}_{\text{fall-bak accretion}}.
%\end{equation}
%%
%%
%Next, even though in principle the effect of radiation pressure on ejecta oughtto 
%be considered, in case where radioactive heating, total energy input 
%$\int \dot{Q}_{r,\upsilon}dt < E_{kin,0}$ of the ejecta \citep{Metzger:2010,Rosswog:2013kqa} 
%%\cite{(Metzger et al 2011; Rosswog et al 2013)}, 
%this effect can be neglected.
%Meanwhile, central engine might provide enough energy to modify the free expansion 
%model. Then the equation for the central shell velocity evolution reads 
%%
%\begin{equation}
%\label{eq:theory:mkn:velcenteng}
%\frac{d}{dt}\Bigg(\frac{M\upsilon_0^2}{2}\Bigg) = M\upsilon_0\frac{d\upsilon_0}{dt} = \frac{E_{\upsilon_0}}{R_0}\frac{dR_0}{dt}
%\end{equation}
%%
%Here, the term with $E_{\upsilon_0}$ balances the $PdV$-\textit{loss} term in the 
%thermal energy equation (for $dE_{\upsilon}/dt$)
%%
%To compute the emitted radiation, first assume the black-body emission, 
%the thermal emission, with effective temperature 
%%
%\begin{equation}
%T_{eff} = \Bigg(\frac{L_{tot}}{4 \pi \sigma R_{ph}^2}\Bigg)^{1/4}
%\end{equation}
%%
%where $L_{tot} = \sum(L_{\upsilon dm_{\upsilon}})$ is the total luminosity 
%(cumulative for all mass shells). At the point where optical depth 
%$\sum(\kappa_{\upsilon}dm_{\upsilon})=1$ the photosphere is located with 
%radius $R_{ph}(t)$. 
%%
%%The flux density of the source at photon frequency $\nu$ is given by 
%%%
%%\begin{equation}
%%F_{\nu}(t) = \frac{2\pi h \nu^3}{c^2} \frac{1}{\exp\Big(\frac{h\nu}{kT_{eff}}\Big) - 1} \frac{R_{ph}^2}{D^2}
%%\end{equation}
%%%
%%where, $D$ is the distnace to the source. (Cosmological effects are neglected here).
%%
%Additionally, the opacity $\kappa_{\upsilon}$ depends on the temperature of the 
%layer $T_{\upsilon}$, that itself can be computed as 
%%
%\begin{equation}
%T_{\upsilon} = \Bigg(\frac{3E_{\upsilon}}{4\pi a R^{3}_{\upsilon}}\Bigg)^{1/4}
%\end{equation}
%%
%assuming that the ejecta internal energy is dominated by the radiation. 
%%
%%
%Finally, in order to compute the \ac{EM} emission from the ejecta, 
%the equation Eq.~\eqref{eq:theory:mkn:energ} ought be solved for $E_{\upsilon}$ 
%(and $L_{\upsilon}$) for every shell with $dM_{\upsilon}$ and $\upsilon>\upsilon_0$. 
%%
%The velocity distribution can be assumed fixed 
%(\eg $M_{\upsilon} = M(\upsilon/\upsilon_0)^{-\beta}$,  
%if only the internal heating are important. If however, the central engine energy 
%input is important, the the velocity of the central layer evolves according to
%Eq.~\eqref{eq:theory:mkn:velcenteng}.
%%
%Initial kinetic energy of the ejecta is quickly removed by the adiabatic expansion 
%and the thermal energy of the ejecta, when its emission peaks, is dominated by the heating.

%% -----------------
%% \paragraph{Opacity}
%% -----------------

In this simplified description thee key components can be identified for modeling the \ac{kN}.
These are 
%\begin{itemize}
    %\item
    (i) ejecta geometry and properties, $M$, $\upsilon$, $Y_e$; 
    %\item 
    (ii) composition of the expanding ejecta and its optical opacity; 
    %\item 
    (iii) dominant sources of energy within the ejecta, heating rate $\dot{Q}(t)$, and how 
    efficiently this energy thermalizes.
%\end{itemize}
%
%We discuss the properties of the ejecta from \ac{BNS} mergers in the chapter~\ref{ch:bns_sims}.
%Here we focus on two other ingredients.

Regarding opacity, there are several sources of opacity that affect photons with
different energies that include 
%%%% --- free-free opacities
%For the photons of the lowest energy (frequency), \ac{FIR}, the free-free absorption 
%in the ionized gas is the dominant source of opacity. Expansion, the recombination removes 
%free $e$, also decreases $\rho$, and thus $\kappa_{ff}$.
(i) free-free transitions, especially in lanthanides and actinides with their 
valence, partially filled $f$-shell, affecting photons of lowest energies, in \ac{FIR} band; 
%%%% --- bound-bound opacities
%For \ac{NIR}/optical photons, the bound-bound transitions are the main source of opacity. 
%Here the dependency on the ejecta composition strongly affects the \textit{effective} 
%continuum opacity. If most complex atoms in the ejecta belong to the iron group, with 
%valence $d$-shell electrons, then the opacity is moderate. However, if elements with 
%valence, partially filled $f$-shell, (lanthinides \& actinides), are present, then the 
%opacity increases by up to two orders of magnitude
%\citep{Kasen:2013xka,Tanaka:2013ana,Fontes:2015,Fontes:2017zfb}. 
%Bound-bound opacity rises with photon $\nu$ (as the number of lines).
%
%The \textit{plank mean expansion opacity} can be approximated as 
%$\kappa_r = 200 (T/4000K)^{5.5}$ cm$^2$g$^{-1}$ for $T\in(1-4)\times10^3$~K and just 
%$\kappa_r = 200$ cm$^2$g$^{-1}$ for $T\in(4-10)\times10^{3}$~k, motivated by 
%Figure~10 it \citet{Kasen:2013xka}. More accurate opacity estimations are plagued by the 
%complexity of the atomic structure. The quantum mechanics models of high-$Z$ atoms exists, 
%but has to be calibrated to the so far absent experimental data.
(ii) bound-bound transitions, that are the main source of opacities for 
\ac{NIR}/optical photons, 
%%%% --- Line opacity -> Effectve continoum opacity
%Additional complexity arises in approximation the line opacities to the effective opacity.
%One common way is to consider the line expansion opacity formalism \citep{Pinto:2000}, that is 
%based on the Sobolev approximation. This method was applied to \acp{kN} modeling by 
%\citet{Barnes:2013wka} and \citet{Tanaka:2013ana}. However, it is unreliable if line width is 
%large \ie, if line spacing of strong lines become comparable to the intrinsic thermal line width 
%\citep{Kasen:2013xka,Fontes:2015,Fontes:2017zfb}. 
(iii) certain optically thick lines, broadened by relativistic effects, 
%%%% --- clumping
%In addition, clumping that might occur when $T\leq10^3$~K might affect escaping 
%emission \citep{Takami:2014oqa}. The formation of the ``\rproc{} dust'' may have a complex 
%effect on optical/\ac{UV} opacity. The process of dust formation is complex and not 
%well understood in general \citep{Cherchneff:2009sj,Lazzati:2016}
(iv) ejecta clumping, 
%%%% --- Ejecta re-ionisation
%For higher energy photos, \ac{UV}/X-ray, bound-free transitions dominate the opacity. 
%For that ejecta ought to become mostly neutral, which occures naturally as it cools, unless 
%there is a source of ionizing radiation, \eg, the remnant. 
%See for details in \eg, \citet{Metzger:2013cha}. 
%Even though very large luminocities are required from the engine initially, 
%they decrease rapidly as ejecta expands. Thus, at late times it is possible that ejecta would be 
%re-ionized, especially in case of a long-lived remnant \citep{Metzger:2013cha}. The re-ionisation 
%can reduce the optical opacity, reducing the prominence of \ac{FIR} peak, 
%a generally regarded distingushed feature of a Kilonova.
and (v) ejecta re-ionisation by \eg, high energy photons from the central engine.
%%%% --- X-ray,. Gamma-rays, thermalization
%At even higher energies, hard X-ray photons, an important source of opacity is the 
%electron scattering with Klein-Nishina corrections. Notably, if the wavelength of photons 
%become smaller then the scale of an atom, the contribution from both, free and bound into 
%nuclei electrons ought to be considered. At high energies, the scattering of photons is inelastic. 
%However, these processes are important as ejecta opacity to very high energy photons, 
%$\gamma$-rays with energy in order of MeVs, determines the thermalisation of the \rproc{} 
%decay products.
%%
%%% pair-creation
%For extremely high photon energies $h\nu \gg m_e c^2$a pair creation becomes important. 
%In particular this is so if remnant is a magnetar with large spin-down luminosity. 
%Then, the at peak of the \ac{kN} emission, the pair creation might prevent pair-creation 
%photons from escaping the kilonova. 

%The calculation of opacities is complicated by the fact that the 
%Additional complexity arises in approximation the line opacities to the effective opacity.

%%%% --- compliecations in calculations 
An additional complication in opacity calculations arises from the fact that the ejecta 
is rapidly expanding, which implies that the matter ``sees'' incoming radiation as 
Doppler shifted. 
One common approach is to consider  %\textit{plank mean expansion opacity}, 
%\red{in MKN it is mean plank opacities} 
%that for 
%$\kappa_r = 200 (T/4000K)^{5.5}$ cm$^2$g$^{-1}$ for $T\in(1-4)\times10^3$~K and just 
%$\kappa_r = 200$ cm$^2$g$^{-1}$ for $T\in(4-10)\times10^{3}$~k, motivated by 
%Figure~10 it \citet{Kasen:2013xka}.
calculation of which, however, still requires complex atomic calculations.
%
The Sobolev approximation is commonly used in the line expansion opacity formalism 
\citep{Pinto:2000}.
This method was applied to \acp{kN} modeling by \citet{Barnes:2013wka} and \citet{Tanaka:2013ana}. 
However, it is unreliable if line width is large \ie, if line spacing of strong 
lines becomes comparable to the intrinsic thermal line width 
\citep{Kasen:2013xka,Fontes:2015,Fontes:2017zfb}. 


%% ----------------------------
%% \paragraph{\rproc{} heating}
%% ----------------------------

The main source of energy in expanding ejecta shell of mass, $dM_{\upsilon}$,
and velocity, $\upsilon$, that has a fraction of \rproc{} elements, $X_{r,\upsilon}$, 
is given by the specific heating, $\dot{e}_r(t)$, and can be approximated as 
%
\begin{equation}
\dot{Q}_{r,\upsilon} = dM_{\upsilon}X_{r,\upsilon}\dot{e}_{r}(t).
\end{equation}
%
The heating occurs through a combination of $\beta$- and $\alpha$-decays, 
and fission \citep{Metzger:2010,Barnes:2016umi,Hotokezaka:2017dbk}. 
The decay products then thermalize with an efficiency, $\varepsilon_{th,\upsilon}$, 
that depends on interactions between them and thermal plasma. 
Neutrinos can freely escape the ejecta. Very high energy photos, gamma rays, 
are also free after about ${\sim}1$~day as the Klein-Nishina opacity decreases 
\citep{Hotokezaka:2017dbk,Barnes:2016umi}.
%
The $\alpha$ and $\beta$ particles however interact efficiently with the matter via 
ionization \citep{Barnes:2016umi} and Coulumb scattering \citep{Metzger:2010}.
%\red{Here it repeats the Barns et al findings on thermalization efficiency}
For a fixed energy $\alpha$-particles thermalize more efficiently, then 
$\beta$-particles. For charged particles, the process depends on the magnetic 
field strength and configuration \citep{Barnes:2016umi}. 
%
Additionally, if actinides are produced in \rproc{}, their decay, 
involving $\alpha$-particles which thermalize with high efficiency. 
%\red{double check with before, Barnes}. 
Nuclear phsyics input that determines the amount of actinides, 
have a strong effect on the thermalization efficiency of \rproc{} elements in the ejecta
\citep{Barnes:2020nfi}.
%
%For a neutron-rich ejecta, $Y_e\leq0.2$, the heating rate is dominated by a large statistical 
%ensemble of nuclei, and the following can be assumed \citep{Korobkin:2012uy},
%%
%\begin{equation}
%\dot{e}_r = 4\times 10^{18} \varepsilon_{th,\upsilon}(0.5 - \pi^{-1} \arctan[(t-t_0)/\sigma])^{1.3} \text{ erg } \text{s}^{-1} \text{g}^{-1}
%\label{eq:kilonova:heat_korob}
%\end{equation}
%%
%Here $t_0=1.3$~s, $\sigma=0.11$~s constants. 
%The $\varepsilon_{th,m}$ is the thermalisation efficiency.
%In general, both $\varepsilon_{th,\upsilon}$ and 
%%
%The heating rate prescribed by this equation has first a constant segment, $\propto1$~s, 
%(depletion of free neutrons by $r$-process) and a decrease segment $\propto t^{-1.3}$, 
%(when heavy nuclei decay to stability) \citep{Metzger:2010,Roberts:2011}. 
%Notably, at higher $Y_e$, the nuclear heating is dominated by specific nuclei and has a complex form.
%%
%It was shown however, that on a timescale relevant for \ac{kN}, and $Y_e$ present in \ac{BNS} 
%($Y_e\leq0.4$), the heating rate can be assumed constant within the accuracy of a few.
%(see \eg, Fig.~7 in \citet{Lippuner:2015gwa}).
%%
%The dependency of $\dot{Q}_{r.\upsilon}$ on the nuclear physics models was shown to be week, 
%unlike the $r$-process abundances themselves \citep{Eichler:2014kma,Wu:2016pnw,Mumpower:2015ova}.
%See however \citet{Barnes:2020nfi} for a more recent assessment.
%
%Regarding the thermalization efficiency of the energy released, \citep{Barnes:2016umi} provides 
%a recipe that sets the $\varepsilon_{th,\upsilon}$ decreasing from $\sim0.5$ at around 
%$1$~day to $\sim0.1$ at around $1$~week. 
%
%\begin{equation}
%\varepsilon_{th,\upsilon}(t) = 0.36 \Bigg[ \exp(-a_{\upsilon}, t_{day}) + \frac{\ln(1+2b_{\upsilon} 
%    t_{day}^{d_{\upsilon}})}{2b_{\upsilon}t_{day}^{d_{\upsilon}}} \Bigg]
%\end{equation}
%
%where $t_{day}$ is the time in days, ${a_{\upsilon}, b_{\upsilon}, d_{\upsilon}}$ are the constants that 
%depend on the ejecta layer properties, mass and velocity. 

%%%% -------------------------------
%\subsection{\ac{kN} properties}
%%%% -------------------------------
%
%From a simple, toy model \citep{Metzger:2016pju}, the ejecta properties can be translated 
%into the properties of light curves. 
%%
%Consider the case where the lanthanides-rich ejecta is present, \eg, in tidal outflows of 
%\ac{BNS} and \ac{NSBH} mergers. There a toy model predicts a light curve from such low-$Y_e$ 
%ejecta that peaks in \ac{NIR} (in relatively good agreement with radiation transport model of 
%\citet{Barnes:2016umi}) on a timescale of several days (week). This is so-called ``red \ac{kN}''.
%The disagreement with radiation transfer models most noticeable in the post-peak period, 
%where toy model predicts sharp decay, while the radiation transport models predict smooth decline. 
%The reason for it is the toy model's assumption of optically thick black-body emission. 
%As ejecta expands cools and become optically thin, this assumption breaks down.
%It is important to note, that the toy model does not take into account other emergent sources 
%of opacity at late times, such as clumping, dust formation, photo-ionization from central engine. 
%These may smooth the post-peak light curves.
%
%Next, consider the ejecta with high electron fraction, \eg, \nwind, or \ac{SWW}, reprocessed by neutrinos.
%Such ejecta would have negligible amount of elements of lanthanides group \citep{Metzger:2014ila} 
%and thus have a different \ac{EM} signature. The emission from such ejecta peaks in optical/\ac{IR} 
%bands on a time scale of days. It is $2-3$ magnitudes brighter then red \ac{kN}. This component is called ``blue \ac{kN}''
%This emission is assumed to be of polar origin and contribute to the total \ac{EM} signature of the BNS ejecta. 


%\subsection{Other \ac{EM} counterparts}
%\red{commented}
%\subsubsection{Free Neutron Precursor}
%
%The sufficiently high density and low veloicyt of the bulk of the ejecta assures that there is enough time for the $r$
%-process to remove free neutrons. However, a small fraction of the ejecta was shown to have hight enough velocity to retain its free neutrins and escape the dens slow part, \textit{e.g.,} \cite{(Bauswein et al 2013a)}. The origin of this component is the shock-heated intefrace between two neutron stars as they collide. 
%The outer layers of the ejecta then can be \red{superheated} by this \red{'neutron skin'}, modifying the Kiloniva signal. \cite{(Metzger et al 2015a; Lippuner and Roberts 2015)}
%
%Here we consider such ejecta, that contains free neutrons. 
%Consider layer $dM_{\upsilon}$, that contain a fraction $X_{n;\upsilon}$ of free neutons, specific heating rate of which $\dot{e}_n(t)$. Then, the heating rate in the layer reads 
%
%\begin{equation}
%\dot{Q}_{r;\upsilon} = dM_{\upsilon} X_{n,\upsilon}\dot{e}_n(t).
%\end{equation}
%
%The initial mass fraction of neutrons $X_{n,\upsilon}$ is defined as 
%
%\begin{equation}
%X_{n,\upsilon} = \frac{2}{\pi}(1 - Y_e)\arctan\Big(\frac{M_{n}}{M_{\upsilon}}\Big),
%\end{equation}
%
%is an arbitrary assumed interpolation between the neutron rich ($M\ll M_{n}$) inner layers with $X_{n} = 1-2Y_e$ and neutron-free ($M\gg M_n$) layers.
%
%Assuming the averabe neutron half-0ife of $900$~s, the specific heating rate $\dot{e}_{n}$ is 
%
%\begin{equation}
%\dot{e}_n = 3.2 \times 10^{14} \exp[-t/\tau_n] \text{ erg } \text{s}^{-1} \text{g}^{-1},
%\end{equation}
%
%Simultaneously, as fraction of free neutrons increases in outermost layrs, the fraction of $r$-process elements decreases as $X_{r,\upsilon} = 1 - X_{n.\upsilon}$, which has to be accounted for in Eq.~\eqref{eq:theory:mkn:energ}.
%
%The effect of free neutrons on Kilonova lightcurves is the following.
%Even for a very small mass, $\sim 10^{-4}M_{\odot}$ of freen-nutron ejecta, with $Y_e\sim0.1$, the UVR luminocities are increased considerably, in the first hours after merger.
%The reason for that is, $\dot{e}_{n} > \dot{e}_r$ by at loeast an order of magnitude on a tiemscales up to 1 hour postmerger. Also, this timescale is comaprable with the diffusion time scale for the neutron mass layer. 
%
%\begin{equation}
%t_{peak,\upsilon} \approx \Bigg(\frac{M_{\upsilon}^{4/3}\kappa_{\upsilon}}{4\pi M^{1/3}\upsilon_0 c}\Bigg)^{1/2} \sim 3.7 \text{ hours}.
%\end{equation}
%
%And 
%
%\begin{equation}
%L_{peak} \approx \frac{E_n \tau_n}{t_{peak;\upsilon}^2} \propto 3\times10^{41}
%\end{equation}
%
%which is insensitive to the mass of the layer itself. 
%This emission is expected to peak in optical/UV band due to the high ejecta temperature during the first hours after merger. 
%
%%%
%
%\subsection{Engine Power}
%
%Additional heating for kilonova might come from the object, left after the merger. In case of BNS it might be the MNS. In case of NSBH it is a BH with accretion disk. This is expected to make Kilonova more luminous than what $r$-porcess products decay might produce.
%
%A large fraction of Short GRB, $\sim(15-25)\%$ is followed by the prolonged ($10-100$~s) hump of $X$-ray emission, \cite{Norris and Bonnell 2006; Perley et al 2009 Kagawa et al 2015}
%
%It is however uncertain, how much energy does the central engine provides. Here some examples are considered.
%
%\subsubsection{Fall-Back Accretion}
%
%A merger leaves a finite amount of mass bound gravitationally to the central object, that falls back of a time timescale of seconds to days \cite{(Rosswog 2007; Rossi and Begelman 2009; Chawla et al 2010; Kyutoku et al 2015)}.
%
%\gray{This is a part of the outflow that was not energetic enough to leave the system. It eventually falls back on the central object. This is not the disk itself...}
%
%The rate of fall-back at late timnes $t\gg 1$~s can be approximated by a power law 
%
%\begin{equation}
%\dot{M}_{fb} \approx \Bigg( \frac{\dot{M}_{fb}(t=0.1~\text{s})}{10^{-3}M_{\odot}\text{s}^{-1}} \Bigg) \Bigg( \frac{t}{0.1 \text{s}} \Bigg)^{-5/3}
%\end{equation}
%
%The value $10^{-3}M_{\odot}\text{s}^{-1}$ is the normalization chosen for BNS. For NSBH it be different by an order of magnitude \cite{(Rosswog 2007)}. 
%
%Notably, the fall-back of the material removed on a dynamical timescales, can be stalled by the continous winds from the disk \cite{(Fernandez et al 2015b)}
%
%Additionally an onset of $r$-process heating in the disk might provide an additional source of outflow that would stall the fall-back on a seconds to minutes timescale \cite{Metzger et al 2010a)}.
%
%On a longer timescale, days to weeks, there seems to be no mechansm that can suppress the fall back completely. This it might still be a relevant source of energy for Kilonva.
%
%The matter that reaches the central objects accrets. This is super-Eddington accetion that releases energy, $L_{acc} \propto \dot{M}_{fb} c^2$ that can heat the ejecta and enhance  the Kilonova emission. Additionally, the accretion might result in the formation of a relativistc jet (similar to GRB) that might account for the extended $X$-ray emission that sometiems follow the GRB.
%As accretion flow subsides, the jet power decreases and it becomes unstable to the magnetic Kink instability \cite{(Bromberg and Tchekhovskoy 2016)}. Then the energy is dissipated pramarely via heating up the ejecta, by magnetic reconnections instead of non-thermal emission. 
%
%The fall-back acctretion can power a mildly relativist, wide-angle disk wind. As the wind collides with the (ejected prior) ejecta shells, its energy thermalizes. 
%
%Overall, the ejecta heating rate due to fall-back accretion can be described as 
%
%\begin{equation}
%\dot{Q}_{fb} = \varepsilon_{j}\dot{M}_{fb}c^2
%\end{equation}
%
%where $\varepsilon_{j}$ is the jet/disk wind efficiency factor. See \cite{Tchekhovskoy et al 2011). Kisaka and Ioka (2015)} for the discussion of efficiency.
%
%For instace the 130603B, was detected with an \ac{NIR} excess. It was initially attributed t othe radiactive heating \cite{Tanvir et al (2013)} \cite{Berger et al (2013)}. On the contrary, \cite{Kisaka et al (2016)} suggested that it might be attributed to the absorbed and re-emitted (reprocessed) $X$-ray emission. 
%
%%%
%
%\subsubsection{Magnetar Remantns}
%
%The outcome of the NSBH merger is always a black hole. Meanwhile an outcome of the BNS merger depends sensetively on the maximum allowed mass for a non-rotating NS ($M_{max}(\Omega=0)$). 
%This value is bounded, \textit{e.g.,} $\geq 2M_{\odot}$ \cite{(Demorest et al 2010; Antoniadis et al 2013)} and $< 3M_{\odot}$, where the upper limit is given by the casuality constrants on the EOS. 
%Withing this boundaries the fate of the remnant is uncertain. Incidently, the observations shows that NS has mass $\sim 1.4M_{\odot}$. Merger of two of this objects thus result in a remnant of mass $\sim2.5M_{\odot}$ ($\approx7.5\%$ of the mass was lost to GW and neutrinos \cite{(Timmes et al 1996)}). If the resulting mass is lower then $M_{max}(\Omega=0)$, it promptly collapses. Otherwise a stable (short- or long-lived) remnant can be formed.
%
%Consider a rotating remnant. An upper limit on a rotating object, is the object that is rotating close to the mass-shedding limit. 
%
%Given the remnant's moment of inertia $I$ and \red{angular velocity} $\Omega$, then rotational period $P = 2\pi / \Omega$. 
%
%Such object has energy 
%
%\begin{equation}
%E_{rot} = \frac{1}{2}I\Omega^2 \approx 10^{53} \Big(\frac{I}{I_{LS}}\Big)\Big(\frac{M_{ns}}{2.3M_{\odot}}\Big)^{3/2}\Big(\frac{P}{0.7\text{ms}}\Big)^{-2} \text{ ergs }
%\end{equation}
%
%Here the remnatn's moment of inertial, $I$ is normmalized to $I_{LS} \approx 1.3\times 10^{45}(M_{NS}/1.4M_{\odot})^{3/2}$ g cm$^{2}$ (motivated by Fig.1 \cite{Lattimer and Schutz (2005}).
%
%This energy exceeds by a factor of $10^3$ the ejecta kinetic energy of radioactive decay energy.
%If this energy can be extracted via channels other then GW (\textit{e.g.,} EM torques), then the EM signal accompanying the merger would be significantly enhanced, \cite{(Gao et al 2013; Metzger and Piro 2014; Gao et al 2015; Siegel and Ciolfi 2016a)}. For a remnant that is supported by the differential roatation, only a part of the ritational energy is availalbe (as the rmenant would eventually collapse loosing it). The Fig.8 shows the dependency of the \textit{extractable} rotational energy as a function of the remnants mass.
%
%The electromagnetic tourques allows to extract the totatinal energy from a remnant with strong magnetic fields. Such fies are expected for the merging NS, due to amplifications, reaching values found in galactic magnetars \cite{(Price and Rosswog 2006; Zrake and Mac-Fadyen 2013; Kiuchi et al 2014)}. However, this amplification occures at small scats and at early times post merger, producing a complex field topology, that evolves with time \cite{(Siegel et al 2014)}. The magnetic field strength at the end of the differential rotation phase is however uncertain. There are speculations that it might remain at $10^{15-16}$~G.
%
%Consider an aligned dipole rotator (different from a vacuum the vacuum dipole). Its spin-down luminocity is \cite{Spitkovsky (2006); Philippov et al (2015)} 
%
%\begin{equation}
%L_{sd} = 
%\begin{cases}
%\frac{\mu^2 \Omega^4}{c^3} = \frac{(B R_{ns}^3)^2 \Omega^4}{c^3} &\text{ if } t< t_{coll} \\
%0 &\text{ if } t> t_{coll}
%\end{cases}
%\end{equation}
%
%The charactersitic 'spin-down timescale' over which an order of unity fraction of the rotational energy is removed is 
%
%\begin{equation}
%t_{sd} = \frac{E_{rot}}{L_{sd}}\Bigg|_{t=0}
%\end{equation}
%
%which is of an order of $\sim 150$~ms for a remnant of the mass $M=2.3M_{\odot}$, $I=I_{LS}$, $B=10^{15}$~G and $P_0=0.7$~ms, 
%
%where $P_0$ is the initial spin-period.
%The mass-shedding limit of this remnant is $P=0.7$~ms. 
%
%The lifetime of the unstable remanant can be estimated as 
%
%\begin{equation}
%L_{extract} = \int_0^{t_{coll}} L_{sd} dt
%\end{equation}
%
%where $t_{coll}$ is the time of the collpase, that marks olse the end of the extraction of the rotational energy. $L_{extract}$ is the total amount of energy extracted from rotation.
%The $t_{coll}$ falls rapidly with the remnant mass, after it passes the stale NS upper limit.
%
%Long-lived magnetar can power the 'prompt-like' X-ray emission (found in sGRB \textit{e.g.,} \cite{(Gao and Fan 2006)Metzger et al (2008b); Bucciantini et al (2012)} ). Additionally, the sGRB with extended emission were explaiend by phenomenological models of magnetar spind-down \cite{(Gompertz et al 2013)}. The observed X-ray and optical plateus were discusssed in \cite{(Rowlinson et al 2010, 2013; Gompertz et al 2015)} and the late-time excess emission was adressed in \cite{(Fan et al 2013; Fong et al 2014a)}. Notably, all models requrie rather large magnetic fields of $\sim 10^{16}$~G.
%
%The formation of the jet and sGRB is subjected to uncertainties. 
%It was argues that magnetar model is not viable due to heavy baryonic pollution in the polar region above the surface \cite{(Murguia-Berthier et al 2014, 2016).}. This led to the develpment of the model, in which GRB is generated after the remnant collapse to a BH, which might happen minutes after the merger. And while this still allow to explain the extended X-ray emission (magnetar spin-down and radiation diffussion through the ejecta). 
%However, if in spin-down the remnant raches a solid-body rotation, a collapse of such a remnatn is not predicted to leave a massive disk, sufficient to power the GRB \cite{Margalit et al (2015)}. The disk that was formed after the merger is expected to be either accreted or spread out (via ) too much for short GRB to be generated. 
%
%There are observational evidences that BH is not mandatory for producing a jet. For instance, the (e.g., Circinus X-1; \cite{Fender et al 2004}), galactic acretring NS.
%While indeed the region above the neutron star is polluted by neutrino-driven wind on a time scale of seconds postmerger \cite{(Dessart et al 2009; Murguia-Berthier et al 2014, 2016)}, the expected strong magnetic field $B\gg 10^{15}$~G, small scale magnetic flux bundles (that dominate dynamically over the thermal or ram pressure of the wind) could confine the plasma \cite{(Thompson 2003)}. Then, originating from the disk open field lines, carrying the Poyntim flux of the GRB jet, would be relatively free of baryonic matter due to centrifugal barrier. 
%Note, that sheer rotation of the NS would result in periodicity in openning of the polar field lines. This, in turn, might lead to a variability in the transinet (without requiring baryon pollution at all). 
%The presence of the NS \textit{after} the GRB is suported by observations: extended X-ray emission that does not follow the model of the fall-back accreting BH. 
%\gray{the early X-ray varaiablility is sometimes attributed to the afterglow phase, \cite{(Holcomb et al 2014)}, but it is too rapid for a foward or reverse shocks}
%
%Magnetic spind-down power, injected into the merger ejecta (behind it) could enhance the Kilonova emission (\cite{Yu et al (2013)}). Similar mechanism was considered for the SLSN \cite{(Kasen and Bildsten 2010; Woosley 2010; Metzger et al 2014)}. This in essence, reminds one of a fall-black powered emission considered before.
%
%A pulsar injects a relativistc wind of $e^{\pm}$ pairs into the surrounding environment (\textit{e.g.,} Crab Nebula). Near the termination shock, wind undegoes the shock dissipation, forming the so-called 'magnetar wind nebulae' of relativistic particles \cite{(Kennel and Coroniti 1984)}. The high density of the BNS merger environment assures a rapid cooling of these pairs (via synchrotron or inverse-compton emission) \cite{(Metzger et al 2014; Siegel and Ciolfi2016a,b)} generating the broadband emission (akin the emission from pulsar wind nebulae \textit{e.g.,} \cite{Gaensler and Slane 2006)}). The inner walls of the expanding ejecta would absorb, UV and X-ray photons, reprocess and emit in optical/IR \cite{(Metzger et al 2014)} contributing and enahncing Kilonova.
%
%Notably, the magnetar wind-nebulae emission does not necessarly undergoes thermalization within the ejecta. If the spectral windows allow, \textit{e.g.,} for instance for hard X-ray
%\footnote{where the bound-free transitions lie at lower energies. Additionally this is possible for hight enerngy $\gg$ MeV photons, that fall into the gap between declining Klein-Nishina cross-section and before the rise of $\gamma-\gamma$ opacities}
%, the emission will escape the ejecta without being reprocessed.
%Additionally, low-mass ejecta can undergo complete ionsiation and allow even lowere energies photones to pass without thermalization. This 'leaking radiation' might be an important EM signal to mergers \cite{(Metzger and Piro 2014; Siegel and Ciolfi 2016a,b; Wang et al 2016).}. 
%
%consider the ejecta heating rate provided by the magnetar spin-down as 
%
%\begin{equation}
%\dot{Q}_{sd} = \varepsilon_{th}L_{sd}
%\end{equation}
%
%where $\varepsilon_{th}$ is the thermalization efficiency, that ranges between $1$ when the ejecta is very opaque (hearly times) to a low value, for low opacities.
%
%Notably, there is another sink for spin-down radiation that is of it utmost importance at early times, where high energy $\gamma$-rays are present in the nebula behind the ejecta \cite{Metzger and Piro (2014)}. These $\gamma$-rays create $e^{\pm}$ pairs (when compactness of the cloud is high). Coming in 'seed photons' can then be compton up-scattered on these particles, becoming energetic enought to produce a new $e^{\pm}$ pair. This initializes a 'pair cascade'. High fraction ($\leq 0.1$) of puslar spin-down power $L_{sd}$ falls into the rest pass of the $e^{\pm}$ pairs \cite{(Svensson 1987; Lightman et al 1987)}. Hence, for the spind-down radaition to reach (and theramlize within) the ejecta, it must diffuse through the 'pair cloud', experiencing $PdV$ adiabatic losses.
%To paramterize the effect, introduce the Thompson optical depth of the pair cloud $\tau_{es}^n$. If This optical depth exceeds the optical depth of the ejecta itslef, then only a fraction of the actual magnetar spin-down power can be thermalzied within the ejecta. 
%This effect of 'pair cloud' can be approximated by suppressing the observed luminosity. Floowing \cite{Metzger and Piro (2014) and Kasen et al (2015),} 
%
%\begin{equation}
%L_{obs} = \frac{L}{1 + (t_{life}/t)}
%\end{equation}
%
%where $L$ is the Kilonova luminocity, computed from the energy equation \eqref{eq:theory:mkn:energ} (with magnetar heat source).
%The $t_{life}/t$ is the caracteristic 'lifetime' of a non-thermal photon in the nebula, relative to the ejecta expansion timescale, written as
%
%\begin{equation}
%\frac{t_{life}}{t} = \frac{\tau_{es}^{n} \upsilon}{c(1 - A)}
%\end{equation}
%
%where $\tau_{es}^n\propto Y L_{sd}$ and $A$ is the frequency averaged albedo of the ejecta ($A\propto 0.5$).
%
%Overall, the pair trapping is able to reduce the effective luminocity of the magnetar powered kilonova by several orders of magnitude (due to reduce thermalization efficenty) at early times.
%
%Energy input from the magnetar spind down, can in itself raise the observed peak luminocities. Note however, that in case of the only temporarly stable remnant, the energy import would be terminated at collapse.
%
%%%
%
%\subsection{Implications}
%
%sGRB is a good smoking gun for Kilonova searches.
%However, it, and its afterglow should not outshine the Kilonova. For instance, in GRB 130603B \cite{(Berger et al 2013; Tanvir et al 2013)} the observed \ac{NIR} excess would require ejecta of $0.05-0.1M_{\odot}$ to be explaiend. This is generally too high for dynamical ejecta only \cite{(Hotokezaka et al 2013b; Tanaka et al 2014; Kawaguchi et al 2016).}, but might be achieved with winds from the disk and remnant \cite{(Metzger and Fernandez 2014)} see also \cite{Kasen et al 2015)}. However, high observed high luminocity might not be a result of radioactive heating alone, but hits towards the contribution from the central engine, fall-back accretion or spin-down luminocity.
%
%Discussion on how different properties of the Kilonova affect detection possibilities and different biasas might araise.
%
%\red{This might serve as a gread introduction to the thesis!}



%%%% -------------
%% AT2017gfo
%%%% -------------

%The final composition of the ejecta determines its optical opacity, that vary by orders of 
%magnitude if $3$rd peak elements, lanthanides $(58\leq Z \leq71)$ and actinides 
%$(90\leq Z \leq 103)$, are present due to their open $f$-shell and hence a plethora of 
%absorption lines \citep{Tanaka:2013ana,Kasen:2013xka}.
%
The properties and geometry of different \ac{BNS} merger ejecta 
that have various optical opacities and heating rates, determining the \ac{kN}, 
emission \citep{Metzger:2019zeh}.
%
Generalizing, two main \ac{kN} components can be distinguished: 
``blue'' and ``red'' depending on whether the fraction of 
lanthanides and actinides is low or high.
%
The former corresponds to the high $Y_e$ material that produces emission 
that peaks in \ac{UV}/optical bands on a timescale of hours-days, 
while the latter is related to low-$Y_e$ material that generates the 
emission peaking on a significantly 
longer timescale, tens of days, in \ac{IR} and \ac{NIR} bands
\citep{Barnes:2013wka,Grossman:2013lqa,Lippuner:2015gwa}.

Both ``blue \ac{kN}'' and ``red \ac{kN}'' were observed for \GW{}, confirming the general 
picture and implying a diverse composition of the ejected material
\citep{Arcavi:2017xiz,Coulter:2017wya,Drout:2017ijr,Evans:2017mmy,Hallinan:2017woc,
    Kasliwal:2017ngb,Nicholl:2017ahq,Smartt:2017fuw,Soares-santos:2017lru,Tanvir:2017pws,
    Troja:2017nqp,Mooley:2018dlz,Ruan:2017bha,Lyman:2018qjg}.

%%%% <<< Also mentioned in the End of Ejecta paragraph >>>
%However, estimated mass of the ejecta required to explain 
%the red component is larger then what is predicted by numerical relativity simulations \red{refs}. 
%%It is believed that the most contribution to this component comes from the low $Y_e$, slow but 
%%massive outflow from the degenrate disk on a secular timescale \red{refs}.
%%
%Semi-analytic two-components (red and blue) spherical \ac{kN} models to the \AT{} observations 
%provided estimates for the ejecta properties for these two components. 
%%
%Specifically, for the langhinide poor (rich) \ie, blue (red) components, the required mass is 
%$2.5\times10^{-2}M_{\odot}$ ($5.0\times10^{-2}M_{\odot}$) and velocity $0.27$c ($0.15$c)
%\citep{Cowperthwaite:2017dyu,Villar:2017wcc}. 
%See however \citep{Waxman:2017sqv} for an alternative interpretation.
%See also \citep{Siegel:2019mlp} for the compiled data on the \ac{kN} models.
%%
%Similar estimates are obtained with $1$D radiation transport \ac{kN} models
%\citep{Tanvir:2017pws,Tanaka:2017qxj}.
%
%A very high energy emission from the non-thermalized radiation is weak and can be 
%detected only for a sufficiently close event \cite{Hotokezaka:2015cma}.
% 
Notably, prior to \AT{}, there were other candidates based on the detection of \ac{SGRB}, 
with infrared excess \eg, 
GRB130603B, \citep{Berger:2013wna,Tanvir:2013pia}, 
GRB060614 \citep{Jin:2015txa,Yang:2015pha}, 
GRB050709 \citep{Jin:2016pnm}.
%GRB200522A \citep{WRONG}  Bruni:2021ilp
However the exact nature of the observed signals were not well constrained. 
%
%The search for \ac{EM} counterparts to mergers continues, involving observatories around the world 
%\citep{Law:2009,Singer:2014qca,Bellm:2014,Kasliwal:2016uhu}.





%% =============================================================
%%
%% G R B 
%%
%% =============================================================

\subsection{Gamma-ray burst and kilonova afterglows}\label{sec:intro:afterglow}

\acp{GRB} are irregular pulses of gamma-ray radiation with broken power-law 
(non-thermal) spectrum, peaking at KeV-MeV \citep{Band:1993,Kouveliotou:1993,Meegan:1992xg}.
%
With respect to the duration, \acp{GRB} are split into two categories: \acp{SGRB}, 
that last ${\leq}2\,$s and long \acp{GRB} that last ${\gtrsim}2\,$s. 
It is generally accepted that the latter are the 
result of the collapse of massive ${\geq}15\,M_{\odot}$ stars, while the former, at 
least in part, are attributed to mergers of compact objects. 
%
The detection of \ac{SGRB} \GRB{} that accompanied 
the \ac{GW} event \GW{} \citep{TheLIGOScientific:2017qsa} confirmed that. 
However, the exact physical origin of different duration \acp{GRB} is not fully understood.
%
%%%% LGRB
%Indications that long \ac{GRB} are associated with core-collapse supernovae, \acp{SN}, 
%are two fold. These \acp{GRB} are typically observed in star-forming regions of their 
%host galaxies \citep[\eg][]{Bloom:2000pq,Bloom:2002hc,Fruchter:2006py,Christensen:2004yx,CastroCeron:2006jh} 
%and several \acp{GRB} are spectroscopically associated with Type Ic \acp{SN}, albeit 
%these \acp{GRB} were significantly less bright and might not be typical \acp{GRB} 
%\citep[\eg][]{Liang:2006ci,Bromberg:2011fm}. Additionally, the late time behaviour 
%of some \acp{GRB} includes a \acp{SN}-like "bump" in the optical and spectral changes 
%that might imply that underlying \acp{SN} flux becomes dominant over \acp{GRB} 
%\citep[\eg][]{Bloom:1999,Woosley:2006fn}.

%%%% Cosmology implications
The \acp{GRB} are distant events, most of which were localized to outside the local 
group \citep[\eg][]{Mao:1992,Piran:1992,Fenimore:1993}. 
%
Particularly useful for distance estimations are the observations of \ac{GRB} afterglow
(fading X-ray, optical and radio emission), 
that allow to estimate the Redshift \citep[\eg][]{Costa:1997cg,Frontera:1997ae}.
%
%%%% Mechanism of radiaiton
Analysis of the multi-wavelength afterglow data for \acp{GRB} \citep[\eg][]{Panaitescu:2001bx} suggested that the mechanism behind the afterglow 
emission is the synchrotron radiation from the external forward-shock, which forms 
when \ac{GRB}-ejecta sweeps-up the \ac{ISM} medium\footnote{
    The specific indications are the power law decay of the light curves, 
    $F_{\nu}\propto \nu^{-1}$ and power-law spectrum $F_{\nu}\propto\nu^{-0.9\pm 0.5}$.
} 
\citep{Rees:1992ek,Paczynski:1993gz,Meszaros:1993ju,Meszaros:1996sv}.
%

%%%% Jet break
%The temporal behavior of many (but not all) \acp{GRB} shows a change, a steepening 
%of the light-curve (to $F_{\nu}\propto t^{-2.2}$) at $\sim 1$~day after the burst. 
%This is usually attributed to the 
%%\gray{deceleration of the colimated GRB-outflow, jet, and decrease on the realtivisitc beaming. This in turn makes the edge of the jet visible to an observer.} 
%finite angular extend of the \ac{GRB}-ejecta, jet \citep[\eg][]{Rhoads:1999wm,Sari:1999mr}. 
%When jet decelerates and relativistic beaming decreases (and the jet edge becomes visible), 
%the optical and X-ray lightcurves decay achromatically faster. This achromatic transition 
%from slow to faster decay is called "jet-break".
%
%%% PROBLEMO -- jet-break is not a universal feature.
%Notably, this jet-break is not observed in all \acp{GRB} for the reason that is not fully 
%understood \citep[\eg][]{Fan:2006pj,Panaitescu:2006,Liang:2007ti,Sato:2006jg,Liang:2007rn,Curran:2007cp,Racusin:2008bx}
%%
%%% PROBLEMO -- GRB density seems uniform, but SSE models predict wind-like profile
%Models of the broadband emission of \acp{GRB} with jet-break showed that the 
%\ac{CBM}, is uniform with number density \red{$\sim 10^{-3}$} \citep{Panaitescu:2001bx}. 
%If \acp{GRB} produced in collapse of massive stars \citep{Woosley:1993,Paczynski:1997yg}, 
%this contradicts the expected density profile from stellar winds, \eg, $\rho\propto r^{-2}$ 
%\citep[\eg][]{Dai:1998iz,Chevalier:1999jy,Chevalier:1999mi,Ramirez-Ruiz:2001} 
%\red{this might be very outdated.}
%
%% sGRB
The origin of \acp{SGRB} was first linked to elliptical galaxies, with dominant 
older stellar population 
\citep[\eg][]{Gehrels:2005qk,Fox:2005kv,Barthelmy:2005bx,Berger:2005dr,Panaitescu:2005er,Bloom:2005qx,Guetta:2005bb,Nakar:2007yr}, 
and thus with \ac{BNS} mergers. A more direct evidence came with the \GRB{}
\citep{Savchenko:2017ffs,Alexander:2017aly,Troja:2017nqp,Monitor:2017mdv,Nynka:2018vup,Hajela:2019mjy}, 
detected by the space observatories Fermi \citep{TheFermi-LAT:2015kwa} and INTEGRAL \citep{Winkler:2011}.
%
%Generally \acp{SGRB} show a complex time behavior of early afterglow X-ray emission, 
%in particular a presence of a plateau ($F_{x}\propto t^{-1/2}$), after the initial sharp 
%decrease ($F_{x}\propto t^{-3}$) which a standard forward shock model does not predict. 
%This implies that early X-ray afterglow is shaped by a variety of physical processes 
%\citep{Zhang:2005fa}.
%

%%%% EARLY emission problem
%Two main questions stem from these observations: is the mechanism behind the prompt 
%$\gamma$-ray emission and early afterglow emission is the same (or do they originate from 
%the same outflow), and is the early X-ray radiation produced by the external shock (just a 
%blast wave takes long time to become self-similar) or does it originate from an internal 
%shock? An indication that the long-lived central engine activity might affect the 
%afterglow came from the observed sharp increase in X-ray flux (flares) on a 
%scale of minutes to hours after the end of the \ac{GRB}
%\citep{Burrows:2005ww,Chincarini:2007fp,Chincarini:2010,Margutti:2011}, 
%which could not be attributed to the inhomogeneities in the \ac{CBM}.
%%% PROBLEMO!
%Thus, the early X-ray behaviour of \acp{GRB} $t < 10^{4}$~s post-burst is not well 
%understood and seems to be in tension with standard afterglow forward shock emission model.

%\textcolor{red}{
%    One of the foremost unanswered questions about GRBs is the physical mechanism
%    by which prompt $\gamma$-rays the radiation that triggers detectors on board
%    GRB satellites are produced. Is the mechanism the popular internal shock
%    model 6 \cite{(Rees and Meszaros, 1994)}, the external shock model, or something
%    entirely different? Are $\gamma$-ray photons generated via the synchrotron process
%    or inverse-Compton process, or by a different mechanism? Answers to these
%    questions will help us address some of the most important unsolved problems
%    in GRBs  how is the explosion powered in these bursts? Does the relativistic
%    jet produced in these explosions consist of ordinary baryonic matter, electron positron
%    pairs, or is the energy primarily in magnetic fields?
%}

%Once again, while it is suggested that the high energy emission, after the propmt phase is produced 
%by the synchrotron process in the external forward shock, \citep{Kumar:2009,Ghisellini:2010}, 
%the mechanism behind the high and low energy $\gamma$-ray emission in the prompt phase remains unknown. 
%Possible mechanisms include: inverse Compton and synchrotron emission in internal and external shocks
%\citep[\eg][]{Rees:1992ek,Dermer:1998py,Lyutikov:2003ih,Zhang:2011} and 
%photospheric radiation with contribution from multiple \ac{IC} scatterings
%\citep[\eg][]{Thompson:1994zh,Ghisellini:1998jy,Meszaros:1999gb,Peer:2005qoc,Peer:2008udu,Giannios:2006jb,Ioka:2007qk,Asano:2009gi,Lazzati:2010,Beloborodov:2010,Toma:2011}.

\GRB{} was dimmer then any other event of its class. Different interpretations for its 
dimness and slow rising flux were proposed including off-axis jet, cocoon or structured jet. 
Now it is commonly accepted that \GRB{} was a structured jet\footnote{
    Structured jet, contrary to the top-hat jet has the angular dependency of the 
    matter energy and Lorentz factor. Such structure is believed to appear 
    when jet drills its way through the merger ejecta \citep{Lamb:2017ych}.
} observed off-axis 
\citep[\eg][]{Fong:2017ekk,Troja:2017nqp,Margutti:2018xqd,Lamb:2017ych,Lamb:2018ohw,Ryan:2019fhz,Alexander:2018dcl,Mooley:2018dlz,Ghirlanda:2018uyx}.
%
The \GRB{} late emission, the afterglow, provided information on the energetics of the 
event and on the properties of the \ac{ISM} \citep[\eg][]{Hajela:2019mjy}. 

%% ---------------
%% Kilonova Afterglow
%% ---------------


In addition to the \ac{GRB} beamed emission, more isotropic non-thermal,
emission is expected from electrons accelerated in shocks formed between the 
(mildly) relativistic ejecta and the \ac{ISM} \citep{Nakar:2011cw}. This emission 
is expected to peak in radio band and continue on a time scale of tens of years 
after merger. Notably, all ejecta components contribute to the emission, 
but depending on the ejecta velocities and kinetic energy, 
the brightness in different frequencies and timescales varies. 
%
Various ejecta components interact with each other and with \ac{ISM}. The latter 
generates a long-lived blast wave. The shock, propagating upstream, amplifies 
(random) magnetic fields and accelerates electrons, that subsequently emit 
synchrotron radiation. This process in phenomenologically similar 
to the one responsible for the \ac{GRB} afterglow and \ac{SN} early remnants. 
%
Numerical simulations of \ac{BNS} mergers showed the presence of mildly relativistic 
ejecta, %(see Sec.~\ref{sec:bns_sims:method:ejecta}). 
and several studies on the possible non-thermal \ac{EM} emission of this ejecta 
have been carried out 
\citep[\eg][]{Piran:2012wd,Hotokezaka:2015eja,Hotokezaka:2018gmo,Radice:2018pdn}. 
%Notably, the observed non-thermal emission from \GW{}, was first interpreted 
%as the non-thermal emission from the ejecta \citep{Mooley:2017enz}.
%This interpretation was however disproved by the emergence of jet break
%
%
%Specifically, a strong radio emission is expected from \ac{BNS} ejecta \citep{Piran:2012wd,Hotokezaka:2015eja}. 
%The \ac{BNS} merger radio remnant is expected to peak on a time-scale 
%of ${\sim}$years after the merger.% and be visible over a similar timescale. Notably, this is 
%Assuming that the ejecta is expanding into the unshocked \ac{ISM}, as it was shown 
%that if the \ac{ISM} is pre-shocked by the jetted outflow, and the \ac{ISM} density is 
%reduced, the kilonova afterglow would be delayed. 
%
%In \citet{Piran:2012wd}, the synthetic \ac{kN} afterglow \acp{LC} are calculated 
%for a set of \ac{NR} \ac{BNS} merger simulation with properties typical to the 
%Galactic binary population and \ac{ISM} density usually found in the Galactic disk, 
%$\nism{\sim1}\ccm$. Authors showed that the from a binary of two $1.4\,\Msun$ \acp{NS}, 
%the kilonova afterglow would peak $\sim10$~years after the merger if $\nism = 0.1\,\ccm$ 
%and $\sim3$~years if $\nism = 1\,\ccm$ in radio bands, $\nu=1.4$~GGz and $\nu=150\,$MGz. 
%Notably, both values of the \ac{ISM} density are larger than what is inferred for \GW{}. 
%Indeed, jet fitting models and dependent analysis of the diffuse emission suggest 
%$\nism\in(10^{-4},10^{-2})\,$\gcm \citep{Hajela:2019mjy}.
%In \citet{Piran:2012wd} authors focus on the observational prospects of this afterglow 
%and compare it to other \ac{EM} emission expected for \ac{BNS}. 





%%%% -------------------------------------------------------
%\section{Afterglow theory}  A F T E R G L O W  T H E O R Y 
%%%% -------------------------------------------------------

%Consider a moving source of radiation and an observer with a line of sight to the source.
%Let $\upsilon$, $\Gamma$ and $\theta$ be the source velocity, 
%\ac{LF} and angle with the line of sight.
%%
%Consider three frames of reference, the comoving frame (usually denoted with a prime $'$),
%the lab frame, where the source is seen as moving with $\upsilon$ and observer frame.
%Then, if two photons are emitted in the comoving frame with time difference of $\delta t'$,
%which is in the lab frame $\delta t = \Gamma \delta t'$, the observer sees the two 
%photons arrive with 
%%
%%\begin{eqnarray}
%%\delta t_{obs} &= \delta t + \frac{(d - \upsilon\cos(\theta) \delta t)}{c} - \frac{d}{c} \\
%%&= \delta t (1 - \upsilon \cos(\theta) / c) \\
%%&= \delta t' \Gamma (1 - \upsilon \cos(\theta) / c)\\
%%&= \delta t' \mathcal{D}^{-1}
%%\end{eqnarray}
%%
%$\delta t_{obs} = \delta t' \mathcal{D}^{-1}$
%%
%where $d$ is the distance to the source, and 
%%
%\begin{equation}
%\mathcal{D} = \frac{1}{\Gamma(1 - (\upsilon/c) \cos(\theta))} = \frac{1}{\Gamma(1 - \beta\cos(\theta))}
%\label{eq:afterglow:dop_fac}
%\end{equation}
%%
%is the \ac{DF}. 
%
%
%Next, we consider the transformation of the photon frequencies. 
%%
%The Lorentz transformation of the photon $4$-momentum in comoving frame, \eg,, 
%$\nu'(1, \cos(\theta'), \sin(\theta'),0)$ to the lab frame $4$-momentum 
%$\nu(1, \cos(\theta), \sin(\theta), 0)$
%%
%%\begin{equation}
%%\nu = \nu' \Gamma(1+\upsilon \cos(\theta')/c) \text{ \& } \nu\cos(\theta) = \nu' \Gamma (\cos(\theta') + \upsilon/c)
%%\end{equation}
%%%
%%or 
%%
%\begin{equation}
%\nu = \frac{\nu'}{\Gamma (1 - \upsilon\cos(\theta)/c)} = \nu' / \mathcal{D}
%\label{eq:afterglow:dop_fac_freq}
%\end{equation}

%The robust way to model the \ac{GRB}/\ac{kN} afterglows is to perform 
%multidynamisional radiation \ac{HD} or \ac{MHD} studies. These, however, 
%are very numerically expensive, as the timescales for the problem range 
%from days to tens of years.
%
When computing the afterglow emission from the structured relativistic 
(mildly relativistic) source, several key components are required: 
(i) special relativistic effects (\eg, beaming and Doppler shift),
(ii) emission mechanism (\eg, \ac{IC}, synchrotron, cyclotron), 
and (iii) the dynamical evolution of the emitting matter over the 
relevant timescale. 

When talking about special relativistic effects, it is convenient 
to introduce the frame of reference comoving with the fluid (usually denoted with prime, $'$) 
and observer frame. 
For an element of the fluid moving with velocity, $\upsilon$, and \ac{LF}, 
$\Gamma$, at angle, $\theta$, from the line of sight, the Lorentz transformation 
reads 
%
\begin{equation}
\delta t_{obs} = \frac{\delta t'}{\mathcal{D}}, \hspace{3mm}
\mathcal{D} = \frac{1}{\Gamma(1 - \beta\cos(\theta))}, \hspace{3mm}
\nu = \frac{\nu'}{\Gamma (1 - \upsilon\cos(\theta)/c)} = \nu' / \mathcal{D}\, ,
\label{eq:afterglow:dop_fac}
\end{equation}
%
where $\delta t_{obs}$ is the time interval between successively emitted photons, 
in observer frame, $\mathcal{D}$ is \ac{LF}, and $\nu=\nu'/\mathcal{D}$ is the 
classical Doppler shift formula for the frequency of the radiation.

%
%which is a standard Doppler shift formula.
%
%\subsubsection{Relativistic beaming of photons}
%
%%%% ---------------------------------------
%%%% --- On the Angular size of the source
%%%% ---------------------------------------
%
%We have shown that $\nu = \nu' \mathcal{D}$, but also 
%$\sin(\theta) = \sin(\theta')/\mathcal{D}$. Then the transverse component of the 
%momentum is invariant under the Lorentz transformation, \eg, 
%$\nu_{\perp}' = \nu'\sin(\theta') = \nu\sin(\theta) \nu_{\perp}$. 
%%
%For a beam of photons it implies that the angular size of the beam is smaller 
%in the lab frame than in the comoving frame by $\propto \Gamma$.
%%
%The solid angle of a conical beam of photons, $d\Gamma$ then 
%%
%\begin{equation}
%d\Gamma = \sin(\theta)d\theta d\phi = \sin(\theta') d\theta' d\phi' / \mathcal{D}^2 = d\Omega'/\mathcal{D}^2
%\end{equation}
%%
%is smaller in the lab frame than in the comoving frame.
%%
%Next, consider a frequency integrated total energy radiated per 
%unit time over the $4\pi$ steradians, denoted as $P$. 
%%
%The power in the lab frame $P = P'\Gamma\delta t'/(\Gamma\delta t') = P'$. 
%Hence, power radiated by particles is \magenta{Lorentz invariant}.

%%%% -------------------------------------------------------------
%%%% Transformation of specific luminosity and specific intensity
%%%% -------------------------------------------------------------
%
%Consider a spherically symmetric source, expanding with \ac{LF} $\Gamma$.
%%
%The \magenta{specific luminosity} is defined as the total energy that passes 
%through the surface enclosing the source per unit time, per unit frequency, 
%$L_{\nu} = dE / d\nu dt_{obs}$. 
%As $d\nu dt_{obs} = d\nu' dt'$ and $E=\Gamma E'$, the Lorentz transformation 
%of luminosity is
%%
%\begin{equation}
%L_{\nu} = \frac{dE}{d\nu dt_{obs}} = \Gamma \frac{dE'}{d\nu' dt'} = \Gamma L_{\nu}'
%\end{equation}
%%
%assuming that the $3$-momentum is zero (as the source is spherically symmetric).
%%
%The \magenta{specific intensity} is defined as a flux per unit frequency 
%and per unit solid angle, mediated by photons, traversing surface $dA$, 
%perpendicular to the conical beam, confining the photons, 
%%
%\begin{equation}
%I_{\nu} = \frac{dE}{d\nu dt_{obs} dA d\Omega}
%\end{equation}
%%
%that has a Lorentz transformation $I_{\nu} = \mathcal{D}^3 I_{\nu'}'$ 
%as $d\nu dt_{obs} dA$ is the Lorentz invariant.
%
%%%% ------------------------------------------------------------
%%%% Observed \ac{LC} from a source that is suddenly turned off
%%%% ------------------------------------------------------------
%
%
Thus, considering an extended emission source that is variable on a short 
timescale (a transient), the observed emission should be computed by integrating 
the emission from different parts of the source, accounting for their respective 
motion, angles and \acp{LF}. This is called the \ac{EATS} integration. For a spherical 
thin shell, described by $(r, \theta, \phi)$ coordinates, radiation emitted 
at $(r=\upsilon t, \theta,\phi)$ arrive at the observer with a time delay with 
respect to a photon emitted at $r=0$ of
%
\begin{equation}
t_{obs} = t - \frac{r \cos(\theta)}{c} = t\Big(1-\frac{\upsilon\cos(\theta)}{c}\Big) = \frac{t}{\Gamma\mathcal{D}}\, .
\label{eq:afterglow:tobs}
\end{equation}
%
Then, the total emission at a given, doppler shifted frequency, from the entire 
shell at the observer frame is obtained by integrated over all elements with the 
same $t_{obs}$.

%
%Now, consider the observed emission from the source at frequency $\nu$. 
%The starting time is $t_{0;obs}\approx(R_0 2 c \Gamma^2)$, \red{check!} 
%at which photons, emitted from $(R_0, 0, 0)$ arrive, At later times, 
%$t_{obs}>t_{0;obs}$, the observer still sees photons emitted when $r < R_0$. 
%%
%Assume that the intrinsic emission spectrum is $I_{\nu'}' = I'\nu^{'-\beta}$.
%Then, at $t_{obs} > t_{0;obs}$ the radiation from $\theta > \theta_t$ 
%(where $\theta_t$ corresponds to $t_{obs} = R_0 (1/\upsilon - \cos(\theta_t)/c)$) 
%reaches the observer.
%%
%The observed flux \eg, $f_{\nu} \propto \int I_{\nu} d\Omega$, has the following 
%Lorentz transformation 
%$f_{\nu}\propto\int_{\theta_t} d\theta \sin(\theta_t) \mathcal{D}^{-(3+\beta)}$.

%%%% ---------------------------
%%%% A More Regorous Derivation
%%%% ---------------------------
%Now, consider a more rigorous derivation of the transformation of the specific 
%flux in observer frame from relativistic source with comoving specific intensity 
%$I_{\nu'}'$ and spectrum $\propto \nu^{' -\beta}$
%%
%\begin{equation}
%f_{\nu}(t_{obs}) = \int d\Omega_{obs} I_{\nu} \cos(\theta_{obs}) = 2\pi \int d\theta_{obs} \frac{ I_{\nu'_0}' \nu_{0}^{'\beta}\sin(2\theta_{obs})[(1+z)\Gamma]^{-(3+\beta)} }{ 2\nu^{\beta} [ 1-\upsilon\cos(\theta + \theta_{obs}) / c ]^{3+\beta} }
%\end{equation}
%%
%where $\nu_0 '$ is a frequency that lies on the power law segment of the spectrum for 
%$I_{\nu'}'$. The Lorentz transformation of the specific intensity was made above. 
%The factor $(1+z)^{3+\beta}$ accounts for the Redshift on the frequency. 
%%
%Assuming that $\sin(\theta)/d_{A} = \sin(\theta_{obs})/r$, the above integral writes 
%%
%\begin{equation}
%f_{\nu} \approx \frac{ 2\pi I' \nu' _0 \nu_{0}^{'\beta}\nu^{-\beta} }{[(1+z)\Gamma]^{3+\beta}} \Big( \frac{R_0}{d_A} \Big)^2 \int_{\theta_t}^{\pi / 2} d\theta \frac{\sin(\theta)\cos(\theta)}{(1-\upsilon\cos(\theta)/c)^{3+\beta}},
%\end{equation}
%where $\theta+\theta_{obs}$ in the denominator was replaced with $\theta$ as $\theta_{obs}\ll\theta$.
%The integral is simple to compute. Ir yields
%
%\begin{equation}
%f_{\nu}(t_{obs}) \propto (1 - \upsilon\cos(\theta_t)/c)^{-(2 + \beta)}\nu^{-\beta} \propto t_{obs}^{-(2+\beta)} \nu^{-\beta},
%\end{equation}
%%
%This equation shows, that the observed radiation does not immediately turns off 
%when the source switches off. The flux falls off rapidly with time and vanishes 
%when $\theta_t$ exceeds the angular size of the source $(\theta_j)$.
%
%
%The \ac{EATS} integration can be performed as follows.
%The expanding thin shell is composed of infinitesimal elements,
%each of which is evolved within its own $(d\phi, d\theta)$
%cell, center of which has coordinates $(\phi_c, \theta_c)$.
%%
%Then for each cell, the observational angle, $\mu$, is computed as 
%%
%\begin{equation}
%    \cos(\mu) = \sin(\alpha) \sin(\theta_c) \sin(\phi_c) 
%     + \cos(\alpha) \cos(\theta_c).
%\end{equation}
%%
%Then the radiation at time $t$ is observed if $t_{obs}$
%%
%\begin{equation}
%    t_{obs} = t_{lab} +\frac{r}{c}(1 - \cos{\mu})
%\end{equation}
%%
%where 
%%
%\begin{equation}
%    t_{lab} = \int \frac{1}{\beta c} dr
%\end{equation}
%%
%is the time measured in the laboratory frame of reference.
%%
%Then we interpolate the part of the dynamical evolution
%of the infinitesimal jet that corresponds to this $t_{obs}$,
%and emitted the radiation during this time.
%The flux, $F_i$, for a given Doppler-shifted frequency, 
%$\nu_{obs}'$, is then obtained for each infinitesimal segment.
%Integrating over the $F_i$ we obtain the total flux emitted by the 
%blast wave and observed at time $t_{obs}$


%%%% ------------------------
%% \subsection{Synchrotron radiation}
%%%% -----------------------

%Consider an electron moving in the magnetic field, perpendicular to the 
%field lines. Let $\gamma_e$, $\upsilon_e$ be the electron's \ac{LF} and 
%velocity and $B$ the magnetic field strength.
%%
%The electric field in the electron rest-frame is $E=\gamma_e \upsilon_e B /c$. 
%The electron acceleration in this field yields radiation, total power of which, 
%according to the Larmor's formula, 
%%
%\begin{equation}
%P_{syn} = \frac{2q^4E^2}{3c^3m_e}=\frac{2q^4B^2\gamma_e^2\upsilon_e^2}{3c^5m_e^2} = \frac{\sigma_TB^2\gamma_e^2\upsilon_e^2}{4\pi c}
%\end{equation}
%%
%where $\sigma_T = 8\pi q^4 / (3m_e^2c^4)$ is the Thompson cross section. 
%%
%The $P_{syn}$ is the Lorentz invariant 
%(as electric dipole radiation is Lorentz invariant).
%%
%Note, that for an isotropic pitch angle distribution, 
%the average power $\langle P_{syn} \rangle = (2/3)P_{syn}$.
%%
%The angular speed of the electron (\eg its Larmor frequency), is
%%
%\begin{equation}
%\omega_L = \frac{q B}{\gamma_e m_e c}
%\end{equation}
%%
%Within the magnetic field, an electron is moving on a spiral trajectory. 
%The relativistic beaming of emitted radiation leads to a distant observer 
%being able to see this radiation, only when the electron velocity vector is 
%within $\angle \sim \gamma_e^{-1}$ from the line of sight. Correspondingly, 
%only a fraction of orbital time, $t\sim1/(\pi\gamma_e)$, contributes to the 
%observed radiation, which appears as a repeated pulse. 
%The duration of this pulse is
%%
%\begin{equation}
%\delta t_{obs} \sim \frac{2}{\gamma_e \omega_L}\frac{1}{2\gamma_e^2}\sim \frac{m_e c}{q B \gamma_e^2}
%\end{equation}
%%
%where we used $\delta t' = \delta t / \gamma_e$. 
%%
%Then the characteristic frequency of the synchtrontron radiation is given 
%by an inverse of $\delta t$ and reads 
%%
%\begin{equation}
%\omega_{syn} \sim \frac{q B \gamma_e^2}{m_e c} \text{ and } \nu_{syn} = \frac{\omega_{syn}}{2\pi} \sim \frac{q B \gamma_e^2}{2\pi m_e c}
%\end{equation}
%%
%where $\nu_{syn}$ is the cyclic frequency.
%%
%Note that here the factor $(3/2)\sin(\alpha)$, where $\alpha$ is the pitch 
%angle between the electron's velocity and the magnetic field is \red{ommited}.
%%
%The synchrotron spectrum peaks at $\sim \nu_{syn}$. At $\nu < \nu_{syn}$ the
%$P_{syn}(\nu)\propto\nu^{1/3}$ (which is determined by the Fourier transform of 
%the synchrotron pulse profile). At $\nu > \nu_{syn}$ the power decays exponentially. 
%See \citet{RybickiLightman:1985} for the calculation of synchrotron spectrum. 
%%
%The power per unit frequency at the peak of the spectrum is given 
%%
%\begin{equation}
%P_{syn}(\nu_{syn}) \sim \frac{P_{syn}}{\nu_{syn}} \sim \frac{\sigma_T B m_e c^2}{2 q},
%\end{equation}
%
%%%% ------------------
%%%% Shortened
%%%% ------------------
%
The power of the synchrotron radiation, $P_{syn}$, emitted by an electron moving with 
the speed, $\upsilon_e$, corresponding to the \ac{LF}, $\gamma_e$, in the magnetic field, $B$, perpendicular 
to the field lines is given by the Larmor's formula.
%
As within the magnetic field, an electron is following the spiral trajectory,
the characteristic frequency of the synchtrontron radiation, $\nu_{syn}$, is given 
by the angular speed of the electron (\eg, its Larmor frequency).
%
The power per unit frequency at the peak, $P_{syn}(\nu_{syn})$, can be computed as 
\citep{RybickiLightman:1985}
%
\begin{equation}
P_{syn} = \frac{\sigma_TB^2\gamma_e^2\upsilon_e^2}{4\pi c}, 
\hspace{5mm} 
\nu_{syn} \sim \frac{q B \gamma_e^2}{2\pi m_e c},
\hspace{5mm}
P_{syn}(\nu_{syn}) \sim \frac{\sigma_T B m_e c^2}{2 q},
\end{equation}
%
where $\sigma_T = 8\pi q^4 / (3m_e^2c^4)$ is the Thompson cross section.

%Now consider the distribution of electrons.
%Commonly adopted is the power-law distribution, $dn_e/d\gamma_e \propto \gamma_e^{-p}$, which results in emission spectrum $f_{\nu}\propto\nu^{-(p-1)/2}$,
%which is a consequence of 
%%
%\begin{equation}
%f_{\nu} = \int_{\gamma_{\nu}}^{\infty} d\gamma_e \frac{dn_e}{d\gamma_e}P_{syn}(\nu) \propto \nu^{-(p-1)/2}
%\end{equation}
%%
%as $P_{syn}(\nu) \propto (\nu/\nu_{syn})^{1/3}$ for $\nu < \nu_{syn}$\red{where is this from?}.
%%
%Here
%%
%\begin{equation}
%\gamma_{\nu} \sim \Bigg(\frac{2\pi\nu m_e c}{qB}\Bigg)^{1/2}
%\end{equation}
%%
%is the minimum \ac{LF}, above which electrons contribute to the specific flux, 
%$f_{\nu}$ \red{why?} \gray{This seems to be an equation for $\nu = f(\gamma)$ 
%    inverted, -- so is the $\nu$ a critical frequency?
%}
%
%%%% ------------------
%%%% Shortened
%%%% ------------------
%
The synchrotron radiation spectrum, emitted by an ensemble of electrons that have a 
distribution function, $dn_e/d\gamma_e$ 
%(with $\gamma_e$ being the electron \ac{LF}), 
is given by convolving the distribution function with the power spectrum of a single 
electron, $P_{syn}(\nu)$, as 
%
\begin{equation}
f_{\nu} = \int_{\gamma_{\nu}}^{\infty} d\gamma_e \frac{dn_e}{d\gamma_e}P_{syn}(\nu), 
%\propto \nu^{-(p-1)/2}
\label{eq:afterglow:sync_power}
\end{equation}
%
where $\gamma_{\nu}$ is the minimum \ac{LF} above which electrons contribute to the 
specific flux.
%
%%%% ------------------------------------------------------
%%%% Effect of synchrotron cooling on electron distribution
%%%% ------------------------------------------------------
%
%Consider the effects of electrons cooling. 
%The characteristic frequency associated with it is $\nu_c$ and $\gamma_c$.
%Electrons with \ac{LF} $\gamma_e > \gamma_c$ can efficiently loose their energy to 
%synchrotron radiation. Then, after the time $t_0$, their $\gamma_e$ drops below 
%$\gamma_c$, 
%%
%\begin{equation}
%c^2 \frac{dm_e}{dt} \gamma_e = -\frac{\sigma_T}{6\pi} B^2 \gamma_e^2 c
%\end{equation}
%%
%\begin{equation}
%\gamma_c \sim \frac{6 \pi m_e c}{\sigma_T B^2 t_0}
%\end{equation}
%%
%The corresponding characteristic frequency is called the synchrotron cooling 
%frequency. 
%%
%\begin{equation}
%\nu_c = \frac{3}{4\pi} \gamma_c^2 \frac{q B}{m_e c}
%\end{equation}
%%
%At $\nu_c$ the spectrum of the synchrotron radiation is changing, as electrons 
%with $\gamma_e > \gamma_c$, the effects of cooling modify the electron distribution. 
%%
%%
%Consider the continuity equation for electrons in the energy space 
%%
%\begin{equation}
%\frac{\partial }{\partial t}\frac{d n_e}{d\gamma_e} + \frac{\partial}{\partial \gamma_e}\Big[ \dot{\gamma_e}\frac{dn_e}{d\gamma_e} \Big] = S(\gamma_e)
%\end{equation}
%%
%where $\dot{\gamma_e} = -\sigma_T B^2 \gamma_e^2 / (6\pi m_e c)$ is the rate at 
%which electron \ac{LF} changes due to losses, $S(\gamma_e)$ is the injection 
%rate of electrons into the system.
%%
%Assume that the minimum \ac{LF} of injected electrons is $\gamma_m$, \eg, 
%where $S(\gamma_e) = 0$ for $\gamma_e < \gamma_m$.
%Then if $\gamma_c < \gamma_e < \gamma_m$ the solution to the equation 
%$dn_{e}/d\gamma_e \propto \dot{\gamma_e}^{-1} \propto \gamma_e^{-2}$.
%%
%Then, for this electron distibution the synchrotron spectrum is 
%$f_{\nu}\propto\nu^{-1/2}$ 
%\footnote{If $B=f(t)$, then the distribution function for $\gamma_e$ 
%    evolves with time and is not a simple pwoer law with index $2$, see \citet{Uhm:2013gwa}.
%}
%%
%For $\gamma_e > \gamma_c > \gamma_m$, the solution to the equation is 
%$dn_e/d\gamma_e \propto \gamma_e^{-p-1}$ (assuming the constant $B$ field, 
%the steady state). Then the synchrotron spectrum reads $f_{\nu}\propto\nu^{-p/2}$.
%
%%%% ------------------
%%%% Shortened
%%%% ------------------
%
%Depending on their \ac{LF} and the magnetic field strength, electrons can loose 
%energy, cool, with different efficiency. For instance, if $\gamma_e > \gamma_c$,
%where $\gamma_c$ is a certain characteristic \ac{LF}, after a finite $t_0$, the 
%$\gamma_e$ will drops below $\gamma_c$, 
%%
%\begin{equation}
%c^2 \frac{dm_e}{dt} \gamma_e = -\frac{\sigma_T}{6\pi} B^2 \gamma_e^2 c 
%\, \rightarrow\, 
%\gamma_c \sim \frac{6 \pi m_e c}{\sigma_T B^2 t_0}, \,
%\nu_c = \frac{3}{4\pi} \gamma_c^2 \frac{q B}{m_e c}
%\end{equation}
%%
%where $\nu_c$ is called cooling frequency. 
%%
%Thus, the synchrotorn spectrum differ for electrons with $\gamma_e > \gamma_c$.
%%
%Consider the continuity equation for electrons in the energy space 
%%
%\begin{equation}
%\frac{\partial }{\partial t}\frac{d n_e}{d\gamma_e} + \frac{\partial}{\partial \gamma_e}\Big[ \dot{\gamma_e}\frac{dn_e}{d\gamma_e} \Big] = S(\gamma_e)
%\end{equation}
%%
%where $\dot{\gamma_e} = -\sigma_T B^2 \gamma_e^2 / (6\pi m_e c)$ is the rate at 
%which electron \ac{LF} changes due to losses, $S(\gamma_e)$ is the injection 
%rate of electrons into the system.
%%
%Consider steady-state solution $\partial_t/\partial t = 0$.
%%
%Assume that the minimum \ac{LF} of injected electrons is $\gamma_m$, \eg, 
%where $S(\gamma_e) = 0$ for $\gamma_e < \gamma_m$.
%%Then if $\gamma_c < \gamma_e < \gamma_m$ the solution to the equation 
%%$dn_{e}/d\gamma_e \propto \dot{\gamma_e}^{-1} \propto \gamma_e^{-2}$.
%%%
%%For $\gamma_e > \gamma_c > \gamma_m$, the solution to the equation is 
%%$dn_e/d\gamma_e \propto \gamma_e^{-p-1}$ (assuming the constant $B$ field, 
%%the steady state).
%%
%Then the electron distribution function then reads
%%
%\begin{equation}
%dn_e/d\gamma_e \propto 
%\begin{cases}
%\gamma_e^{-2} &\text{ if } \gamma_c < \gamma_e < \gamma_m, \\
%\gamma_e^{-p-1} &\text{ if } \gamma_e > \gamma_c > \gamma_m
%\end{cases}
%\label{eq:afterglow:elec_dist}
%\end{equation}
%%
%with the former usually referred as \textit{slow cooling} and the latter
%\textit{fast cooling} regimes.
%
%
%%%% ------------------
%%%% Shortened x2
%%%% ------------------
The electron distribution function, $dn/d\gamma_e$ can be obtained from 
the continuity equation for electrons in the energy space 
%
\begin{equation}
\label{eq:intro:electron_dist_cont_eq}
\frac{\partial }{\partial t}\frac{d n_e}{d\gamma_e} + \frac{\partial}{\partial \gamma_e}\Big[ \dot{\gamma_e}\frac{dn_e}{d\gamma_e} \Big] = S(\gamma_e)\, ,
\end{equation}
%
where $\dot{\gamma_e} = -\sigma_T B^2 \gamma_e^2 / (6\pi m_e c)$ is the rate at 
which electron \ac{LF} changes due to energy losses, $S(\gamma_e)$ is the injection 
rate of electrons into the system.
%
Assuming that the injection of electrons is constant (steady-state solution,
$\partial_t = 0$), and has a minimum, $\gamma_m$, such that, 
$S(\gamma_e) = 0$ for $\gamma_e < \gamma_m$, the solutions to the Eq.~\eqref{eq:intro:electron_dist_cont_eq} reads 
%
\begin{equation}
\frac{dn_e}{d\gamma_e} \propto 
\begin{cases}
\gamma_e^{-2} &\text{ if } \gamma_c < \gamma_e < \gamma_m, \\
\gamma_e^{-p-1} &\text{ if } \gamma_e > \gamma_c > \gamma_m
\end{cases}
\label{eq:afterglow:elec_dist}
\end{equation}
%
where $\gamma_c$ is the cooling \ac{LF}, above which electrons loose their 
energy to radiation efficiently over a certain characteristic time.
%
The $\gamma_c < \gamma_e < \gamma_m$ regime is usually referred as \textit{slow cooling} and $\gamma_e > \gamma_c > \gamma_m$ as \textit{fast cooling}.
%
%%%% ------------------------------------------------------
%%%% Synchrotron self-absorption frequency
%%%% ------------------------------------------------------
%If the photon absorption by the inverse-synchrotron process is important, 
%another characteristic frequency, $\nu_a$, can be determined. Consider the 
%Kirchhoff's law, \red{stating that the emergent specific flux cannot exceed the 
%    black-body flux corresponding to the appropriate electron temperature} which is
%%
%\begin{equation}
%k_BT\approx \max(\gamma_a,\min[\gamma_m,\gamma_c])m_e c^2 / 2.7
%\end{equation}
%%
%where $\gamma_m$, $\gamma_c$ and $\gamma_a$ are electron Lorentz factors 
%corresponding to $\nu_m$, $\nu_c$ and $\nu_a$.
%Then the \red{synchrotron self-absorption frequency $\nu_a$ is the 
%    frequency where the emergent synchrotron flux is equal to the black body flux
%}
%%
%\begin{equation}
%\frac{2m_ec^2\max(\gamma_a,\min[\gamma_m,\gamma_c])\nu_a^2}{2.7c^2}\approx\frac{\sigma_T B m_e c^2 N_>}{4 \pi q}
%\end{equation}
%%
%where the \ac{LHS} is the Plank function in the Rayleigh-Jeans limit and 
%$N_{>}$ is the column density of electrons with Lorentz factor larger then 
%$\max(\gamma_a\min[\gamma_m,\gamma_c])$.
%%
%Finally, the order of characteristic frequencies determines the emergent 
%synchrotron spectrum for a distribution of electrons. 
%See Fig.X for fast and slow cooling regimes \citet{Sari:1997qe}.
%
%%%% ------------------
%%%% Shortened
%%%% ------------------
%
The photon absorption by the inverse-synchrotron process can be computed by evaluating 
the black-body flux corresponding to the appropriate electron temperature 
and equating it to the emergent synchrotron flux. Alternatively, it can be accounted for 
via flux attenuation \citep{Dermer:2009}.
%

%%%% ------------------
%% Maximum energy of synchrotron photons
%%%% ------------------
%
%\subsubsection{Maximum energy of synchrotron photons}
%%
%Consider a shock front. Scattering back and forth, particles within it accelerate 
%via the \magenta{first order Fermi process}, increasing their energy $\times 2$ times, 
%at every front of the shock.
%In order to determine what is the maximum energy a particle can reach consider the 
%following. A charged particle of mass $m$ accelerates while crossing the shock front 
%on a timescale $\sim$ Larmor time, $t'_L = mc\gamma/(qB')$, where primed quantities 
%are measured in the rest frame of the fluid and $\gamma$, \ac{LF} on a particle in 
%the frame, comoving with the shock. 
%%
%The particle can accelerate to $\gamma$ only if it losses less then half of its 
%energy to synchrotron emission in $t'_L$. Then 
%%
%\begin{equation}
%\frac{4 q^4 B^{'2}\gamma^2 t'_L}{9 m^2 c^3} < \frac{m c^2\gamma}{2} \text{ or } \nu\propto \frac{q B' \gamma^2}{2\pi m c} < \frac{9 m c^3}{16\pi q^2}
%\end{equation}
%%
%Thut, for electron the maximum synchrotron photon energy is $\sim 50$~MeV and for 
%proton it is $\sim 100$~GeV in the shocked fluid comoving frame. \red{assuming 
%    Bohm diffusion limit.}
%This limit can be exceeded in case of highly inhoogeneous magnetic field 
%\citep{Kumar:2012}.
%%
%%%%% ------------------
%%% Inverse-Compton radiation
%%%%% ------------------
%%
%\subsubsection{Inverse-Compton radiation}
%
%The \ac{IC} scattering is the scattering of photons by electrons of larger energy, resulting in increase in photon energy on average.
%%% ---
%Consider electrons with $\gamma_e$ and photons with frquency $\nu$. Let $h\nu\gamma_e \ll m_e c^2$. The average frequency of scattered photons then $\nu_s\sim\nu\gamma^2_e$.
%\gray{
%    This can be seen from considering the scattering in the rest frame of the electron.
%    Let the incident photon have frequency $\nu' \sim \nu\gamma_e$. (See eq.for Doppler Shift). If $h\nu'\ll m_e c^2$, the scattering is elastic (electron recoil is negligible) and the post-scattering angle distribution is a dipol function. 
%    Then, transforming the $\nu'$ into the original frame results in $\nu_s\sim\nu\gamma_e^2$.
%}
%
%%% Single electron, Radiation Field
%Consider a radiation field with photon density $u_{\gamma}$, and an electron moving through it. 
%Then, the power in \ac{IC}-scattered photons is (assuming $h\nu\gamma_e\ll m_e c^2$)
%
%\begin{equation}
%P_{ic} \sim \sigma_T \int d\nu \frac{u_{\nu} c}{h\nu} h\nu\gamma_e^2 \sim \sigma_T u_{\gamma}\gamma^2_e c;
%\end{equation}
%
%where $u_{\nu}d\nu$ is the energy density in photons of frequency between $\nu$ and $\nu+d\nu$, such $\int d\nu u_{\nu} = u_{\gamma}$. From $P_{sync}$ (see eq.above.somewhere) and this equation $P_{sync}/P_{IC} \sim u_{B}/u_{\gamma}$, where $u_{B}- B^2 / 8\pi$.
%
%Now consider, that the radiation field is generated by the synchrotron process, \ie, photons are produced by and scattered on the same electrons (to typically much higher energies). This process is called \magenta{synchrotron-self-Compton} or \ac{SSC}.
%The relative importance of \ac{IC} process is specified by the Compton paramter $Y$ for a population of energetic electrons. 
%Consider an energy density in photons for synchrotron process
%
%\begin{equation}
%u_{\gamma} = \int dr \int d\gamma_e \frac{P_{syn}}{c}\frac{dn_e}{d\gamma_e} = \frac{\sigma_T (\delta R) B^2}{6\pi} \int d\gamma_e \gamma^2_e \frac{d n_e}{d\gamma_e} = \frac{\sigma (\delta R) n_e B^2}{6\pi}\langle\gamma_e^2\rangle
%\end{equation}
%
%where $\delta R$ is the radial width of the source, and 
%
%\begin{equation}
%\langle \gamma_c^2\rangle = \frac{1}{n_e} \int d\gamma_e \gamma_c^2\frac{dn_e}{d\gamma_e}.
%\end{equation}
%
%Invoking the formula \red{which} for the $u_{\gamma}$ for synchrotron radiation, the Compton parameter reads 
%
%\begin{equation}
%Y \sim P_{IC} / P_{syn} \text{ where } \tau_e = \sigma_T (\delta R) n_e
%\end{equation}
%
%is the optical depth of the source to Thompson scattering.
%
%\paragraph{IC spectrum.}
%
%In order to obtain IC radition spectrum, the seed photon spectrum is to be convolved with electron distribution \citep{RybickiLightman:1985}
%
%\begin{equation}
%f_{IC}(\nu_{IC}) \approx \frac{3\sigma_T (\delta_R)}{4} \int d\nu \frac{\nu_{IC}}{\nu^2}f_{syn}(\nu) \int \frac{d\gamma_e}{\gamma_e^2}\frac{dn_e}{d\gamma_e}F\big( \nu_{IC} / 4 \gamma_c^2\nu \big)
%\end{equation}
%
%where 
%
%\begin{equation}
%F(x) \approx \frac{2}{3}(1-x), \text{ and } x = \frac{\nu_{IC}}{4\gamma_e^2\nu}
%\end{equation}
%
%To qualitatively asses the spectrum, assume that the seed photon spectrum is a $\delta$-function around frequency $\nu_0$. Electron distribution is power law with index $p$.
%%% ---
%Consider the low energy side, where the spectrum is cut off at $\gamma_m$, is proportional to $\nu_{IC}$ for $\nu_{IC} < 4\gamma_m^2\nu_0$. Then, if \ac{SSA} is neglected, then the \ac{IC} spectrum at low energies is much steeper than the hardest synchrotron spectrum $\propto\nu^{1/3}$.
%Now, consider the high energy side, $\nu_{IC} > 4 \gamma_m^2\nu_0$. There, the \ac{IC} spectrum approaches $\propto \nu_{IC}^{-(p-1)/2}$, same as the synchrotron process spectrum.
%
%\paragraph{IC in Klein-Nishina regime}
%
%The assumed non-elastic scattering of photons is only valid as long as photon energy is lower then $m_e c^2$ in the comoving frame. When this condition is not longer valid, the electron recoil in the scattering can no longer be ignored. Additioanlly, the cross-section becomes smaller then $\sigma_T$ (decreasing with rising photon energy). 
%The electron recoil also leads the the change in upper limit of the scattered photon energy, $\sim m_e c^2 \gamma_e / 2$. See \citet{RybickiLightman:1985} for equations.
%
%
%\subsubsection{Hadronc processes}
%\red{very brief}
%
%Under the hadronic processes one understands the followign processes.
%The photon-pion process, \ie, the production of pions ($\pi^0, \pi^+$ and $\pi^-$), the decay of $\pi^+$ produces $p^+$ with high lorentz factor that can cool via synchrotron processes.
%The Bethe-Heitler pair production process.
%Others...


%
%%%% ----------------------
%%   \subsection{Dynamics}
%%%% ----------------------
%
%Consider a dynamics of a relativistic blast wave propagating through a \ac{CBM}. 
%Such scenario is a universal part of the \ac{GRB} theory, that can be treated  independently. Assume that such "fireball" has a \ac{LF} $\Gamma_0$ and a total
%"isotropic equivalent" energy $E$. The \ac{CBM} has a density profile described 
%by $n(R) = (A/m_p)R^{-k}$.
%%
%The theory of relativist shocks with applications to \ac{AGN} jets was developed 
%by \citep{Blandford:1976}. Later, the theory was successfully applied to \ac{GRB} 
%afterglows \citep{Costa:1997cg,vanParadijs:1997wr,Frail:1997qf}. 
%%
%Importantly, the power law behaviour of the afterglow \acp{LC} is naturally 
%reproduces the self-similar nature of the self-similar blast wave solution
%%
%%\red{here the physical understanding is emphasized, not the derivation}
%%
%Consider the reference frame comoving with the shocked fluid. Then, $\Gamma$ 
%is the \ac{LF} of this fluid with respect to the unshocked one. The density of 
%the unshocked fluid in this reference frame is $\Gamma n$, and its particles 
%are seen as streaming towards the shocked fluid with \ac{LF} $\Gamma$. Generally, 
%for unshocked fluid it is assumed that the thermal energy of its particles is 
%much lower than the rest mass, or in other words, that the \ac{CBM} is cold. 
%%
%Passing through the cold \ac{CBM}, shock front randomizes the particle, protons, 
%velocity vectors, raising their thermal energy to $\Gamma^2 m_p c^2$ 
%(while their \ac{LF} remains unchanged). 
%In the \red{lab frame the average energy of each down stream proton is $\Gamma^2 m_p c^2$, from which it follows that at radius $R$, the total energy in the shocked plasma }
%%
%\begin{equation}
%E \approx \frac{4\pi A R^{3-k}c^2 \Gamma^2}{3-k}
%\end{equation}
%%
%where $AR^{-k}$ is the density of the \ac{CBM} at radius $R$ and 
%$4\pi A R^{3-k}/ (3-k)$ is the total swept up mass.
%%
%From this equation the dynamic of a blast wave can be inferred. For instance, 
%assuming $E=\text{const}$, $k=0$ (uniform density in \ac{CBM}), the blast wave 
%\ac{LF} $\Gamma\propto R^{-3/2}$.
%%
%The radius from the \ac{CoE} at which $\Gamma = 1/2 \Gamma_0$, initial value, 
%when also $1/2 E_0$ is deposited into \ac{CBM}, is called the 
%\textit{deceleration radius}, $R_d$. 
%For the uniform \ac{CBM}, it is $R_d\propto E^{1/3} n^{-1/3} R_{0,2}^{-2/3}$.
%%
%Additionally, shock compresses the plasma. For \red{highly relativistc shocks}, the compression is $4\Gamma$, giving the density in the comoving frame $4 \Gamma n$.
%It also accelerates the inbound particles to a power-law distribution function. 
%Additionally, it amplifies the magnetic fields. 
%%
%Essentially, this is all that is required for computing the afterglow emission.
%%
%%
%%
%\red{Now consider a slightly more rigorous derivation of $R_d$ and compression ratio and entropy generation by the blast wave.}
%%
%Now, consider a relativistic shock propagating into a cold upstream medium.
%The evolution of physical properties of the shock is governed three conservation laws: 
%baryon number, $n' \Gamma c$, energy and momentum fluxes across the shock front. 
%The latter two are a embedded into the fluid energy momentum tensor 
%%
%\begin{equation}
%    T^{\mu\nu} = (\rho' c^2 + p') u^{\mu} u^{\nu} + p' g^{\mu\nu},
%\end{equation}
%%
%where $\rho' c^2$ is the total energy density, $p'$ is the pressure 
%(in the plasma rest frame), $u^{\mu}$ is the $4$-velocity and $g^{\mu\nu}$ 
%is the metric tensor.
%%
%Through some magic the conservation equations mentioned above can be written as \citep{Blandford:1976,Rezzolla:2013} 
%%
%\begin{eqnarray}
%\frac{e_2'}{n_2'} = (\gamma_{21} - 1)m_p c^2 \\
%\frac{n_2'}{n_1'} = \frac{\hat{\gamma}\gamma_{21} + 1}{\hat{\gamma}-1} \\
%\gamma_{1s}^2 = \frac{(\gamma_{21} + 1) [\hat{\gamma}(\gamma_{21}-1)+1]^2}{\hat{\gamma}(2-\hat{\gamma})(\gamma_{21}-1)+2}
%\end{eqnarray}
%%
%\begin{figure*}[t]
%    \centering 
%    \includegraphics[width=0.45\textwidth]{Fig_8_KZ.pdf}
%    \caption{
%        This is a schematic sketch of a pair of shocks produced when a relativistic
%        jet from a \ac{GRB} collides with the \ac{CBM}, as viewed from the
%        rest frame of unshocked \ac{CBM}. Regions 2 \& 3 represent shocked \ac{CBM} and \ac{GRB}
%        jet respectively. They move together with the same \ac{LF} ($\gamma_2$, as viewed
%        by a stationary observer in the unshocked \ac{CBM}), and have the same pressure but
%        different densities.
%        (Adapted from \citet{Kumar:2014upa}, Fig.~8)
%    }
%    \label{fig:aafg:theory:sr8}
%\end{figure*}
%%
%Here, in Fig.~\ref{fig:aafg:theory:sr8} 
%\red{same as in Nava picture that you should put} the $2$ and $1$ subscripts 
%stand for downstream and upstream respectively, $e'$ is the internal energy density, 
%$n'$ is the proton number density, (both in the local fluid rest frame), 
%$\gamma_{21}$ is the relative \ac{LF} of plasma in region $2$ with respect to the 
%region $1$, $\gamma_{1s}$ is the relative \ac{LF} of plasma in region $1$ with 
%respect to the shock front and $\hat{\gamma}$ is the adiabatic index of the fluid.
%%
%For $\Gamma\gg1$ that usually describes early stage of the \ac{GRB} afterglow
%\citep{Piran:1999kx}, the $\hat{\gamma}=4/3$ 
%\red{recall that in subrelativistc it is $\hat{\gamma}=5/3$}
%%
%Then, $n_2'/n_1' = 3 ((4/3)\gamma_{21} + 1) = 4\gamma_{21} + 3 \approx 4\gamma_{21}$
%which implies that the downstream plasma is compressed with compression ratio
%$4\gamma_{21}$.
%%
%Similarly, approximating, the conservation equation 
%$\frac{e_2'}{n_2'} \approx \gamma_{21} m_p c^2$ and the last equation can be 
%simplified to $\gamma_{1s} \approx \sqrt{2} \gamma_{21}$, which implies that the 
%shock front travels faster then the downstream fluid.
%%
%Next, consider the self-similar deceleration phase of the blast wave in the 
%constant density \ac{CBM}. There, the energy conservation reads 
%%
%\begin{equation}
%E = \frac{4 \pi }{3 } R^3 n m_p c^2 \Gamma^2 = \text{ const}
%\end{equation}
%%
%where $\Gamma = \gamma_{21}$ is the \ac{LF} of the blastwave with respect to the 
%unshocked medium, $R$ is the radius of the blast wave (from \ac{CoE}). 
%%
%Note, that in the comoving frame the average proton thermal energy is 
%$m_p c^2 \Gamma$. In the lab frame it is $m_p c^2 \Gamma^2$. 
%Overall, we observe that $\Gamma^2 R^3 = \text{ const}$ or 
%%
%\begin{equation}
%\Gamma \propto R^{-3/2}
%\end{equation}
%%
%Now, consider the elapsed time in the observer frame. 
%As both the blast wave and emitted photons are moving in the same direction with the speed difference of $\sim 1/2 \Gamma^2$, 
%%
%\begin{equation}
%t_{obs} \sim \frac{R}{2\Gamma^2 c} \propto R^4 \propto \Gamma^{-8/3}
%\end{equation}
%%
%and 
%%
%\begin{equation}
%\Gamma \propto R^{-3/2} \propto t_{obs}^{-3/8}, \hspace{3mm} R\propto t_{obs}^{1/4}
%\end{equation}
%%
%
%%%______________________________________
%%% on the non-uniform CBM
%
%Next, we consider \red{power-law stratified density profile}, 
%
%\begin{equation}
%n = n_0 \Big(\frac{R}{R_0}\Big)^{-k}
%\end{equation}
%
%and obtain similar scaling relations.
%\red{I do not really need this. I can go directly to Peer model and Nava model}
%
%\begin{equation}
%E = \int n_0 \Big( \frac{R}{R_0} \Big)^{-k} m_p c^2 \Gamma^2  4\pi R^2 dR = \text{ const}
%\end{equation}
%
%where $R^{3-k} \Gamma^2 \text{ const}$. 
%
%After some derivation 
%
%\begin{equation}
%\Gamma\propto R^{(k-3)/2}\propto t_{obs}^{(k-3)/(8-2k)}
%\end{equation}
%
%And if $k=0$, the previously derived relation for constant density CBM follows.
%
%A particularly useful case is the \ac{CBM} filled with a free wind with constant mass loss rate $\dot{M}$ and wind speed $\upsilon_w$, that gives $\dot{M}=4\pi R^2 n \upsilon_w = \text{ const}$, or $n\propto R^{-k}$, \ie, the case of $k=2$ with $\Gamma\propto R^{-1/2}\propto t_{obs}^{-1/4}$.
%
%%%_______________________________________
%%% On the energy injection
%
%Now consider the case where the energy of the blast wave is continuously increasing. A possible physical scenario here is a long-lasting Poynting-flux dominated jet, feeding the fireball \gray{and suppressing the reverse shock}. 
%Then the energy of the outflow from the central engine has to be included into the energy equation of the blast wave
%
%\begin{equation}
%E_{tot} = E_0 + E_{inj}
%\end{equation}
%
%Consider a central engine with time dependent luminocity  $L(t) = L_0 (t_{obs}/t_0)^{-q}$. Then the energy equation reads
%
%\begin{equation}
%E_{tot} = E_0 + E_{inj} = E_0 + \int_{0}^{t_{obs}} L(t)dt = E_0 + \frac{L_0 t_0^q}{1-q}t_{obs}^{1-q}
%\end{equation}
%
%where $E_0$ is the initial energy of the blast wave and $E_{inj}$ is the injected energy into the blast wave from the central engine.
%
%Consider the case when energy injection increases with time noticeably, $q<1$.
%
%Then, when $E_{inj} \gg E_0$ for $q<1$, the blast wave scaling 
%
%\begin{equation}
%E_{tot} \sim E_{inj} \propto t_{obs}^{1-q}.
%\end{equation}
%
%and for the constant density \ac{CBM}, $\Gamma^2 R^3 \propto t_{obs}^{1-q}$ which eventually leads to $\Gamma\propto R^{-(2+q)/(4-2q)}\propto t_{obs}^{-(2+q)/8}$.
%
%And it is easy to see that if $q\rightarrow 1$, the 'no injection' case is resored.
%
%%%____________________________________
%%% Lorentz factor stratification of the ejecta as the Energy injection
%
%The energy can be added to the blast wave in a form of velocity stratified ejecta, when the wave decelerates, \eg,
%
%\begin{equation}
%E\propto \gamma^{1-s}\propto\Gamma^{1-s}
%\end{equation}
%
%where $\gamma$ is the \ac{LF} of the ejecta and $\Gamma$ is the \ac{LF} of the blast wave.
%Here the effects of the reverse shock can also be neglected as energy injection comes when $\Gamma\sim\gamma$ \red{[How does this work in the Peer/Nava model?]}
%
%This method is equivalent to the long-lived central engine with time dependent luminosity (at least for the dynamics of the blast wave), and the coefficient $s$ can be expressed in terms of $q$ \cite{Zhang:2005fa}. 
%
%For a uniform density \ac{CBM} the scaling relation reads
%
%\begin{equation}
%\Gamma\propto R^{-3/(1+s)}\propto t_{obs}^{-3/(7+s)}, \hspace{3mm} R\propto t_{obs}^{(1+s)/(7+s)}
%\end{equation}
%
%which then gives $s = (10-7q)/(2+q)$ and $q=(10-2s)/(7+s)$
%
%\red{The question is, can I add L(t) to the dE/dr of the Nava model and it is all?..}

A universal part of the afterglow theory is the dynamics of the \trans{} 
\blast{} propagating through the \ac{ISM}, that is also called "fireball".
%
The theory of relativist shocks with applications to \acp{AGN} jets was 
developed by \citet{Blandford:1976}. Later, the theory was successfully 
applied to \ac{GRB} afterglows \citep{Costa:1997cg,vanParadijs:1997wr,Frail:1997qf},
and \ac{kN} afterglows \citep[\eg][]{Nakar:2011cw,Hotokezaka:2015eja,Hotokezaka:2018gmo}.

%%
%Consider a \blast{} propagating though the cold \ac{CBM} with a power-law
%density profile, $n(R)$ with a \ac{LF} $\Gamma$.
%%
%The the total energy in the shocked plasma then is 
%$E \approx \frac{4\pi A R^{3-k}c^2 \Gamma^2}{3-k}$
%where $n(R) \propto AR^{-k}$ is the density of the \ac{CBM} at radius $R$ and 
%$4\pi A R^{3-k}/ (3-k)$ is the total swept up mass.
%%
%The conservation of the total energy governs the dynamics of the \blast{}. 
%The point at which the \ac{LF} and energy of the \blast{} decreases by half, and when 
%is referred to as \textit{deceleration radius}, $R_d$. \red{IS it?}.
%%
%Passing through the cold \ac{CBM}, shock front randomizes the velocity vectors,
%of particle, protons, raising their thermal energy, while their \ac{LF} remains
%unchanged. Additionally, shock compresses the plasma and accelerates the inbound 
%particles to a power-law distribution function, and amplifies the magnetic fields. 
%
Withing a shock front, particles are being scattered back and forth, and get accelerated 
via the first order Fermi process. This increases their energy ${\times}2$ 
times, at every front of the shock.
%
Given a relativistic shock propagating into a cold upstream medium, 
the evolution of physical properties of the shock is governed by three conservation
laws:  baryon number, $n' \Gamma c$, and energy and momentum fluxes across the shock front.
%
The latter two are a embedded into the fluid energy momentum tensor 
(see Sec.~\ref{sec:theory:grhd}, and Eq.~\ref{eq:theory:tmunu_perf} there).
%
These equations can be written as \citep{Blandford:1976,Rezzolla:2013} 
%
\begin{equation} % subequations
\frac{e_2'}{n_2'} = (\gamma_{21} - 1)m_p c^2, \hspace{3mm}
\frac{n_2'}{n_1'} = \frac{\hat{\gamma}\gamma_{21} + 1}{\hat{\gamma}-1}, \hspace{3mm}
\gamma_{1s}^2 = \frac{(\gamma_{21} + 1) [\hat{\gamma}(\gamma_{21}-1)+1]^2}{\hat{\gamma}(2-\hat{\gamma})(\gamma_{21}-1)+2}\, ,
\label{eq:afterglow:blast}
\end{equation} % subequations
%
\begin{figure*}[t]
    \centering 
    \includegraphics[width=0.45\textwidth]{Fig_8_KZ.pdf}
    \caption{
        This is a schematic sketch of a pair of shocks produced when a relativistic
        jet from a \ac{GRB} collides with the \ac{CBM}, as viewed from the
        rest frame of unshocked \ac{CBM}. Regions 2 \& 3 represent shocked \ac{CBM} and \ac{GRB}
        jet respectively. They move together with the same \ac{LF} ($\gamma_2$, as viewed
        by a stationary observer in the unshocked \ac{CBM}), and have the same pressure but
        different densities.
        (Adapted from \citet{Kumar:2014upa}, figure~8)
    }
    \label{fig:aafg:theory:sr8}
\end{figure*}
%
where subscripts $2$ and $1$ stand for downstream and upstream respectively, 
shown in Fig.~\ref{fig:aafg:theory:sr8}, 
$e'$ is the internal energy density, $n'$ is the proton number density, 
$\gamma_{21}$ is the relative \ac{LF} of plasma in region 
$2$ with respect to the region $1$
$\gamma_{1s}$ is the relative \ac{LF} of plasma in region $1$ with respect to the shock front,
$\hat{\gamma}$ is the adiabatic index of the fluid, which is $\hat{\gamma}=4/3$ for the ideal relativistic 
fluid and $\hat{\gamma}=5/3$ for subrelativisitc fluid.
%The $2$ and $1$ regions are also shown in Fig.~\ref{fig:aafg:theory:sr8} 
%(see also \citep{Nava:2013} for a more comprehensive take).
%
Solving the system Eq.~\eqref{eq:afterglow:blast} gives the full evolution 
of the \blast{}. 



























%% --------------------------------------------------------------------------
%%               Nucleo & EM counterts
%% --------------------------------------------------------------------------

\subsection{\ac{BNS} Merger Ejecta} \label{sec:intro:ejecta}


%% --------------------------------------------------------------------------
%%               M M  S I G N A T U R E S
%% --------------------------------------------------------------------------

%\subsection{Multimessenger Signatures} % -> EM signatures

%As \ac{PC} cases are 
%not expected to ejecta large amount of material and be \ac{EM}-loud (in case of 
%comparable mass binary), the \GW{} is believed to be not a \ac{PC} case, 
%\cite{Margalit:2017dij,Bauswein:2017vtn}. 

During and after the merger, neutron-rich material is ejected on the 
dynamical \citep{Rosswog:1998hy,Hotokezaka:2013b,Bauswein:2013yna,Wanajo:2014wha,Radice:2018pdn} 
and secular \citep{Lee:2009,Perego:2014fma,Fernandez:2015use,Siegel:2017nub,Fujibayashi:2017puw,Fernandez:2018kax,Miller:2019dpt} 
timescales via a number of processes. %(see Sec.~\ref{sec:intro:bns:ejecta}), 
The matter then 
undergoes \rproc{} \nuc{}, synthesizing neutron-rich unstable elements 
\citep{Eichler:1989ve,Wanajo:2014wha,Cowan:2019pkx},
that subsequently decay, releasing energy that thermalizes and can be observed as 
an \ac{EM} transient, \ac{kN} with quasi-thermal spectrum \citep{Metzger:2019zeh},
and later, non-thermal \ac{kN} afterglow as cold ejecta interacts with \ac{ISM} \citep{Piran:2012wd}.
%
%The non-thermal emission associated with \ac{BNS} mergers arise primarily from the
%interaction between the fast ejecta and \ac{ISM} \citep{Kumar:2014upa,Piran:2012wd}. 
%%This includes the \ac{kN} afterglow, produced by dynamical/secular ejecta and 
%For example, the collimated relativistic outflow, jet, can be observed as bright flashes 
%of high energy $\gamma$-rays, prompt emission, and late time afterglow across all \ac{EM} 
%bands \citep{Eichler:1989ve,Berger:2013jza} 
%Notably, the origin of prompt emission is not yet fully understood \citep{Kumar:2014upa},
%as well as the mechanism responsible for launching the jet. Several possibilities are 
%considered in the literature. \gray{The \ac{BZ} mechanism, that extracts the rotational energy 
%    from spinning \ac{BNH} in a strongly magnetized environment} 
%\citep{Blandford:1977ds,Ruiz:2016rai}, magnetized winds 
%from the remnant engulfed in strong \acp{MF} 
%\citep{Zhang:2000wx,Bucciantini:2011kx} and neutrino-antineutrino powered 
%fireballs \citep{Eichler:1989ve}.

%Both, \ac{kN} and \ac{SGRB} were observed for \GW{}, and are called respectively, \AT{} and \GRB{} 
%%\cite{kilonova observation list}
%%\cite{grb observation list}
%The $\gamma$-ray flash was detected $1.7\,$s after the \ac{GW} trigger by INTEGRAL and 
%Fermi satellites \citep{Monitor:2017mdv}. The \GRB{} appeared dimmer than other transients of its class, 
%which was attributed to the \ac{GRB} jet being a structured (not top-hat) jet with central axis
%misaligned with the line-of-sight and the 
%\citep[\eg][]{Lazzati:2017zsj,Xie:2018vya}%\cite{Refs of a struct jet model}

%% Neutrinos
%\ac{BNS} mergers are expected to be strong neutrino emitters, on par with \acp{CCSN}, if 
%the remnant suvives sufficiently long for trapped neutrinos to escape \cite{72}. However, 
%detection of this emission is limited to the events in our galaxy, where the merger rate is 
%once every $\mathcal{O}(10^4)\,$years.
%The IceCube and VERITAS facilities can detect GeV/TeV neutrinos and photons that might originate 
%in \ac{SGRB} jet or in the remnant magnetosphere \cite{130,131} if the source is up to 
%${\sim}10\,$Mpc away. The origin of this very high energy emission, however, is not fully 
%understood, and their detection would be a ``new messenger'' from \ac{BNS} mergers.

%% postmerger GWs
%A large amount of energy ${\lesssim}0.1\,\Msun\eqsim 2\times 10^{53}\,$erg is radiated 
%over the ${\sim}10-20\,$ms after merger in a form of kHz \acp{GW} \cite{107, 60}.
%While this emission is below the sensitivity of current generation facilities 
%\cite{97,132,80,133}, it bay be observed by $3$rd gen detectors for close enough events 
%\cite{57}. 
%The main peak of \pmerg{} \ac{GW} signal, $f_2$, \cite{132,133,134,135} was shown to 
%correlate with $R_{1.6}$ \cite{132,80}, (and not with \ac{NS} spin and \mr{} \cite{134,135}).
%A more tight constraints can be obtained on the $\kappa_2^T$ (or $\tilde{\Lambda}$) from 
%$f_2$ via the quasi-universal realtion \cite{80,56,57,58,136} 

%% Phase transition and GWs
%Notably, the \pmerg{} observations of $f_2$, that itself is set by the orbital frequency 
%that depends on the $k_2^T$ allows to constrain the low-density physics (fixing the $R_{1.6}$ 
%and $k_2^T$) \cite{44}. Thus, $f_2$, is almost independent on the merger remnant star 
%deformations (until the collapse), as the rate of angular momentum loss is correlate with the rate of binding energy loss to \acp{GW} \cite{56,107,135,20}. 
%This trend was also snown to be robust with the presence of the $2$nd order phase transition 
%after merger, that allows for na additional energy release and amplification of \acp{GW} signal 
%\cite{9}. This is not exactly the case if the $1$st order phase transition is present, that 
%leads to the remnant contraction on the dynamical timescale and instead of angular 
%velocity, it is angular momentum that is conserved, \cite{11}.
%This such a phase transition can be undetified as disagreement between the $k_2^T$ obtained 
%from the inspiral \acp{GW} and \pmerg{} \acp{GW} \cite{11} (assuming \ac{PC} did not 
%occur\cite{10}).


%\subsubsection{Mass Ejection, Kilonovae, and Nucleosynthesis}



%\ac{BNS} mergers result in ejection of material that can be observed, providing the 
%information of the merger dynamics and properties of the binary, and enrich the 
%\ac{ISM} with heavy \rproc{} elements \citep{Shibata:2019wef}. The quasi-thermal emission 
%associated with the decay of heavy \rproc{} elements is qusi-isotropic and thus have 
%higher changes to be observed \citep{Metzger:2019zeh}



%%%% Ejecta
%Generally, the ejecta is classified based on the timescale over which it occurs. 
The \textit{dynamical ejecta} is generally divided into 
the cold, low-$Y_e$, \textit{tidal}, produced by strong tidal torques at merger 
and prominent in very assymetric binaries
\citep{Rosswog:1998hy,Radice:2016dwd,Dietrich:2016hky},
and shock-heated, high-$Y_e$, \textit{shocked}, produced 
by the core \bnc{} induced shock waves, propagating through the \pmerg{} debris 
\citep{Radice:2018pdn}. %Hotokezaka:2013b, Bauswein:2013yna, Sekiguchi:2016bjd, Dietrich:2016hky,
%
The ejecta mass and velocity, as estimated by \ac{NR} simulations, lie in 
$(10^{-4}-10^{-2})\,\Msun$ and $(0.1-0.3)\,c$ respectively 
\citep{Hotokezaka:2013b,Bauswein:2013yna,Sekiguchi:2016bjd,Radice:2018pdn}.
% Depending on the binary \mr{} (that determins the relative controbutions of two
%components of the dynamical ejecta The outcome of the \rproc{} in the dynamical ejecta is 
%broadly consistent with the solar \rproc{} abundances.

%%%% Secular ejecta
On a longer, \textit{secular}, timescale additional matter is ejected in a form of 
massive winds \citep{Lee:2009,Perego:2014fma,Fernandez:2015use,Siegel:2017nub,
    Fujibayashi:2017puw,Fernandez:2018kax,Miller:2019dpt}, % & Nedora V, et al. Astrophys. J. 886:L30 (2019)
that could could be more massive than dynamical one, liberating $(10-40)\%$ of the disk mass. 
%according to the numerical simulations of the disk evolution 
%(see \eg, \citet{Radice:2018xqa} for estimates). 
Various mechanisms contribute to 
the matter ejection, \eg, neutrino irradiation of the polar region, generating high-$Y_e$, 
low-mass winds \citep{Perego:2014fma,Miller:2019dpt}; nuclear recombination in the outer 
part of the disk (after it expanded due to viscous and thermal processes) that can eject 
${\sim}(10-20)\%$ of the disk mass at ${\sim}0.1\,c$ \citep{Lee:2009,Fernandez:2015use,Fahlman:2018llv};
and \ac{MHD} effects facilitating matter ejection \citep{Fujibayashi:2017puw,Radice:2018xqa}.
% & 141 Nedora V, et al. Astrophys. J. 886:L30 (2019)

%%%% Ejecta -> EM signal
The \pmerg{} disk and remnant properties, and the lifetime of the latter, set by the 
binary parameters and \ac{EOS} (see Sec.~\ref{sec:intro:remnant}) \citep{Radice:2018xqa,Perego:2019adq} 
determine the properties of the secular ejecta and hence, the \ac{kN} signal 
\citep[\eg][]{Radice:2018pdn}.
%The disk mass dependency on the \ac{EOS} and the remnant lifetime was examined in \cite{72,70},
%where it was pointed out that short-lived remnant is usually associated with less massive disk.
Additionally, the presence of the remnant modifies the ejected material by means of 
neutrino irradiation \citep{Fernandez:2015use}, providing a possibility to infer the nature of 
the remnant from \ac{EM} observations. Modeling this process, however, requires very long-term 
$3$D ab-initio \ac{BNS} merger simulations with complete physics, that are absent in presence.

%%%% GW170817 EM signature
\ac{EM} followup of \GW{} showed the presence of both blue and red \acp{kN}\footnote{
    See, however, \citet{Waxman:2017sqv} for a different interpretation
} \citep{Villar:2017wcc}.
Simplified \ac{kN} fitting models suggest that former requires ${\sim}0.02\,\Msun$ of high-$Y_e$ 
material ejected at ${\sim}0.25\,c$, and the latter requires ${\sim}0.04\,\Msun$ of low-$Y_e$ 
material ejected at ${\sim}0.1\,c$. Due to its large mass, the red component is 
generally thought to originate in secular ejecta with the contribution from the dynamical, while 
the origin of the blue component is less clear. Specifically, the required amount of high-$Y_e$,
material (that is somewhat lower if sophisticated radiation transport \ac{kN} models are considered),
is in tension with the \ac{NR} \ac{BNS} merger simulations 
\citep{Sekiguchi:2016bjd,Siegel:2019mlp,Perego:2017wtu,Kawaguchi:2018ptg}.
Additional ejection mechanisms have been proposed to address the discrepancy, \eg, 
magnetic effects prior to merger \citep{Metzger:2018qfl,Fernandez:2018kax,Radice:2018ghv}. 
%\gray{spiral waves shocking the accretion 
%    disk by a long-lived \ac{NS} remnant \cite{141 Nedora V, et al. Astrophys. J. 886:L30 (2019)}}. 
Future observations and long-term multiphysics \ac{NR} models will undoubtedly shed more light on this tension \citep{Metzger:2018qfl}.

%%%% Fast tail of the ejecta -- UV precourser
\ac{NR} simulations show that within the velocity distribution of the dynamical ejecta, 
there is ${\sim}(10^{-6}-10^{-5})\,\Msun$ of matter ejected at ${\sim}0.8\,c$ 
\citep{Metzger:2014yda,Hotokezaka:2018gmo,Radice:2018pdn,Radice:2018ghv}, due to shocks launched 
at core bounces \citep{Radice:2018pdn}.
This matter is sufficiently fast to avoid the neutron capture on seed nuclei 
%(see Sec.~\ref{sec:intro:nucleo}) 
and will eventually undergo free-neutron decay, emitting 
\ac{UV} radiation on a timescale of hours \citep{Metzger:2014yda}. 

%%%% Synchrotron remnant
%Expanding into the \ac{ISM}, \ac{BNS} merger ejecta generates shocks that amplify 
%random magnetic fields and accelerates electrons that in turn emit non-thermal radiation 
%via synchrotron mechanism (see Sec.~\ref{sec:intro:afterglows}). This \ac{kN} afterglow 
%can be observed on a very long, 
%moths to years, timescale \citep{Nakar:2011cw,Hotokezaka:2018gmo} and provide information 
%about merger dynamics and ejecta properties quasi-independent of uncertainty in \rproc{} \nuc{} 
%affecting the thermal, \ac{kN}, emission. 
%%
Since ${\sim}160\,$days after the merger the non-thermal emission from \GW{} have been 
consistent with \ac{SGRB} afterglow. However, ${\simeq}1243\,$days after the merger, a 
change in spectral and temporal behaviour of the afterglow was observed, one of the 
possible explanations of which is the emergence of the \ac{kN} afterglow\footnote{
    See however \citet{Troja:2021xsw} for the alternative explanation
} \citep{Hajela:2021faz}.
The exact nature of this change remains at present unclear. 


%% =====================================================================================
%%
%%              O R G A N I S A T I O N
%%
%% =====================================================================================

\section{Aims and organization of this thesis}

\ac{BNS} mergers and especially the \pmerg{} evolution hosts a number of interesting 
and yet poorly understood physics. Although there are many published \ac{NR} simulations of \ac{BNS} mergers, 
most of them are either very short, or neglect important physics, \eg, neutrino reabsorption, 
and effects of the \ac{MF}-induced turbulence. 
Moreover, the statistical analysis of up-to-date sample of thesis 
simulations is a pending task.
%
We intend to improve upon this situation by performing a large number of long-term ab-initio 
\ac{NR} \ac{BNS} simulations with neutrino reabsorption and viscosity, focusing on the 
\pmerg{} evolution of the remnant and astrophysical aspects, \ie, matter ejection, 
\rproc{} \nuc{} and \ac{EM} signals.
%
All simulations have chirp masses targeted to \GW{}, and with this work we aim to further 
constrain the \ac{NS} \ac{EOS} and the properties of the binary and shed light on 
some of the aforementioned discrepancies, namely the 
\begin{itemize}
    \item what is the long-term evolutionary trajectory of the \ac{HMNS} merger remnant
    \item whether the \ac{BNS} mergers are dominant cite of \rproc{}, 
    \item what is the additional ejecta component required to explain the blue \ac{kN} of \AT{}, 
    \item whether \ac{kN} afterglow from the ejecta of ab-initio \ac{NR} models is consistent with \GRB{} rebrightening.
\end{itemize}

The thesis is organized as follows.
In the Chapter \ref{ch:nr_methods} we provide a brief overview of the methods used to model  \ac{BNS} mergers, 
that are implemented in the code \wisky{}, used throughout this work.
In Chapter \ref{ch:bns_sims} we present the results of the simulations and discuss the \pmerg{} dynamics and ejecta.
In Chapter \ref{ch:nucleo} we compute the \rproc{} \nuc{} in the ejecta and compare the result with 
solar abundances.
In Chapter \ref{ch:kilonova} we compute the quasi-thermal \ac{EM} counterpart, \ac{kN}, and compare 
the synthetic light curves with the observations of \AT{}.
In Chapter \ref{ch:afterglow} we compute the non-thermal, \ac{kN} afterglow and compare the 
result the recently detected re-brightening of \GRB{}.
Finally, in Chapter \ref{ch:conclusion} we conclude our work and provide an outlook.