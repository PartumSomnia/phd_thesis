% Chapter Template

\chapter{Non-thermal emission from \ac{BNS} mergers} \label{ch:afterglow} 

%In this chapter we discuss models of the non-thermal counterpart 
%of the \ac{BNS} mergers, the \ac{kN} afterglow.

In Sec.~\ref{sec:intro:afterglow} we discussed the origin of 
the non-thermal emission from \ac{BNS} mergers, focusing on 
\ac{GRB} and \ac{kN} afterglows. 
%
In this chapter we expand upon this discussion. First, in 
Sec.~\ref{sec:intro:afterglow_modelling}, we recall the basic methods 
of computing the synchrotron emission from a relativistic \blast{} 
expanding into \ac{ISM}. 
%
Then, in Sec.~\ref{sec:afterglow:code}, we describe specific methods 
that we implemented numerically in \pyblast{} code, and verify the 
code performance by comparing our results with published ones. 
%
Finally, in Sec.~\ref{sec:afterglow:results}, we present 
\ac{kN} afterglow \acp{LC} from ejecta from our \ac{BNS} merger 
simulations and compare them with observations. 


%% =============================================================
%%
%% P H Y S I C S 
%%
%% =============================================================

\section{Overview of the afterglow modeling methods}\label{sec:intro:afterglow_modelling}

The theory of relativist shocks with applications to \acp{AGN} jets was 
developed by \citet{Blandford:1976} 
and further expanded by \citep{vanEerten:2011bf,Nava:2013,Peer:2012}. 
The theory was successfully applied to \ac{GRB} afterglows 
\citep[\eg][]{Costa:1997cg,vanParadijs:1997wr,Lamb:2017ych}
(see \citet{Kumar:2014upa} for a review) 
and \ac{kN} afterglows \citep[\eg][]{Nakar:2011cw,Hotokezaka:2015eja,Hotokezaka:2018gmo}.
%
%A way to compute the non-thermal emission from an expanding \blast{} 
%is to perform multidimensional 
%radiation transport \ac{HD} or \ac{MHD} simulation. However, this 
%approach is numerically very expensive due to broad spatial and 
%time ranges involved. 
%
%It is possible, however, to approximate the \blast{} evolution  
%with a semi-analytic one-zone model, neglecting the internal structure 
%of the \blast{} \citep{Nava:2013,Peer:2012,Kumar:2014upa}. 
%
%When computing the afterglow emission from the structured relativistic 
%(mildly relativistic) source, 
The key components of the afterglow modeling are (i) \blast{} dynamics, 
(ii) electron energy distribution, (iii) \ac{EM} emission. 



\subsection{Dynamical evolution of a \blast{}}

A universal part of the afterglow theory is the dynamics of a  
\blast{} propagating through the \ac{ISM}, that is also called ``fireball''.

Analytical studies and numerical simulations showed that an 
expanding into \ac{ISM} a \blast{} generates a pair of shocks: a forward shock, that 
propagates through the upstream \ac{ISM}, and a reverse shock that moves black, 
through ejecta \citep[\eg][]{Blandford:1976,Ayache:2021six}. 
%%
\begin{figure*}[t]
    \centering 
    \includegraphics[width=0.45\textwidth]{Fig_8_KZ.pdf}
    \caption{
        Schematic sketch of a pair of shocks produced when a relativistic
        jet from a \ac{GRB} collides with the \ac{CBM}, as viewed from the
        rest frame of unshocked \ac{CBM}. Regions 2 \& 3 represent shocked \ac{CBM} and \ac{GRB}
        jet respectively. They move together with the same \ac{LF} ($\gamma_2$, as viewed
        by a stationary observer in the unshocked \ac{CBM}), and have the same pressure but
        different densities.
        (Adapted from \citet{Kumar:2014upa}, Fig.~8)
    }
    \label{fig:aafg:theory:sr8}
\end{figure*}
%%
In Fig.~\ref{fig:aafg:theory:sr8}, this pair of shocks is schematically depicted as 
boundaries between regions $1-2$ and $3-4$ respectively. 
Overall, there are four district regions: unshocked and shocked \ac{ISM} (regions $1$ and $2$), 
and shocked and unshocked \ac{GRB} or \ac{kN} ejecta (regions $3$ and $4$).
The comoving- and observer-frame quantities are marked with and without a superscript prime ($'$) 
respectively. 

The evolution of physical properties of a shock is governed by three conservation laws: 
baryon number, and energy and momentum fluxes across the shock front. 
The latter two are a embedded into the fluid energy momentum tensor, Eq.~\eqref{eq:theory:tmunu_perf}. 

Neglecting the internal structure of the \blast{} it is possible to express 
these conservation laws as \citep{Blandford:1976,Rezzolla:2013} 
%%
\begin{equation}
\frac{e_2'}{n_2'} = (\gamma_{21} - 1)m_p c^2, \hspace{5mm}
\frac{n_2'}{n_1'} = \frac{\hat{\gamma}\gamma_{21} + 1}{\hat{\gamma}-1}, \hspace{5mm}
\gamma_{1s}^2 = \frac{(\gamma_{21} + 1) [\hat{\gamma}(\gamma_{21}-1)+1]^2}{\hat{\gamma}(2-\hat{\gamma})(\gamma_{21}-1)+2},
\label{eq:afterglow:blast}
\end{equation}
%%
where subscripts $2$ and $1$ stand for downstream and upstream respectively, 
shown in Fig.~\ref{fig:aafg:theory:sr8}, 
$e'$ is the internal energy density, $n'$ is the proton number density, 
$\gamma_{21}$ is the relative \ac{LF} of a fluid in region $2$ with respect to region $1$, 
$\gamma_{1s}$ is the relative \ac{LF} of a fluid in region $1$ with respect to the shock front,
$\hat{\gamma}$ is the adiabatic index of a fluid, which is $\hat{\gamma}=4/3$ 
for the ideal relativistic fluid and $\hat{\gamma}=5/3$ for a subrelativisitc fluid.

Solving the system Eq.~\eqref{eq:afterglow:blast} for the 
forward and reverse shocks gives the full evolution of the quantities describing the \blast{}. 

A shock propagating through \ac{ISM} compresses it. 
For a relativistic case with $\hat{\gamma}=4/3$, the compression ratio in downstream fluid is 
$n_2'/n_1' = 3 ((4/3)\gamma_{21} + 1) = 4\gamma_{21} + 3 \approx 4 \gamma_{21}$. 
%
Additionally, a shock front randomizes the velocity vectors,
of particle, protons, raising their thermal energy, while their \ac{LF} remains
unchanged.


\subsection{Electron distribution}

As the shock compresses the fluid and amplifies random magnetic fields, it accelerates the 
inbound particles into a power-law distribution in energy space. 
%
The continuity equation for electrons in this space reads \citep{Kumar:2014upa} 
%%
\begin{equation}
\label{eq:intro:electron_dist_cont_eq}
\frac{\partial }{\partial t}\frac{d n_e}{d\gamma_e} + \frac{\partial}{\partial \gamma_e}\Big[ \dot{\gamma_e}\frac{dn_e}{d\gamma_e} \Big] = S(\gamma_e)\, ,
\end{equation}
%
where $dn/d\gamma_e$ is the electron distribution function, 
$\dot{\gamma_e} = -\sigma_T B'^2 \gamma_e^2 / (6\pi m_e c)$ is the rate at 
which electron \ac{LF} changes due to energy losses, $S(\gamma_e)$ is the injection 
rate of electrons into the system.

Electrons with \acp{LF} $\gamma_e > \gamma_c$ can efficiently loose their energy to 
synchrotron radiation. Then, after a certain characteristic time, $t_0$, 
their $\gamma_e$ drops below $\gamma_c$, 
%
\begin{equation}
c^2 \frac{dm_e}{dt} \gamma_e = -\frac{\sigma_T}{6\pi} B^2 \gamma_e^2 c
\hspace{5mm}
\gamma_c \sim \frac{6 \pi m_e c}{\sigma_T B^2 t_0}.
\end{equation}


Assuming that the injection of electrons is constant (steady-state solution,
$\partial_t = 0$), and has a minimum, $\gamma_m$, such that, 
$S(\gamma_e) = 0$ for $\gamma_e < \gamma_m$, the solutions to the Eq.~\eqref{eq:intro:electron_dist_cont_eq} reads 
%
\begin{equation}
\frac{dn_e}{d\gamma_e} \propto 
\begin{cases}
\gamma_e^{-2} &\text{ if } \gamma_c < \gamma_e < \gamma_m, \\
\gamma_e^{-p-1} &\text{ if } \gamma_e > \gamma_c > \gamma_m. 
\end{cases}
\label{eq:afterglow:elec_dist}
\end{equation}
%
The $\gamma_c < \gamma_e < \gamma_m$ regime is usually referred as 
\textit{slow cooling} and $\gamma_e > \gamma_c > \gamma_m$ as \textit{fast cooling} 
\citep{Sari:1997qe}.





\subsection{Synchrotron emission}

The power of the synchrotron radiation, $P'_{syn}$, emitted by an electron moving with 
the speed, $\upsilon_e$, corresponding to the \ac{LF}, $\gamma_e$, 
in the magnetic field, $B'$, perpendicular to the field lines is given by the Larmor's formula.
%
As within the magnetic field, an electron is following a spiral trajectory,
the characteristic frequency of the synchtrontron radiation, $\nu'_{syn}$, is given 
by the angular speed of the electron (\eg, its Larmor frequency).
%
The power per unit frequency at the peak, $P'_{syn}(\nu_{syn})$, can be computed as 
\citep{RybickiLightman:1985}
%
\begin{equation}
P'_{syn} = \frac{\sigma_T B'^2\gamma_e^2\upsilon_e^2}{4\pi c}, 
\hspace{5mm} 
\nu'_{syn} \sim \frac{q B' \gamma_e^2}{2\pi m_e c},
\hspace{5mm}
P'_{syn}(\nu_{syn}') \sim \frac{\sigma_T B' m_e c^2}{2 q},
\end{equation}
%
where $\sigma_T = 8\pi q^4 / (3m_e^2c^4)$ is the Thompson cross section.


The synchrotron radiation spectrum, emitted by an ensemble of electrons that have a 
distribution function $dn_e/d\gamma_e$ 
%(with $\gamma_e$ being the electron \ac{LF}), 
is given by convolving the distribution function with the power spectrum of a single 
electron, $P_{syn}(\nu)$, as 
%
\begin{equation}
f'(\nu') = \int_{\gamma_{m}}^{\gamma_M} d\gamma_e \frac{dn_e}{d\gamma_e}P'_{syn}(\nu'), 
%\propto \nu^{-(p-1)/2}
\label{eq:afterglow:sync_power}
\end{equation}
%
where $\gamma_{m}$ and $\gamma_M$ are the minimum and maximum \acp{LF} 
within which electrons contribute to the specific flux.


\subsection{Relativistic effects}

Consider a spherical coordinate system, $(r,\theta,\phi)$, where 
$r$ is the distance to the coordinate center, $\theta$ and $\phi$ 
are the latitudinal and azimuthal angles respectively. 
The \ac{BNS} merger remnant is located at $r=0$. 
Its axis coincides with $\theta = 0$. 
The observer lies on the $\phi = \pi / 2$, and the $\theta_{\rm obs}$ is 
the angle between \ac{LOS} and remnant axis. 

For a fluid element on moving with velocity $\upsilon$ (and \ac{LF}, $\Gamma$) 
and emitting photons at angle, $\theta$, from the \ac{LOS}, the time between 
two consecutive emissions in the observer frame, $\delta t_{\rm obs}$, 
and in the comoving frame, $\delta t'$, is related via the Lorentz transformation as 
%
\begin{equation}
\delta t_{\rm obs} = \frac{\delta t'}{\mathcal{D}}, \hspace{3mm}
\mathcal{D} = \frac{1}{\Gamma(1 - \beta\cos(\theta))}, \hspace{3mm}
\nu = \frac{\nu'}{\Gamma (1 - \upsilon\cos(\theta)/c)} = \frac{\nu'} {\mathcal{D}}\, ,
\label{eq:afterglow:dop_fac}
\end{equation}
%
where $\mathcal{D}$ is the Doppler factor, and $\nu=\nu'/\mathcal{D}$ is the 
classical Doppler shift formula for the frequency of the radiation.


For a thin spherical shell, radiation emitted at $(r=\upsilon t,\theta,\phi)$ 
arrives at the observer with a time delay with respect to a photon emitted at 
$r=0$ of
%
\begin{equation}
t_{\rm obs} = t - \frac{r \cos(\theta)}{c} = t\Big(1-\frac{\upsilon\cos(\theta)}{c}\Big) = \frac{t}{\Gamma\mathcal{D}}\, .
\label{eq:afterglow:tobs}
\end{equation}
%
Then, the total emission at a given, doppler shifted frequency, from the entire 
shell at the observer frame is obtained by integrated over all elements with the 
same $t_{obs}$.

Finally, the observed flux at a frequency $\nu_{\rm obs}$ from a spherical thin shell
in our coordinate system can be obtained by integrating 
Eq.~\eqref{eq:afterglow:sync_power} over the \ac{EATS} 
%%
\begin{equation}
    F_{\nu_{\rm obs}} = \frac{1+z}{4\pi D_L^2} \int_{\rm (EATS)} \mathcal{D}^3 f'(\nu') d\Omega
    \label{eq:afterglow:sync_power_obs}
\end{equation}
%%
where $d\Omega = \sin\theta d\theta d\phi$ is the solid angle, $z$ is the red-shift, 
$D_L$ is the luminosity distance. 


%% ------------------
%%  C O D E 
%% ------------------

\section{\pyblast{}}\label{sec:afterglow:code}

\def\eq{\text{equation}}
\def\eqs{\text{equations}}

%We calculate the non-thermal 
%radiation arising from the dynamical ejecta propagating into the cold \ac{ISM} 
%with the semi-analytic code \texttt{PyBlastAfterglow}. 

Considering the spherical coordinates we introduced in the previous section, 
we discretize it into $[N_{\theta},N_{\phi}]$ elements.
%
Extracted from \ac{BNS} merger simulations (Chapter~\ref{ch:bns_sims}), 
\ac{DE} angular profiles are then mapped onto this grid to provide the 
initial conditions for the evolution. 
%
Each elements of this structured \blast{} is then evolved independently 
as follows. 
%
We adopt the \blast{} dynamics formalism developed by \citet{Nava:2013} that casts 
Eqs.~\eqref{eq:afterglow:blast} into a set of \acp{ODE} for the \blast{} \ac{LF}, 
energy and swept-up mass that read 
%
\begin{subequations}
    \label{eq:afterglow:odes_nava}
    \begin{align}
    \frac{d\Gamma}{dr} &= -\frac{(\Gamma_{\rm eff} + 1)(\Gamma - 1) c^2\frac{dm}{dr} + \Gamma_{\rm eff}\frac{dE'_{\rm ad}}{dr}}{(m_0 + m) c^2 + E'_{\rm int}\frac{d\Gamma_{\rm eff}\rm }{d\Gamma}}, \\
    \frac{dE'_{\rm tot}}{dr} &= \frac{dE_{\rm sh}}{dr} + \frac{dE'_{\rm ad}}{dr} + \frac{dE'_{\rm rad}}{dr} \\
    \frac{dm}{dr} &= 2 \pi \rho (1 - \cos(\theta)) r^2;
    \end{align}
\end{subequations}
%
where $\Gamma$ is the \blast{} \ac{LF}, $r$ is its radius, 
$\Gamma_{\rm eff} = (\hat{\gamma})\Gamma^2-\hat{\gamma}+1 / \Gamma$ is the effective 
\ac{LF} (see \citet{Nava:2013}) 
with $\hat{\gamma}$ being the fluid adiabatic index, 
$E'_{\rm tot}$ and  $E'_{\rm int}$ are the total and internal energies respectively, 
$dE'_{\rm ad}$ and $dE'_{\rm rad}$ denote the adiabatic and radiative losses respectively, 
$m_0$ is the initial mass, 
$\theta$ is the opening angle of the \blast{}, 
and $\rho$ is the \ac{ISM} density.
%
Eqs.~\ref{eq:afterglow:odes_nava} are solved via 
explicit Runge-Kutta method of order $8(5,3)$ by \citet{Dormand:1980} 
(with stepsize control). 
%
The effects of radiation losses, discussed in \citet{Nava:2013} and lateral 
spreading of the \blast{} \citep[\eg][]{Granot:2012}, are turned off,
\ie, $d E'_{\rm rad}/dr = d\theta / dr = 0$.
%
The adiabatic index, $\hat{\gamma}$, is computed from the approximation to the 
numerical study of the \trans{} fluid \citep{Service:1986}
%\citep[\eq~5 in][]{Service:1986} as a function 
%of normalized temperature \citep[\eq~11 in][]{Peer:2012}. 
%
\begin{eqnarray}
\hat{\gamma} \approx (5 - 1.21937z &+ 0.18203z^2 - 0.96583z^3 + \\
2.32513z^4 &- 2.39332z 5 + 1.07136z^6)/3.
\end{eqnarray}
% 
where $z \approx T/(0.24 + T)$ with $T$ being the normalization temperature
\citep{Peer:2012}.
%
It smoothly connects the 
$\hat{\gamma}=4/3$ and $\hat{\gamma}=5/3$ regimes. 


We adopt a common assumption that a fixed fraction of the \blast{} 
energy, $E'_{\rm int},$ is being deposited into the electrons, 
$\varepsilon_e$, and magnetic field, $\varepsilon_B$, \citep[\eg][]{Dermer:1997pv}. 
%
We adopt the power-law electron distribution, Eq.~\eqref{eq:afterglow:elec_dist}, 
with $p$, the spectral index, being a free parameter.
%
We compute the characteristic \acp{LF}, $\gamma_c$ and $\gamma_m$ using the standard
prescriptions \citep[\eg][]{Dermer:2008ev}
%\citep[\eqs~A3 and A4 in][respectively]{Johannesson:2006zs}.
%
\begin{equation}
\label{eq:afterglow:gmin_gc}
\gamma_{min} = \frac{p - 2}{p - 1}  \varepsilon_e (\Gamma - 1) , \hspace{5mm} \gamma_c = \frac{6 \pi m_e c }{\sigma_T \Gamma B^{'2} t_{\rm obs}}\, 
\end{equation}
%
where where $t_{obs} = \int dr / (\beta c)$ is the time in the 
observer frame and $B'$ is the magnetic field strength.

%
The synchrotron emission in the comoving frame, 
Eq.~\eqref{eq:afterglow:sync_power}, for the slow and fast cooling 
regimes is approximated with a smooth broken power-law 
\citep{Johannesson:2006zs}
%
\begin{equation}
\begin{aligned}
P'(\nu') &= P_{\rm max;\, f}' \Bigg[\Big(\frac{\nu'}{\nu_c'}\Big)^{-\kappa_1/3} + \Big(\frac{\nu'}{\nu_c '}\Big)^{\kappa_1/2}\Bigg]^{-1/\kappa_2} \Bigg[1 + \Big(\frac{\nu'}{\nu_m '}\Big)^{(p-1)\kappa_2/2}\Bigg]^{-1/\kappa_2}, \\
P'(\nu') &= P_{\rm max;\, s}' \Bigg[\Big(\frac{\nu'}{\nu_m'}\Big)^{-\kappa_1/3} + \Big(\frac{\nu'}{\nu_m '}\Big)^{\kappa_3(p-1)/2}\Bigg]^{-1/\kappa_3} \Bigg[1 + \Big(\frac{\nu'}{\nu_c '}\Big)^{((1-p)/2 + p/2)\kappa_4}\Bigg]^{-1/\kappa_4},
\end{aligned}
\end{equation}
%
where $\nu_i ' = \chi_p \gamma_i^2 (3 B' / 4 \pi m_e c)$ 
are characteristic frequencies, and the 
$P_{\rm max;\, i}'$ are the maximum values of the power density, 
%
\begin{align}
P' _{\rm max;\, f} &= 2.234 \phi_p \frac{q_e^3 n' B'}{m_e c^2}, \\
P' _{\rm max;\, s} &= 11.17 \phi_p \frac{p-1}{3p-1}\frac{e^3 n' B'}{m_e c^2}.
\end{align}
%
The, $\phi_p$, $\chi_p$, and $\kappa_i$ are polynomials that 
describe the $p$-dependence \citep{Johannesson:2006zs}.

%\citep[\eqs~A2 and A6 in][respectively]{Johannesson:2006zs}, and the 
%the characteristic frequencies are obtained from the characteristic \acp{LF} 
%$\gamma_{m}$ and $\gamma_c$ via their \eq~A5.
%
%The synchrotron self-absorption is included via flux attenuation \citep[\eg][]{Dermer:2009}. %
%However, for the applications discussed in this paper, the self-absorption is not 
%relevant as the ejecta remains optically thin for the emission ${\geq3}\,$GHz 
%\citep[\eg][]{Piran:2012wd}.
%

The flux in the observer frame is obtained by integrating over the \ac{EATS}, 
Eq.~\eqref{eq:afterglow:sync_power_obs}.
%that are obtained by comping the $t_{obs}$, \ie, Eq.~\eqref{eq:afterglow:tobs} 
%for each segment. The relativist effects are taken into account via 
%Doppler factor, Eq.~\eqref{eq:afterglow:dop_fac}.
%See \citet{Salmonson:2003} for the detailed discussion of the method, 
%and \citet{Lamb:2018ohw,Fernandez:2021xce} for the recent implementations





\subsection{Method validation}

We verify the performance of \pyblast{} by comparing the synthetic \acp{LC} to 
those available in the literature. Specifically, we consider the \ac{kN} afterglow 
light curves presented in \citet{Radice:2018pdn} (see their Fig.~$30$ and Fig.~$31$) 
computed for \ac{DE} from a set of \ac{BNS} merger models. As we aim to conduct a 
similar study, and our \ac{BNS} merger simulations were computed with the same 
\ac{NR} code, this gives us a natural point of comparison. 
%
\begin{figure}[t]
    \centering 
    \includegraphics[width=0.70\textwidth]{ejecta_afterglow_vs_hotokezaka.png}
    \caption{
        Comparison between \ac{kN} afterglow \acp{LC} computed with 
        \pyblast{} and the code of \citep{Hotokezaka:2015eja} for \ac{DE}
        from a set of \ac{BNS} merger models of \citep{Radice:2018pdn}. 
%        (see their Fig.~$30$ and Fig.~$31$). 
%        Cumulative kinetic energy distribution for a selected set of models (\textit{top panel}) 
%        and its angular distribution for a BLh $q=1.00$ model (\textit{bottom panel}).
%        %% Also shown as a solid black line is the slow quasi-spherical model of \cite{Mooley:2017enz}.
%        The vertical light green line marks the $\upsilon_{\text{ej}}=0.6$.
%        The top panel shows that equal mass models have a more extend high energy tail,
%        while the bottom panel shows that the angular distribution of the ejecta is not 
%        uniform.
    } 
    \label{fig:afg_test}
\end{figure}
%
The result is shown in Fig.~\ref{fig:afg_test}. Overall, in most cases we observe 
a good agreement between our results and those shown in \citep{Radice:2018pdn}. 
%
We find the level of agreement sufficient considering very different treatments 
of the \blast{} dynamics and synchrotron emission. Furthermore, these discrepancies 
are considerably smaller than those, introduced by uncertain microphysics parameters 
and ejecta properties (discussed below).

%in comparison with large 
%uncertainties on microphysical parameters and ejecta properties (discussed below) these 
%discrepancies can be neglected.
%In Fig.~\ref{fig:afg_test} we show the \ac{kN} afterglow \acp{LC} from 
%Figure~$30$ and Figure~$31$ of \citet{Radice:2018pdn}. The plot shows that that 
%two codes are in a good agreement. The discrepancies observed can be attributed 
%for different physical treatment of the \blast{} dynamics and synchrotron radiation. 
%Notably, considering the large uncertainties on microphysical parameters and 
%ejecta properties (discussed below) these discrepancies can be neglected.



\section{\ac{kN} afterglow for \GRB{} rebrightening} \label{sec:afterglow:results}


%Fig.~\ref{fig:ejecta_vel_hist}. 
%Notably, since the largest part (in mass) of the ejecta is equatorial it eludes the 
%interaction with the \ac{GRB} collimated ejecta and expands into an unshocked \ac{ISM}.
%The latter can decrease the \ac{ISM} density and delay the peak of the 
%synchotron emission \citep{2020MNRAS.495.4981M}.
%\red{move that to afterglow and rephrase}


%/bns_em_prj/scripts/plot_afgs.py
%%cp best_xray_obs_representative_all_eos.pdf ~/GIT/GitHub/phd_thesis/v4/Figures/kn_afterglow/

\begin{figure*}[t]
    \centering 
    %% ---\includegraphics[width=0.49\textwidth]{./figs/scatter_lightcurve_peaks.pdf}
    %% \includegraphics[width=0.48\textwidth]{./figs/xray_obs_representative_all_eos.pdf}
    %% \includegraphics[width=0.48\textwidth]{./figs/radio_obs_representative_all_eos.pdf}
    %% --- \includegraphics[width=0.49\textwidth]{./figs/scatter_q_lam_ideaplot.pdf}
    \includegraphics[width=0.50\textwidth]{kn_afterglow/best_xray_obs_representative_all_eos.pdf}
    \hspace{-4mm}
    \includegraphics[width=0.50\textwidth]{kn_afterglow/best_radio_obs_representative_all_eos.pdf}
    \caption{
        \ac{NR}-informed \ac{kN} afterglow \acp{LC} in X-ray (\textit{left panel}) and 
        radio (\textit{right panel}) for a set of \ac{BNS} merger models. Different 
        colors represent different \acp{EOS}, while various line styles indicate three 
        \mr{}s. The \acp{LC} are compute with microphysical parameters reported in  Tab.~\ref{tab:pars}. 
        The observational data (depicted with gray circles) are obtained from 
        \citet{Hajela:2021faz,Balasubramanian:2021kny}.
        (Adapted from \citet{Nedora:2021eoj}).
%        Representative kilonova afterglow \acp{LC} for \ac{NR} models, 
%        in X-ray (\textit{left panel}) and in radio (\textit{right panel}), where 
%        %% There, every marker is annotated with two numbers, the tops is the peak time in years and the bottom is the peak flux in nJy ($10^{-9}$~Jy). The red color means that the value is below latest observations, that are $t=3.31$~years and $F_{\nu}=0.34$~$n$Jy. And the green color means that the model peak values are above the observational lower limit. 
%        the gray circles are the observational data from \citet{Hajela:2021faz,Balasubramanian:2021kny}.
%        %% The gray circles are the observational  in the X-ray band $0.3-10$~keV obtained from \citet{Fong:2017ekk,Hajela:2019mjy,Hajela:2020}.
%        The synthetic \acp{LC} are computed with varying 
%        micrphysical parameters and \ac{ISM} density within the 
%        range of credibility to achieve a better fit to observational data (see Tab.~\ref{tab:pars} for details).
%        %% The synthetic \acp{LC} are computed with the following parameters: 
%        %% $\epsilon_e = 0.1$, $\epsilon_B\in(10^{-3},10^{-2})$, $n_{\text{ISM}}\in(10^{-3},10^{-2})$~$\ccm{}$.
%        %% The last two parameters are varied for each model to achieve a better fit to the data. 
%        %% ---
%        The plots show that, within allowed parameter ranges, the \acp{LC} 
%        from all models are in agreement with observations. 
%        Models with moderately stiff \ac{EOS} and $q<1<1.8$ are tentatively preferred,
%        as their flux is rising at $t\geq10^3$~days, in agreement with observations.
%        %% --- 
%        %% The plot shows that while for all models the peak time and magnitude are
%        %% relatively close to the X-ray observed data, the models with moderately stiff \ac{EOS} and $q<1<1.8$ are tentatively preferred. 
    } 
    \label{fig:lightcurves}
\end{figure*}

\begin{table}
    \begin{center}
        \caption{
            Microphysical parameters and \ac{ISM} density used in \ac{kN} afterglow 
            calculations for \ac{BNS} merger models depicted in Fig~\ref{fig:lightcurves} and 
            Fig.~\ref{fig:lightcurve_peaks}, where the data for the latter 
            is shown in the bottom row. 
%            List of parameters for synthetic \acp{LC} shown in the Fig~\ref{fig:lightcurves} 
%            and Fig.~\ref{fig:lightcurve_peaks}.
%            For the former the microphysical and \ac{ISM} density are adjusted model-wise 
%            to achieved the good agreement with observations. For the latter,
%            (the last row of the table) the parameters are the same for all models shown.
%            Other parameters, such as observational angle, are the same everywhere (see text).
            (Adapted from \citet{Nedora:2021eoj})
        } \label{tab:pars}
        \scalebox{0.70}{
            \begin{tabular}{l | l l l l}
                Fig~\ref{fig:lightcurves} & $p$ & $\epsilon_e$ & $\epsilon_b$ & $n_{\text{ISM}}$ \\ \hline 
                BLh q=1.00    & 2.05 & 0.1          & 0.002        & 0.005            \\
                BLh q=1.43    & 2.05 & 0.1          & 0.003        & 0.005            \\
                BLh q=1.82    & 2.05 & 0.1          & 0.01         & 0.01             \\
                DD2 q=1.00    & 2.05 & 0.1          & 0.005        & 0.005            \\
                LS220 q=1.00  & 2.05 & 0.1          & 0.01         & 0.005            \\
                LS220 q=1.43  & 2.05 & 0.1          & 0.001        & 0.005            \\
                SFHo q=1.00   & 2.05 & 0.1          & 0.001        & 0.004            \\
                SFHo q=1.43   & 2.05 & 0.1          & 0.01         & 0.005            \\
                SLy4 q=1.00   & 2.05 & 0.1          & 0.001        & 0.004            \\
                SLy4 q=1.43   & 2.05 & 0.1          & 0.004        & 0.005            \\ \hline
                Fig.~\ref{fig:lightcurve_peaks}  & 2.15 & 0.2         & 0.005        & 0.005           
            \end{tabular}
        }
    \end{center}
\end{table}

In order to compute \ac{kN} afterglow \acp{LC}, 
several free parameters of the model need to be set. 
Ideally, these parameters have to be obtained by fitting the \acp{LC} to the 
observational data via \eg{}, Bayesian methods. Instead we opt to consider the 
parameters inferred for \GRB{} afterglow by prior studies and investigate, whether 
within their range of uncertainty they can lead to \acp{LC} compatible with observations, 
leaving the more rigorous analysis to future works.

Specifically, we consider the \ac{ISM} density to be uniform with 
$\nism\in(10^{-3}, 10^{-2})$ $\ccm$ \citep{Hajela:2019mjy}. 
The observational angle, 
%(the angle between the line of sight and the polar axis of the \ac{BNS} system) 
is set to $\theta_{\text{obs}}=30\,$deg \citep{TheLIGOScientific:2017qsa}.
The luminosity distance of NGC 4993, the host galaxy of \GW{}, is $41.3\times10^{6}\,$pc 
with the redshift $z=0.0099$ \citep{Hjorth:2017yza}.
%
The index of the electron energy distribution, $p$, and microphysical parameters are 
chosen based on the recent observations of \GRB{}, where the spectral evolution 
was detected \citep{Hajela:2021faz}.  
%(see however \citet{Troja:2021xsw}). 
%
We consider 
$\varepsilon_e\in(0.1, 0.2)$,
$\varepsilon_B\in(10^{-3}, 10^{-2})$, 
$p\in[2.05,2.15]$.

%/bns_em_prj/scripts/afterglow.py
%cp /home/vsevolod/GIT/overleaf/601033a6dcbe0b2c7e66cddf/figs/Plot2_1.pdf KN-afterglow-NR-equalmass.pdf
%cp /home/vsevolod/GIT/overleaf/601033a6dcbe0b2c7e66cddf/figs/Plot2.pdf KN-afterglow-NR-massratios.pdf

\begin{figure}
    \begin{center}
        \includegraphics[scale=0.51]{kn_afterglow/KN-afterglow-NR-massratios.pdf}
        \hspace{-5mm}
        \includegraphics[scale=0.51]{kn_afterglow/KN-afterglow-NR-equalmass.pdf}
        \caption{
            The effect of the cahnging $p$ from $2.15$ (lower boundary of colored bands) to 
            $2.05$ (upper boundary of colored bands) shown in a set of \ac{kN} afterglow 
            X-ray \acp{LC} for \ac{DE} from a sample of \ac{NR} \ac{BNS} merger simulations.
            %% ---
            In \emph{upper panel} the models with different \mr{} are shown and the 
            $\nism=6\times10^{-3}$~cm$^{-3}$, and microphysical parameters, 
            $\varepsilon_{\rm e}=10^{-1}$, $\varepsilon_{\rm B}=10^{-2}$.
            %% --- 
            In \emph{lower panel} the models with $q=1$ are shown and afterglow
            parameters are adjusted to fit observations, 
            with $\varepsilon_{\rm e}=0.1$ fixed and  
            $\nism\sim 6\times 10^{-3}, 5\times 10^{-3},5\times 10^{-3}\,\rm{cm^{-3}}$
            $\varepsilon_{\rm B}\sim 10^{-2},2\times 10^{-3},10^{-3}$ for models with 
            LS220, BLh and SFHo \acp{EOS} respectively.
            (Adapted from \citet{Hajela:2021faz})\red{rephrased}
        } \label{fig:kn_afterglow}
    \end{center}
\end{figure}

In Chapter~\ref{ch:bns_sims}, Sec.~\ref{sec:bns_sims:fast_de}, we discussed the properties 
of the fast tail of \ac{DE}. Here we examine how these properties translate to \ac{kN} 
afterglow signatures. 
%
Fig.~\ref{fig:lightcurves} shows the \acp{LC} in X-ray and radio bands for several 
representative \ac{BNS} merger models together with the latest \GRB{} data. 
%% --- lightcurve shape
We observe, that ejecta velocity and angular distribution defines primarily the shape of 
afterglow \acp{LC}. 
Specifically, broad velocity distribution found in equal mass \ac{BNS} models 
with soft \ac{EOS}, (\eg, SLy4 $q=1.00$ model, shown in Fig.~\ref{fig:ejecta_vel_hist}) 
translates into wide \acp{LC}, with an early rise time, comaptible with that of the early \GRB{} afterglow. 
This behaviour is guverned by the deceleration of the fastest ejecta shells, 
emission from which peaks early. If the velocity distribution is rather narrow, with most of the 
material moving at ${\leq}0.2\,c$ (\eg, LS220 $q=1.43$ model) 
the \ac{LC} rise is steeper and occurs later (${\sim}10^2$~days after merger). 

%% --- general agreement with observations
Fig.~\ref{fig:lightcurves} also shows that \ac{kN} afterglow \acp{LC} computed 
for most of our \ac{BNS} merger models are in a good agreement with the 
changing behaviour of \GRB{} within the uncertainties introduced by microphsical 
parameters and \ac{ISM} density.
%
Specifically, this agreement is particularly good for models with moderately stiff 
\acp{EOS} and $1.00<q<1.82$, considering the \ac{LC} peak time.

Spectral analysis of the changing behaviour of \GRB{} indicated a possible 
change in electron distribution index $p$ \citep{Hajela:2021faz}. Specifically, 
while previous \GRB{} analysis suggested $p=2.15$ with high degree of confidence 
\citep[\eg][]{Hajela:2019mjy}, the latest observations suggest lower value of $p$, 
$p=2.05$ \citep{Hajela:2021faz}. 
%
%From \GRB{} model fitting the electron power law index $p$ is well constrain to $p=2.15$
%\citep[\eg][]{Hajela:2019mjy}. The new emergent comonent in \GRB{} was found to have a lower 
%$p=2.05$ \citep{Hajela:2021faz}. 
The effect of the decrease in $p$ is shown in 
Fig.~\ref{fig:kn_afterglow}. Notably, the parameters $p$, $\varepsilon_e$. $\varepsilon_b$ and 
$\nism$ are very degenerate, meaining that the change in one can be offset by the change in 
another withing these parameters' ranges of credibility inferred for \GRB{}. 

\begin{figure}%%[t]
    \centering 
    %%     \includegraphics[width=0.49\textwidth]{./figs/scatter_thetarms_vej06.pdf}
    %% \includegraphics[width=0.49\textwidth]{figs/scatter_lightcurve_peaks_vs_lambda.pdf}
    \includegraphics[width=0.65\textwidth]{kn_afterglow/scatter_lightcurve_tpeak_vs_lambda.pdf}
    \caption{
        The time of \ac{kN} afterglow \ac{LC} peak for a set of \ac{BNS} merger simulations 
        as a function of binary parameters of these simulations, 
        tidal deformability $\tilde{\Lambda}$ and \mr{}, $q$ (color-coded). 
        %
        The microphysical parameters and \ac{ISM} density for all models are fixed and 
        given in the Tab.~\ref{tab:pars}.
        % 
        The time of the latest \GRB{} observation is shown as black dashed line.
        The black arrow indicates that at the time of the observation the flux appears 
        to be rising, \ie, the peak has not yet been reached.
        (Adapted from \citet{Nedora:2021eoj})
%        Peak time, $t_p$, for \ac{LC} for all considered \ac{NR} simulations. 
%        Dashed black line corresponds to the last observation of \GRB{} afterglow,
%        where the rising flux implies that it is a lower limit on the kilonova 
%        afterglow.
%        The microphysical parameters and \ac{ISM} density for all models are fixed and 
%        given in the Tab.~\ref{tab:pars}.
%        The plot shows that in general the $t_p$ increases with mass ration and with 
%        softness of the \ac{EOS}, except for the softest, DD2 \ac{EOS}. 
    } 
    \label{fig:lightcurve_peaks}
\end{figure}

%% --- Peak FLUX --- [UPDATED] --- IN CASE IT IS NEEDED [ BUT WITH NO FIGURE ]
If we fix the \ac{ISM} density and microphysical parameters to 
$\nism=5\times10^{-3}$~\gcm, $\varepsilon_e=0.1$ and $\varepsilon_b=5\times10^{-3}$, 
we observe that the \ac{LC} peak flux, $F_{\nu;p}\,$, is the highest 
for models with soft \acp{EOS} such as SLy4. 
In general, however, we do not find a strong dependency between \acp{EOS} and $F_{\nu;p}$.
With respect to the \mr{} we find that for models with stiff \acp{EOS}, 
the larger the \mr{}, the smaller is the $F_{\nu;p}$. 
%
This behaviour can be attributed to the overall dependency of the ejecta mass-averaged 
velocity on the \mr{} (see Sec.~\ref{sec:bns_sims:ejecta}, Fig.~\ref{fig:ejecta:dyn:dsfits}).
As the mass-averaged velocity decreases when \mr{} increases, the 
kinetic energy budget of the fast ejecta of these models decreases. 
Slower, more massive ejecta have lower peak flux.
Notably, for models with stiffer \acp{EOS} the dependency on \mr{} is not clear. 

%% --- UNCERTANTIES --- dominant --- microphysics 
%% It is however important to note that the \ac{LC} fluxes and the $F_{\nu,p}$ 
%% depend strongly on the shock michrophsyics. Within the error bars provided by 
%% \citet{Hajela:2019mjy}, they can vary by more then one order of magnitude.  
%% Additionally, the slope of the electron distribution, 
%% $p$, was shown to be lower for the emerging new component of \GRB{} afterglow 
%% (Hajela et al.~in prep). The change from $p=2.15$ to $p=2.05$ translates to the
%% inclrease in $F_{\nu,p}$ by $\sim2$.
%% --- uncertanties -- subdominat -- ejecta
We find that the the \ac{LC} shape and peak time do not depend strongly on the 
uncertain microphysical parameters and \ac{ISM} density. With respect to the latter, 
the peak time changes by a factor of a few when $\nism$ varies between 
$10^{-3}\, \ccm$ and $10^{-2}\,\ccm$.
Finite resolution effects that are present in ejecta properties do affect the 
afterglow \acp{LC}. Specifically, the $t_p$ changes by a factor of 
${\leq2}$, and $F_{\nu;p}$ changes withing a factor of ${\leq4}$. However, 
our analysis shows that the uncertainty in $\nism,\,\varepsilon_e,\,\varepsilon_b$ 
and $p$ have stronger effect on the \ac{LC} properties. 
%% --- Robust feature
%% Specifically, the \ac{LC} shape and the peak time appear to be robust 
%% both with resolution and with microphysics, as they set by the dynamics 

%\section{Discussion}
%
%In this section we considered the synchrotron afterglow arising from the interaction 
%of \ac{DE} and \ac{ISM} for a set of \ac{NR} \ac{BNS} models. 
%%
%The recent observations of \GRB{} by Chandra, ${\sim}10^3$~days after the \GW{} event 
%showed the emergence of the rising flux component. Our analysis suggests that this 
%rebrightening can be attributed to the emergence of the \ac{kN} afterglow, as its 
%properties, such as time and flux are naturally reproduced by the \ac{kN} afterglow 
%from ab-initio \ac{NR} \ac{BNS} simulations.
%%
%In the analysis we evolved the \ac{DE} with semi-analytic code and computed its 
%synchrotron emission. 
%We found that the synthetic \acp{LC} are in agreement with the emerging new component 
%in the \GRB{} afterglow within the range of credibility of the microphisycal parameters 
%and of the \ac{ISM} density, $n_{\text{ISM}}$. 
%% The \ac{LC} shape and the time of the peak depends primarily on the 
%% ejecta velocity distribution. If massive fast ejecta component is present, the \ac{LC} 
%% is broader and peaks before $10^3$~days postmerger, while afterglow from 
%% the ejecta with small/absent fast tail peaks on the timescale of $>10^3$~days 
%% (see Fig.\ref{fig:lightcurves}). 

%% --- Note on the observations
%% The latest \GRB{} observations by Chandra \newtxt{(and possibly VLA)} 
%% at $1234$~days show a rising flux (Hajela \textit{et al.} (in prep.)). 

Comparing the \GRB{} observations and synthetic \acp{LC} we observe, that the 
changing afterglow at $1243$~days after merger has the following implications:
the \ac{kN} afterglow peak should be (i) later and (ii) brighter than what is
currently observed. 
%
The condition (ii) is weak as the \ac{LC} peak flux is not well constrained due to 
uncertain microphysical parameters.
%% With respect to our models, the (ii) condition suggests that models soft \ac{EOS} and large \mr{} are disfavoured, 
%% as their $F_{\nu,p}$ is lower then the observations.
%% However, an increase in $\epsilon_B$ by a factor of $2$ changes this result.
The condition (i), however, is more robust from that point of view and allows to 
asses, afterglow from which models is in a better agreement with observations.

%% --- PEAK TIME 
We show the peak time of afterglow \acp{LC} of our \ac{BNS} merger models in 
Fig.~\ref{fig:lightcurve_peaks}. There, the microphysical parameters and $\nism$ are fixed 
and listed in Tab.~\ref{tab:pars} (last row). 
Here we consider all models, including those that do not have fast ejecta tail 
(see Sec.~\ref{sec:bns_sims:fast_de}) as their ejecta is still energetic enough 
to produce bright afterglow. 
%
The \ac{LC} peak times are ${\sim}10^3$~days for all models that do not undergo \ac{PC}. 
The latter, (BLh $q=1.82$ model) produces massive and 
slow ejecta that is characterized by the late afterglow 
with \ac{LC} peak of ${\sim}10^4\,$days.
%
Otherwise, we find $t_{p}<10^3\,$days for models with $q\sim1$ and 
$t_p>10^3$~days for models with larger \mr{}. 
%
This relation appears more prominent for 
models with soft \acp{EOS}, as the ejecta in these models has a strong contribution from 
shocked, fast component of \ac{DE} (when \mr{} is small), and the 
kinetic energy of the ejecta fast tail increases with 
growing contribution from the shocked component (see Fig.~\ref{fig:ejecta_v06}). 
And the afterglow of faster, less massive ejecta peaks earlier 
\citep[\eg][]{Hotokezaka:2015eja}.
Indeed, the time of the \ac{LC} peak depends primarily on the ejecta 
dynamics, the so-called deceleration time \citep[\eg][]{Piran:2012wd}.
%% --- 
%% In the Fig.~\ref{fig:lightcurve_peaks} we show the peak time, $t_{p}$, for each model, 
%% alongside the lower limit (the time of the latest observation).
%% --- 

In Fig.~\ref{fig:lightcurve_peaks} we also show the time of the latest observation 
of the rising flux in \GRB{} (horizontal line). This provides the lower limit on $t_p$
in accordance with (ii).
%
We observe that synthetic \acp{LC} of models with 
moderate amount of fast ejecta, \eg, models with 
\acp{EOS} of mild stiffness and \mr{}, lie above the limit, while models with very 
energetic fast tails, found in $q=1$ models with very stiff \acp{EOS}, peak earlier. 
%% ---
This provides a new avenue to constrain binary parameters and perform \ac{MM} analysis.

%%%% <<< moved to overall conclusion >>>
%The main conclusion of our work is that the observed rising flux in the afterglow of \GRB{} 
%at $10^3$~days can be explained by the \ac{kN} afterglow produced by ejecta in 
%ab-initio \ac{NR} \ac{BNS} simulations targeted to \GW{}. 
%%% ---
%Specifically, models with moderately stiff \acp{EOS} and moderately large \mr{}, 
%that produce a mild amount of fast ejecta, are favored.
%%% ---
%Out results are subjected to uncertainties, the dominant among which are introduced 
%by ill-constrained microphysical parameters. 
% Additionally, the systematic effects due to the finite resolution, neutrino treatment 
%and \acp{EOS} might be important. 
%%% ---
%A larger set of observations, that allows for a better assessment of shock microphysics, 
%and a larger sample of high resolution \ac{NR} simulations are required to investigate 
%these uncertainties further. We leave this to future works.