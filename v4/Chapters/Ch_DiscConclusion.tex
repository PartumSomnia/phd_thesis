\chapter{Discussion and Conclusions}\label{ch:conclusion} % Main chapter title



%In this Chapter, in Sec.~\ref{sec:conclusion:summary} we summarize, 
%how our results presented apply to \GW{} analysis and allow to 
%infer additional information about the event and by 
%an extension, to constrain the \ac{NS} \ac{EOS}. 
%%
%In Sec.~\ref{sec:conclusion:conclusion} we conclude our 
%work, providing an outlook and discuss the future directions 
%of our research. 
%%In this chapter we summarize how the results discussed in this thesis 
%%provide additional information on the \GW{}, the only \ac{BNS} merger 
%%observed so far in \ac{GW} and \ac{EM} spectra.







%\section{Conclusion}\label{sec:conclusion:conclusion}

%% =======================================================
%%
%%                   Conclusion
%%
%% =======================================================


The scope of this thesis was to advance our understanding of the \ac{BNS} mergers,
and fundamental physics, by using the state-of-the-art \ac{NR} simulations with
advanced physics and \ac{EM} models in tandem with \mm{} observations of the 
\GW{}. 



%% \section{Conclusion}

%In this part we discussed the long-term dynamics of the $37$ \ac{BNS} merger simulations
%focusing on the long-term evolution and hydrodynamics. 
%The binaries span the range total mass $M\in[2.73, 2.88]\,\Msun$, 
%mass ratio $q\in[1,1.8]$, and had fixed chirp mass $\mathcal{M}_c=1.188\,\Msun$.
%We considered five microphysical \ac{EOS} compatible with the current nuclear and
%astrophysical constraints. 
%%% ---
%Each \ac{BNS} model was computed at several resolutions. 
%In total $76$ simulations were considered, some of which were ${\sim}100$~ms 
%long. Together with our previous data
%\citep{Bernuzzi:2015opx,Radice:2016dwd,Radice:2016rys,Radice:2017lry,Radice:2018xqa,Radice:2018pdn,Perego:2019adq,Endrizzi:2019trv,Bernuzzi:2020txg}
%these simulations form the largest sample of merger simulations with microphysics
%available to date. 
%\red{Our ejecta data are publicly available \citep{vsevolod_nedora_2020_4159619}.
%}

%\subsection{Disk \& Remnant}

Considering the \pmerg{} evolution of \ac{BNS} merger remnant, we find an
overall strong dependency on the system \mr{} and \ac{EOS}, and on the 
finite temperature effects in the latter. One of the key affected 
parameters is the remnant lifetime. We find that models with 
soft \acp{EOS} or/and large \mr{}s produce short-lived \ac{NS} remnants that 
collapse within few ${\sim}10\,$ms after merger. 
%
More symmetric models with stiffer \acp{EOS} produce long-lived, possibly 
stable remnants. The lifetime of the \ac{NS} remnant appears to be correlated with the 
disk mass for the $q\sim 1$ models, in agreement with previous findings 
\citep{Radice:2017lry,Radice:2018pdn}.
Binaries with larger \mr{}s tend to have more massive disks and more massive tidal 
compoents of the \ac{DE}.
%
%The disk temperature and composition show dependency on the \mr{}. For instance, 
%models with $q{\sim}1$ form a disk primarily from the material squeezed out at the 
%\acp{NS} collisional interface. Such disk is characterized by relatively high 
%temperatures $O(10\, {\rm MeV})$. Its composition evolves from being 
%relatively proton reach, with electron fraction set by the 
%neutrino irradiation and shocks to $Y_e \simeq 0.25$, to being neutron rich with 
%$Y_e\simeq0.1$ as disks settles on a quasi-steady state. 
%This is in agreement with the theory of neutrino dominated accretion 
%flows \citep{Beloborodov:2008nx,Siegel:2017jug}. 
%
The long-term evolution of \pmerg{} \ac{NS} remnants is governed by both accretion, 
induced by the neutrino cooling and viscous stresses, and mass shedding that originates 
in gravitational and hydrodynamical torques and neutrino reabsorption (heating). 
%\citep[spiral waves][]{Radice:2018xqa}
Notably, the newly formed \ac{NS} remnant with mass exceeding the 
maximum of the uniformly rotating configuration, \ac{HMNS}, does not necessarily collapse 
to a \ac{BH}. Instead, massive winds, such as \ac{SWW}, can efficiently remove the 
excess in mass (alongside the angular momentum), bringing the \ac{NS} remnant 
to a rigidly rotating configuration.
%
%In order to asses the ultimate fate of the \ac{BNS} mergers with masses that fall 
%in-between the maximum for a nonrotating \ac{NS} and those leading to a prompt collapse,
%the long-term $3$D neutrino-radiation \ac{GRMHD} simulations are required.

Considering the matter ejected during and after mergers, we find 
two distinct types. The \ac{DE} and the \pmerg{} \ac{SWW}. 
%% --- dynamical
With respect to the former 
%we extend the previous analysis done by 
%\citet{Radice:2018pdn} with more advanced physics input and larger parameter 
%space of \ac{BNS} models. We 
we augment the analysis by considering all 
available \ac{BNS} merger models in the literature with various physics inputs. 
The statistical analysis of \ac{DE} properties highlights the strong 
dependency of these properties on the physics input, 
Specifically, the neutrino reabsorption has a systematic effect on the 
\ac{DE} properties. Its inclusion raises the ejecta mass and velocity. 
Meanwhile, the composition of \ac{DE} (its electron fraction) 
from our models, computed with an approximated M0 neutrino scheme, 
is similar to that found in simulations with more sophisticated neutrino treatment 
methods \citep{Sekiguchi:2016bjd,Vincent:2019kor}. 
Taking the largest-to-date set of \ac{BNS} simulations, we link the \ac{DE} properties 
back th binary parameters considering a variety of fitting formulae, updating 
these very important for \ac{MM} astronomy relations. 
Notably, a simple two parameter polynomial, \polql{}, shows a comparable or 
better statistical performance than other fitting formulae.

%
%(i) consistent inclusion of neutrino reabsorption via the M0 scheme,
%(ii) addition of the effects of the \ac{MHD} turbulence in most simulations 
%via the \ac{GRLES} subgrid model, 
%(iii) direct focus on the \GW{} with all models having corresponding chirp mass,
%whilst still considering models with larger \mr{} than in the aforementioned study.
%We also augment the analysis of our own simulaltions with the largest-to-date compiled
%set of \ac{BNS} merger simulations in the literature. 
%
%We find that the neutrino reabsorption has a systematic effect on the \ac{DE} properties.
%Its inclusion raises the ejecta mass and velocity.
%%This result is in agreement with previous studies \citep{Sekiguchi:2015dma,Radice:2018pdn}.
%The ejecta electron fraction, elevated by the neutrino reabsorption, is similar to that 
%found in simulations with different approximation schemes for neutrinos
%\citep{Sekiguchi:2016bjd,Vincent:2019kor}. 
%This suggests that the current neutrino schemes in \ac{NR} simulations are able to 
%reproduce the main neutrino effects consistently.
%%Our set of simulations shows that the properties of the \ac{DE} depend on the \mr{}.
%We also find that the ejecta velocity and electron fraction decrease with \mr{}, while 
%the total mass increases. However the statistical significance of the latter is weak.

%% --- SWW
In cases where \ac{NS} merger remnant is long-lived, we identify a new ejecta component, 
the \ac{SWW}. The wind is driven by energy and angular momentum injected into the 
disk by the remnant subjected to bar-mode and one-armed dynamical instabilities.
%that are present in \ac{MNS} remnants formed in mergers
%\citep{Shibata:1999wm,Paschalidis:2015mla,Radice:2016gym}
We find that within the simulation time, (up to ${\sim}100$~ms) \ac{SWW} does not 
saturate, unless the \ac{NS} remnant collapses to a \ac{BH}.
%The mass-loss rate via the \ac{SWW} is ${\sim}0.1{-}0.5\, \Msun\, {\text{s}}^{-1}$.
\ac{SWW} have a broad distribution in electron fraction, being on average higher,
then that of \ac{DE}, and narrow distribution in velocity, and can unbind
${\sim}0.1{-}0.5\, \Msun$ within a second.
%% --- Nu-Wind
A part of \ac{SWW}, channeled along the polar axis and exhibiting the highest
electron fraction we identify as \nwind{}. Contrary to other studies of the 
neutrino-driven outflows, \citep[\eg][]{Dessart:2008zd,Perego:2014fma,Fujibayashi:2020dvr}
the \nwind{} in our simulations saturates shortly after merger.
%
%However, while in the early \pmerg{} 
%tje properties of the \nwind{} in our simulations are similar to that found in the 
%\citet{Dessart:2008zd,Perego:2014fma,Fujibayashi:2020dvr}, within few tens 
%of milliseconds after merger as polar region above \ac{MNS} remnant 
%becomes polluted by the material lifted from the disk by thermal pressure. 
Notably, the steady state $\nu$-wind is generally referred to the outflow 
that emerges on a timescales hundreds of milliseconds longer than ours. 
Additionally, it is plausible that the approximated neutrino reabsorption 
scheme used in our simulations is insufficient in this case. 
Long-term simulations employing more advanced 
neutrino transport schemens are requred to asses the properties of \nwind.
%In the early \pmerg{}, the properties 
%of the wind are similar to $\nu$-wind found in 
%\eg~\citet{Dessart:2008zd,Perego:2014fma,Fujibayashi:2020dvr}.
%However, in our simulations the \nwind{} mass flux saturates within few tens 
%of milliseconds after merger, as polar region above \ac{MNS} remnant 
%becomes polluted by the material lifted from the disk by thermal pressure. 
%Notably, the emergence of the steady state $\nu$-wind is generally associated 
%with longer timescales that those, simulated here. It is thus possible, that the 
%conditions for the onset of \nwind{} have not been achieved in our simulations.
%Additionally, it is possible that the approximate neutrino treatment 
%employed in our simulations is not sufficient. 
%The emergence of the \nwind{} should be further investigated with long-term 
%simulations employing more advanced neutrino transport schemens.
%
Additionally, the effects of magnetization are important for the polar outflow 
\citep{Siegel:2017nub,Metzger:2018uni,Fernandez:2018kax,Miller:2019dpt,Mosta:2020hlh}.
Our simulations do not include magnetic fields and thus we cannot asses their 
effect on the outflow.
% However, simulations include turbulent viscosity 
%that have been show to produce a similar effect on the bulk of the secular 
%outflow to the full-\ac{MHD} \citep{Fernandez:2018kax}.

%% --- Nucleo
We asses the outcome of \rproc{} \nuc{} in the ejected matter via the precomputed 
parameterized model, based on the \ac{NRN} \texttt{SkyNet} \citep{Lippuner:2015gwa}.
%
The \rproc{} in \ac{DE} depends strongly on \mr{}, following the electron fraction,
with large amounts of lanthanides and actinides produced in high-$q$ cases.
%We find that as the \ac{DE} electron fraction depends on the binary \mr{}, so do the 
%\rproc{} yields. For instance, the ejecta from binaries with large $q$ is very neutron-rich 
%and the nucleosynthesis in it produces large amount of lanthanides and actinides.
Models with the highest \mr{}, that undergo \ac{PC}, show the 
actinides abundances in their \ac{DE} similar to solar.
%
Binaries with $q \sim 1$ produce less neutron-rich \ac{DE} and the 
final abundances show significant production of lighter elements,
%Ejecta from binaries with small $q{\sim}1$, on the other hand, is less neutron-rich 
%and the final abundances show significant production of lighter elements.
If the merger remnant is a \ac{NS}, the final \rproc{} abundances in total ejecta 
(that include \ac{DE} and \ac{SWW}) show large amount 
of both heavy and light elements. 
The abundance pattern in these ejecta is similar to solar, down to the $A\simeq 100$.
This result further implies the importance of \ac{BNS} mergers in cosmic 
chemical evolution.

%% Kilonova
%We compute the \ac{kN} emission from the decay of \rproc{} elements in ejecta using 
%the semi-analytic, anisotropic, multi-component model with nuclear heating rates 
%and opacities inferred from dedicated studies.
Considering the thermal emission from the decay of \rproc{} elements in ejecta, \ac{kN}, 
from our models with find that, when spherically symmetric \ac{kN} models are 
considered \citep{Villar:2017wcc}, none of our models can explain 
the \AT{} bolometric light curves.
However, when an anisotranisotropic multi-components \ac{kN} models are considered, 
that take into account properties and geometry of ejecta, 
certain key features of \AT{} are recovered.
Specifically, we find that the early blue emission can be explained 
when both \ac{DE} and \ac{SWW} are considered and when a binary produces 
a long-lived \ac{NS} remnant.
However, high electron fraction material was also shown to be present in the outflows 
from \ac{BH}-torus systems and thus does not necessarily require a long-lived \ac{NS} 
\citep{Fujibayashi:2020qda}.
The late time red kilonova component requires massive, ${\sim}20\%$ of the disk mass, 
low-$Y_e$ outflows. Such outflows can be driven by viscous processes and nuclear recombination 
on a timescale of seconds \citep[\eg][]{Metzger:2008av}.
%The presence of the high electron fraction, low opacity massive outflow, \ac{SWW} 
%eases the tension found between the \ac{kN} models applied to \AT{} and \ac{NR} 
%simulations, especially with respect to the early blue emission. And while other 
%interpretation exists they are not free from internal inconsistencies. 
%Standard kN models applied to the early AT2017gfo light curve are in
%tension with ab-initio simulations conducted so far.
%While alternative interpretations have been proposed, they are either
%disfavored by current simulations and observations (e.g. jets) \citep{Bromberg:2017crh,Duffell:2018iig},
%or require the presence of large-scale strong magnetic 
%fields which might not be formed in the postmerger
%\citep{Metzger:2018uni,Fernandez:2018kax,Radice:2018ghv,Ciolfi:2019fie}. 
%We identified a robust dynamical mechanism for mass ejection that
%explains early-time observations without requiring any fine-tuning.
%The resulting nucleosynthesis is complete and produces all
%$r$-process elements in proportions similar to solar system abundances.
%Methodologically, our work underlines the importance of employing
%NR-informed ejecta for the fitting of light-curves.
%Meanwhile, the \ac{SWW} of robust dynamical origin provides a natural explanation
%without requiring model fine-tuning and underlines the \ac{NR}-informed ejecta for 
%the fitting of light-curves.
%
%Further work in this direction should 
%include better neutrino-radiation transport and magnetohydrodynamic effects
%\citep{Siegel:2017nub,Fujibayashi:2017puw,Radice:2018xqa,Radice:2018pdn,Miller:2019dpt}. 
%\red{The dependency of the ejecta properties on the binary parameters can be used 
%    in the \ac{kN} parameter estimation. We asses this possibility and provide the 
%    updated statistical investigations of the ejecta properties in the section \ref{sec:ejecta:statistics}}.

% Kilonova Afterglow
Considering the synchrotron afterglow from the interaction between the cold  ejecta and 
\ac{ISM}, we find that the recently observed change in the afterglow of \GRB{} 
$10^3\,$days after merger can be explained by the \ac{kN} afterglow produced by ejecta in 
ab-initio \ac{NR} \ac{BNS} simulations targeted to \GW{}. 
%
Specifically, models with moderately stiff \acp{EOS} and moderately large \mr{}, 
that produce a mild amount of fast ejecta, are favored.
%
Our results are subjected to uncertainties, the dominant among which are introduced 
by ill-constrained microphysical parameters. 
Additionally, the systematic effects due to the finite resolution, neutrino treatment 
and \acp{EOS} might be important. 
%%
%A larger set of observations, that allows for a better assessment of shock microphysics, 
%and a larger sample of high resolution \ac{NR} simulations are required to investigate 
%these uncertainties further. We leave this to future works.

%% --- Targeted to GW 170817
%Comparing the ejecta properties from our simulations to those, iferred from the 
%fitting of \AT{} bolometric light curves \citep{Villar:2017wcc}, 
%we observe that none of our models is sufficient. 
%However, when an anisotranisotropic multi-components kilonova models, iformed by the 
%ejecta properties and geometry are employed, certain key features of \AT{} are recovered.
%Specifically, we find that the early blue emission can be explained 
%when both \ac{DE} and \ac{SWW} are considered for a binary with long-lived \ac{MNS} remnant.
%However, high electron fraction material was also shown to be present in the outflows 
%fron \ac{BH}-torus systems and thus does not requires a long-lived \ac{MNS} 
%\citep{Fujibayashi:2020qda}.
%The late time red kilonova component requires massive, ${\sim}20\%$ of the disk mas, 
%low-$Y_e$ outflows. Such outflows can be driven by viscous processes and nuclear recombination 
%on a timescale of seconds \citep[\eg][]{Metzger:2008av}.

%% --- overall conclusion :: loocing forward
\section*{Future work}

In order to investigate the \pmerg{} dynamics in more detail, \eg: 
(i) asses the remnant lifetime and the ultimate fate, 
(ii) verify the presence of the \ac{SWW} and \nwind{} and their properties, 
%more simulations are required. 
%%
%Specifically, 
high resolution long-term (several seconds) $3$D neutrino-radiation \ac{GRMHD} 
simulations, computed with advanced microphysical \acp{EOS} with finite temperature effects 
are required.
This might become possible as new, more advanced \ac{NR} codes become available 
\citep[\eg][]{Daszuta:2021ecf}.
Special attention should be given to the neutrino treatment methods as they affect strongly the 
properties and composition of ejecta.
Several methods that are now in development, such as gray or spectral M1 \citep{Foucart:2016rxm,Roberts:2016lzn},
or Monte-Carlo methods, are steps in this direction. 
%
Current leakage-based schemes, such as 
the M0 scheme used in this thesis and M1-leakages scheme of \citet{Sekiguchi:2015dma,Fujibayashi:2017puw}
cannot adequately treat the diffusion of neutrinos from the interior of the
\pmerg{} \ac{NS} remnant.
Additionally, the \ac{MHD} effects need to be re-examined. While it is 
apparent that \ac{MHD} is crucial for launching the relativistic jet, its effects on the 
ejecta and nucleosynthesis is not yet clear \citep{Siegel:2017jug, Fernandez:2018kax}.

On the other front, the growing number of observational facilities and their increasing 
sensitivity requires continuous advancements in modeling \ac{EM} counterparts to mergers.
%
For instance, while \ac{EM} follow-up of \GW{} has started ${\sim}11\,$hours after the 
\ac{GW} trigger and thus missed the very early emission, if such emission is observed 
in future events it would provide very important information on the ejecta properties 
and merger dynamics. 
%
Specifically, prompt $\gamma$-ray emission from a \ac{SGRB} allows to 
gauge the energetics of the event and properties of the system.
The \ac{UV}-precursor emission can hint the presence of the very 
fast ejecta component that can be later verified with afterglow observations. 


With respect to \ac{kN} models the attention needs to be given to 
(i) the geometry of ejecta 
(ii) the dynamical evolution of ejecta 
(iii) the non-\ac{LTE} effects.
%
The latter are especially important as with the launch of \ac{JWST}, 
the late \ac{kN} emission in \ac{IR} band would become observable for an event in the 
relative vicinity.


As \GRB{} has demonstrated, observations of \acp{SGRB} originating from 
\ac{BNS} mergers can shed light on the properties of the \ac{BNS} 
progenitor system, its astrophysical environment, as well as jet properties 
and jet physics \citep[\eg][]{Hajela:2019mjy}. 
%
Development of \ac{GRB} models that can take advantage of multi-epoch observations, 
including the motion of the flux centroid \citep[\eg][]{Fernandez:2021xce}, models that 
allow for an arbitrary jet structure and complexity are required.
%
Additionally, as sume \acp{SGRB} have shown to exhibit distributive features 
in their X-ray \acp{LC}, such as plateau \citep{Kumar:2014upa}, it is important 
to account for these features in \ac{GRB} modeling as they can provide crucial 
information on the \pmerg{} remnant \citep[\eg][]{Gibson:2017dep}. 


Several processes in the pre-merger and \pmerg{} stages of the 
\ac{BNS} system can produce \ac{EM} signals that could be their respective 
``smoking guns''. 
%
For instance, the \ac{EM} emissions from inspiraling strongly magnetized \acp{NS} 
\citep[\eg][]{Beloborodov:2020ylo}, 
fall-back accretion onto a \ac{BH} \citep[\eg][]{Desai:2018rbc}
and \ac{NS} \citep[\eg][]{Gibson:2017dep}.


Finally, combining the aforementioned methods and models would result in a 
surrogate \ac{EM} model of \ac{BNS} mergers that in tandem with 
already actively developing surrogate \ac{GW} models would allow for 
the most informed inference of binary properties. 
%
An example of such model is the \texttt{NMMA} pipeline \citep{Dietrich:2020efo} that 
was recently used to obtain constant on $R_{1.4}$, 
employing models of \ac{kN}, \acp{GW} and incorporating 
data from nuclear physics, pulsar observations and \GW{}.

Growing sample of observed \ac{BNS} mergers in \acp{GW} and \ac{EM} emission 
in the next decade will provide an unprecedented amount of information 
that would require constant re-analysis with ever-advancing models, 
techniques and our understanding of the merger processes.
%
It is a formidable challenge and interesting at that. 
%that would benefit from state-of-the-art 
%tools and methods, such as machine learning. 
%


%Finally, to infer the binary parameters and \ac{NS} \ac{EOS}, the \mm{} studies, that 
%bring together the state-of-the-art \ac{NR} simulations and \ac{EM} counterpart models are
%required. Such are the \texttt{NMMA} pipeline \citep{Dietrich:2020efo} that was recently
%used to provide the constant on the $R_{1.4}$, incorporating data from nuclear physics, 
%pulsar observations and \GW{}.
%
%
%Further work is required to overcome the limitations of this study.
%In order to asses weather the \AT{} \red{and its afterglow} can be explained from the 
%first principles, the long-term (several seconds), \ac{NR} \ac{BNS} merger simulations
%are required. Special attention should be given to the neutrino treatment, and more 
%accurate transport schemes, such as gray or spectral M1 \citep{Foucart:2016rxm,Roberts:2016lzn},
%or Monte-Carlo methods are required. Current methods leakage-based schemes, such as 
%the M0 scheme used in this work and M1-leakages scheme of \citet{Sekiguchi:2015dma,Fujibayashi:2017puw}
%cannot adequately treat the diffusion of neutrinos from the interior of the \ac{MNS} remnant.
%Additionally, the \ac{MHD} effects needs to be re-examined. While it is 
%apparent that \ac{MHD} is crucial for launching the relativistic jet, its effects on the 
%ejecta and nucleosynthesis is not yet clear \citep{Siegel:2017jug, Fernandez:2018kax}.
%
%
%Future work \& other counterparts \& MM pipeline \& PI and inference
%
%%
%In a high velocity tale of the \ac{DE}, the neutrons might avoid being captuired on a seed nuclide and freely 
%decay, producing a short, a few hours, and bright \ac{UV} precursor \citep{Metzger:2014yda}. 
%Unfortunately, in case of \AT{}, the \ac{EM} followup started \red{11} hours after the \ac{GW} detection.
%%
%See Figure 6 from \cite{Metzger:2016pju} for an examples of lightcurves of Kilonova and precursor.

