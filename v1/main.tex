\documentclass[11pt,a4paper,headinclude=true,DIV=14,BCOR=8mm,chapterprefix,listof=totoc,twoside,openright,abstracton]{scrbook}

\usepackage[headsepline]{scrpage2}
\usepackage[utf8]{inputenc}
\usepackage{geometry}
\usepackage{amssymb}
\usepackage{amsthm}
\usepackage{enumerate}
\usepackage{graphicx}
\usepackage{float}
\usepackage[intlimits]{amsmath}
% \usepackage{siunitx}
% \usepackage{color}
\usepackage{xcolor}
\usepackage{verbatim}
\usepackage{appendix}
\usepackage{hyperref}
\usepackage{hyperref}
\usepackage{mathtools}
% \usepackage[style=authoryear]{biblatex}
\usepackage{natbib}
% \usepackage{newtxtext}
% \usepackage{newtxmath}
% \usepackage{harvard}
\setcitestyle{aysep={}} 
\bibliographystyle{apalike}
\usepackage{xr}
\usepackage{wrapfig}
% \bibliographystyle{agsm}
%\usepackage{feynmf}
%\usepackage{tensor}
\usepackage[framemethod=tikz]{mdframed} % for a block of text

\setlength{\parindent}{0pt}
\geometry{a4paper, tmargin=3cm, bmargin=3cm, lmargin=3cm, rmargin=3cm, headheight=3em, headsep=2em, footskip=1cm}

\setcitestyle{citesep={,}}

\newcommand{\todo}[1]{\textcolor{red}{$\blacksquare$ TODO: #1}} 
\newcommand{\red}[1]{\textcolor{red}{#1}} 
\newcommand{\gray}[1]{\textcolor{gray}{#1}} 

\newmdenv[linecolor=cyan,backgroundcolor=cyan!20]{sidenote}


\geometry{a4paper, tmargin=2cm, bmargin=2cm, lmargin=1cm, rmargin=1cm, headheight=2em, headsep=2em, footskip=1cm}

\title{PhD thesis}
\author{Vsevolod Nedora}
\date{today}

\begin{document}
    
    \maketitle

%% --------------- 
%%
%% Theory
%%
%% ---------------

\chapter{Numerical Approximation of Conservation Laws}
\label{ch:nummethods}

Laws of conservation form a foundation of physics. Their complex matematical from, however, makes them challanging ti treat numerically. Thus, approximations are required. In this chapter we briefly outline high-order, state-of-the-art numerical methods for the solution of conservation laws. Owing to their key importance in physics, the amount of literature concerning the topic is extensive. Thus, we shall limit our discussion to what is relevant to this thesis, whilst keep the discussion brief. \gray{A particular focus will be set on general-relativistic hydrodynamics} \\

This chapter is organized as following. \gray{In section ... ... ... }

\section{Theoretical Background}
\label{sec:theory:conserv_laws:theorback}

In this section we outline the mathematical theory of conservation laws, and then, techniques of how they can be implemented numerically. Of particular importance, stand topics of weak and entropic solution for conservation laws, their existance and uniqueness. With respect to the numerical applications, we discuss the consistency, stability and convergence, as well as Lax-Richtmeyer theorem. We conclude with a brief overview of the case of non-linear equations. The brief discusstion that we provide here is based on the fundamental works by \cite{LeVeque:1992} and \cite{Tadmor1998}. We refer to them for more in-depth discussion. 

\subsection{Conservation Laws}

If we consider a basic form of conservation laws, 

%% eq 2.1 and 2.2 in the SOURCE 
\begin{align} 
\partial_t\boldsymbol{u} + \nabla\cdot\boldsymbol{f}(\boldsymbol{u}) = 0, \hspace{10mm} &(t,x)\in \text{I\!R}_{+}\times\text{I\!R}^d , \\
\boldsymbol{u}(0, x) = \boldsymbol{u}(x), \hspace{18mm} &x\in \text{ I\!R},
\label{eq:theory:conservlaws}
\end{align}

where $\boldsymbol{u}$ is the vector of $m$ unknowns, $\boldsymbol{f}=(\boldsymbol{\boldsymbol{f}^1,...,\boldsymbol{f}^m})$ is a $d$-dimensional flux and $\boldsymbol{u_0}\in\big[L^{\infty}(\text{I\!R}^d)\big]^m$ is the initial data. \\

It can be shown that the solution of the system \ref{eq:theory:conservlaws} can exhibit discontinuities (shocks) even if the initial data is smooth. Thus, a common approach is to view such a system on the notion of distributions. \\

A solution that satisfies an equation \ref{eq:theory:conservlaws} in a prescribed sense, for which, however, not all the derivatives exists, is called \textit{weak solution}. It can be shown, however, that even a simple case of a scalar conservation law may have multiple weak solutions. To select a "physically preferred" solution a concept of an "entropy function" $\eta$ is introduced. This is a special (convex) function that allows to symmetrizes the spatial Jacobian, $\nabla_{\boldsymbol{u}}f^i$. Together with the entropy flux, $\psi$, derived from compatibility relation, the $\eta\psi$ constitute and \textit{entropy pair}. Then, a weak solution for which the entropy pair exists and admits

\begin{equation}
\partial_t\eta(\boldsymbol{u}) + \nabla\cdot\boldsymbol{\psi}(\boldsymbol{u})\leq 0,
\label{eq:theory:nummeth:entropic}
\end{equation}

is the \textit{entropic solution}. The criterion can de understood, as requiring that the process, that led to the formation of a shock is irreversible \cite{LeVeque:1992}. In a scalar case $m=1$, the existence and uniqueness was shown possible for broad range of conditions \cite{Kruzkov:1970}. The extension to the  measure-valued solutions was stated by \cite{DiPerna:1985} and to the conservation laws on manifolds \cite{Benartzi:2007}. \\

In case of a system of conservation laws, however, the existence of an entropy flux $\psi$ is not guaranteed and the uniqueness and stability of entropic solution has not been proven. 
Thus, while in one dimension and with a particular equation of state such properties can be found \textit{e.g.,} \cite{Chen:2009}, for a general equations the existence and uniqueness of the Reimann problem is not assured \cite{Curtis:1972}.

Another examples, of where the weak solution existence was found are the relativistic case \cite{Glimm:1965} with however ultrarelativistic equation of state only \cite{Smoller:1993} and strictly hyperbolic systems with a smooth enough initial conditions \cite{Lax:1957}.

For a recent review on the topic, see \textit{e.g.,} \cite{Chen:2006}. \\

%%
%% []
%%

\subsection{Consistency, Stability and Convergence}

For the discussion of a numerical treatment of conservation laws, several key concepts ought be to be introduced. For simplicity, we limit ourselves to the $m=1$, one-dimensional case, where $\boldsymbol{u}$ of equation \ref{eq:theory:conservlaws} are scalar functions $u$. Let us further simplify notations, by viewing the solution to the 1D version of equation \ref{eq:theory:conservlaws} as a a function of time only $u(t)$, a bounded "curve" in infinite dimension vector space. Then, the \ref{eq:theory:conservlaws} can be viewed as an infinite system of ordinary differential equations (ODEs), written symbolically as

\begin{equation}
\frac{\text{d}u(t)}{\text{d}t} = \mathcal{L}[u(t)], \hspace{10mm} u(0) = u_0,
\label{eq:theory:conservlawsode}
\end{equation}

where we associate the operator $\mathcal{L}(\cdot)$ with the $-\nabla\cdot\boldsymbol{f}(\cdot)$. 
Importantly, the solution to \ref{eq:theory:conservlawsode} is not necessarily smooth in time and thus ought to be considered in terms of distributions. We however, assume for the purpose of this discussion that the $u(t)$ is a smooth function. In addition, by saying $u(0) = u_0$ we have broken a certain mathematical rigor. This is acceptable for our task, however, for more details we refer to the \cite{Kruzkov:1970}. 

Now we discuss the numerical approximation to \ref{eq:theory:conservlawsode}. This begins with introducing a discretization parameter $\Delta$ into the equation \ref{eq:theory:conservlawsode} such that $u^{\Delta}\approxeq u $, $L^{\Delta}\approxeq \mathcal{L}$ and $u^{\Delta}(0) \approx u_0$. In the latter, there is an additional error associated projection operator, but it is usually negligible in comparison with the one associated with discretization. One of these errors is the \textit{truncation error}, $r^{\Delta}$, which is due to approximating $\mathcal{L}[u(t)]$ with $L^{\Delta}[u(t)]$, where $u(t)$ is the exact solution to \ref{eq:theory:conservlaws}. A scheme is called \textit{consistent} if truncation error converges to zero as $\Delta \rightarrow 0$ in some (problem- and method dependent) norm for all possible initial data $u_0$. A scheme is said to be of order $r$ if $|| r^{\Delta}(t) || = \mathcal{O}(\Delta^r)$. A scheme is said to be \textit{stable} if this norm is limited $|||L^{\Delta}||| \leq C$, where $C\geq 0$ is a constant independent of $\upsilon$. And a scheme is said to be \textit{convergent} if a limit of $||u^{\Delta}(t)-u(t)|| = 0$ when $\Delta\rightarrow 0$. Lax-Richtmeyer equivalence for a linear equations state that numerical approximation of well-posed problems is convergent if and only if the scheme is stable and consistent \cite{Lax:1956}, thus relating consistency, stability and convergence. For a non-linear case a stronger criterion \textit{non-linear stability} is however required. 

\subsection{Non-Linear Equations and Non-Linear Stability}









%% --------------- 
%%
%% References
%%
%% ---------------

\newpage

\bibliography{../references}

\end{document}
