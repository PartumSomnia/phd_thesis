%% ==============================================================================
%% ==============================================================================
%% ==============================================================================
%%
\part{Numerical relativity simulations of neutron star mergers}
%% \label{sec:part1}
%%
%% ==============================================================================
%% ==============================================================================
%% ==============================================================================

%% In this part to discuss
%% GR Hydro 
%% Numerical methods for GR Hydro
%% Radiation
%% M0 scheme for neutrinos
%% WhiskyTHC & Lorene
%% GW1708017 targeted models
%% Remannt/Disk dynamics of model 


%% ===========================================================================
%%
%% Intorduction
%%
%% ===========================================================================

\chapter{Introduction}

\todo{Write me}

%% ===========================================================================
%%
%% Theoretical Background
%%
%% ===========================================================================

\chapter{Numerical Relativity} %% [ Based on the thesis of David Radice ]

%% ===========================================================================
%%
%%

In this thesis we perform and analyze numerical relativity simulations of merging neutron stars. These simulations are performed via solving the equations of general relativity, hydrodynamics and radiation, neutrino, transport via special numerical schemes. 

In this chapter we provide a brief description of the main equations and methods used to produce simulations analyzed in this thesis. For the sace of bravity we limit the discussion to the main results and implication important for our work.

For the underlying principles of the Eintein's theory of General Relativity, for which we here the reado to \red{[GR refs]}.

For the discussion and derivation of general relativistic hydrodynamics and refer the interested reader to \red{[GRHD refs]}.

For the Discussion on the radiation transport we refer to \red{GR-Rad refs}


%% ===========================================================================

\section{Basics of numerical relativity}

%% ===========================================================================






\section{Radiation}




\subsection{Neutrino leackage scheme}



\subsection{Neutrino M0 scheme}



\section{Initial Data}

\red{To explain: 
    pseudo-spectral code; 
    neutrino-less beta-equilibrium}

The initial data is computer via pseudo-spectral code \texttt{Lorene} \citep{Gourgoulhon:2000nn}
The code generates the binary neturon stars in a qusi-circular orbit, that revolve only few times before merging.
The EOS for the minimum temperature slice of the EOS table used for the evolution assuming neutrino-less beta-equilibrium.
Assuming constant temperature, we also remove the photon energy contribution to the pressure.

In the evolution code, passing the initial data, the mapping is done from the zero tem
In the evolution code, the electron fraction is set by the beta equilibrium condition. 
The specific internal energy is reset in accoradance with minimum temperature slice of the EOS table used for evolution.

Errors present in the initial data in intriduced during the mapping result in a small oscillations of netron stars.
In terms of relative changes in central density these amounts to $\sim2-3\%$ \cite{Radice:2018pdn}

\section{WhickyTHC}

\texttt{WhiskyTHC} is ...
\cite{Radice:2012cu,Radice:2013xpa,Radice:2013hxh,Radice:2015nva}


%% GR Hydrodynamics
The code evolves the proton and neutron number densities, $n_n$ and $n_p$
respectively, as 

\begin{equation}
\label{eq:wthc:pndens}
    \nabla_\nu (n_p u^\mu) = R_p^\mu \ \ , \ \ 
    \nabla_\nu (n_n u^\mu) = R_n^\mu \ .
\end{equation}

Here $u^{\mu}$ is the fluid four-velocity, $R_p = -R_n$ is the net
lepton number deposition rate due to the absorption and emission of neutrinos 
and antineutrinos (\red{see Section XXX})

The number densities are related as $n_p=Y_e n$ where $n = n_p + n_e$ is the baryon 
number density and $Y_e$ is electron fraction.

The matter of a neutron star is approximated with ideal fluid with stress-energy tensor

\begin{equation}
T_{\mu\nu} = \rho h u_{\mu} u_{\nu} + Pg_{\mu\nu}
\end{equation}

where $\rho=m_{\rm b} n$ is the baryon rest-mass density, 
$n$ the baryon number density, $m_{\rm b} \simeq 10^{-24}\,$g 
the neutron mass, 
$h=1+\epsilon + P/\rho$ the specific enthalpy, 
$\epsilon$ the specific internal energy (energy density),
and $P$ is pressure

The Euler equation for balance of energy and momentum reads

\begin{equation}
\label{eq:wthc:euler}
\nabla_\nu T^{\mu\nu} = Q u^\mu \ ,
\end{equation}

where $Q$ is the net energy deposition rate doe to absorption
and emission of neutrinos (\red{see Section XXX})


%% Numerical methods
High resolution shock capturing methods are used to discritize equations 
\eqref{eq:wthc:euler} and \eqref{eq:wthc:pndens}.
Specifically, central Kurganov-Tadmor type scheme \cite{Kurganov:2000} with 
HLLE flux formula \cite{Einfeldt:1988}
and non-oscillatory reconstruction of the primitive variables with the MP5 scheme of
\cite{Suresh:1997}.

Shock capturing schemes require the presence of a low density atmosphere around neutron stars.
The constant value of $\rho_0 = m_p n \approx 6\times 10^4$~\gcm.


\begin{sidenote}
    \textbf{David:}  \\
    
    I've been reading that for high-resolution shock-capturing, the finite differencing techniques are more efficient and simpler in implementation. However, most codes for MHD and radiation MHD that I find are using finite-volume methods. I am very curious why?..
    
    THC actually has both FD and FV schemes implemented
    FV is exactly conservative and there is a better way to do AMR with it
    FD is better because it is much simpler at higher order
    with THC when we want to do high order precision things we use FD
    when we do messy simulations with microphysics, for which robustness and conservation are more important than formal order of convergence, we use FV
    to be more precise we actually use the Kurganov-Tadmor central scheme, not a Godunov-type FV scheme
    but people always mixes the two (see e.g., the discussion in my PhD thesis)
    
    So, our simulations with microphysical eos andneutrinos are performed using the KT FV scheme?
    
    yes KT FV
\end{sidenote}

To increase a formal order of accuracy in a current generation numerical codes, flux-conservative finite-difference HRSC schemes is the simplest approach. Its direct competitor, high-order finite volume schemes are more computationally expensive, as they require solution of multiple Riemann problems at the interface between regions \cite{Reisswig:2009us,Shu:2001rep}, as well as complex averaging and de-averaging procedures \cite{Tchekhovskoy:2007zn} 

In this thesis we adopt \gray{ere David presents a new code}, the Templ
ated-Hydrodynamics Code (THC), developed using Cactus framework \cite{Goodale:2003} \red{cite David}. In \texttt{THC}, the state-of-the-art flux-vector splitting scheme are employed. 
The reconstruction in characteristic fields is available for up to 7th order, as well as, the Roe flux split with a entropy-fix prescription. 

The "tempalted" in the code name stands for a modern paradigm in C++ programming, the tempalated programming, which means, that part of the code can be generated from the prescribed templates at compiling time. 
This paradigm allows for a creation of complex modular codes avoiding computational costs, that plague classical polymorphism. 
The "templated" programming allows to inline all the needed functions and classes at compiling time, \cite{Yang:2001}. 

The following reconstruction schemes are implemented:
MP5, classical monotonicity preserving \cite{Suresh:1997,Mignone:2010},
the weighted essentially non oscillatory (WENO) schemes WENO5 and WENO7 \cite{Liu:1994,Jiang:1996,Shu:1997},
and two bandwidth-optimized WENO schemes WENO3B and WENO4B \cite{Martin:2006,Taylor:2007}, constructed for modeling the compressible turbulence. 
Note, that the number in scheme name stands for a formal order of accuracy. 

In this \red{section} we briefly state the details of the \texttt{THC} algorithm and highlight the results of the comparison between different reconstruction schemes for modeling relativistic turbulence. 

\gray{
The section is structured as following. First we overview several detain of \texttt{THC} code, discussing the numerical algorithms, in particular with respect to the equations of Newtonian and special relativistic hydrodynamics. Then we state several results. Then we view the linear and non-linear development of the relativistc Kelvin-Helmholtz instability (KHI) in 3D. 
}


\subsubsection{The \texttt{THC} code }
\red{Presentation and tests inf HYDRODYNAMIC part of the code only}

Here the infrastructure of \texttt{THC} is presented in addition to the formulation of Newtonian and special-relativistc HD. 

\textcolor{gray}{Here I will outline some results for my own understanding. This is not to be put in the thesis, as I am not working with the code development.}
\textcolor{red}{NOT REFPHRASED}

\begin{itemize}
    \item \textbf{Strong shock}. Classical one-dimensional shock tube. Even at this fairly low resolution, all the schemes are able to capture well both the shock wave and the rarefaction wave, showing the good behavior of the entropy fix. The contact discontinuity is resolved, but not without oscillations (due to the high Mach number of the shock wave, i.e., $\mathcal{M}=360$.
    \item \textbf{Blast wave}. larger density contrast at the contact discontinuity. The MP5 scheme is able to properly capture the constant state between the shock wave and the contact discontinuity, while the WENO schemes result in more “rounded” solutions.
    \item \textbf{Rotated Sod test}. Three-dimensional shock-tube test in Newtonian hydrodynamics. All the schemes are able to properly capture the main features of the solution: the discontinuities are captured within 1 or 2 gridpoints and both WENO5 and MP5 are able to capture the plateau in the velocity. Overall, these tests demonstrate the accuracy of the dimensionally unsplit approach that we use to treat the multi-dimensional case.
    \item \textbf{Double Mach reflection test} Our algorithm is able to introduce enough numerical dissipation to avoid the odd-even decoupling. All things considered, we find that the best performance is given by the MP5 scheme.
\end{itemize}

and in special relativity 

\begin{itemize}
    \item \textbf{Adiabatic smooth flow} Test code with the smooth solutions. One-dimensional, large-amplitude, smooth, wave propagating in an isentropic fluid. A good-enough approximation of the exact solution was obtained by computing it on a very fine Lagrangian grid (1e6 points) and interpolated on the Eulerian grid. Instead of the third-order SSP-RK scheme, we adopt here a fourth-order RK time integrator. Our schemes approach the expected convergence order only asymptotically, at very high resolution. The reason for this behaviour is in the “kinks” ahead and behind the pulse, where the numerical error is largest. These regions are “misinterpreted” as discontinuities by the shock-detection part of our schemes, unless they are resolved with enough gridpoints. The best performing scheme in this test is the MP5 one. Formation of the shock gradually degrades the overall convergence order to the 1st.
    \item \textbf{Blast wave}. Relativistic fluids can exhibit much stronger shock waves. MP5 scheme requires twice as small CFL as other schemes to prevent large oscillations and yields non-physical values. \textcolor{gray}{there is a finite-volume code \texttt{Whisky}, \cite{Baiotti:2010zf,Baiotti:2004wn} with the HLLE approximate Riemann solver \cite{Toro:1999} and PPM reconstruction \cite{Colella:1984}. } If the timestep is sufficiently small, on the other hand, the MP5 algorithm results in very accurate solutions, as in the Newtonian case. \texttt{THC} here performs better then \texttt{Whisky}.
    \item \textbf{Shock-heating} relativistic effects can enhance the density contrasts in shock waves. shocks whose collision compresses the fluid. Kinetic energy into thermal energy, that is, through “shock heating”. For a Lorentz factor of a 1000, $\Gamma=4/3$, for a Newtonian fluid the compression ratio $\approx7$, while for spec. relativ. it is $\approx 4000$. The WENO5 and WENO7 solutions are affected by some small wall-heating effect, slight underdensity. The MP5 scheme, on the other hand, yields a solution which is essentially free from oscillations.
    \item \textbf{Transverse shock}. the equations for the momentum in the different directions are coupled together by the Lorentz factor: even in one-dimensional problems the application of a transverse velocity can change completely the solution. This feature was first pointed out by \cite{Pons:2000} and \cite{Rezzolla:2002ra}, and then used by \cite{Rezzolla:2002cc} and \cite{Aloy:2006rd} to discover a new physical effect, see also [\cite{Mignone:2005ns}, \cite{Zhang:2005qy}] for a description of the numerical consequences of this property]. The MP5 scheme overestimates slightly the density contrast, but all of the algorithms are able to capture the correct location of the shock wave.
    \item \textbf{Spherical explosion}. No analytic solution is known in this case. As in the one-dimensional case, a small timestep is necessary in order to avoid numerical oscillations with the MP5 algorithm, while the other schemes appear to be stable even with a timestep which is twice as large.
    \item \textbf{Kelvin-Helmholtz instability in 2D}. The instability is seeded by adding a small perturbation in the transverse component of the velocity. we use periodic boundary conditions in all the directions. Compare first growth rate of the transverse velocity during the linear-growth phase of the KHI. Important to including the contact wave in the approximate Riemann solver in the case of a finite-volume code. We also note the importance of avoiding excessive dissipation in the contact discontinuity. The behaviour of the MP5 scheme, as well as that of the bandwidthoptimized WENO schemes, is more surprising: all of these schemes overestimate the growth of the RMS transverse velocity at low resolution. Some insight about the numerical viscosity can be gained by looking at the topology of the flow during the linear-growth phase of the KHI. These secondary instabilities, although only numerical artifacts (see below), appear only in schemes able to properly treat the initial contact discontinuity. They are not to be genuine features of the solution and, rather, tend to disappear as the resolution is increased. Conclusion: secondary instabilities are triggered by the non-linear dissipation mechanism of the different schemes, emerge neatly when computed with numerical schemes that treat properly the initial contact discontinuity, but do not have a physical meaning. Solution: adding more numerical viscosity [219] or as David suggests, physical viscosity. A more quantitative way of estimating the numerical viscosity of the code: The one-dimensional power spectrum can be used to quantify the typical scale of structures, such as the secondary vortices discussed above, stretched in the direction of the bulk shear flow. Even more unexpected is the ability of the MP5 scheme to resolve small scales structures and that, on the basis of the argument about the development of the secondary instabilities, should be more dissipative than WENO4B, but which instead appears to yield more small-scale structures in the rest-mass density.
\end{itemize}

\paragraph{The relativistic Kelvin-Helmholtz instability in 3D}
\textcolor{red}{Important for GRBs}

analysis is meant to assess how the different methods reproduce the same turbulent initial-value problem and to provide some insight on the spectral properties of the different schemes. The relativistic KHI [see, e.g., [51]] is of particular interest because of its relevance for the stability of relativistic jets [see, e.g., [251, 250]], and because of its potential role in the amplification of magnetic fields in gamma-ray bursts [see, e.g., [338]], and binary neutron-star mergers [25, 143, 240, 274].

\begin{itemize}
    \item \textbf{The linear evolution of the instability} Consider the evolution of the instability during its linear-growth phase. As expected, all the numerical schemes, with the exception of MINMOD, are in very good agreement with the 2D solution up to the end of the linear-growth phase, when 3D effects become important and turbulence starts to play an important role in the dynamics. It is interesting to note that MINMOD, which is the most dissipative of the schemes we are using, is actually overestimating the growth of the KHI.
    This suggests that \textbf{some care should be taken when interpreting the results from under-resolved simulations}. secondary vortices are produced in more least dissipative methods.
    \item \textbf{The non-linear evolution of the instability} The linear-growth phase of the KHI instability ends when the primary vortices become unstable to secondary instabilities and the flow starts the transition to turbulence. Three-dimensional effects dominate. Use the tracer scalar field to track the evolution.
    \item \textbf{The non-linear evolution of the instability}. when the primary vortices become unstable to secondary instabilities and the flow starts the transition to turbulence. three-dimensional effects dominate.
    \item \textbf{Fully-Developed turbulence} By far the most interesting quantity to study is the three-dimensional velocity power spectrum. conclusions. importance of the use of high order    schemes (avoid bottle-neck, otherwise power-spectrum shows an excess due to viscous effects.) use of WENO4B over WENO5 is well justified, since WENO4B is roughly twice as expensive as     WENO5 in 3D. Tthe main differences between the bandwidth-optimized schemes and their traditional counterparts seem to lay in the bottleneck region WENO3B and WENO4B have a much less pronounced bottleneck with respect to WENO5, WENO7 and MP5.
\end{itemize} 

\subsubsection{Driven Relativistic Turbulence}

Consider an idealized model of an ultrarelativistic fluid. The fluid is modeled as perfect. We evolve the equations describing conservation of energy and momentum in the presence of an externally imposed Minkowskian force. To solve the equations of relativistic hydrodynamics in 3D we use the THC code described in this chapter and published in \cite{Radice:2012cu}. In particular, here, we use the MP5 reconstruction in local characteristic variables [165].

\begin{itemize}
    \item \textbf{Basic flow properties} All in all, this is one of our main results: the velocity power spectrum in the inertial range is universal, that is, insensitive to relativistic effects, at least in the subsonic and mildly supersonic cases. Note that this does not mean that
    relativistic effects are absent or can be neglected when modelling relativistic turbulent flows.
    \item \textbf{Intermittency} local appearance of anomalous, short-lived flow features.
    \item \textbf{Conclusion} \textcolor{red}{We have presented THC, a new multi-dimensional, finite-difference, high-resolution shock-capturing code for classical and special-relativistic hydrodynamics... -- [FULL description of THC]}
\end{itemize}



\subsection{Finite-Differencing Methods: General Spacetimes}
\red{Presentation of GR+Hydro part -- whiskythc}


Goal is to model the inspiral of BNS to produce accurate waveforms. 
\textcolor{red}{here, we describe our new high order, high-resolution shockcapturing, finite-differencing code: \texttt{WhiskyTHC}, which constitutes the extension to general relativity of the \texttt{THC} code.}


\subsubsection{WhiskyTHC}
\textcolor{red}{marginally rephrased}

\begin{itemize}
    \item \textbf{Numerical Methods}. 
    \textcolor{gray}{[high order, high-resolution shockcapturing, finite-differencing code]} 
    
    \texttt{WhiskyTHC} is a result of combination of two \texttt{Whisky} \cite{Baiotti:2004wn} and \texttt{THC} \cite{Radice:2012cu}. High-order flux-vector splitting finite-differencing techniques has come from the former, while the module for the recovery of the primitive quantities as well as the equation of state framework from the latter \cite{Galeazzi:2013mia}. Tabulated temperature and composition dependent equation of states can be used \textcolor{gray}{however David used only polytrops}. 
    
    Overall, \texttt{WhiskyTHC} solves the equations of general-relativistic hydrodynamics in conservation form \ref{eq:theory:grhdeq_thc}. using a finite difference scheme \textcolor{red}{we however are using FV? Be carefull with which methods are used exactly}. 
    
    The flux reconstruction is done in
    local-characteristic variables using the MP5 scheme, see \textit{e.g.,} \cite{Rezzolla:2013}. The space-time is evolved using the CCZ4 formulation \ref{eq:theory:ccz4equations}, solved via finite difference code publicly available through \texttt{Einstein Toolkit}, \cite{McLachlan,Loffler:2011ay}. There, the central stencil is used throughout, and only terms associated with the advection along the shift vector are treated using the upwinded by one grid point stencil. The accuracy of the scheme is availalbe at 6th and 8th order, while 4th is commonly employed. 
    In addition, the fifth order Kreiss-Oliger style artificial dissipation \cite{Kreiss:1973} is added to aid with non0linear stability. 
    The code is build on the \texttt{Carpet} AMR driver \cite{Schnetter:2003rb} from the \texttt{Cactus} computational toolkit \cite{Goodale:2003}, incorporating a provided by \texttt{Carpet} Berger-Oliger-style mesh refinement \cite{Berger:1989,Berger:1984} with sub-cycling in time and re-fluxing. 
    \textcolor{red}{in Thesis it is said, -- no refluxing was done yet}
    
    
    \item \textbf{Atmosphere Treatment} The atmosphere is referred to an artificial density floor in the simulation domain. It is introduced in order to tackle the challenges arising when considering boundary between the fluid and vacuum in Eulerian (relativistic) hydrodynamics codes \cite{Galeazzi:mThesis:2008,Kastaun:2006,Millmore:2009dk}. 
    The defining property of the atmosphere is that the rest mass density and coordinate velocity are reset to a floor values once the former falls below a certain threshold value during the evolution \cite{Font:2001ew,Baiotti:2004wn}.
    While showing a reasonable results in second order codes, in higher order ones the numerical oscillations lead to the creation of vacuum nonetheless, that in light of the aforemention atmosphere effect result in the mass and energy violation \cite{Radice:2011qr}. 
    For codes that rely on characteristic variables, the degeneracy in low-density, low-temperature limits also plagues the computation. This problem is the main reason behind the popularity of robust shock capturing codes, even though they are of first order in the general-relativistic hydrodynamics codes. Vacuum treatment for higher order codes is of main challenges to overcome. 
    
    \begin{itemize}
        \item \textit{Standard Atmosphere Treatment} or \textit{"ordinary MP5 approach"} is based on setting density that falls below $(1+\epsilon)\rho_{\text{atmo}}$ to the atmosphere density, velocity to zero and internal energy to the one prescribed by the polytropic EoS. The $\rho_{\text{atmo}}$ is usually related to a certain characteristic density, \textit{e.g.,} maximum density at the beginning of the simulation as $\rho_{\text{atmo}} = 10^{-7,-9}\rho_{\text{max}}$. The tolerance parameter $\epsilon$ is usually set to $10^{-2}$ and accounts for excessive oscillations of the fluid–vacuum interface. 
        
        \item \textit{An Improved Atmosphere Treatment} or \textit{"MP5+LF"} In this approach the component-wise Lax-Friedrichs flux split is turned on when a certain density is reached. This increases the dissipation of the scheme and allows to avoid problems arising in characteristic reconstruction, associated with the degeneracy of the characteristic variables close to vacuum. Unfortunately, if the \red{ejection of low velocity and density matter is concerned}, this approach may yield oscillatory solutions and thus creates artifacts. 
        
        \item \textit{Positivity Preserving Limiter} a novel approach based on the use of PPL proposed in \cite{Hu:2013}. Here we provide a brief overview. 
        
        Consider a simple scalar conservation law in 1D
        
        \begin{equation}
        \frac{\partial u}{\partial t} + \frac{\partial f(u)}{\partial x} = 0
        \label{eq:theory:whickythc:atmo:conslaw}
        \end{equation}
        
        Since for a SSP time integrator a time update is convex combination of Euler steps, for which the positivity of $u$ is guaranteed for any scheme, the general discrete from of \ref{eq:theory:whickythc:atmo:conslaw} can be written as 
        
        \begin{equation}
        \frac{u_{i}^{n+1} - u_{i}^{n}}{\Delta^0} = \frac{f_{i-1/2} - f_{i+1/2}}{\Delta^1}
        \end{equation}
        
        And if $\lambda = \Delta^0/\Delta^1$, then 
        
        \begin{equation}
        u_{i}^{n+1} = \frac{1}{2}(u_{i}^{+} + u_{i}^{-}) = \frac{1}{2}\Big[ (u_{i}^{n} + 2\lambda f_{i-1/2}) + (u_{i}^{n} - 2\lambda f_{i+1/2})\Big].
        \end{equation}
        
        where then $u_{i}^{n+1} = u_{i}^{+} + u_{i}^{-}$ and $u_{i}^{n} = u_{i}^{n} - 2\lambda f_{i+1/2}$. Notably, the $u_{i}^{+}$ and $u_{i}^{-}$ as well as $u_{i}^{n+1}$ are positive. 
        In \cite{Hu:2013} it was pointed out that if a first-order Lax-Friedrichs scheme with $\lambda\leq 1/2a$ (with $a$ being the maximum propagation speed) is used for evaluating $f_{i\pm 1/2}$, the $u_{i}^{\pm}\geq \text{min}_i u_{i}^{n}$ \cite{Zhang:2010}. \textcolor{red}{not understand that}. Then the suggested point is ti change the $f_{i+1/2}$ to be 
        
        \begin{equation}
        f_{i+1/2} = \theta f_{i+1/2}^{\text{HO}} + (1-\theta)f_{i+1/2}^{\text{LF}},
        \end{equation}
        
        where $f_{i+1/2}^{\text{HO}}$ is the high-order flux of the original scheme, and $f_{i+1/2}^{\text{LF}}$ is the flux associated with the first order Lax-Friedrichs scheme, and $\theta\in[0,\:1]$. If the spatial location is far from vacuum, then the original high accuracy scheme can be used, so the $\theta$ remains $1$. However, in the vicinity of the vacuum, the $\theta$ decreases, to assure that $u_{i}^{\pm}$ remains positive. This is always possible since the Lax-Friedrichs scheme, used for $f_{i+1/2}^{\text{LF}}$ is positivity preserving.
        
        In a multidimensional case the the component-vise extension is employed. \textcolor{red}{formula that I will not used for $u_{i,j,k}^{n+1}$}.
        
        In \cite{Hu:2013} the extension of the method to the system of conservation laws was also proposed. 
        
        The complications however are present when the source terms are treated. While for a simplified case of classical gas dynamics it might require a lower timestep, in the general relativistic case and general tabulated EOS, the positivist of pressure is difficult to assure due to complexity of the energy source terms. It can be mitigated by enforcing a floor value on the pressure.
        
        Note, that adopting a positivity preserving limiter to treat the transition between matter and vacuum, still implies replacing the vacuum with low density fluid at rest, is not a physically accurate approach. That would rely on treating the transition as a free boundary (see \textit{e.g.,} \cite{Kastaun:2006}) The advantage of positivity preserving limiter with respect to a classical atmosphere treatment, is that it allows to have a value of $\rho_{\text{atmo}}$ that does not require further tuning and can be arbitrary small, and assure that the solution is locally conserved. 
        
        \textcolor{red}{In our models} we employ this approach as follows, at the meginning of the simulations we set the floor density, relying in the subsequent evolution on a positivity preserving limiters to ensure the atmosphere well behaviour. Due to negligeble density of the atmosphere its accretion has a negligeble effect on the evolved object. 
        
        
    \end{itemize}
    
    \item \textbf{Single Neutron Stars: Fixed space-time} here the atmosphere test showed that using the standard atmosphere leads to the mass conservation violation on a small degree, however, it also shows an appearance of a "jet+-like structure along the axes where grid points are aligned with the star's surface. These aritifical outflows are driven by the numerical oscillations creating an imbalance at the surface MP5+LF on the other hand shows no artificial matter streams due to its conservative nature
    
    \item \textbf{Single Neutron Stars: Full-GR}
    
    \item \textbf{Non-linear Oscillations: the Migration Test} Here the setup is the following, a neutron star in Full GR is set with an initial oscillation, that forces a star to first contract and then re-bounce. This re-bounce creates the ejecta. Different prescriptions show this ejecta, but the MP5-LF is inadequate in this test, introducing the structure in the outflow (numerical osculations/fragmentation). Origin: component-wise reconstruction in low density regions.
    
    \item \textbf{Gravitational Collapse to Black-Hole}
    
    \item \textcolor{red}{\textbf{Binary Neutron Stars} [Copied. Not rephrased]} 
    Models having an initial small separation of 45 km. Compare it to \texttt{Whisky} code, that is a second-order finite-volume code, with high-order primitive reconstruction and implements several different approximate Riemann solvers, \textcolor{red}{David used PPM reconstruction [95] and of the HLLE Riemann solver \cite{Harten:1983}, \cite{Einfeldt:1988}].} 
    
    The initial data we consider describes two neutron stars in quasi-circular orbit. It is computed in the conformally-flat approximation using the Lorene pseudo-spectral code \cite{Gourgoulhon:2000nn} and has been made publicly available by the Meudon group \cite{Lorene}. 
    The EoS assumed for the initial data is polytropic. \textcolor{gray}{In our case it is cold EOS, a slice from finite temperature EOS} while the evolution is performed using the ideal-gas EoS to allow for thermal effects in the merger phase. \textcolor{gray}{In our case it is finite temperature EOS}. 
    Discussion on baryonic masses and compactness $c=M_{\infty}/R_{\infty}$, where $R_{\infty}]$ is the areal radius.
    
    \begin{itemize}
        \item \textit{Small separation} Grid discussion: extend, symmetries \textit{e.g.,} we assume reflection
        symmetry across the $xy$ plane and $\pi$ symmetry across the yz plane. Number of refinement levels. Static grid or AMR. 
        Evolution via CCZ4 with damping constants $\kappa_1=?$, $\kappa_2=?$ and $\kappa_3=?$ and with beta-driver $\eta=?$. The space time evolved. Space-time is evolved via fourth order finite-differencing and with fifth order Kreiss-Oliger artificial dissipation \textcolor{red}{I need to find what is used in our runs}.
        
        Study the graviational radiation via looking ad the $l=2$ $m=2$ mode of the $Weyl$ scalar $\Psi_4$ extracted at a fixed coordinate radius of $r=450M_{\odot}$. Strain is not computed as it involves other uncertainties \cite{Boyle:2009vi,Reisswig:2009us,Reisswig:2009rx,Reisswig:2010di}. 
        
        The dynamics of the inspiral and merger of BNS has been described many times and in great detail in the literature \textit{e.g.,} \cite{Baiotti:2008ra}. We only mention that the two neutron stars
        inspiral for about 2:5 orbits, touch and quickly merge into a single black-hole. For this particular model no significant disk is left behind. The gravitationalwave signal consists.
        For GW plot 22 mode of $\Psi_4$ as extracted at $r=450M_{\odot}$ and as a function of the retarded time $t-r_*$ where $r_* = r + 2M_{\text{ADM}}\log(r/(2M_{\text{ADM}})-1)$.
        
        Results: 1. treatment of the neutron star surface is not a leading source of error in binary neutron star simulations, as far as the inspiral GW signal is concerned. \texttt{WhiskyTHC} shows a smaller dephasing significantly smaller de-phasing: the difference between the low and the high resolution is about 0.6 radians, which is a factor four smaller than the one shown by \texttt{Whisky}. 
        Observation: merger happens earlier as we increase the resolution. For each run we compute the phase, $\phi$, of the 22 mode of $\Psi_4$ from its definition, $(\Psi_4)_{22} = A e^{i\phi}$. 
        
        We should stress that this error estimate only reflects the numerical truncation error. Other systematic errors and, in particular, finite extraction radius effects and inaccuracies in the initial data, are also present and might be relevant (especially for WhiskyTHC). On the other hand, here we are interested only in evaluating the accuracy of the two numerical methods.
        
        \item \textit{Large separation} [mostly skipped]
        Notice that  contact happens before the bare contact angular frequency \cite{Damour:2012yf} 
        \begin{equation}
        0.15276 = M\omega_{\text{contact}} := 2C^{3/2}, \hspace{5mm} \omega:=\dot{\phi}
        \end{equation}
        is reached. This is in any case expected because this approximation of the contact frequency does not take tidal deformations into account.
    \end{itemize}
    
    \item \textbf{Conclusion}
    
\end{itemize}



