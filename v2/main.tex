\documentclass[11pt,a4paper,headinclude=true,DIV=14,BCOR=8mm,chapterprefix,listof=totoc,twoside,openright,abstracton]{scrbook}


%% packages
\usepackage[headsepline]{scrpage2}
\usepackage[utf8]{inputenc}
\usepackage{geometry}
\usepackage{amssymb}
\usepackage{amsthm}
\usepackage{enumerate}
\usepackage{graphicx}
\usepackage{float}
\usepackage[intlimits]{amsmath}
\usepackage{xcolor}
\usepackage{verbatim}
\usepackage{appendix}
\usepackage{hyperref}
\usepackage{hyperref}
\usepackage{mathtools}
\usepackage{natbib}
\usepackage{xr}
\usepackage{wrapfig}
\usepackage[framemethod=tikz]{mdframed} % for a block of text
\usepackage{etoolbox}

%% properties
\setlength{\parindent}{0pt}
\setcitestyle{aysep={}} 
\bibliographystyle{apalike}
\setcitestyle{citesep={,}}
\newtoggle{Full}
\togglefalse{Full}
\geometry{a4paper, tmargin=3cm, bmargin=3cm, lmargin=3cm, rmargin=3cm, headheight=3em, headsep=2em, footskip=1cm}

%% commands / macro
\newcommand{\todo}[1]{\textcolor{red}{$\blacksquare$ TODO: #1}} 
\newcommand{\red}[1]{\textcolor{red}{#1}} 
\newcommand{\gray}[1]{\textcolor{gray}{#1}} 
\newcommand{\magenta}[1]{\textcolor{magenta}{#1}} %% For terms/concepts to remember 
\newcommand{\swind}{spiral-wave wind}
\newcommand{\nwind}{$\nu$-component}
\newcommand{\gcm}{g cm$^{-3}$}
\newcommand{\dd}{\text{d}}
\newmdenv[linecolor=cyan,backgroundcolor=cyan!20]{sidenote}

%% head
\title{PhD thesis}
\author{Vsevolod Nedora}
\date{today}


%% ==================== START ================
\begin{document}

\maketitle

\mainmatter

%% ================= Targeted structure ====
%%
%% Part 1: Numerical relativity simulations of neutron star mergers
%%         Ch.1. Introduction
%%         Ch.2. Theoretical Background
%%               Sec.1. General-relativistic hydrodynamics
%%               Sec.2. Numerical Approximations of conservation laws
%%               Sec.3. High order numerical methods for Rel.Hydro.
%%               Sec.4. Relativistiv Radiation transport
%%               Sec 5. Microphysical equation of starts of nuclear matter
%%         Ch.3. Numerical relativity simulations
%%               Sec.1. Evolution code and Initial Data code
%%               Sec.2. Data analysis methods
%%         Ch.4. Results and Conclusions
%% Part 2: Electromagnetic counterparts to neutron star mergers
%%         Ch.1. Introduction
%%         Ch.2. Theoreticak Background
%%               Sec.1. Nucleosynthesos in binary neutron star ejecta
%%               Sec.2. Kilonova
%%               Sec.3. Short Gamma Ray Burst afterglow
%%               Sec.4. Ejecta aftergkiw
%%         Ch.2. Methods for modelling Kilonova and non-thermal afterglows
%%               Sec.1. MKN code
%%               Sec.2. PyBlast code (my code)
%%         Ch.3. Results and Conclusions
%% ======================


%% ================== ABSTRACT ===============
\begin{center}
    \textbf{Abstract} \\[1cm]
\end{center}


%% ============================================================================ PART 1 :: MODELS / Simulations
\part{Numerical relativity simulations of neutron star mergers}

%% ============================================================= Chapter :: Introduction 
\chapter{Introduction}

In this thesis we perform and analyze numerical relativity simulations of merging neutron stars. 
These simulations are performed via solving the equations of general relativity, hydrodynamics and radiation, neutrino, transport via special numerical schemes. 

In this chapter we provide a brief description of the main equations and methods used to produce simulations analyzed in this thesis. 
For the sace of bravity we limit the discussion to the main results and implication important for our work.
For the underlying principles of the Eintein's theory of General Relativity, for which we here the reado to \red{[GR refs]}.
For the discussion and derivation of general relativistic hydrodynamics and refer the interested reader to \red{[GRHD refs]}.
For the Discussion on the radiation transport we refer to \red{GR-Rad refs}

%% from GRLES Raduce paper
Multimessenger observations of BNS mergers are starting to constrain the poorly known properties of
matter at extreme densities [11,12,20–36] and the physical processes powering short g-ray bursts (SGRBs)
[37–42]. They are also beginning to reveal the role played by compact binary mergers in the chemical
enrichment of the galaxy with r-process elements [8,13,43–62]. The key to the solution of some of the most
pressing open problems in nuclear and high-energy astrophysics – such as the origin of heavy elements,
the nature of neutron stars (NSs), and the origin of SGRBs – is encoded in these and future observations.
However, theory is essential to turn observations into answers.

%% ============================================================ Chapter :: Theory/Methods 
\chapter{Numerical Relativity}

\section{Theoretical Background}

\subsection{General Relativistic Hydrodynamics}

\subsection{General Relativistic Radiation Transport}

\subsection{Numerical Methods}

%% ======================================== EOS used
\subsection{Microphysical EOS}

This subsection is based introduction in the recent publications 
\cite{Radice:2018pdn,Perego:2019adq,Bernuzzi:2020txg,Nedora:2020pak}

\begin{figure}[t]
    \centering 
    \includegraphics[width=0.49\textwidth]{./figs/tov_mr.pdf}
    \caption{Mass-radius relations for the EOSs used in this work. 
        Markers along the sequences indicate the NSs smulated in this work.}  
    \label{fig:method:tov_mr}
\end{figure}

For our models we employ $5$ finite-temperature, composition-dependent equations of state, namely the 
HS(DD2) (hereafter DD2) \cite{Typel:2009sy,Hempel:2009mc}, 
BLh, \cite{Bombaci:2018ksa}, 
LS220, \cite{Lattimer:1991nc}, 
HS(SFHo), (hereafter SFHo) \cite{Steiner:2012rk} and 
SLy4-SOR EOS (hereafter SLy4) \cite{daSilvaSchneider:2017jpg}.

All EOS include neutrinos $(n)$, protons $(p)$, nuclei, electrons, positrons, and photons
as important thermodynamic degrees of freedom.

The radii and maximum masses of neutron stars composed of the cold, neutrino-less $\beta$-equilibrium matter,
from these EOS, fall in line with the current astrophysical constraints, 
\textit{e.g.,} LIGO/Virgo constrain on tidal deformability 
\cite{TheLIGOScientific:2017qsa,Abbott:2018wiz,De:2018uhw,Abbott:2018exr}

All EOS models have symmetry energy at saturation density that are in argeement with experimental limits.
Notably, the LS220 has a especially steep dependency of its symmetry energy on density (\textit{e.g.,} \cite{Lattimer:2012xj,Danielewicz:2013upa}. Thus, this EOS might predict too low symmetry energy below the saturation density. 

%% LS220
The LS220 EOS is based on a non-relativistic (liquid droplet) Skyrme model.
The absolute value of the nuclear bulk incompressibility is set to $220$~MeV, hence, the name.
The EOS includes surface effects and it models $\alpha$-particles as an ideal, classical
non-relativistic gas. For heavy nuclei, the single nucleus approximation is sued. 
%Thus, the the compressible, liquid-drop model with surface effects and composed of ideal gas of 
%particles and heavy nuclei is used to represent the non-homogeneous nuclear matter.
%Heavy nuclei are considered with single nucleus approximation. 
The Gibbs construction is used to model the transition between homogeneous and non-homogeneous matter.
LS220 does not satisfy the constraints from Chiral effective field theory \cite{Hempel:2017ikt}

%% DD2
DD2 (and SFHo) employs the statistical equilibrium to treat the ensemble of several thousands nuclei.
The high-density nuclear matter is treated via RMF approach for unbound nucleons \cite{Hempel:2009mc}.
The excluded volume mechanism is utilized for the phase transition from nuclei to homogeneous
nuclear matter (when densities approch the nuclear saturation density).
DD2 employs the linear, but density dependent coupling for modeling the mean-field nuclear interactions \cite{Typel:2009sy}.
It was however noted that DD2 is not in very good agreement with the so-called flow-constraint \cite{Danielewicz:2002pu}.

%% SFHo [more rephrasing needed!]
Similar to DD2, SFHo combines a statistical ensemble of numerous nuclei, under the assumption of nuclear
statistical equilibrium (NSE) to treat the homogeneous matter, 
with the relativistic mean field approach for the unbound nucleons to treat high-density homogeneous nuclear matter.
It however employs a different parameterizations and values for modeling the mean-field nuclear interactions, 
which is motivated by neutron star radius measurements from low-mass X-ray
binaries (\cite{Steiner:2012rk} and references therein).

The DD2 and SFHo are based on nuclear statistical equilibrium, but 
different parameterizations of the covariant Lagrangian which models the mean-field nuclear interactions.
In these EOSs a finite volume correction coupled to a relativistic mean field theory for treating high-density nuclear matte. However, between these EOSs mean-field nuclear interactions have different parameterizations and values.

%% SLy4
The SLy4 EOS eomplyed in this work is the finite temperature extension \cite{daSilvaSchneider:2017jpg}
of the basic SLY4 Skryme parametrisation for cold nuclear NS matter \cite{Douchin:2001sv}.
The extension includes the non-local isospin asymmetric terms as well as more sophisticated 
treatment of nuclear surface properties and consistent treatment of heavy nuclei size. 
Which is a more advanced version of LS220 treatment (model).
When the phase transition, between the uniform and non-uniform phases, occurs, the phase with lowest 
free energy is chosen, (first order transition).

%% BLh
BLh is a new finite temperature EOS \cite{Logoteta:2020yxf}
This EOS is a finite temperature extension of the cold, $\beta$-equilibrium EOS, \cite{Bombaci:2018ksa},
which was applied to model the BNS merger in \cite{Endrizzi:2018uwl}.
The equation of state is derived in the framework of non-relativistic Brueckner-Hartree-Fock approach.
where microphysical approach based on a specific nuclear interaction is employed for the homogeneous nuclear phase.
The interactions between nucleos are described through a potential derived perturbatively 
in Chiral-Effective-Field theory \cite{Machleidt:2011zz}
The two body interactions are modeled up to second (from the leading term) order, that are used to calculate the local potential. This potential includes $\Delta$-resonances possible excitation. 
Further, the potential is augmented with the three-nucleon force, with the addition of $\Delta$-excitations.
The three-nucleon force was calibrated to reproduce the symmetric nuclear matter at saturatuin density \cite{Logoteta:2016nzc}.
The non-homogeneous phase of the EOS is treated by smoothly connecting the high density BLh EOS to the low-density oart of the SFHo EOS.




\subsubsection{Finite temperature treatment}

Thermal effects are included a different way in these EOS. 
In particular particle correlations beyond the mean field approximation are included only in the BLh EOS.
In other EOS thermal effects enter in the nucleon effective mass, that depend on temperature and density.
These effects, however, are important for the thermal evolution of the NS matter.

%% LS220 and SLy4
In the EOSs that are based on the Skyrme effecrive nuclear interatcions, \textit{e.g.,} LS220 and SLy4
the thermal effects are added in the following form. 
Consider a zero temperature internal energy functional, which depends explicitly on the nuclear density.
The part of this functional that is resposible for interactions is divided into the 
subpart that represents the two-body nucleon-nucleon interactions, (it is quadratic in nuclear density) and a 
subpart that mimics the effect of many body nuclear forces (proportional to the nuclear density in certain power).
Then, both the single particle potentials as well as kinetic energy effective mass dependence play a role
in the temperature dependence of the nuclear effective interaction.
The single particle potential are computed via the variation of the internal energy with respect to the 
neutron and proton densities.
Thus, the smaller the effective masss the larger are kinetic energies and hence, hgiher matter temperature. 
If the entropy remains unchainged.
Thus, the finite temeperature behaviour of these EOSs is largely set by the nucleon effective mass.
%Then, the smaller the effective mass, the higher the temperature (assuming that entropy is unchanged).
For LS220, the nucleon mass is its bare nucleon mass (for any densities) \gray{so... $m_{N}^*=m_{N}$??}.
For SFHo, $m_N ^* / m_N = 0.76$ at saturation density. 
For DD2 $m_N^*/m_N = 0.56$ 
For BLh $\red{None}$ 
For SLy4 $m_N^*/m_N=0.70$ at saturation density. 
where $m_{N}^*$ is the effective nucleon mass and $m_N$ is the bare nucleon mass.

%% SFHo (and DD2??)
For the SFHo and DD2 EOS, the relativistic Lagrangian considers the $\sigma-$, $\omega-$ and $\rho-$
meson exchanges for descibing nuclear interactions, and the mean-field approximation is used
to solve the resulted Euler-Largrange equations. 
The thermal effects for various species are introduced via Fermi-Dirac distributions at finite temperatures.
Then, self consitent solution of the mean filed eqution introduces the temperature dependence into other 
thermodynamic quantities (through fiurst mesons and nucleon fields)

%% BLh
The BLh EOS employs a different approach to incorporate temperature effects.
The method is based on evaluating the free energy in the Brueckner-Hartree-Fock, which in turn requires,
the effective in-medium nuclear interactions to be defined (starting from the bare nuclear potential).
\gray{This effective interaction is obtained by solving
the Bethe-Goldstone integral equation which describe the nucleonnucleon
scattering in the nuclear medium and properly takes into
account the Pauli principle}.
Then, the nucleon single particle potentials are evaluated (via integration of the on-shell effective inteaction matrix)
, which represent the mean field that a nucleon with a certain momenum experience sorrounded by other nucleons.
The nucleon single particle potentials, then, allow to evaulate the free energy, and subsequently, 
other thermodynamic quantities.
Notable difference with outher EOS disucssed here is that the many-body correlations extand beyong the mean field approximation, and are not present in other EOS. 
This EOS was first employed for the BNS merger simulations in \cite{Bernuzzi:2020txg}.

\subsubsection{TOV}

To characterize these EOS, that employs very different microphyscis and finite temperature properties and their relation to the electron fraction, we consider the TOV solutions, presented on the figure \ref{fig:method:tov_mr}.
The maximum mass of a non-rotating NS that these EOSs support are $2.06$, $2.06$, $2.42$ $None$ $None$ 
for SFHo, LS220, DD2, BLh and SLy4 respectively. The NS radii, $R_{1.4}$, then $11.9$, $12.7$, $13.2$ $\red{None}$ $\red{None}$, which in turn is related to the pressure at half saturation density \cite{Lattimer:2012nd}. Thus we adopt the following naming convention for EOS. Those that lead to a NS with smaller radii are called "softer" and those that lead to a NS with larger radii are referred to as "stiffer" EOS.
Among considered, the DD2 is the stiffest EOS, while SLy4 is the softest.

%% ======================================== Code used
%% \section{Used codes and setup}

\subsection{Initial Data}

For Each EOS considered, we compute the irrotational BNS configurations in quasi-circular orbit.
We employ the pseudo-spectral code \texttt{Lorene} \citep{Gourgoulhon:2000nn}, that 
solves the general relativistic initial data problem.
The initial separation (of the qusi-circular orbit) is chosen $\sim40$~km and that corresponds to $~2-3$ orbits before merger.

The EOS table used for the initial data computation is the minimum temperature slice
$(T\sim 0.5 - 0.1)$~MeV of the finite temperature EOS table used for the evolution.
The assumpution of the neutrino-less beta-equilibrium is made.
At constant temperature, at lowest densities, the photon energy (radiation) is a dominant contribution to 
pressure. Thus, we substruct this contribution from the tables.
%Addiitionally, the assuming constant temperature, we also remove the photon (radiation) energy contribution to the pressure (which dominates at the lowest densities)

The EOS table for the minimum temperature slice of the EOS table used for the evolution assuming neutrino-less beta-equilibrium.
Assuming constant temperature, we also remove the photon energy contribution to the pressure.

%In the evolution code, passing the initial data, the mapping is done from the zero temerature
In the evolution code, the electron fraction is set by the beta equilibrium condition. 
The specific internal energy is reset in accoradance with minimum temperature slice of the EOS table used for evolution.

Errors present in the initial data in introduced during the mapping result in a small oscillations of netron stars.
In terms of relative changes in central density these amounts to $\sim2-3\%$ \cite{Radice:2018pdn}


\subsection{Evolution with WhiskyTHC}


\texttt{WhiskyTHC} is ...
\cite{Radice:2012cu,Radice:2013xpa,Radice:2013hxh,Radice:2015nva}


\subsection{Hydrodynamics}

The code evolves the proton and neutron number densities, $n_n$ and $n_p$
respectively, as 

\begin{equation}
\label{eq:wthc:pndens}
\nabla_\nu (n_p u^\mu) = R_p^\mu \ \ , \ \ 
\nabla_\nu (n_n u^\mu) = R_n^\mu \ .
\end{equation}

\gray{in Radice2016dwd it is $\nabla_{\alpha}(n_e u^{\alpha}) = R$}

\gray{in Galezzi2013, for Whisky, the equations are separate for baryon and leptons: $\nabla_{\alpha}(n_bu^{\alpha})=0$ and $\nabla_{\alpha}(n_eu^{\alpha})=N$, where the $n_b$ and $n_e$ are the baryon and electron number densities respectively.}

Here $u^{\mu}$ is the fluid four-velocity, $R_p = -R_n$ is the net
lepton number deposition rate due to the absorption and emission of neutrinos 
and antineutrinos (\red{see Section XXX})

The $R_{p,n}$ is computed according to the neutrino M0 scheme \cite{Radice:2016dwd,Radice:2018pdn}

The number densities are related as $n_p=Y_e n$ where $n = n_p + n_e$ is the baryon 
number density and $Y_e$ is electron fraction.

The matter of a neutron star is approximated with ideal fluid with stress-energy tensor

\begin{equation}
T_{\mu\nu} = \rho h u_{\mu} u_{\nu} + Pg_{\mu\nu}
\end{equation}

where $\rho=m_{\text{b}} n$ is the baryon rest-mass density, 
$n$ the baryon number density, $m_{\text{b}} \simeq 10^{-24}\,$g 
the neutron mass, 
\gray{if Galezzi13 it is nucleon mass which is actrually related to the EOS.}
$h=1+\epsilon + P/\rho$ the specific enthalpy, 
$\epsilon$ the specific internal energy (energy density),
and $P$ is \gray{total isotropic} pressure.

\gray{Galezzi:13:
    pressure and specific internal energy contain contributions of baryons, electrons, photons and trapped neutrinos
    \begin{align}
        p &= p_e + p_b + p_{\gamma} + p_{\nu_e,\bar{\nu}_e}... \\
        \epsilon &= \epsilon_e + \epsilon_b + \epsilon_{\gamma} + \epsilon_{\nu_e,\bar{\nu}_e} + ...
    \end{align}
    where the contribution from trapped electron-neutrinos $\nu_e$ and
    antineutrinos $\bar{\nu}_e$ in the dense baryonic component can be evaluated from the thermodynamic state of the fluid and assuming that the neutrinos are following a Fermi-Dirac distribution 
    \begin{align}
        p_{\nu_e,\bar{\nu}_e} &= p_{\nu_e} + p_{\bar{\nu}_e} = \frac{4\pi}{3}T^4[F_3(\eta_{\nu_e}) + F_3(\eta_{\bar{\nu}_e})], \\
        \epsilon_{\nu_e,\bar{\nu}_e} &= \epsilon_{\nu_e} + \epsilon_{\bar{\nu}_e} = \frac{1}{3}\frac{p_{\nu_e,\bar{\nu}_e}}{\rho}
    \end{align}
    where $\eta_{\nu_e} = \mu_{\nu_e}/T$ and $\eta_{\bar{\nu}_e} = \mu_{\bar{\nu}_e}/T$ are the degeneracy parameters for the electron-neutrinos and antineutrinos, $\mu_{\nu_e}$, $\mu_{\bar{\nu}_e}$ the corresponding chemical potentials and $T$ is the temperature, $F_3(\eta_{\nu_e})$ is the Fermi integral.
    There, the contributions of trapped neutrinos to pressure and internal energy is neglected, 
    $p_{\nu_{e},\bar{\nu}_e}=\epsilon_{\nu_e,\bar{\nu}_e}=0$.
    For the computation of the neutrino source term $N$, the neutrinos emission rates per baryond are introduced $R_{\nu_e}$ and $R_{\bar{\nu}_e}$ for electron neutrinos and antineutrinos respectively. Then in the fluid rest-frame the change in electrom fraction is $u^{\alpha}\nabla_{\alpha}(Y_e)=\matcal{R} = R_{\bar{\nu}_e} - R_{\nu_e}$. 
    Similarly, for the source term $\Psi^{\beta}$ that describes the radiative losses of energy and momentum due to neutrinos, the neutrino emissivity $Q$ is intorcued. Assumptions: emission is isotropic in the fluid's rest frame. Then the covarient equation reads 
    \begin{equation}
        \Psi^{\beta} = -\rho m_b^{-1}Qu^{\beta} = \rho m_b^{-1}\sum_I Q_I u^{\beta} = -\rho m_b^{-1} (Q_{\nu_e} + Q_{\bar{\nu}_e} + Q_{\nu_{\tau,\mu}}+Q_{\bar{\nu}_{\tau,\mu}})u^{\beta}
    \end{equation}
    Here, the emissivity due to the $\tau$ and $\mu$ neutrinos into a single contribution $ Q_{\nu_{\tau,\mu}}$.
    These equations, for numerical evolution, the equations are cast into a flux conservative formulation, based on the Valencia formulation, representing Eurler equtions and lepton/baryon number conservations as balance laws.
    \begin{equation}
        \partial_t (\sqrt{\gamma}\boldsymbol{q}) + \partial_t\Big( \sqrt{\gamma} \boldsymbol{f}^{(i)}(\boldsymbol{q}) \Big) = s(\boldsymbol{q}).
    \end{equation}
    where $\gamma$ is the determinant of the three-metric, while $\boldsymbol{f}^{(i)}(\boldsymbol{q})$ and $s(\boldsymbol{q})$ are the flux vectors and source terms, respectively \cite{Font:2008fka}.
}


Written in a covariant form, the Euler equation for balance of energy and momentum reads

\begin{equation}
\label{eq:wthc:euler}
\nabla_\nu T^{\mu\nu} = Q u^{\mu} \ ,
\end{equation}

\gray{in Radice2016dwd it is $\nabla_{\beta}T^{\alpha\beta}=\Psi^{\alpha}$
with $\Psi^{\alpha} = Q u^{\alpha}$.
}
\gray{In the Galezzi:2013 it is $\nabla_{\alpha}T^{\alpha\beta}=\Psi^{\beta}$.
There the $T^{\alpha\beta}$ accounts for the ordinary matter and for trappend neutrinos and photons, but it does not include free-streaming neutrinos. Assumed to be similar to the 'test-fluid' they are neglected in constracting RHS of the Eistein equations.
}

where $Q$ is the net energy deposition rate doe to absorption
and emission of neutrinos also treated with the M0 scheme.
 

\subsection{Numerical methods}


High resolution shock capturing methods are used to discritize equations 
\eqref{eq:wthc:euler} and \eqref{eq:wthc:pndens}.
Specifically, central Kurganov-Tadmor type scheme \cite{Kurganov:2000} with 
HLLE flux formula \cite{Einfeldt:1988}
and non-oscillatory reconstruction of the primitive variables with the MP5 scheme of
\cite{Suresh:1997}.

Shock capturing schemes require the presence of a low density atmosphere around neutron stars.
The constant value of $\rho_0 = m_p n \approx 6\times 10^4$~\gcm.

The rest-mass consirvation in the presence of artificial atmosphere is assured via 
positivity-preserving limiter from \cite{Radice:2013xpa}

The local number densities of neutrons and protons separately, are assured via 
multi-fluid advection method of \cite{Plewa:1998nma}

The outflow properties are extracted when the density exceeds the atmosphere density
by several orders of magnitude.

%% Spacetime evolution
The spacetime is evolved using the Z4c formulation of Einstein's equations
\cite{Bernuzzi:2009ex,Hilditch:2012fp} as implemented in the \texttt{CTGamma} code
\cite{Pollney:2009yz,Reisswig:2013sqa} which is part of the \texttt{Einstein Toolkit} 
\cite{Loffler:2011ay}.

The non-linear stability of evolution is assured via Kreiss-Oliger dissipation. 
The spacial discritisation is done via fourth-order finite-differencing implemented in \texttt{CTGamma}.

The method of lines, MOL, couples the space-time evolution and hydrodynamics. 

Time integrator of choice is strongly-stability preserving third-order Runge-Kutta scheme \cite{Gottlieb:2009}.
The timestep is regulated by the Courant-Friedrichs-Lewy (CFL) condition, that required CFL factor 
to be $<0.25$ for numerical stability. To assure taht the positivity-preserving limiter implemented in \texttt{WhiskyTHC} maintains the density positive, the CFL factor is set to $0.15$.


\subsection{AMR}

The code uses the Berger-Oliger conservative adaptive mesh renement (AMR) \cite{Berger:1984} with 
sub-cycling in time and \red{refluxing (Davids thesis does not have refluxing)} \cite{Berger:1989,Reisswig:2012nc} as provided by the \texttt{Carpet module} of the \texttt{Einstein Toolkit} 
\cite{Schnetter:2003rb}. 



\subsection{Neutrino scheme}


This subsection is based on the recent papers by \cite{Radice:2016dwd} where the implemented 
into the \texttt{WhiskyTHC} leackage and M0 schemes are described 
and \cite{Radice:2018pdn} where the scheme is summarized. 
%% \gray{for now we neglect the long fundamental paper \cite{Galeazzi:2013mia} describing the leakage scheme}

For other methods see \cite{vanRiper:1981mko,Ruffert:1995fs,Rosswog:2003rv,OConnor:2009iuz,Sekiguchi:2010ep,Neilsen:2014hha,Perego:2015agy,Ardevol-Pulpillo:2018btx}.

Neutrino transport is of prime importance for shaping the composition of the ejecta
\cite{Wanajo:2014,Sekiguchi:2015dma,Foucart:2015vpa,Foucart:2015gaa},
affecting the nucleosynthesis in the ejecta \cite{Wanajo:2014,Goriely:2015fqa} and the thermal
electromagnetic counterpart, Kilonova (Macronova) \cite{Metzger:2014ila,Lippuner:2015gwa}


In this subsection a brief overview of the neutrino radaition transport, implemented in the 
\texttt{WhiskyTHC} is outlined.

We begin by considering the so-called gray (energy-averaged) leackage scheme that descibes the 
neutrino cooling.
This scheme is very popular in core-collapse supernovae models as well as binary neutron star models.
\cite{vanRiper:1981mko,Ruffert:1995fs,Rosswog:2003rv,OConnor:2009iuz,Perego:2015agy}


\subsubsection{Leackage scheme}

During the neutron star mergers, the thermodynamic conditions are such that the powerfull 
neutrino bursts can originate from the hot and shock heated NS matter \cite{Sekiguchi:2011zd}.
At high densities and temperatures reaching several MeV, the weak interaction become increasingly important, moving the material away fron the original chemical equilibrium with respect to the 
$\beta$-processes, and emission of numerous neutrinos with neutrino luminocity reaching $\sim10^{53}$erg s $^{-1}$. It was believed that such strong neutrino flux can power a GRB, however the baryon pollusion of the polar region found in BNS simulations might make it difficult \red{[refs]}.

Here the "neutrino leakage" scheme, implemented in the \texttt{WhiskyTHC} is briefly described following \cite{Galeazzi:2013mia}.
The leakage scheme allows to estimate the effect of neutrino radiation transport, specifically tracking the evolution of the local lepton number and the association energy loss via neutrino radiation.

Neutrino interactions depend on the matter composition, its density and temperature and on the neutrino energies. For instance, at rest-mass density $10^{12}$\gcm and temperature $\sim10$~MeV, the neutrino scattering on matter becomes so efficient, that they fall into the thermal equilibrium with it. 
The mean free path of these neutirnos becomes of order of $\sim 50$~m. These neutrinos are considered trapped.

At low densities, $<10^{11}$~\gcm, neutrinos with energy $<10$~MeV are no longer coupled to matter and their mean free path can reacn tens of kilometers. Such neutrinos are considred free-streaming.

If there is a sharp density gradient present, \textit{e.g.,} a surface of a neutron star where density falls by several orders of magnitude, then the neutrinos can be effectively divided into trapped and free-streaming. The transition region, however, is more difficult to treat and requires more accurate neutrino treatment, \textit{e.g.,} Monte Carly methods for solving Boltzmann equations \cite{Abdikamalov:2012zi}. Another alternative is the approximate, "ray-by-ray" \cite{Scheck:2007gw} multi-energy-group neutrino schemes [\textit{e.g.,} multigroup fluxlimited diffusion \cite{Mezzacappa:1993gn} and isotropic diffusion source approximation \cite{Liebendoerfer:2007dz}].

\gray{
    Notably, the $E == n_{\alpha}n_{\beta}T^{\alpha\beta}$, 
    $S_i = -\gamma_{i\alpha}n_{\beta}T^{\alpha\beta}$ and $S_{ij} = \gamma_{i\alpha}\gamma_{j\beta}T^{\alpha\beta}$ are the matter source terms where $n_{\alpha}=(-\alpha, 0, 0, 0)$ is the future pointing four-vector orthonormal to the space-like hypersurface, $S=S_{i}^{i}$ is the trace.
}

GR is evolved following BSSNOK using \texttt{McLachlan} code.

GRHD is a solved using \texttt{Whisky} code that considerers perfiect compressible fluid with an additional source terms to account for the composition, energy and momentum changes due to neutrino radiation.

Since (trapped) neutrinos are assumed to be in equilibrium with the baryonic matter, the number density and energy distribution of neutrinos is not evolved -- their contribution to the source terms $\Psi^{\beta}$ and $N$ is obtained from the matter properties.
\gray{only the electron neutrinos are considered there, and only their degrees of freedom is accounted for by the electron fraction $Y_e=n_e/n_b$.
The electron fraction is changed only by the neutrinos (antineutrinos) according to the source term $N$.}



The foundation of the scheme described here is presented in \cite{Galeazzi:2013mia}.
The method is similar to that of the \cite{Ruffert:1995fs}.
And, as in the \cite{Rosswog:2003rv}, the opacities are computed on the basis local thermodynamical equilibrium chemical potential for the neutrinos.


The goal of the leackage scheme is to describe a series of effective emissivities $R^{\text{eff}}$ and $Q^{\text{eff}}$ for electron neutrinos, $\nu_e$, anti-electron neutrinos $\bar{\nu}_e$ and the heavy-lepton neutrinos, which we collectively label as $\nu_x$.
The optical depth is then used to reduce the intrinsic emissivites, mimicking the effect of the diffusion of radiation from the optically thick regions.
This scheme also includes the heating effects by the free-streaming neutrinos, \textit{i.e.,} neutrino absorption. \gray{ which was shown to be important for altering the ejecta composition }.


\begin{table}
    \caption{
        Weak reactions employed in our simulations and references for their implementation.
        In the left column, $\nu \in \{\nu_e, \bar{\nu}_e, \nu_{x}\}$ denotes any neutrino species, 
        $\nu_{x}$ any heavy-lepton neutrinos, $N \in\{n, p\}$ a nucleon, and $A$ any nucleus.
        In the central column the role of each reaction is highlighted, with "P" standing for 
        production, "A" for absorption opacity and "S" for scattering opacity. When two roles are
        indicated, the second refers to the inverse ($\leftarrow$) reaction.}
    \label{tab:leakage}
    \begin{center}
        \begin{tabular}{lll}
            \hline\hline
            Reaction & Role &  Ref. \\ 
            \hline
            $p + e^- \leftrightarrow \nu_e + n $          & P,A & \cite{Bruenn:1985}  \\
            $n + e^+ \leftrightarrow \bar{\nu}_{e} + p $  & P,A & \cite{Bruenn:1985}  \\
            $e^+ + e^- \rightarrow \nu + \bar{\nu}$       & P & \cite{Ruffert:1995fs} \\
            $\gamma + \gamma \rightarrow \nu + \bar{\nu}$ & P & \cite{Ruffert:1995fs} \\
            $N + N \rightarrow \nu + \bar{\nu} + N  + N$  & P & \cite{Burrows:2004vq} \\
            $\nu + N \rightarrow \nu + N$                 & S & \cite{Ruffert:1995fs} \\
            $\nu + A \rightarrow \nu + A$                 & S & \cite{Shapiro:1983du} \\
            \hline\hline
        \end{tabular}
    \end{center}
\end{table}

%% Leackage
%The change in material composition ($Y_e$) is treated according to the leakage scheme \cite{Galeazzi:2013mia,Radice:2016dwd}. 

The scheme consideres three neutrino species: electron neutrino $\nu_e$, electron antineutrino $\bar{\nu}_e$ and an single species $\nu_x$ for heavy-lepton neutrinos.
The reactions traced by the scheme are listed in the table \ref{tab:leakage}.

Following the sources listed in the table, the neutrino production rate $R_{\nu}$ for $\nu\in\{\nu_e,\bar{\nu}_e,\nu_x\}$ are computed alongside the respective energy release $Q_{\nu}$ and
opacities: $\kappa_{\nu;a}$ and $\kappa_{\nu;s}$ for absorption and scattering respectively.

Neutrinos are assumed to obey the Fermi-Dirac distribution. Their chemical potential is governed by the $\beta$-equilibrium with thermal neutrinos \cite{Rosswog:2003rv}.
Thus, for the calculation of the opacities, local thermodynamical equilibrium chemical potential
is assumed for the neutrinos.

Additionally, there are two types of opacities, the density weighted opacities $\kappa_{\nu;a}^0$ and $\kappa_{\nu;s}^0$ and energy density weighted opacities $\kappa_{\nu;a}^1$ and $\kappa_{\nu;s}^1$. 
The former are related to the rate at which neutrinos escape the material, while the latter set the rate 
at which energy escapes the material as neutrinos escape \cite{Ruffert:1995fs}.

The optical depth, $\tau_{\nu}^{\alpha}$ is computed taking into account total neutrino opacities $\kappa_{\nu;a}^j + \kappa_{\nu;s}^j$ \cite{Neilsen:2014hha}.

Optical depth is then used to estimate the effective emission rates \cite{Ruffert:1995fs}

\begin{equation}
    R_{\nu}^{\text{eff}} = \frac{R_{\nu}}{1 + t_{\text{diff}}^0(t^0_{\text{loss}})^{-1}}
    \label{eq:method:whisky:Rnueff}
\end{equation}

whete $t_{\text{diff}}$ is the diffusion time

\begin{equation}
    t_{\text{diff}}^{0} = \mathcal{D}\frac{(\tau_{\nu}^0)^2}{\kappa_{\nu;a}^0 + \kappa_{\nu;s}^0}
\end{equation}

and the neutrino emission timescale 

\begin{equation}
    t_{\text{loss}}^0 = \frac{R_{\nu}}{n_{\nu}}
\end{equation}

and $n_{\nu}$ is the neutrino number density estimated based on the beta equilibrium with \red{(thermal?)} neutrinos.
The $\mathcal{D}$ is a tuning parameter set to $6$.

Similarly, the effective energy emission rate $Q_{\nu}^{\text{eff}}$ are computed with $\tau_{\nu}^1$, $\kappa_{\nu;a}^1$ and $\kappa_{\nu;s}^1$.

Further, the neutrinos are divided into two categories. 
The neutrinos that cannot escape, trapped, 
The neutrinos that escape according to the effective rate $R_{\nu}^{\text{eff}}$,
with and average energy $U_{\nu}^{\text{eff}}/R_{\nu}^{\text{eff}}$ are the free streaming neutrinos $n_{\nu}^{\text{fs}}$. 
These neutrinos are treated afterwards according to the M0 scheme \cite{Radice:2016dwd}.

The rest are the trapped neutrinos. Their effect on the pressure is neglected as it was shown to be weak 
in the NS conditions \cite{Galeazzi:2013mia}.

%% M0 Appendix A from Radice:2016dwd, Neutrin transport details
\subsubsection{The Boltzmann equations for Free-Streaming Neutrinos}

Neutrinos are considered as massless particles, propagating in the fluid
with four-velocity is $u^{\alpha}$.
The neutrino four-momentum, $p^{\alpha}$, can be decomposed as \cite{Thorne:1981}

\begin{equation}
    p^{\alpha} = (-p_{\beta}u^{\beta})(u^{\alpha} + r^{\alpha}),
\end{equation}

where $E_{\nu}=-p_{\alpha}u^{\alpha}$ is the neutrino energy measured by \red{Eulerian} observer,
comoving with the fluid, $r^{\alpha}$ is the unit space-like vector, orthogonal to the 
fluid's $u^{\alpha}$, or in other words $r_{\alpha}r^{\alpha}=1$ and $u_{\alpha}r^{\alpha}=0$.
Additionally, $r^{\alpha}$ can be interpreted as the direction along which neutrinos move,
as seen by the \red{Eulerian} observer comoving with the fluid.

The four-vector of the neutrinos is

\begin{equation}
    \label{eq:method:whisky:neut:k}
    k^{\alpha} = u^{\alpha} + r^{\alpha}
\end{equation}

with $k_{\alpha}k^{\alpha} = 0$.

Next, the affine parameter, that parameterizes the neutrino's worldline is introduced

\begin{equation}
    l = \int (-p_{\alpha u^{\alpha}})\dd s \text{ so that } \Big( \frac{\partial}{\partial l} \Big)^{\alpha} = k^{\alpha}.
\end{equation}

Now, with $l$, the Boltzmann equations for the neutrino radiation transport read \cite{Thorne:1981}

\begin{equation}
    \frac{D F}{D l} = \mathbb{C}[F]
\end{equation}

where $F$ is the distribution fuction for a give neutrino species, and $\mathbb{C}$ is the
'collisional operator' \red{See Rdice Thesis} the contains information regarding the interactions between neutrinos and the background fluid (in the frame of the fluid). 
The $D/Dl$ is the total derivative in phase space along $p^{\alpha}$ and it reads 

\begin{equation}
    \label{eq:method:whisky:neut:bolzeq}
    \frac{DF}{Dl} = k^{\alpha} \Big[ \frac{\partial F}{\partial x^{\alpha}} - \Gamma^{\delta}_{\:\:\alpha\beta}p^{\beta}\frac{\partial F}{\partial p^{\delta}} \Big],
\end{equation}

where $\Gamma^{\delta}_{\:\:\alpha\beta}$ are the Christoffel symbols.


\subsubsection{Neutrino number density evolution}

Neutrinos are split into two categories. 
Trapped neutrinos are treated with leackage scheme.
Free streaming neutrinos are evolved according to the moment scheme. Specifically, the propagation
of the neutrinos along the radial rays is considered tracking the change in their average energy.

Free streaming neutrino evolve according to the Boltzmann equation \eqref{eq:method:whisky:neut:bolzeq}.
The source term is given by the neutrino leakage scheme, taking the effective emissivity from it.
The collisional term is approximated in a way that it only includes neutrino absorption and emission. Scattering is neglected.

For the evaluation of the absorption opacities, (of $\nu_{e}$ and $\bar{\nu}_{e}$) the local thermodynamic equilibrium is assumed.

The neutrino number density (in the fluid rest frame) for neutrinos of a give flavor, $n_X$, can be expressed
through the neutrino number current $J_{X}^{\alpha}$ that reads \cite{Lindquist:1966},

\begin{equation}
    J_{X}^{\alpha} = \int F p^{\alpha} \frac{\dd^3 p}{-p_0}
\end{equation}

as $n_X = - u_{\alpha} J_{X}^{\alpha}$.

The balance equation (between absorption and emission of neutrinos) can be obtained from the 
first moment of the Boltzmann equation \cite{Thorne:1981,Shibata:2011kx}

\begin{equation}
    \label{eq:method:whisky:neut:balanseq}
    \nabla_{\alpha}J_{X}^{\alpha} = R_{X}^{\text{eff}} - \kappa_X n_X,
\end{equation}

where $\kappa_X$ is the absorption opacity and $R_X^{\text{eff}}$ is the effective neutrino emission rate.

While the equation \eqref{eq:method:whisky:neut:balanseq} is exact, in order to be solved, it requires closure. 
In this method the closure is given by considering neutrinos only propagating radially and at a speed of light, \textit{e.g.,}

\begin{equation}
    J_{X}^{\alpha} = n_X k^{\alpha},
\end{equation}

where $k^{\alpha}$ is the fiductial null vector, \eqref{eq:method:whisky:neut:k}, under the assumption that $r^{\alpha}$ is the radial null-vector, orthogonal to the fluid $u^{\alpha}$.
This translates into the assumption that the free-streaming neutrinos are moving radially in a frame instantaneously comoving with the fluid.
Then, the balance equation for $n_X$ reads

\begin{equation}
    \label{eq:mehtod:whisky:neut:balanseq2}
    \partial_t(\sqrt{-g}n_X k^t) + \partial_r(\sqrt{-g}n_X k^r) = \sqrt{-g}(R^{\text{eff}}_{X} - \kappa_X n_X)
\end{equation}

where $g$ is the determentatn of the $4$-metric \gray{(in spherical coordinates)}.

%% numerics
The eqution \eqref{eq:mehtod:whisky:neut:balanseq2} is solved on a series of independent radial rays using a first order, fully-implicit finite volume method.
Coupling with the hydrodynamics is done via interpolation at every timestep.
\gray{For our models 2048 radial rays uniformly spaced in latittude and longitude were used with radial resolution of $244$~m [CONFIRM!]}.


\subsubsection{Neutrino average energy evolution}

Computation of the neutrino average energy is required to evaluate the matter composition and temperature changes.
Here the additional assumtion is made, that the space-time is stationary, \textit{e.g.,} $t^{\alpha}:=(\partial_t)^{\alpha}$ is a Killing vector, which leads to the $(-p_{\alpha}t^{\alpha})$ to be a conserved quantity. Assuming that there is no interaciton with the fluid and the along the neutrino worldlines, \textit{e.g.,}

\begin{equation}
    \frac{\dd(-p_{\alpha}t^{\alpha})}{\dd l} = 0
\end{equation}

Then, the quentiy $\mathcal{E}_X = -p_{\alpha}t^{\alpha}$ is the energy of the neutrinos of species $X$ \gray{as seen by the coordinate observer, -- an unphysical observer with four-velocity $t^{\alpha}$}.

Consider 

\begin{equation}
    \mathcal{E}_X = -p_{\alpha}t^{\alpha} = -E_{X} k_{\alpha}t^{\alpha} =: E_X \chi
\end{equation}

then, the eqution for the average neutrino energy reads

\begin{equation}
    \frac{\dd \mathcal{E}_X}{\dd l} = \frac{R^{\text{eff}_X}}{n_X}\Big( \chi \frac{Q_X^{\text{eff}}}{R_{X}^{\text{eff}}} - \mathcal{E}_X \Big) 
\end{equation}

where $Q_{x}^{\text{eff}}$ is the effective neutrino energy source (taken from the leackages scheme).

For neutrinos radially moving, the equation reads

\begin{equation}
    n_X k^{t} \partial_t \mathcal{E}_X + n_{X} k^{r}\partial_{r}\mathcal{E}_X = (\chi Q_{X}^{\text{eff}} - \mathcal{E}_X R_X^{\text{eff}}).
\end{equation}

This eqution is solved on the same spherical grid using hte first order finite differencing method.

\subsubsection{Coupling between hydrodynamics and neutrinos}

Equations \eqref{eq:wthc:euler} and \eqref{eq:wthc:pndens} are solved via operator split approach.
The deposition of momentum by neutirnios is accounted for via the first-order explicit time update.
\red{FIND OUT WHAT EXPLICIT AND IMPLICIT TIMESTEP MEANS!}
The evolution of the $n_p$ and conserved energy is treaded with semi-implicit time-step \gray{update formula}.

for a single timestep, in the operator split formulism, the $n_p$ and conserved energy evolution update is given by the equation 

\begin{equation}
    \frac{\dd u}{\dd t} = f
\end{equation}

where $f$ can be large while, being the energy or a number density, $u$, should remain positive.
The treatment of EOS here is such, that the specific internal energy per baryon and the evolved energy density are always $>0$. However, when there is a strong cooling, $f\ll0$ and $u<|f|\Delta t$, 
the equation becomes stiff and aa update of $u$ required special treatment. 
The semi-implicit scheme is implemented as 

\begin{equation}
    \frac{u^{k+1}-u^{k}}{\Delta t} = \theta^k u^{k+1}
\end{equation}

where $\theta = f/u$.

As $k$ stands for the function $t=k\Delta t$,

\begin{equation}
    u^{k+1} = \frac{u^{k}}{1-\theta^k \Delta t}
\end{equation}

that assures the positvity of the solution.






%% M0 scheme
Computation of the $n_{\nu_e}$, $n_{\bar{n}_e}$, $E_{\nu_e}$ and $E_{\bar{\nu}_e}$ is accomplished via the zeroth momentum (M0) of the free-streaming neutrino distribution function on a set of individual radial rays, with the closure adopted to post-merger geometry.

In the M0 scheme the evolution of the number density of free steaming neutrinos is done under the assumption.
Neutrons are assumed to be moving along the radial null rays with four vector $k^{\alpha}$. The vector is normalized such that $k^{\alpha}u_{\alpha}=-1$. 
Then the number density of the free neutrinos in the fluid rest frame $n_{\nu}^{\text{fs}}$ follows \cite{Radice:2016dwd}

\begin{equation}
    \label{eq:method:whisky:eq7}
    \nabla_{\alpha}[n_{\nu}^{\text{fs}}k^{\alpha}] = R_{\nu}^{\text{eff}} - \kappa_{\nu;a}^{\text{eff}}n_{\nu}^{\text{fs}},
\end{equation}

where $R_{\nu}^{\text{eff}}$ is the effective luminosity (\red{emission }) \eqref{eq:method:whisky:Rnueff}. 

The effective absorption rate then

\begin{equation}
    \kappa_{\nu,a}^{\text{eff}} = e^{-\tau_{\nu}^0}\Big( \frac{E_{\nu}^{\text{fs}}}{E_{\nu}^{\beta}} \Big)^2 \kappa_{\nu,a}^0.
\end{equation}

where $E_{\nu}^{\text{fs}}$ is the average energy of the free-streaming neutrinos, and 
$E_{\nu}^{\beta}$ is the average energy on the neutrinos that are in $\beta$-equilibrium.
$E_{\nu}^{\text{fs}}$ and $E_{\nu}^{\beta}$ are defined in the restframe of the fluid.

The energy of the free-sctreaming neutrinos is computed assuming the \red{stationarity of the metric}.
Having the $\partial_t$ killing vector thus allows to have $p_{\nu}^{\alpha}(\partial_t)_{\alpha}$ conserved, 
where $p_{\nu}^{\alpha}$ is the neutrinos four-momentum.

Then, the average energy density of free-streaming neutirions obays

\begin{equation}
   \label{eq:method:whisky:eq9}
    k^t\partial_t(E_{\nu}^{\text{fs}}\chi) + k^{r}\partial_r(E_{\nu}^{\text{fs}}\chi) = \frac{\chi}{n_{\nu}^{\text{fs}}}(Q_{\nu}^{\text{eff}}-E_{\nu}^{\text{fs}}R_{\nu}^{\text{eff}}),
\end{equation}

where $\chi=-k^{\alpha}(\partial_t)_{\alpha}$.

%% Coupling, neutrinos and hydro
The coupling between the matter and neutirnos is done via operator split approach \cite{Radice:2016dwd}.
For equation \eqref{eq:wthc:pndens} reads

\begin{equation}
    R_p = (\kappa_{\nu_e;a}^{\text{eff}}n_{\nu_e}^{\text{fs}} - \kappa_{\bar{\nu}_e;a}^{\text{eff}}n_{\bar{\nu}_e}^{\text{fs}}) - (R_{\nu_e}^{\text{eff}} - R_{\bar{\nu}_e}^{\text{eff}}).
\end{equation}

For the Euler equation \eqref{eq:wthc:euler} reads

\begin{equation}
    Q = (\kappa_{\nu_e;a}^{\text{eff}}n_{\nu_e}^{\text{fs}}E_{\nu_e} + 
    \kappa_{\bar{\nu}_e;a}^{\text{eff}}n_{\bar{\nu}_e}^{\text{fs}}E_{\bar{\nu}_e}) - 
    (Q_{\nu_e}^{\text{eff}} + Q_{\bar{\nu}_e}^{\text{eff}} + Q_{\nu_x}^{\text{eff}}).
\end{equation}

The M0 scheme of \cite{Radice:2016dwd}, while less complex then frequency-integrated M1 schemes used by 
\cite{Sekiguchi:2015dma} and \cite{Foucart:2015vpa}, it is advantagous with respect to the computational efficiency.
It also includes approximations of the Doppler and gravitaional effects. 
Additionally, the unphysical radaition shocks above the merger remnant, that commonly present in M1 schemes \cite{Foucart:2018gis}, do not develop in M0 scheme.


\subsubsection{Viscosity}


This subsection is based on the GRLESS method paper \cite{Radice:2017zta}, its extension  \cite{Radice:2020ids}, and its application to the BNS \cite{Radice:2017lry} for dynamical ejecta and summary of this study in \cite{Radice:2018pdn}.

%% Motivation

The fluid flow inside remnants of binary neutron star mergers is expected to be turbulent. 
This is because the magnetohydrodynamics instabilities,
such as the Kelvin-Helmholtz (KH) instability and the magnetorotational instability (MRI), are  
activated at scales too small to be resolved in simulations.
%The effect of these instabilities can be assessed via large eddy simulation technique \cite{Radice:2017zta}, which allow to investigate the subgrid-scale turbulent transport in GR.

Large scale magnetic fields are of prime importance for studying the emergence of magnetically and \red{reconnection} driven outflows and collimated jets \cite{Rezzolla:2011da,Bucciantini:2011kx,Siegel:2014ita,Ruiz:2016rai,Metzger:2018uni}.

Additionally, \gray{global} magnetic fields, and magnetic stresses induced by them are crucial for the self-consistent treatment of the angular momentum transport in the merger remnant \cite{Duez:2006qe,Kiuchi:2014hja,Guilet:2016sqd,Kiuchi:2017zzg}.

Such effects are studied with high-resolution general-relativistic magnetohydrodynamics (GRMHD) simulations. However, despite rapid progress of GRMHD simulations \cite{Rezzolla:2011da,Kiuchi:2014hja,Ruiz:2016rai},
the degree to which the magnetoturbulence affects the structure and the lifetime of the remnant 
before collapse is still poorly constrained.
And while the magnetorotational instability (MRI) \cite{Balbus:1991} is believed to be present within the 
massive neutron stars, and is responsible for the redistribution of angular momentum, affecting the 
remnant lifetime, \cite{Duez:2006qe,Siegel:2013nrw}, the fastest growing modes of the MRI remain beyond 
reach even at very high resolutions \cite{Kiuchi:2014hja}.

Setting artificially large initial magnetic allows to raise the cutoff length scales associated with some of these instabilities and study their effects, but even then simulations fail to capture the dynamics of the turbulent cascade at the viscous scale, at which neutrino viscosity and drag damps the turbulent eddies \cite{Guilet:2016sqd}. 
% This, however, is of large importance for BNS merger simulations.

THere is a possibility of including the turbulent angular momentum transport via effective viscosity
\cite{Duez:2004nf}. 
This approach has theoretical limitations and numerical difficulties, such as the emergence of unphysical effects in the Navier-Stokes equations describing relativistic viscous flows (\textit{e.g.,} \cite{Hiscock:1985}).

Another possible alternative is to consider the effective viscosity via an effective model based on 
general relativistic extension of the Newtonian large-eddy simulation (LES) (\textit{e.g.,} \cite{Miesch:2015les}). The model does return Navier-Stokes equations in the Newtonian limit, but it is not a relativistic theory of viscous flows \cite{Radice:2017zta}.
In essence, general-relativistic large-eddy simulation, (GRLES), main idea is to evolve the coarse-grained GRHD equations with a turbulent closure models.
The results from these simulations, however, depend on the adopted subgrid model. 
These models are calibrated to capture the effect of turbulence operating at sub-grid scales,
using either the dimensional analysis and linear perturbation theory \cite{Radice:2017zta}, or a very high resolution GRMHD simulations where most of the relevant unstable scales of MRI are resolved (with however large initial magnetic field), \textit{e.g.,} \cite{Kiuchi:2017zzg}.

Recently a further facilitation of the mathematical basis behind the \gray{GRlES} method discussed here was published by Eyink and Drivas \cite{Eyink:2017zfz}.

It is however, not the only approach to study turbulence while avoiding ultra-high resolution GRMHD simulations. An approach based on the Israel-Stewart formalism was proposed by Shibata and collaborators \cite{Shibata:2017jyf}.
A method that extends further to GRMHD and includes more rigorous formulation \gray{with terms neglected here} was proposed in \cite{Carrasco:2019uzl,Vigano:2020ouc}. 
Machine learning was suggested as method to calibrate the subgrid turbulence for 2D MHD \cite{Rosofsky:2020}. 
A version of the GRLES approach was implemented into the Spectral Einstein Code (SpEC) for 2D axisymmetric simulations \cite{Jesse:2020oss}.


%% Method 
Next, we briefly summarize the GRLES model. We begin with the recalling the Valencia formalism of the GRHD \cite{Banyuls:1997}. The fluid four-velocity is represented as a sun of the vector on the hypersurface $t=\text{const}$, and a vector orthogonal to it, $n^{\mu}$ and reads

\begin{equation}
    u^{\mu} = (-u_{\mu}n^{\mu})(n^{\mu}+\upsilon^{\mu}) = W(n^{\upsilon} + \upsilon^{\mu})
\end{equation}

where $W$ is the Lorentz factor, $\upsilon^{\mu}$ is the fluid three-velocity.

The proton and neutron \red{currents} can be expressed as

\begin{eqnarray}
    J^{\mu}_n = n_n W(n^{\mu} + \upsilon^{\mu}) := D_n (n^{\mu} + \upsilon^{\mu}) \\
    J^{\mu}_p = n_p W(n^{\mu} + \upsilon^{\mu}) := D_p (n^{\mu} + \upsilon^{\mu})
\end{eqnarray}

respectively.


%To account for the effect of subgrid-scale turbulent angular momentum transport, 
%the general-relativistic large eddy simulations method (GRLES; \cite{Radice:2017zta}) is employed.
%We briefly review here the method.

Consider the stress energy tensor of a perfect fluid

\begin{equation}
    T_{\mu\nu} = \rho h u_{\mu} u_{\nu} + pg_{\mu\nu},
\end{equation}

where $\rho$ is the density, $h$ is the specific enthalpy, $u_{\mu}$ is the fluid four-velocity, and 
$g_{\mu\nu}$ is the spacetime metric.

Following the $3+1$ decomposition, 
where the spacetime is divided into space-like slices with 
the normal to the space-like slice hyper-surface, $n^{\mu}$,
the decomposition of the $T_{\mu\nu}$ with respect to the $n^{\mu}$ reads

\begin{equation}
    T_{\mu\nu} = En_{\mu}n_{\nu} + S_{\mu}n_{\nu} + S_{\nu}n_{\mu} + S_{\mu\nu}
\end{equation}

where the first term of the RHS is

\begin{equation}
    E = T_{\mu\nu}n^{\mu}n^{\nu} = \rho h W^2 - p
\end{equation}

and the last two,

\begin{eqnarray}
    & S_{\mu} = -\gamma_{\mu\alpha}n_{\beta}T^{\alpha\beta} = \rho h W^2 \upsilon_{\mu} \\
    & S_{\mu\nu} = \gamma_{\mu\alpha}\gamma_{\mu\beta}T^{\alpha\beta} = S_{\mu}\upsilon_{\nu} + p \gamma_{\mu\nu} \gray{ + \tau_{\mu\nu}}
\end{eqnarray}

where while $\gamma_{\mu\nu}$ is the spatial metric, $\upsilon^{\mu}$ is the three velocity and $W$ is the Lorentz factor, and $p$ is the pressure. 

Neglecting the neutrino source terms \gray{to simplify notations}, the equations of energy and momentum conservation, the GRHD equations, are 

\begin{eqnarray}
\label{eq:method:whisky:emomcons_lk}
    \partial_t(\sqrt{\gamma}D_n) + \partial_j\Big[ \alpha\sqrt{\gamma}(\upsilon^j + n^j)D_n \Big] = 0, \\
    \partial_t(\sqrt{\gamma}D_p) + \partial_j\Big[ \alpha\sqrt{\gamma}(\upsilon^j + n^j)D_p \Big] = 0, \\
    \partial_t(\sqrt{\gamma}S_i) + \partial_j\Big[ \alpha \sqrt{\gamma} (S_i^{\; j} + S_i n^j) \Big] = 
    \alpha \sqrt{\gamma}\Big( \frac{1}{2} S^{jk} \partial_i \gamma_{jk} \frac{1}{\alpha} S_k \partial_i \beta^k - E\partial_i \log(\alpha) \Big) \\
    \partial_t(\sqrt{\gamma}E) + \partial_j\Big[ \alpha \sqrt{\gamma} (S^{j} + E n^j) \Big] = 
    \alpha \sqrt{\gamma}\Big( K_{ij}S^{ij} - S^i\partial_i \log(\alpha) \Big) 
\end{eqnarray}

where $\alpha$ is the lapse function, $\beta^i$ is the shift vector, $\gamma_{ij}$ is the three metric and $K_{ij}$ is the extrinsic curvature, and $\sqrt{\gamma}$ is the spatial volume element.

These equations are then closed with the EOS and \red{Euler equations} for conservation of baryon and 
lepton numbers \red{\textit{e.g.,} eq.\eqref{eq:wthc:euler} and neutirno equation ???}.

However, while modes of all scales are present in the equations \eqref{eq:method:whisky:emomcons_lk}, 
only the 'resolved' modes can evolve in numerical simulations. 
\gray{In other works, in numerical applications the 'coarse-grained' version of hydrodynamic equations
is considered \cite{Radice:2017zta}.}.

Following the LES model, the linear filtering operator, $u\rightarrow \bar{u}$ is introduced, that 
discards the modes \gray{features} below \gray{smaller} a given scale $\Delta$.
As these equations are disctitized via finite volume scheme, the cell-averaging operator was chosen to
perform filtering. These modifies the equations \eqref{eq:method:whisky:emomcons_lk}, as
$S\rightarrow\bar{S}$, $E\rightarrow\bar{E}$.

Assuming that the metric, a being a large scale quantity, is not changed during the averaging, the resulted system of averaged equations reads \cite{Radice:2017zta}

\begin{eqnarray}
    \label{eq:method:whisky:emomcons_lk_filt}
    \partial_t(\sqrt{\gamma}\overline{D_n}) + \partial_j\Big[ \alpha\sqrt{\gamma}(\overline{D_n\upsilon^j} + \overline{D_n}n^j) \Big] = 0, \\
    \partial_t(\sqrt{\gamma}\overline{D_p}) + \partial_j\Big[ \alpha\sqrt{\gamma}(\overline{D_p\upsilon^j} + \overline{D_p}n^j) \Big] = 0, \\
    \partial_t(\sqrt{\gamma}\overline{S_i}) + \partial_j\Big[ \alpha \sqrt{\gamma} (\overline{S_i^{\; j}} + \overline{S_i} n^j) \Big] = 
    \alpha \sqrt{\gamma}\Big( \frac{1}{2} \overline{S^{jk}} \partial_i \gamma_{jk} \frac{1}{\alpha} \overline{S_k} \partial_i \beta^k - \overline{E}\partial_i \log(\alpha) \Big) \\
    \partial_t(\sqrt{\gamma}\overline{E}) + \partial_j\Big[ \alpha \sqrt{\gamma} (\overline{S^{j}} + \overline{E} n^j) \Big] = 
    \alpha \sqrt{\gamma}\Big( K_{ij}\overline{S^{ij}} - \overline{S^i}\partial_i \log(\alpha) \Big) 
\end{eqnarray}

Notably, the equations are not closed. 
The equations are non-linear, and $\overline{D_{n}}$, $\overline{D_p}$, $\overline{S_i}$ and $\overline{E}$ are not sufficient to express all the terms.
A closure is thus required.

\begin{eqnarray}
    \bar{S_i\upsilon_j} &= \bar{S_i}\bar{\upsilon_j} + \tau_{ij}, \\
    \overline{D\upsilon^i} &= \overline{D}\overline{\upsilon^i} + \mu^i
\end{eqnarray}

where $\tau_{ij}$ is the so-called subgrid-scale turbulence tensor \cite{Radice:2017zta} or stress,
and $\mu^i$ is the sub-scale rest-mass diffusion.

Notably, these terms are intrinsic to numerical discretization of the GRHD equations. 

Additionally, the three velocity $\overline{\upsilon^i}$ is a non-linear function of the filtered quantities, and thus requires closure. So does the pressure, as the adopted equations of state are non-linear:

\begin{equation}
    \overline{p} = p(\overline{D_{n,p}},\overline{S_i},\overline{E}) + \Pi.
\end{equation}

In the considered formulation \cite{Radice:2020ids}, these corrections are neglected on the basis of post-merger dynamics having subrelativistic and subsonic turbulence that can be effectively described by $\tau_{ij}$.

However, see \cite{Carrasco:2019uzl,Vigano:2020ouc} for a more general treatment.

% When $\tau_{ij}=0$, there is no subgrid turbulence. 
following the analogy with Newtonian closure of \cite{Smagorinsky:1963}, the $\tau_{ij}$ can be computed \cite{Radice:2017zta}
 
\begin{align}
    \tau_{ij} = &-2\nu_T\rho h W^2\Big[ \frac{1}{2}(\nabla_i\overline{\upsilon_j} + \nabla_j\overline{\upsilon_i}) - \frac{1}{3}\nabla_k\overline{\upsilon^k}\gamma_{ij} \Big] = \\
    &-2 \nu_T (\epsilon + p)W^2\Big[ \frac{1}{2} (\nabla_i\overline{\upsilon_j} + \nabla_j\overline{\upsilon_i}) - \frac{1}{3}\nabla_k\overline{\upsilon^k}\gamma_{ij} \Big]
\end{align}

where $\nabla$ is the covariant derivative compatible with $\gamma_{ij}$, which is the spatial metric, and $\nu_T = l_{\text{mix}}c_s$ is the turbulent viscosity, expressed through the mixing length $l_{\text{mix}}$, which is characteristic length scale of turbulence, and characteristic velocity, $c_s$, which is the sound speed.

%The $\tau_{\mu\nu}$ is a purely spatial tensor representing the effect of the subgrid turbulence. 
%It reads \cite{Radice:2017zta}

% \begin{equation}
%    \tau_{ij} = -2\nu_T(\rho + p)W^2 \Big[ \frac{1}{2}(D_i\upsilon_j + D_j\upsilon_i) - \frac{1}{3}D_k\upsilon^k\gamma_{ij} \Big]
%\end{equation}

%where $\nu_T = l_{\text{mix}}c_s$ is the turbulent viscosity, $c_s$ is the sound speed.
%The $D_i$ here are the \red{covariant derivatives compatible with spatial metric}. 

The mixing length parameter, $l_{\text{mix}}$, is related to the length over which the effects of turbulence are present. 

Together, enegery and momenta consideration equations with averaging operator, subgrid-scale turbulence tensor, equation for $l_{\text{mix}}$, EOS and continuity equations comprise the GRLES system.


The $\nu_T$ is not a physical viscosity and by definition it relates to the numerical grid and 
Eulerian observer $n^{\mu}$. 
\gray{Indeed, in relativity, for a certain scale to be resolved or not depends on the observer}.
%$\nu_T$ can be calibrated based on high resolution simulations \red{EXTEND Here To 2020 Paper}

%is a free parameter that can be varied to study the impact of the turbulence on the results. \red{Alternatively it can be set as a function of density. 
%See Radice Paper}

%For now $l_{\text{mix}}$ is free parameter.

With respect to the MRI, it is natural to set $l_{\text{mix}} \sim \lambda_{\text{MRI}}$, where $\lambda_{\text{MRI}} \sim \Omega^{-1}B$ with $\Omega$ being the angular velocity and $B$ is the magnetic field strength \cite{Duez:2006qe}.
Here $\lambda_{\text{MRI}}$ describes the scale over which turbulence is predominantly driven according to linear theory.


Viscous flows in accretion disks is often described in terms of a dimensionless constant $\alpha$
 (so called $\alpha$-viscosity???) that is related to the mixing length as 

\begin{equation}
    l_{\text{mix}} = \alpha c_s \Omega^{-1}
\end{equation}

where $\Omega$ is the angular velocity of the fluid \cite{Shakura:1972te}.

The value of $\alpha$ can be constrained by very high resolution GRMHD simulations with seed magnetic field $(10^{15}~G)$ strong enough that the magnetorotational instability (MRI) within the remnant is resolved. 
Such simulation, performed by \cite{Kiuchi:2017zzg} yielded average values of $\alpha$ for various rest-mass density shells.
Together with the sound speend and angular velocity, the mixing length can thus be estimated \cite{Radice:2020ids}, (and Fig.1 there) 

\begin{equation}
    l_{\text{mix}} = 
    \begin{cases}
        \alpha \xi \exp(-|b\xi|^{5/2}) \: [m], \: &\text{ if } \xi > 0, \\
        0, &\text{ otherwise }
    \end{cases}
\end{equation}

where 

\begin{equation}
    \xi = \log_{10}\Big( \frac{m_p(n_p + n_n)}{\rho^*} \Big)
\end{equation}

with the constants $a$, $b$ and $\rho^*$ are constants.

It was observed that even for a highly magnetized binary, simulated by \cite{Kiuchi:2017zzg}, the $l_{\text{mix}}$ is rather small. The turbulence appear weaker within the merger remnant at higher densities, as the angular velocity, growing with radius, stabilizes the flow against MRI (\textit{e.g.,} \cite{Radice:2017lry}). 
For lower densities, $\rho<10^{10}$~\gcm, the $l_{\text{mix}}$ is also decreasing, but due to the fitting procedure, the log-linear extrapolation into the region where the $\alpha$ values are not provided by \cite{Kiuchi:2017zzg}.
The turbulence appear the strongest in the NS mantel at densities between $10^{9}$~\gcm and $10^{13}$~\gcm,
\red{the region that we would later call Disk???}







%Together with the $c_s$ and $\Omega$, form \red{our} simulations, the $l_{\text{mix}}\in(0,30)$~m \cite{Radice:2018pdn}

%Additionally, the $l_{\text{mix}})$ can be computed from ....

%In out models the $l_{\text{mix}}\in(0,30)$ is computed according to \red{New Radice Ppaer}

%% Method for inculding the Viscosity term.
The direct application of finite volume Godunov-type methods to discretize the stress-energy tensor 
that includes $\tau_{\mu\nu}$ would lead to the development of the \red{odd-even} decoupling instability \cite{Lowrie:2002}.
\texttt{WhiskyTHC} avoids this problem by discritizing the terms araising from the derivatives of $\tau_{\mu\nu}$
in a flux-conservative fashion via proper combination of left and right biased finite-differencing operators \cite{Radice:2018pdn}.

In the \cite{Radice:2017lry} the effect of the subgrid turbulence on the dynamical ejecta was investigated, where $l_{\text{mix}}$ was a free parameter, varying between $0$ and $50$~m. 

Simulations performed with the code that incorporates these equations are referred to as simulations
with viscosity.

The code that we employ, \texttt{WhiskyTHC} does not have magnetic fields. 
However, the possible effects that angular momentum transport might have on the simulation evolution
can be investigated via an inclusion of \textit{effective viscosity}.
This method has been shown to reproduce main features of MHD dynamics with application to post-merger accretion
disks \cite{Fernandez:2018kax}.

\paragraph{title}


\subsection{Setup}



The simulation domain is a cube of $3.024$~km each side, whose center is at the center of mass pf the binary.
The AMR structure has $7$ refinemnt levels, with the finest convering both compact objects during the inspiral and the remnant postmerger.

We consider several resolution setups. Low resolution (LR) simulations have $h=246$~m, standard resolution (SR) 
have $h=185$~m and high resolution (HR) $h=123$~m for the final refinemnt level.

In the simulations where the neutirno M0 scheme is included, it is switched on shortly before the merger. 
The equations \eqref{eq:method:whisky:eq7} and \eqref{eq:method:whisky:eq9} are solved on the uniform spherical grid
with radius $\approx 756$~km, and resolution $n_r\times n_{\theta}\times n_{\phi} = 3096 \times 32 \times 64$
grid points.

A subset of models discussed in this thesis include the effective treatment of viscosity. 

We consider \red{$33$} distinct binary with total masses $\red{[None,None]}$ and mass-ratio $q\in[1.00,1.82]$.
In all models the neutrino leackage plus M0 scheme. Most models were computed at at least two resolutions. 
Most our models also include the effect of subgrid turbulence, viscosity.

\gray{Summary of all results in given in the table...}

\gray{Each run is nameed as}

\gray{We simulate each model for at least $\red{None}$~ms after the merger or a few milliseconds after BH formation}


 
%% ================================================================= APPENDIXES 
\appendix
%% \include{appendix_gr} %% GR

%% ================================================================= REFERENCES
\backmatter
\bibliography{../references}


%% ========================= END
\end{document}

























\documentclass[11pt,a4paper,headinclude=true,DIV=14,BCOR=8mm,chapterprefix,listof=totoc,twoside,openright,abstracton]{scrbook}
%% \documentclass[11pt,a4paper,headinclude=true,DIV=14,BCOR=8mm,chapterprefix,listof=totoc,twoside,openright,abstracton]{scrbook}

\usepackage[headsepline]{scrpage2}
\usepackage[utf8]{inputenc}
\usepackage{geometry}
\usepackage{amssymb}
\usepackage{amsthm}
\usepackage{enumerate}
\usepackage{graphicx}
\usepackage{float}
\usepackage[intlimits]{amsmath}
% \usepackage{siunitx}
% \usepackage{color}
\usepackage{xcolor}
\usepackage{verbatim}
\usepackage{appendix}
\usepackage{hyperref}
\usepackage{hyperref}
\usepackage{mathtools}
% \usepackage[style=authoryear]{biblatex}
\usepackage{natbib}
% \usepackage{newtxtext}
% \usepackage{newtxmath}
% \usepackage{harvard}
\setcitestyle{aysep={}} 
\bibliographystyle{apalike}
\usepackage{xr}
\usepackage{wrapfig}
% \bibliographystyle{agsm}
%\usepackage{feynmf}
%\usepackage{tensor}
\usepackage[framemethod=tikz]{mdframed} % for a block of text

\setlength{\parindent}{0pt}
\geometry{a4paper, tmargin=3cm, bmargin=3cm, lmargin=3cm, rmargin=3cm, headheight=3em, headsep=2em, footskip=1cm}
%% \geometry{a4paper, tmargin=3cm, bmargin=3cm, lmargin=3cm, rmargin=3cm, headheight=3em, headsep=2em, footskip=1cm}

\setcitestyle{citesep={,}}

\newcommand{\todo}[1]{\textcolor{red}{$\blacksquare$ TODO: #1}} 
\newcommand{\red}[1]{\textcolor{red}{#1}} 
\newcommand{\gray}[1]{\textcolor{gray}{#1}} 
\newcommand{\magenta}[1]{\textcolor{magenta}{#1}} %% For terms/concepts to remember 

\newcommand{\swind}{spiral-wave wind}
\newcommand{\nwind}{$\nu$-component}

\newcommand{\gcm}{g cm$^{-3}$}

\newmdenv[linecolor=cyan,backgroundcolor=cyan!20]{sidenote}

% Display the argument if \cond is True/False
%\newcommand{\IfCond}[2]{
%    \ifnum\pdfstrcmp{\cond}{True}=0
%    \ifnum\pdfstrcmp{}{#1}=0\unskip\else#1\fi%
%    \else
%    \ifnum\pdfstrcmp{}{#2}=0\unskip\else#2\fi%
%    \fi\ignorespaces
%}
%\newif\ifanswers
%\answerstrue % comment out to hide answers

% play the argument if \cond is False
%\newcommand{\IfCondFalse}[1]{\ifnum\pdfstrcmp{\cond}{True}=0 \unskip\else #1\fi\ignorespaces}
\usepackage{etoolbox}
\newtoggle{Full}
% \toggletrue{Full}
\togglefalse{Full}

%% \geometry{a4paper, tmargin=2cm, bmargin=2cm, lmargin=1cm, rmargin=1cm, headheight=2em, headsep=2em, footskip=1cm}

\title{PhD thesis}
\author{Vsevolod Nedora}
\date{today}

\begin{document}

\maketitle

%% =======================
%%
%% MAIN
%%
%% =======================

\mainmatter

\begin{center}
    \textbf{Abstract} \\[1cm]
\end{center}
\todo{Write me when you are ready, son!}

%% ======================
%%
%% Planned STRUCTURE
%%
%% Part 1: Numerical relativity simulations of neutron star mergers
%%         Ch.1. Introduction
%%         Ch.2. Theoretical Background
%%               Sec.1. General-relativistic hydrodynamics
%%               Sec.2. Numerical Approximations of conservation laws
%%               Sec.3. High order numerical methods for Rel.Hydro.
%%               Sec.4. Relativistiv Radiation transport
%%               Sec 5. Microphysical equation of starts of nuclear matter
%%         Ch.3. Numerical relativity simulations
%%               Sec.1. Evolution code and Initial Data code
%%               Sec.2. Data analysis methods
%%         Ch.4. Results and Conclusions
%% Part 2: Electromagnetic counterparts to neutron star mergers
%%         Ch.1. Introduction
%%         Ch.2. Theoreticak Background
%%               Sec.1. Nucleosynthesos in binary neutron star ejecta
%%               Sec.2. Kilonova
%%               Sec.3. Short Gamma Ray Burst afterglow
%%               Sec.4. Ejecta aftergkiw
%%         Ch.2. Methods for modelling Kilonova and non-thermal afterglows
%%               Sec.1. MKN code
%%               Sec.2. PyBlast code (my code)
%%         Ch.3. Results and Conclusions
%% ======================

%% =======================
%%
%% PART 1
%%
%% =======================

%% %% ==============================================================================
%% ==============================================================================
%% ==============================================================================
%%
\part{Numerical relativity simulations of neutron star mergers}
%% \label{sec:part1}
%%
%% ==============================================================================
%% ==============================================================================
%% ==============================================================================

%% In this part to discuss
%% GR Hydro 
%% Numerical methods for GR Hydro
%% Radiation
%% M0 scheme for neutrinos
%% WhiskyTHC & Lorene
%% GW1708017 targeted models
%% Remannt/Disk dynamics of model 


%% ===========================================================================
%%
%% Intorduction
%%
%% ===========================================================================

\chapter{Introduction}

\todo{Write me}

%% ===========================================================================
%%
%% Theoretical Background
%%
%% ===========================================================================

\chapter{Numerical Relativity} %% [ Based on the thesis of David Radice ]

%% ===========================================================================
%%
%%

In this thesis we perform and analyze numerical relativity simulations of merging neutron stars. These simulations are performed via solving the equations of general relativity, hydrodynamics and radiation, neutrino, transport via special numerical schemes. 

In this chapter we provide a brief description of the main equations and methods used to produce simulations analyzed in this thesis. For the sace of bravity we limit the discussion to the main results and implication important for our work.

For the underlying principles of the Eintein's theory of General Relativity, for which we here the reado to \red{[GR refs]}.

For the discussion and derivation of general relativistic hydrodynamics and refer the interested reader to \red{[GRHD refs]}.

For the Discussion on the radiation transport we refer to \red{GR-Rad refs}


%% ===========================================================================

\section{Basics of numerical relativity}

%% ===========================================================================






\section{Radiation}




\subsection{Neutrino leackage scheme}



\subsection{Neutrino M0 scheme}



\section{Initial Data}

\red{To explain: 
    pseudo-spectral code; 
    neutrino-less beta-equilibrium}

The initial data is computer via pseudo-spectral code \texttt{Lorene} \citep{Gourgoulhon:2000nn}
The code generates the binary neturon stars in a qusi-circular orbit, that revolve only few times before merging.
The EOS for the minimum temperature slice of the EOS table used for the evolution assuming neutrino-less beta-equilibrium.
Assuming constant temperature, we also remove the photon energy contribution to the pressure.

In the evolution code, passing the initial data, the mapping is done from the zero tem
In the evolution code, the electron fraction is set by the beta equilibrium condition. 
The specific internal energy is reset in accoradance with minimum temperature slice of the EOS table used for evolution.

Errors present in the initial data in intriduced during the mapping result in a small oscillations of netron stars.
In terms of relative changes in central density these amounts to $\sim2-3\%$ \cite{Radice:2018pdn}

\section{WhickyTHC}

\texttt{WhiskyTHC} is ...
\cite{Radice:2012cu,Radice:2013xpa,Radice:2013hxh,Radice:2015nva}


%% GR Hydrodynamics
The code evolves the proton and neutron number densities, $n_n$ and $n_p$
respectively, as 

\begin{equation}
\label{eq:wthc:pndens}
    \nabla_\nu (n_p u^\mu) = R_p^\mu \ \ , \ \ 
    \nabla_\nu (n_n u^\mu) = R_n^\mu \ .
\end{equation}

Here $u^{\mu}$ is the fluid four-velocity, $R_p = -R_n$ is the net
lepton number deposition rate due to the absorption and emission of neutrinos 
and antineutrinos (\red{see Section XXX})

The number densities are related as $n_p=Y_e n$ where $n = n_p + n_e$ is the baryon 
number density and $Y_e$ is electron fraction.

The matter of a neutron star is approximated with ideal fluid with stress-energy tensor

\begin{equation}
T_{\mu\nu} = \rho h u_{\mu} u_{\nu} + Pg_{\mu\nu}
\end{equation}

where $\rho=m_{\rm b} n$ is the baryon rest-mass density, 
$n$ the baryon number density, $m_{\rm b} \simeq 10^{-24}\,$g 
the neutron mass, 
$h=1+\epsilon + P/\rho$ the specific enthalpy, 
$\epsilon$ the specific internal energy (energy density),
and $P$ is pressure

The Euler equation for balance of energy and momentum reads

\begin{equation}
\label{eq:wthc:euler}
\nabla_\nu T^{\mu\nu} = Q u^\mu \ ,
\end{equation}

where $Q$ is the net energy deposition rate doe to absorption
and emission of neutrinos (\red{see Section XXX})


%% Numerical methods
High resolution shock capturing methods are used to discritize equations 
\eqref{eq:wthc:euler} and \eqref{eq:wthc:pndens}.
Specifically, central Kurganov-Tadmor type scheme \cite{Kurganov:2000} with 
HLLE flux formula \cite{Einfeldt:1988}
and non-oscillatory reconstruction of the primitive variables with the MP5 scheme of
\cite{Suresh:1997}.

Shock capturing schemes require the presence of a low density atmosphere around neutron stars.
The constant value of $\rho_0 = m_p n \approx 6\times 10^4$~\gcm.


\begin{sidenote}
    \textbf{David:}  \\
    
    I've been reading that for high-resolution shock-capturing, the finite differencing techniques are more efficient and simpler in implementation. However, most codes for MHD and radiation MHD that I find are using finite-volume methods. I am very curious why?..
    
    THC actually has both FD and FV schemes implemented
    FV is exactly conservative and there is a better way to do AMR with it
    FD is better because it is much simpler at higher order
    with THC when we want to do high order precision things we use FD
    when we do messy simulations with microphysics, for which robustness and conservation are more important than formal order of convergence, we use FV
    to be more precise we actually use the Kurganov-Tadmor central scheme, not a Godunov-type FV scheme
    but people always mixes the two (see e.g., the discussion in my PhD thesis)
    
    So, our simulations with microphysical eos andneutrinos are performed using the KT FV scheme?
    
    yes KT FV
\end{sidenote}

To increase a formal order of accuracy in a current generation numerical codes, flux-conservative finite-difference HRSC schemes is the simplest approach. Its direct competitor, high-order finite volume schemes are more computationally expensive, as they require solution of multiple Riemann problems at the interface between regions \cite{Reisswig:2009us,Shu:2001rep}, as well as complex averaging and de-averaging procedures \cite{Tchekhovskoy:2007zn} 

In this thesis we adopt \gray{ere David presents a new code}, the Templ
ated-Hydrodynamics Code (THC), developed using Cactus framework \cite{Goodale:2003} \red{cite David}. In \texttt{THC}, the state-of-the-art flux-vector splitting scheme are employed. 
The reconstruction in characteristic fields is available for up to 7th order, as well as, the Roe flux split with a entropy-fix prescription. 

The "tempalted" in the code name stands for a modern paradigm in C++ programming, the tempalated programming, which means, that part of the code can be generated from the prescribed templates at compiling time. 
This paradigm allows for a creation of complex modular codes avoiding computational costs, that plague classical polymorphism. 
The "templated" programming allows to inline all the needed functions and classes at compiling time, \cite{Yang:2001}. 

The following reconstruction schemes are implemented:
MP5, classical monotonicity preserving \cite{Suresh:1997,Mignone:2010},
the weighted essentially non oscillatory (WENO) schemes WENO5 and WENO7 \cite{Liu:1994,Jiang:1996,Shu:1997},
and two bandwidth-optimized WENO schemes WENO3B and WENO4B \cite{Martin:2006,Taylor:2007}, constructed for modeling the compressible turbulence. 
Note, that the number in scheme name stands for a formal order of accuracy. 

In this \red{section} we briefly state the details of the \texttt{THC} algorithm and highlight the results of the comparison between different reconstruction schemes for modeling relativistic turbulence. 

\gray{
The section is structured as following. First we overview several detain of \texttt{THC} code, discussing the numerical algorithms, in particular with respect to the equations of Newtonian and special relativistic hydrodynamics. Then we state several results. Then we view the linear and non-linear development of the relativistc Kelvin-Helmholtz instability (KHI) in 3D. 
}


\subsubsection{The \texttt{THC} code }
\red{Presentation and tests inf HYDRODYNAMIC part of the code only}

Here the infrastructure of \texttt{THC} is presented in addition to the formulation of Newtonian and special-relativistc HD. 

\textcolor{gray}{Here I will outline some results for my own understanding. This is not to be put in the thesis, as I am not working with the code development.}
\textcolor{red}{NOT REFPHRASED}

\begin{itemize}
    \item \textbf{Strong shock}. Classical one-dimensional shock tube. Even at this fairly low resolution, all the schemes are able to capture well both the shock wave and the rarefaction wave, showing the good behavior of the entropy fix. The contact discontinuity is resolved, but not without oscillations (due to the high Mach number of the shock wave, i.e., $\mathcal{M}=360$.
    \item \textbf{Blast wave}. larger density contrast at the contact discontinuity. The MP5 scheme is able to properly capture the constant state between the shock wave and the contact discontinuity, while the WENO schemes result in more “rounded” solutions.
    \item \textbf{Rotated Sod test}. Three-dimensional shock-tube test in Newtonian hydrodynamics. All the schemes are able to properly capture the main features of the solution: the discontinuities are captured within 1 or 2 gridpoints and both WENO5 and MP5 are able to capture the plateau in the velocity. Overall, these tests demonstrate the accuracy of the dimensionally unsplit approach that we use to treat the multi-dimensional case.
    \item \textbf{Double Mach reflection test} Our algorithm is able to introduce enough numerical dissipation to avoid the odd-even decoupling. All things considered, we find that the best performance is given by the MP5 scheme.
\end{itemize}

and in special relativity 

\begin{itemize}
    \item \textbf{Adiabatic smooth flow} Test code with the smooth solutions. One-dimensional, large-amplitude, smooth, wave propagating in an isentropic fluid. A good-enough approximation of the exact solution was obtained by computing it on a very fine Lagrangian grid (1e6 points) and interpolated on the Eulerian grid. Instead of the third-order SSP-RK scheme, we adopt here a fourth-order RK time integrator. Our schemes approach the expected convergence order only asymptotically, at very high resolution. The reason for this behaviour is in the “kinks” ahead and behind the pulse, where the numerical error is largest. These regions are “misinterpreted” as discontinuities by the shock-detection part of our schemes, unless they are resolved with enough gridpoints. The best performing scheme in this test is the MP5 one. Formation of the shock gradually degrades the overall convergence order to the 1st.
    \item \textbf{Blast wave}. Relativistic fluids can exhibit much stronger shock waves. MP5 scheme requires twice as small CFL as other schemes to prevent large oscillations and yields non-physical values. \textcolor{gray}{there is a finite-volume code \texttt{Whisky}, \cite{Baiotti:2010zf,Baiotti:2004wn} with the HLLE approximate Riemann solver \cite{Toro:1999} and PPM reconstruction \cite{Colella:1984}. } If the timestep is sufficiently small, on the other hand, the MP5 algorithm results in very accurate solutions, as in the Newtonian case. \texttt{THC} here performs better then \texttt{Whisky}.
    \item \textbf{Shock-heating} relativistic effects can enhance the density contrasts in shock waves. shocks whose collision compresses the fluid. Kinetic energy into thermal energy, that is, through “shock heating”. For a Lorentz factor of a 1000, $\Gamma=4/3$, for a Newtonian fluid the compression ratio $\approx7$, while for spec. relativ. it is $\approx 4000$. The WENO5 and WENO7 solutions are affected by some small wall-heating effect, slight underdensity. The MP5 scheme, on the other hand, yields a solution which is essentially free from oscillations.
    \item \textbf{Transverse shock}. the equations for the momentum in the different directions are coupled together by the Lorentz factor: even in one-dimensional problems the application of a transverse velocity can change completely the solution. This feature was first pointed out by \cite{Pons:2000} and \cite{Rezzolla:2002ra}, and then used by \cite{Rezzolla:2002cc} and \cite{Aloy:2006rd} to discover a new physical effect, see also [\cite{Mignone:2005ns}, \cite{Zhang:2005qy}] for a description of the numerical consequences of this property]. The MP5 scheme overestimates slightly the density contrast, but all of the algorithms are able to capture the correct location of the shock wave.
    \item \textbf{Spherical explosion}. No analytic solution is known in this case. As in the one-dimensional case, a small timestep is necessary in order to avoid numerical oscillations with the MP5 algorithm, while the other schemes appear to be stable even with a timestep which is twice as large.
    \item \textbf{Kelvin-Helmholtz instability in 2D}. The instability is seeded by adding a small perturbation in the transverse component of the velocity. we use periodic boundary conditions in all the directions. Compare first growth rate of the transverse velocity during the linear-growth phase of the KHI. Important to including the contact wave in the approximate Riemann solver in the case of a finite-volume code. We also note the importance of avoiding excessive dissipation in the contact discontinuity. The behaviour of the MP5 scheme, as well as that of the bandwidthoptimized WENO schemes, is more surprising: all of these schemes overestimate the growth of the RMS transverse velocity at low resolution. Some insight about the numerical viscosity can be gained by looking at the topology of the flow during the linear-growth phase of the KHI. These secondary instabilities, although only numerical artifacts (see below), appear only in schemes able to properly treat the initial contact discontinuity. They are not to be genuine features of the solution and, rather, tend to disappear as the resolution is increased. Conclusion: secondary instabilities are triggered by the non-linear dissipation mechanism of the different schemes, emerge neatly when computed with numerical schemes that treat properly the initial contact discontinuity, but do not have a physical meaning. Solution: adding more numerical viscosity [219] or as David suggests, physical viscosity. A more quantitative way of estimating the numerical viscosity of the code: The one-dimensional power spectrum can be used to quantify the typical scale of structures, such as the secondary vortices discussed above, stretched in the direction of the bulk shear flow. Even more unexpected is the ability of the MP5 scheme to resolve small scales structures and that, on the basis of the argument about the development of the secondary instabilities, should be more dissipative than WENO4B, but which instead appears to yield more small-scale structures in the rest-mass density.
\end{itemize}

\paragraph{The relativistic Kelvin-Helmholtz instability in 3D}
\textcolor{red}{Important for GRBs}

analysis is meant to assess how the different methods reproduce the same turbulent initial-value problem and to provide some insight on the spectral properties of the different schemes. The relativistic KHI [see, e.g., [51]] is of particular interest because of its relevance for the stability of relativistic jets [see, e.g., [251, 250]], and because of its potential role in the amplification of magnetic fields in gamma-ray bursts [see, e.g., [338]], and binary neutron-star mergers [25, 143, 240, 274].

\begin{itemize}
    \item \textbf{The linear evolution of the instability} Consider the evolution of the instability during its linear-growth phase. As expected, all the numerical schemes, with the exception of MINMOD, are in very good agreement with the 2D solution up to the end of the linear-growth phase, when 3D effects become important and turbulence starts to play an important role in the dynamics. It is interesting to note that MINMOD, which is the most dissipative of the schemes we are using, is actually overestimating the growth of the KHI.
    This suggests that \textbf{some care should be taken when interpreting the results from under-resolved simulations}. secondary vortices are produced in more least dissipative methods.
    \item \textbf{The non-linear evolution of the instability} The linear-growth phase of the KHI instability ends when the primary vortices become unstable to secondary instabilities and the flow starts the transition to turbulence. Three-dimensional effects dominate. Use the tracer scalar field to track the evolution.
    \item \textbf{The non-linear evolution of the instability}. when the primary vortices become unstable to secondary instabilities and the flow starts the transition to turbulence. three-dimensional effects dominate.
    \item \textbf{Fully-Developed turbulence} By far the most interesting quantity to study is the three-dimensional velocity power spectrum. conclusions. importance of the use of high order    schemes (avoid bottle-neck, otherwise power-spectrum shows an excess due to viscous effects.) use of WENO4B over WENO5 is well justified, since WENO4B is roughly twice as expensive as     WENO5 in 3D. Tthe main differences between the bandwidth-optimized schemes and their traditional counterparts seem to lay in the bottleneck region WENO3B and WENO4B have a much less pronounced bottleneck with respect to WENO5, WENO7 and MP5.
\end{itemize} 

\subsubsection{Driven Relativistic Turbulence}

Consider an idealized model of an ultrarelativistic fluid. The fluid is modeled as perfect. We evolve the equations describing conservation of energy and momentum in the presence of an externally imposed Minkowskian force. To solve the equations of relativistic hydrodynamics in 3D we use the THC code described in this chapter and published in \cite{Radice:2012cu}. In particular, here, we use the MP5 reconstruction in local characteristic variables [165].

\begin{itemize}
    \item \textbf{Basic flow properties} All in all, this is one of our main results: the velocity power spectrum in the inertial range is universal, that is, insensitive to relativistic effects, at least in the subsonic and mildly supersonic cases. Note that this does not mean that
    relativistic effects are absent or can be neglected when modelling relativistic turbulent flows.
    \item \textbf{Intermittency} local appearance of anomalous, short-lived flow features.
    \item \textbf{Conclusion} \textcolor{red}{We have presented THC, a new multi-dimensional, finite-difference, high-resolution shock-capturing code for classical and special-relativistic hydrodynamics... -- [FULL description of THC]}
\end{itemize}



\subsection{Finite-Differencing Methods: General Spacetimes}
\red{Presentation of GR+Hydro part -- whiskythc}


Goal is to model the inspiral of BNS to produce accurate waveforms. 
\textcolor{red}{here, we describe our new high order, high-resolution shockcapturing, finite-differencing code: \texttt{WhiskyTHC}, which constitutes the extension to general relativity of the \texttt{THC} code.}


\subsubsection{WhiskyTHC}
\textcolor{red}{marginally rephrased}

\begin{itemize}
    \item \textbf{Numerical Methods}. 
    \textcolor{gray}{[high order, high-resolution shockcapturing, finite-differencing code]} 
    
    \texttt{WhiskyTHC} is a result of combination of two \texttt{Whisky} \cite{Baiotti:2004wn} and \texttt{THC} \cite{Radice:2012cu}. High-order flux-vector splitting finite-differencing techniques has come from the former, while the module for the recovery of the primitive quantities as well as the equation of state framework from the latter \cite{Galeazzi:2013mia}. Tabulated temperature and composition dependent equation of states can be used \textcolor{gray}{however David used only polytrops}. 
    
    Overall, \texttt{WhiskyTHC} solves the equations of general-relativistic hydrodynamics in conservation form \ref{eq:theory:grhdeq_thc}. using a finite difference scheme \textcolor{red}{we however are using FV? Be carefull with which methods are used exactly}. 
    
    The flux reconstruction is done in
    local-characteristic variables using the MP5 scheme, see \textit{e.g.,} \cite{Rezzolla:2013}. The space-time is evolved using the CCZ4 formulation \ref{eq:theory:ccz4equations}, solved via finite difference code publicly available through \texttt{Einstein Toolkit}, \cite{McLachlan,Loffler:2011ay}. There, the central stencil is used throughout, and only terms associated with the advection along the shift vector are treated using the upwinded by one grid point stencil. The accuracy of the scheme is availalbe at 6th and 8th order, while 4th is commonly employed. 
    In addition, the fifth order Kreiss-Oliger style artificial dissipation \cite{Kreiss:1973} is added to aid with non0linear stability. 
    The code is build on the \texttt{Carpet} AMR driver \cite{Schnetter:2003rb} from the \texttt{Cactus} computational toolkit \cite{Goodale:2003}, incorporating a provided by \texttt{Carpet} Berger-Oliger-style mesh refinement \cite{Berger:1989,Berger:1984} with sub-cycling in time and re-fluxing. 
    \textcolor{red}{in Thesis it is said, -- no refluxing was done yet}
    
    
    \item \textbf{Atmosphere Treatment} The atmosphere is referred to an artificial density floor in the simulation domain. It is introduced in order to tackle the challenges arising when considering boundary between the fluid and vacuum in Eulerian (relativistic) hydrodynamics codes \cite{Galeazzi:mThesis:2008,Kastaun:2006,Millmore:2009dk}. 
    The defining property of the atmosphere is that the rest mass density and coordinate velocity are reset to a floor values once the former falls below a certain threshold value during the evolution \cite{Font:2001ew,Baiotti:2004wn}.
    While showing a reasonable results in second order codes, in higher order ones the numerical oscillations lead to the creation of vacuum nonetheless, that in light of the aforemention atmosphere effect result in the mass and energy violation \cite{Radice:2011qr}. 
    For codes that rely on characteristic variables, the degeneracy in low-density, low-temperature limits also plagues the computation. This problem is the main reason behind the popularity of robust shock capturing codes, even though they are of first order in the general-relativistic hydrodynamics codes. Vacuum treatment for higher order codes is of main challenges to overcome. 
    
    \begin{itemize}
        \item \textit{Standard Atmosphere Treatment} or \textit{"ordinary MP5 approach"} is based on setting density that falls below $(1+\epsilon)\rho_{\text{atmo}}$ to the atmosphere density, velocity to zero and internal energy to the one prescribed by the polytropic EoS. The $\rho_{\text{atmo}}$ is usually related to a certain characteristic density, \textit{e.g.,} maximum density at the beginning of the simulation as $\rho_{\text{atmo}} = 10^{-7,-9}\rho_{\text{max}}$. The tolerance parameter $\epsilon$ is usually set to $10^{-2}$ and accounts for excessive oscillations of the fluid–vacuum interface. 
        
        \item \textit{An Improved Atmosphere Treatment} or \textit{"MP5+LF"} In this approach the component-wise Lax-Friedrichs flux split is turned on when a certain density is reached. This increases the dissipation of the scheme and allows to avoid problems arising in characteristic reconstruction, associated with the degeneracy of the characteristic variables close to vacuum. Unfortunately, if the \red{ejection of low velocity and density matter is concerned}, this approach may yield oscillatory solutions and thus creates artifacts. 
        
        \item \textit{Positivity Preserving Limiter} a novel approach based on the use of PPL proposed in \cite{Hu:2013}. Here we provide a brief overview. 
        
        Consider a simple scalar conservation law in 1D
        
        \begin{equation}
        \frac{\partial u}{\partial t} + \frac{\partial f(u)}{\partial x} = 0
        \label{eq:theory:whickythc:atmo:conslaw}
        \end{equation}
        
        Since for a SSP time integrator a time update is convex combination of Euler steps, for which the positivity of $u$ is guaranteed for any scheme, the general discrete from of \ref{eq:theory:whickythc:atmo:conslaw} can be written as 
        
        \begin{equation}
        \frac{u_{i}^{n+1} - u_{i}^{n}}{\Delta^0} = \frac{f_{i-1/2} - f_{i+1/2}}{\Delta^1}
        \end{equation}
        
        And if $\lambda = \Delta^0/\Delta^1$, then 
        
        \begin{equation}
        u_{i}^{n+1} = \frac{1}{2}(u_{i}^{+} + u_{i}^{-}) = \frac{1}{2}\Big[ (u_{i}^{n} + 2\lambda f_{i-1/2}) + (u_{i}^{n} - 2\lambda f_{i+1/2})\Big].
        \end{equation}
        
        where then $u_{i}^{n+1} = u_{i}^{+} + u_{i}^{-}$ and $u_{i}^{n} = u_{i}^{n} - 2\lambda f_{i+1/2}$. Notably, the $u_{i}^{+}$ and $u_{i}^{-}$ as well as $u_{i}^{n+1}$ are positive. 
        In \cite{Hu:2013} it was pointed out that if a first-order Lax-Friedrichs scheme with $\lambda\leq 1/2a$ (with $a$ being the maximum propagation speed) is used for evaluating $f_{i\pm 1/2}$, the $u_{i}^{\pm}\geq \text{min}_i u_{i}^{n}$ \cite{Zhang:2010}. \textcolor{red}{not understand that}. Then the suggested point is ti change the $f_{i+1/2}$ to be 
        
        \begin{equation}
        f_{i+1/2} = \theta f_{i+1/2}^{\text{HO}} + (1-\theta)f_{i+1/2}^{\text{LF}},
        \end{equation}
        
        where $f_{i+1/2}^{\text{HO}}$ is the high-order flux of the original scheme, and $f_{i+1/2}^{\text{LF}}$ is the flux associated with the first order Lax-Friedrichs scheme, and $\theta\in[0,\:1]$. If the spatial location is far from vacuum, then the original high accuracy scheme can be used, so the $\theta$ remains $1$. However, in the vicinity of the vacuum, the $\theta$ decreases, to assure that $u_{i}^{\pm}$ remains positive. This is always possible since the Lax-Friedrichs scheme, used for $f_{i+1/2}^{\text{LF}}$ is positivity preserving.
        
        In a multidimensional case the the component-vise extension is employed. \textcolor{red}{formula that I will not used for $u_{i,j,k}^{n+1}$}.
        
        In \cite{Hu:2013} the extension of the method to the system of conservation laws was also proposed. 
        
        The complications however are present when the source terms are treated. While for a simplified case of classical gas dynamics it might require a lower timestep, in the general relativistic case and general tabulated EOS, the positivist of pressure is difficult to assure due to complexity of the energy source terms. It can be mitigated by enforcing a floor value on the pressure.
        
        Note, that adopting a positivity preserving limiter to treat the transition between matter and vacuum, still implies replacing the vacuum with low density fluid at rest, is not a physically accurate approach. That would rely on treating the transition as a free boundary (see \textit{e.g.,} \cite{Kastaun:2006}) The advantage of positivity preserving limiter with respect to a classical atmosphere treatment, is that it allows to have a value of $\rho_{\text{atmo}}$ that does not require further tuning and can be arbitrary small, and assure that the solution is locally conserved. 
        
        \textcolor{red}{In our models} we employ this approach as follows, at the meginning of the simulations we set the floor density, relying in the subsequent evolution on a positivity preserving limiters to ensure the atmosphere well behaviour. Due to negligeble density of the atmosphere its accretion has a negligeble effect on the evolved object. 
        
        
    \end{itemize}
    
    \item \textbf{Single Neutron Stars: Fixed space-time} here the atmosphere test showed that using the standard atmosphere leads to the mass conservation violation on a small degree, however, it also shows an appearance of a "jet+-like structure along the axes where grid points are aligned with the star's surface. These aritifical outflows are driven by the numerical oscillations creating an imbalance at the surface MP5+LF on the other hand shows no artificial matter streams due to its conservative nature
    
    \item \textbf{Single Neutron Stars: Full-GR}
    
    \item \textbf{Non-linear Oscillations: the Migration Test} Here the setup is the following, a neutron star in Full GR is set with an initial oscillation, that forces a star to first contract and then re-bounce. This re-bounce creates the ejecta. Different prescriptions show this ejecta, but the MP5-LF is inadequate in this test, introducing the structure in the outflow (numerical osculations/fragmentation). Origin: component-wise reconstruction in low density regions.
    
    \item \textbf{Gravitational Collapse to Black-Hole}
    
    \item \textcolor{red}{\textbf{Binary Neutron Stars} [Copied. Not rephrased]} 
    Models having an initial small separation of 45 km. Compare it to \texttt{Whisky} code, that is a second-order finite-volume code, with high-order primitive reconstruction and implements several different approximate Riemann solvers, \textcolor{red}{David used PPM reconstruction [95] and of the HLLE Riemann solver \cite{Harten:1983}, \cite{Einfeldt:1988}].} 
    
    The initial data we consider describes two neutron stars in quasi-circular orbit. It is computed in the conformally-flat approximation using the Lorene pseudo-spectral code \cite{Gourgoulhon:2000nn} and has been made publicly available by the Meudon group \cite{Lorene}. 
    The EoS assumed for the initial data is polytropic. \textcolor{gray}{In our case it is cold EOS, a slice from finite temperature EOS} while the evolution is performed using the ideal-gas EoS to allow for thermal effects in the merger phase. \textcolor{gray}{In our case it is finite temperature EOS}. 
    Discussion on baryonic masses and compactness $c=M_{\infty}/R_{\infty}$, where $R_{\infty}]$ is the areal radius.
    
    \begin{itemize}
        \item \textit{Small separation} Grid discussion: extend, symmetries \textit{e.g.,} we assume reflection
        symmetry across the $xy$ plane and $\pi$ symmetry across the yz plane. Number of refinement levels. Static grid or AMR. 
        Evolution via CCZ4 with damping constants $\kappa_1=?$, $\kappa_2=?$ and $\kappa_3=?$ and with beta-driver $\eta=?$. The space time evolved. Space-time is evolved via fourth order finite-differencing and with fifth order Kreiss-Oliger artificial dissipation \textcolor{red}{I need to find what is used in our runs}.
        
        Study the graviational radiation via looking ad the $l=2$ $m=2$ mode of the $Weyl$ scalar $\Psi_4$ extracted at a fixed coordinate radius of $r=450M_{\odot}$. Strain is not computed as it involves other uncertainties \cite{Boyle:2009vi,Reisswig:2009us,Reisswig:2009rx,Reisswig:2010di}. 
        
        The dynamics of the inspiral and merger of BNS has been described many times and in great detail in the literature \textit{e.g.,} \cite{Baiotti:2008ra}. We only mention that the two neutron stars
        inspiral for about 2:5 orbits, touch and quickly merge into a single black-hole. For this particular model no significant disk is left behind. The gravitationalwave signal consists.
        For GW plot 22 mode of $\Psi_4$ as extracted at $r=450M_{\odot}$ and as a function of the retarded time $t-r_*$ where $r_* = r + 2M_{\text{ADM}}\log(r/(2M_{\text{ADM}})-1)$.
        
        Results: 1. treatment of the neutron star surface is not a leading source of error in binary neutron star simulations, as far as the inspiral GW signal is concerned. \texttt{WhiskyTHC} shows a smaller dephasing significantly smaller de-phasing: the difference between the low and the high resolution is about 0.6 radians, which is a factor four smaller than the one shown by \texttt{Whisky}. 
        Observation: merger happens earlier as we increase the resolution. For each run we compute the phase, $\phi$, of the 22 mode of $\Psi_4$ from its definition, $(\Psi_4)_{22} = A e^{i\phi}$. 
        
        We should stress that this error estimate only reflects the numerical truncation error. Other systematic errors and, in particular, finite extraction radius effects and inaccuracies in the initial data, are also present and might be relevant (especially for WhiskyTHC). On the other hand, here we are interested only in evaluating the accuracy of the two numerical methods.
        
        \item \textit{Large separation} [mostly skipped]
        Notice that  contact happens before the bare contact angular frequency \cite{Damour:2012yf} 
        \begin{equation}
        0.15276 = M\omega_{\text{contact}} := 2C^{3/2}, \hspace{5mm} \omega:=\dot{\phi}
        \end{equation}
        is reached. This is in any case expected because this approximation of the contact frequency does not take tidal deformations into account.
    \end{itemize}
    
    \item \textbf{Conclusion}
    
\end{itemize}





%% =======================
%%
%% PART 2
%%
%% =======================

%% %% ==============================================================================
%% ==============================================================================
%% ==============================================================================
%%
\part{Electromagnetic counterparts to neutron star mergers}
%% \label{sec:part1}
%%
%% ==============================================================================
%% ==============================================================================
%% ==============================================================================

%% In this part to discuss
%% GR Hydro 
%% Numerical methods for GR Hydro
%% Radiation
%% M0 scheme for neutrinos
%% WhiskyTHC & Lorene
%% GW1708017 targeted models
%% Remannt/Disk dynamics of model 


%% ===========================================================================
%%
%% Intorduction
%%
%% ===========================================================================

\chapter{Introduction}

\todo{Write me}

%% ===========================================================================
%%
%% Theoretical Background
%%
%% ===========================================================================

\chapter{Theoretical Background}

%% 

\section{Nucleosynthesis in binary neutron star ejecta}

%% =======================
%%
%% APPENDEXES
%%
%% =======================

%% \appendix
%% \include{appendix_gr} %% GR
%% %% ============================
%%
%% Appendix B
%%
%% ============================

\chapter{Nuclear Physcis \& Nuclear Reaction Networks}
\label{AppendixB}
\externaldocument{intro}

blabla
 %% Nuc

%% =======================
%%
%% References
%%
%% =======================

%% \bibliography{../references}

% \listoffigures
% \listoftables

\end{document}
