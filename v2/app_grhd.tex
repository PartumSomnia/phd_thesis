%% ============================
%%
%% Appendix A
%%
%% ============================

\chapter{General relativistic hydrodynamics}


%% ---

\subsection{The Equations of General-Relativistic Hydrodynamics}

%% --- 

In this section we discuss the equations of general relativistic hydrodynamics. 
We consider the fluid on a Lorentzian manifold and how its flow affects the spacetime.

The topics that we are going to touch are:

\begin{itemize}
    \item fluid kinematics,
    \item equations of motion for perfect fluids (assuming that there is no thermal conduction or viscosity)
    \item the “Valencia formulation” of the hydrodynamic equations.
\end{itemize}


We note that the following description is very brief and is based on the following works: \cite{Misner:1973,Schutz:2009a,Gourgoulhon:2006bn,Andersson:2006nr,Rezzolla:2013} to which we refer the reader for more details.

%% P.1 Introduction. Difference in fluid treatment in Newtonain and Relativity cases
In Newtonian physics, a fluid is an "entity" whose dynamics is described by flows of quantities such as energy density, mass, momentum density. 
However, in general and special relativity, these quantities are not well defined and depend on the observer. 
In other words, different observers perceive the the same fluid being in different thermodynamic state. 
Hence, a description of the fluid dynamics in general and special relativity requires a new formulation, a formulation in which a fluid is not represented by a scalar and vector fields, that are observer-dependent, but implicitly by a "flow" in spacetime. 
These are \textit{flux-conservative formulations} of hydrodynamics.

The Eulerian specification of the flow field is a way of looking at fluid motion that focuses on specific locations in the space through which the fluid flows as time passes. Hence, the Eulerian observer is the observer that is at rest in space.

Consider the classical mass density, a scalar $\rho$, usually defined as total umber of particles $N$ of rest-mass $m$ in the volume $V$. Then, the total mass is given by

\begin{equation}
\int_V \rho \text{d}^3x = m\int_V n \text{d}^3 x = mN.
\end{equation}

However, while the number of particles $N$ would be the same regardless of the observer, the $\text{d}^3x$ would be measured differently by observers moving in relation to each other. 
Hence, the $n$ would be different. 
One of the solutions is to chose a frame of reference, say comoving with the fluid and define the $\rho$ there. However, this would hinder our ability to generalize to other reference frames.
An even better solution is to construct a \textit{covariant description in terms of invariant quantities}. 

%% P.2 Covarient descriptoion of the fluid 
%% P.2.1. Define the mass flow
We start by defining the \textit{flow of the fluid density} in space-time, the $3$ pseudo-form $\boldsymbol{\rho}$ that on any three dimensional submanifold describes the \textit{flow of mass}, transverse to the submanifold as

\begin{equation}
\int_{\Sigma} \boldsymbol{\rho},
\end{equation}

where $\Sigma$ be a spacelike hypersurface, $\vec{n}$ is the future-oriented normal vector. 
This is the density measured by an observer with $4$-velocity $\vec{n}$. 

To define a \textit{mass flow} measured by an Eulerian observer across any spacelike surface $\Omega\subset\Sigma$, we need to construct a two-form $\boldsymbol{\rho}(\vec{n}, \cdot, \cdot)$ given by the interior product between the $3$ pseudo-form $\boldsymbol{\rho}$ and $\vec{n}$. 
Then the mass flow is 

\begin{equation}
\int_{\Omega} i_{\vec{n}}\boldsymbol{\rho}.
\end{equation}

The \textit{conservation of the number of particles of the fluid} is expressed by the vanishing exterior product of the density form, i.e. $\text{d}\boldsymbol{\rho}=0$, or in an integral form 

\begin{equation}
\int_{\partial\Omega} \boldsymbol{\rho} = \int_{\Omega}\text{d}\boldsymbol{\rho} = 0,
\end{equation}

that reads as the following: the net flow across any sufficiently regular surface $\partial\Omega$ enclosing a four-dimensional open set $\Omega\subset\mathcal{M}$ is zero.

%% P.2.2. Define the flux associated with the flow, in this case of a rest-mass density four-vector
Next we define a \textit{flux}. First, let us reintroduce the volume pseudo-form

\begin{equation}
\text{Vol}_x ^4 = \sqrt{-g}dx^0 \wedge dx^1 \wedge dx^2 \wedge dx^3,
\end{equation}

where $g$ is the determinant of the spacetime metric. 

On the submanifold $\Sigma$, the intrinsic volume then would be defined as 

\begin{equation}
\text{Vol}_x ^3 = i_{\vec{n}} \text{Vol}_x ^4.
\end{equation}

A \textit{flux of a vector field} can be described by a three-form, for which on a pseudo-Riemannian manifold there exist a vector field associated with it.

A vector field associated with density is called \textit{rest-mass density four-vector} and is denoted by $\vec{j}$.

It is constructed from the one-form by raising indexes, $\vec{j} = {^{\#}\underline{j}}$. 
The one-form $\underline{j}$ is obtained as $\underline{j}\star\boldsymbol{\rho}$, where $\star$ is the \textit{Hodge dual operator} (see \textit{e.g.,} \cite{Frankel:1982dva}). 

Then if the $\boldsymbol{\rho} = i_{\vec{j}}\text{Vol}_x ^4$, the flux of $\vec{j}$ can be shown as 

\begin{equation}
\int_{\Sigma} \boldsymbol{\rho} = - \int_{\Sigma}\vec{j}\cdot\vec{n}\text{Vol}_x ^3,
\end{equation}

where $\vec{n}$ is the future-oriented unit-timelike normal to $\Sigma$.

%% --- for liuville theorem
\textcolor{gray}{
    [Direct copy... maybe not needed] More generally the flux associated with a flow defined by a vector field, $\vec{X}$, across a hypersurface, $\Sigma$, transverse to it and with normal $\vec{\nu}$ (with appropriate sign depending on the signature of the metric and on $\Sigma$), is given 
    \begin{equation}
    \int_{\Sigma} \star\underline{X} = \int_{\Sigma}i_{\vec{X}}\text{Vold}^n = \int_{\sigma}i_{\vec{X}}\big[\underline{\nu}\wedge\text{Vol}^{n-1}\big] = \int_{\Sigma}\vec{X}\cdot\vec{\nu}\text{Vol}^{n-1}
    \label{eq:theory:flux_of_flow}
    \end{equation}
}
\textcolor{red}{this piece is used in Liuille theorem though}

We note that $\vec{j}$ is timelike (or null).
It is given by the flux of particles across any future-oriented spacelike hypersurface is positive (or zero). 
If $\vec{j}$ is timelike, there exists a unique decomposition 

\begin{equation}
\vec{j} = \rho \vec{u},
\label{eq:theory:defofjandu}
\end{equation}

where the scalar $\rho$ can be seen as density in the comoving frame and unit-timelike vector $\vec{u}$ as a fluid four-velocity.

The divergence of vecotor $j$ then gives a familiar mass conservation expression

\begin{equation}
0 = \nabla_{\mu} j^{\mu} = \frac{1}{\sqrt{-g}}\partial_{\mu}[\sqrt{-g}\rho u^{\mu}].
\label{eq:theory:nablamu_jmu}
\end{equation}

\textcolor{gray}{
    Similarly energy and momentum of a fluid can be defined, using the Cartan formalism... but this is a PAIN! and is done to show that div(T)=0 is not really energy/momentum conservation...
}

Next, let us introduce the mixed tensor $\boldsymbol{T}$. 
Since the three-forms are equivalent to vectors, we can define a flow of the $\nu$ momentum across the volume element orthogonal to $dx^{\mu}$ as 

\begin{equation}
{T^{\mu}}_{\nu}=\boldsymbol{T}(dx^{\mu},\partial_{\nu}).
\end{equation}

${T^{\mu}}_{\nu}$ is the stress energy tensor that was already introduced earlier \ref{eq:theory:action1}. 

Note, that if the Einstein equation are satisfied the Bianchi identities dictate that the $\nabla_{\mu}{T^{\mu}}_{\nu}$ must vanish as

\begin{equation}
\nabla_{\mu}{T^{\mu}}_{\nu} = 0= \frac{1}{\sqrt{-g}}\partial_{\mu}(\sqrt{-g}{T^{\mu}}_{\nu}) - {\Gamma^{\alpha}}_{\mu\nu}{T^{\mu}}_{\alpha}.
\label{eq:theory:nablamu_tmunu}
\end{equation}

However, this statement does not imply the conservation of the energy and momentum of the fluid in a general sense. 
The conservation of the $\nu$-momentum requires $\vec{\partial}_{\nu}$ to be a Killing vector.


\paragraph{Dynamics of a Relativistic Fluid}


%% Introduction
In the previous subsection we have introduced the fluid kinematic, and defined the important quantities such as mass, energy and momentum and their "conservation" in \ref{eq:theory:nablamu_jmu} and \ref{eq:theory:nablamu_tmunu}.

In this \gray{thesis} chapter we consider only the \textit{perfect fluid}, meaning that in the co-moving frame, there is not heat conduction and there is no viscosity. 
\todo{actually we do have a viscous part -- you have to add this...}. 

The former criterion implies that the fluid is in \textit{local thermodynamic equilibrium} (LTE). 
The latter however requires more explanation. 
The mathematical formulation of viscous, thermally conducting fluids in general-relativity, especially with respect to the numerical applications is still an active area of research  (see e.g., \cite{Andersson:2006nr} and references therein). 
\textcolor{red}{however in recent years there have been some progress GRELS models and David's implementation I must add!}.

%% Constructing a stress-energy tensor
Consider a stress-energy tensor of a perfect fluid in the comoving frame with the fluid. 
To construct it, we return to the fluid's four velocity $\vec{u}$ from (\ref{eq:theory:defofjandu}). 
If $e_{i}$ is the basis vector, the scalar product $\vec{u}\cdot\vec{e}_i=0$ and $\vec{e_i}\cdot\vec{e}_k = \delta_{ik}$. 
Then the orthonormal tetrad $\{\vec{u},\vec{e}\}$ is comoving with the fluid, and the $\{\underline{u},\underline{e}^i\}$ is the dual basis. 

Tensor $\boldsymbol{T}$ is the stress-energy tensor with the following components: 

\begin{itemize}
    \item $\boldsymbol{T}(\underline{u}, \vec{u})$ energy-density in the rest-frame of the fluid, the scalar $e$
    \item $\boldsymbol{T}(\underline{u}, \vec{e}_i) = 0$ represent the energy flowing transverse to the four-velocity, which we set to $0$ in the absence of the heat-conduction.
    \item $\boldsymbol{T}(\underline{e}^i, \vec{e}_k) = 0$ represent the $k$ component of the force exchanged across the surface element orthogonal to $\underline{e}_i$.
\end{itemize}

Taking into account that the $\boldsymbol{T}$ must be invariant with respect to the rotations of the $\{\vec{e}_i\}$ and that the viscosity is not included, force exchange can be effectively described by a scalar $p$, that we call \textit{pressure} as

\begin{equation}
\boldsymbol{T}(\underline{e}^i,\vec{e}_k) = p {\delta^i}_k,
\end{equation}

Combining the aforementioned description of the components of $\boldsymbol{T}$ we get

\begin{equation}
\boldsymbol{T} = (e + p)\vec{u}\otimes \underline{u} + p\boldsymbol{\delta}.
\end{equation}

Defining the \textit{enthalpy} of the fluid as $h = 1 + \epsilon = p/\rho$, where $\epsilon$ is the specific internal energy, we rewrite $\boldsymbol{T}$ as 

\begin{equation}
\boldsymbol{T} = \rho h \vec{u}\otimes\underline{u} + p\boldsymbol{\delta}
\label{eq:theory:stressenergytensor}
\end{equation}

In addition to the fluid kinematics (eqs. \ref{eq:theory:nablamu_jmu} and \ref{eq:theory:nablamu_tmunu}) and the description of motion (eq. \ref{eq:theory:stressenergytensor}), the relation between the \textit{pressure}, \textit{internal energy} and \textit{density} is needed to fully describe the dynamics of the fluid. 
This relation is usually called the \textit{equation of state}.

The commonly adopted equations (EoS) of state are the the ideal-gas, or gamma-law EoS $\rho = (\Gamma-1)\rho\epsilon$, where $\Gamma$ is the polytropic index of the gas, the polytropic EoS $p = K\rho^{\Gamma}$ and the microphysical equation of state 
\todo{that you need to discuss more..., as we use only it.}

Combined with an EoS, equations \ref{eq:theory:adm}, \ref{eq:theory:nablamu_jmu}, \ref{eq:theory:nablamu_tmunu} and \ref{eq:theory:stressenergytensor} form a hyperbolic
system of equations that can be evolved, once initial data is prescribed. 
The complete evolution of spacetime and the dynamics of the matter requires initial data to be set on the Cauchy surface.


\paragraph{Conservative Formulations}

%% Introcution
In the pioneering works of May and White \cite{May:1966} and Wilson \cite{Wilson:1972} the equations of general relativistic hydrodynamics were solved using the finite-difference (FD) schemes after casting them into a from of non-linear advection-like equations. 

To avoid excessive oscillations at shocks a combination of upwinding and artificial-viscosity methods was employed. 
This however led to several limitations, such as difficulty with tunning the artificial viscosity to still allow shocks to develop, and the limit on a flows being only mildly relativistic \cite{Font:2008fka}. 

A next big advancement in the numerical relativistic hydrodynamics was made after the non-conservative nature of the Wilson’s approach was pointed out \cite{Marti:1991wi} and the conservation formulation was developed. 

%% Introduction :: valencia formuilation
An important example of the conservation formulation that is adopted to $3 + 1$ formalism is the "Valencia formulation" \cite{Banyuls:1997} that can be represented as following

\begin{equation}
\frac{\partial\boldsymbol{F}^{0}(\boldsymbol{u})}{\partial t} + \frac{\partial\boldsymbol{F}^{i}(\boldsymbol{u})}{\partial x^{i}} = \boldsymbol{S}(\boldsymbol{u})
\label{eq:theory:valencia_formalism}
\end{equation}

where $u$ is a “vector” of \textit{primitive quantities}, such as the rest-mass density or the specific internal energy, $\boldsymbol{F}^0$ is a “vector” of \textit{conserved quantities} and $\boldsymbol{F}^i$ and $\boldsymbol{S}$ are their fluxes and sources respectively.

This formulation allowed to study ultra-relativistic flows and resolve shocks without spurious oscillations and without need for artificial viscosity.

It was shown to be especially well suited for use with numerical methods that take into account the conservation laws. 
These are the finite-volume (FV) high-resolution shock capturing (HRSC) methods, that will be discussed in \red{Chapter} \ref{chapter:num_methods}

Many recent advancements in numerical relativistic hydrodynamics and magnetohydrodynamics (MHD) have relied on these methods (\textit{e.g.,} \cite{Giacomazzo:2010bx} \cite{Rezzolla:2011da} and references therein \todo{add recolla/bernuzzi/radice/shibata}).

There are other conservative formulations and methods (see \textit{e.g.,} \cite{Papadopoulos:1999kt})
However, we will limit our focus to the "Valencia formulation".

%% Derivation of the valencia formulation
To begin we split the four-velocity $\vec{u}$ into the component parallel to the normal vector $\vec{n}$ and a purely spatial component as

\begin{equation}
\vec{u} = (-\vec{u} \cdot \vec{n})(\vec{n} + \vec{\upsilon}),
\end{equation}

where naturally the Lorentz factor, measured by the Eulerian observer 
$W = (-\vec{u}\cdot\vec{n})$ emerges, and the $\upsilon$ is the fluid three-velocity measured by the Eulerian observer, 

\begin{equation}
\vec{\upsilon} = \frac{\vec{u}}{W} -\vec{n},
\end{equation}

components of which are

\begin{equation}
\upsilon^i = \frac{u^i}{W}+ \frac{\beta^i}{\alpha}, \hspace{10mm} \upsilon_i= \frac{u_{i}}{W}.
\end{equation}

Divergence of the rest-mass density four-vector $j$, (\ref{eq:theory:nablamu_jmu}) can easily be cast as 

\begin{eqnarray}
0 = \nabla_{\mu}j^{\mu} = \frac{1}{\sqrt{-g}}\partial_{t}[\sqrt{\gamma}\rho W] + \frac{1}{\sqrt{-g}}\partial_{i}[\sqrt{\gamma}\rho(\alpha \upsilon^{i} - \beta^{i})]
\end{eqnarray}

where $D= \rho W = -\vec{j}\cdot \vec{n}$ is the conserved density.

To write the energy and momentum equations we note that for any vector field $\vec{p} $ \cite{Rezzolla:2013}, 

\begin{equation}
\nabla_{\mu}[{T^{\mu}}_{\nu}p^{\nu}].
\end{equation}

To obtain the \textit{Valencia formulation} we set $\vec{p}$ to have zeroth component $-\vec{n}$ and spatial components $\vec{\partial}_i$. 
Then the

\begin{itemize}
    \item ${T^0}_{\nu}p^{\nu}$ represent the conserved quantities,
    \item ${T^i}_{\nu}p^{\nu}$ are associated fluxes,
    \item ${T^{\mu}}_{\nu}p^{\nu}$ are sources
\end{itemize}

with the former being 

\begin{equation}
S_{i} = \alpha {T^0}_{\nu}(\partial_i)^{\nu}=-\boldsymbol{T}(\vec{n},\vec{\partial}_i), \hspace{10mm} E = -\alpha{T^0}_{\nu}n^{\nu} = \boldsymbol{T}(\vec{n},\vec{n})
\end{equation}

for numerical reasons we will replace the total internal energy density $E$ with $\tau = E-D$, where $D$ is the rest mass density. 
This is done to avoid errors emerging due to $E$ being much smaller then $D$. 

Now we can combine the obtained expressions for the conserved quantities, associated fluxes and sources with eq. (\ref{eq:theory:valencia_formalism}) and obtain

\begin{equation}
\frac{1}{\sqrt{-g}}\Big[\frac{\partial\sqrt{\gamma}\boldsymbol{F}^{0}(\boldsymbol{u})}{\partial t} + \frac{\partial\sqrt{-g}\boldsymbol{F}^{i}(\boldsymbol{u})}{\partial x^i}\Big] = \boldsymbol{S}(\boldsymbol{u}),
\label{eq:theory:grhdeq_thc} % used for THC section Code
\end{equation}

where \textit{primitive quantities} being

\begin{equation}
\boldsymbol{u} = [\rho,\: \upsilon_i,\: \epsilon],
\end{equation}

the \textit{conserved ones} are 

\begin{equation}
\boldsymbol{F}^0(\boldsymbol{u}) = [D,\: S_j,\: \tau] = [\rho W,\: \rho h W^2 \upsilon_j,\: \rho h W^2 - p - \rho W],
\end{equation}

and the \textit{associated} fluxes are 

\begin{equation}
\boldsymbol{F}^i(\boldsymbol{u})=\Bigg[D\Big(\upsilon^{i}-\frac{\beta^i}{\alpha}\Big),\: S_{j}\Big(\upsilon^{i}-\frac{\beta^i}{\alpha}\Big)+p{\delta^i}_j ,\: \tau\Big(\upsilon^{i}-\frac{\beta^i}{\alpha}+p\upsilon^i\Big)\Bigg]
\end{equation}

and finally \textit{sources} are

\begin{equation}
\boldsymbol{S}(\boldsymbol{u}) = \Bigg[0,\: T^{\mu\nu}\Big(\frac{\partial g_{\nu j}}{\partial x^{\mu}} - \Gamma^{\delta}_{\nu\mu}g_{\delta j}\Big),\: \alpha\Big(T^{\mu 0}\frac{\partial\log\alpha}{\partial x^{\mu}}-T^{\mu\nu}\Gamma^{0}_{\nu\mu}\Big)\Bigg]^T
\end{equation}

The from of the obtained general relativistic hydrodynamics equations resemble the one of the Newtonian gas dynamics. If the latter is adopted for numerical solutions. 

There are however several complications. 

In particular there is no explicit inverse relation between the primitive quantities and the conserved ones. 
Thus one has to resort to the root-finding algorithms to reconstruct them (More on this in later chapters). 
In addition, it was pointed out that the $W$ couples the equation for the momenta in different direction \cite{Pons:2000,Rezzolla:2002ra,Rezzolla:2002cc,Aloy:2006rd}. 
This leads to the fact that the dynamics of the shock wave can be affected by the non-zero tangential velocity. 
Hence, the increased complexity if he problem of GR hydrodynamics \cite{Mignone:2005ns,Zhang:2005qy}.


%% ---

\subsubsection{Conclusion}

%% ---

%% Fluid Kinetics -- definitions & conserevation
In this section we discuss the equations of general relativistic hydrodynamics. 
We consider the fluid on a Lorentzian manifold and how its flow affects the spacetime; the fluid kinematics, dynamics in the absence of heat conduction and finally, the Valencia formulation".
%% Intro -- Newtonand and Realtivity desciptions
Notably the description of the fluid in the specail and general relativity differs from the one in the Newtonian case, as in the former, the fluid thermodynamic state is observer-dependent.
The "flux-conservative formulations" of hydrodynamics aim to address this issue by describing the fluid implicitly by a flow in space-time. 
The covariant description in terms of invariant quantities of the fluid, that is independent of the frame of reference, is required. 
%% Building the covariant fluid description
First, define the flow of \textit{fluid density} in space-time, $\boldsymbol{\rho}$, and the \textit{mass-flow} measured by an Eulerian observer (an integral of $\boldsymbol{\rho}$ over the spacelike hypersurface $\Sigma$). 
The conservation of the number of particles reads $\text{d}\boldsymbol{\rho}=0$. 
Next, the \textit{rest-mass density four-vector}, $\vec{j}$, which is a vector field associated with density, is defined. This four-vector is time-like or null as it is given by the flux of particles across any future-oriented spacelike hypersurface. 
If the former case there exists unique decomposition: $\vec{j} = \rho \vec{u}$, where $\rho$ can be seen as density in the comoving frame and unit-timelike vector $\vec{u}$ as a fluid four-velocity.
The divergence of vecotor $j$ then gives a mass conservation, $0 = \nabla_{\mu}j^{\mu}$
Next, the mixed tensor $\boldsymbol{T}$ is introduced to described flow of the $\nu$ momentum across the volume element orthogonal to $dx^{\mu}$ as ${T^{\mu}}_{\nu}=\boldsymbol{T}(dx^{\mu},\partial_{\nu})$.
The ${T^{\mu}}_{\nu}$ is the \textit{stress-energy tensor}. Notably, $\nabla_{\mu}{T^{\mu}}_{\nu} = 0$, if Einstein equations satisfies Bianchi identities, which however does not imply the conservation of the energy and momentum of the fluid in a general sense. For that the $\vec{\partial}_{\nu}$ ought to be a Killing vector
The $0 = \nabla_{\mu}j^{\mu}$ and $\nabla_{\mu}{T^{\mu}}_{\nu} = 0$ can be still through of as a mass, energy and momentum conservation.

%% Paragraph : dynamics of the relativistc fluid
We limit the discussion to the ideal fluid, omitting the effects of heat condition and viscosity.
\red{we remark on the effective treatment of viscosity in the ... somewhere} 
Taking the fluid's four velocity, $\vec{u}$, the stress-energy of the perfect fluid $\boldsymbol{T}$, has the following components:
$\boldsymbol{T}(\underline{u}, \vec{u})$ energy-density in the rest-frame of the fluid, the scalar $e$;
$\boldsymbol{T}(\underline{u}, \vec{e}_i) = 0$ represent the energy flowing transverse to the four-velocity, which we set to $0$ in the absence of the heat-conduction;
$\boldsymbol{T}(\underline{e}^i, \vec{e}_k) = 0$ represent the $k$ component of the force exchanged across the surface element orthogonal to $\underline{e}_i$.
The force exchange in the absence of viscosity can be effectively described by a scalar $p$, 
$\boldsymbol{T}(\underline{e}^i,\vec{e}_k) = p {\delta^i}_k$.
Thus, the stress-energy tensor reads
\begin{equation}
\boldsymbol{T} = (e + p)\vec{u}\otimes \underline{u} + p\boldsymbol{\delta}.
\end{equation}
Defining the \textit{enthalpy} of the fluid as $h = 1 + \epsilon = p/\rho$, where $\epsilon$ is the specific internal energy, we rewrite $\boldsymbol{T}$ as 
\begin{equation}
\boldsymbol{T} = \rho h \vec{u}\otimes\underline{u} + p\boldsymbol{\delta}
\end{equation}
In addition to the fluid kinematics,
$0 = \nabla_{\mu} j^{\mu}$ \ref{eq:theory:nablamu_jmu}
$0 = \nabla_{\mu}{T^{\mu}}_{\nu}$\ref{eq:theory:nablamu_tmunu}
and the description of motion
$\boldsymbol{T} = \rho h \vec{u}\otimes\underline{u} + p\boldsymbol{\delta}$ \ref{eq:theory:stressenergytensor},
an additional relation is required, called the \textit{equation of state}.
\red{in this thesis we focus on the microphysical equations of state}
Combined with an EoS, equations \ref{eq:theory:adm}, \ref{eq:theory:nablamu_jmu}, \ref{eq:theory:nablamu_tmunu} and \ref{eq:theory:stressenergytensor} form a hyperbolic
system of equations that can be evolved, once initial data is prescribed. 
The complete evolution of spacetime and the dynamics of the matter requires initial data to be set on the Cauchy surface.

%% The General-Relativistic Boltzmann Equation


%% ---------- 
\subsubsection{The General-Relativistic Boltzmann Equation}
%% -----------

\red{To understand this one has to know:
    manifolds, tangent vectors, tangent bundles, unit vectors and subbundles,
    vector fields, incompressible vector fields, 
    natural topology on a tangent bundle and natural projection;
    coordinate patch, and coordinate transformation between patches;
    connections on tangent bundles; push-forward, and pull-back operations
}

In special relativity the Boltzmann equation was expressed by Synge \cite{Synge:1957}. 
Later Chernikov \cite{Chernikov:1962} and Tauber and Weinberg \cite{Tauber:1961} proposed its extension to the general relativity.

The list of applications of the Boltzmann equation was limited to the relativistic gas at first \cite{Israel:1963}. 
Later the list was supplemented by transient relativistic thermodynamics \cite{Israel:1979wp}, radiative transfer \cite{Lindquist:1966}, core-collapse supernovae \cite{Bruenn:1985} and others (see \textit{e.g.}, \cite{Cercignani:2002} and references therein).

Different formulations of the general relativistic Boltzmann equation exists in the literature.
Lindquist \cite{Lindquist:1966} and Ehlers \cite{Ehlers:1971} proposed a geometrical interpretation. 
Later, a formulation based on Riemannian structure of tangent bundles was proposed by Sasaki \cite{Sasaki:1958,Sasaki:1962}. 
In addition, Debbasch and van Leuuwen \cite{Debbasch:2009a,Debbasch:2009b} recently provided a detailed derivation, albeit strongly focused on the algebraic aspects while eluding simple geometrical interpretation. 

Here we recall the detailed derivation of the general relativistic Boltzmann equation, using modern differential geometry notation by \red{Radice thesis}.

\textcolor{red}{This.Is.Tough. Pure math. Copied from David + his sources.}


\paragraph{The geometry of the tangent bundle}


Let the $\mathcal{M}$ be $4$ dimensional differential \textit{manifold} such that $(\mathcal{M},\: g_{\alpha\beta})$ form the $C^2$ spacetime. 
The set of \textit{tangent vectors} of $\mathcal{M}$ constitutes \textit{tangent bundle} of $\mathcal{M}$, the we denote as $T\mathcal{M}$. 
The set of all \textit{unit vectors} of $\mathcal{M}$ constitute the \textit{subbundle} of $T\mathcal{M}$.

\textcolor{gray}{incompressible vector field}.
\textit{Every Killing vector field of $\mathcal{M}$ is in incompressible vector field}


\paragraph{Extended transformation and extended tensors}

\red{excurse in basic theory}


Let the $T\mathcal{M}$ be the set of all the \textit{tangent vectors} of $\mathcal{M}$. 
The $T\mathcal{M}$ has a \textit{natural topology}, bundle structure with $\mathcal{M}$ and the base - linear vector space $E^i$. 
We call $T\mathcal{M}$ the \textit{tangent bundle} of $\mathcal{M}$. 
There exists a \textit{natural projection}, or a projection map $\pi:\: T\mathcal{M}\rightarrow\mathcal{M}$.

Let $U$ be a coordinate neighborhood, or a \textit{coordinate patch} of $\mathcal{M}$ with $n$ variables $x^{\alpha}$ as coordinates. 
Then, every \textit{tangent vector} of $\mathcal{M}$ at a point $p\in U$ is described with $2n$ variables $(x^i,\upsilon^{\alpha})$. Here $x^{\alpha}$ are coordinates of $p$ with respect to the coordinate patch ${x^{\alpha}}$ and $\upsilon^{\alpha}$ are components of a \textit{tangent vector} in the \textit{natural frame} that constitutes by the vectors $\partial/\partial x^4$ at $q$. Thus, the vector $\vec{p}$ at $q$ can be written as:

\begin{equation}
\vec{p} = p^{\alpha}\frac{\partial}{\partial^{\alpha}}
\end{equation}

and its dual as 

\begin{equation}
\underline{p} = p_{\alpha}dx^{\alpha}:=g_{\alpha\beta}p^{\beta}dx^{\alpha}
\end{equation}

In addition we introduce a \textit{coordiante patch} $TU$, $\{z^A\}$, where $A$ runs from $0$ to $7$ of $T\mathcal{M}$ as 

\begin{equation}
z^{\alpha} = z^{\alpha}, \hspace{10mm} z^{\alpha+4} = p^{\alpha}.
\end{equation}

Now, let the $U(x^{\alpha})$ and $\hat{U}(\hat{x}^{\alpha})$ be the two coordinate patches of $\mathcal{M}$ such that $U\cap\hat{U}$ is not empty. 
Then the \textit{intersection} of the coordinate patches is also not empty. 
For every coordinate transformation of $\mathcal{M}$, there exists a corresponding matrix $\frac{\partial \hat{x}^{\alpha}}{\partial x^{\beta}}$.

The \textit{coordinate transformation} is then

\begin{equation}
\hat{x}^{\mu} = \hat{x}^{\mu}(x), \hspace{5mm} \hat{p}^{\mu} = \frac{\partial\hat{x}^{\mu}}{\partial x^{\nu}}p^{\nu}
\end{equation}

which denotes the extended transforation of the $\hat{x}^{\mu} = \hat{x}^{\mu}(x)$.

The corresponding Jacobian matrix is 
\renewcommand\arraystretch{1.6} %% it stretches the matrix
\begin{equation}
\frac{\partial\hat{z}^A}{\partial z^B} = 
\begin{pmatrix}
\frac{\partial\hat{x}^{\alpha}}{\partial x^{\beta}} & 0 \\
\frac{\partial^2\hat{x}^{\alpha}}{\partial x^{\beta} \partial x^{\gamma}}p^{\gamma} & \frac{\partial\hat{x}^{\alpha}}{\partial x^{\beta}} 
\end{pmatrix}
\end{equation}
\renewcommand\arraystretch{1.0}


\paragraph{Vectors on $T\mathcal{M}$}


As we will need to introduce \textit{connections on a tangent bundle}, here we discuss the \textit{double tangent bundle}, or a second tangent bundle. 
Since $T\mathcal{M}$ is a \textit{vector bundle} on its own right, its \textit{tangent bundle} has the secondary vector bundle structure $TT\mathcal{M}$. 
Let the point $b\in TU$ and $T_b T\mathcal{M}$ be the \textit{tangent space} to $T\mathcal{M}$ at $b$.
Given a vector $\partial/\partial x^{\alpha}$ at a point $b$, it can be "pushed forward" to the point on the $TT\mathcal{M}$ by means of so called \textit{differential of} $\pi$, written as $\pi_*$ \cite{Frankel:2002}.

On a natural basis the \textit{push-forward} acts as 

\begin{equation}
\pi_*\Big[\frac{\partial}{\partial x^{\alpha}}\Big] = \frac{\partial}{\partial x^{\alpha}}, \hspace{5mm} \pi_* \Big[\frac{\partial}{\partial p^{\alpha}}\Big] = 0,
\end{equation}

and the \textit{pull back} as  

\begin{equation}
\pi^* {\text d} x^{\alpha} = {\text d} x^{\alpha}.
\end{equation}

Consider a vector field $\vec{X} \ in TT\mathcal{M}$ in a vicinity of the point $b$, which is associated with the point $q$ of $\mathcal{M}$ and vector $\vec{x}\in T_{q}\mathcal{M}$. 
Let $b{\lambda}$ be the flow of $b$ generated by $\vec{X}$. 
The $b(\lambda)$ is associate with $q(\lambda)$, the \textit{one parameter family of points} of $\mathcal{M}$. 
The $b(\lambda)$ is also associated with $\vec{x}(\lambda)$ the \textit{one parameter family of vectors} on $T\mathcal{M}$.

The vector field $\vec{X}$ is called \textit{vertical} if the $q(\lambda)\in\mathcal{M}$ are constant along the flow.
Similarly, the vector field $\vec{X}$ is called \textit{horizontal} if $\vec{x}(\lambda)\in T_p \mathcal{M}$ is "constant" along the flow, meaning that $\vec{x}(\lambda)$ is just $\vec{x}$ that is \textit{parallel transported} to $q(\lambda)$.

As there is no unique way to perform a \textit{parallel transport}, the linear connection $\nabla$ on $\mathcal{M}$ has to be chosen. 
This choice is akin choosing two vector spaces $\mathcal{O}_b$ and $\mathcal{V}_b$ of the horizontal and vectical vectors respectively at each point $b$ that the direct sum of these spaces yields

\begin{equation}
\mathcal{O}_b\oplus \mathcal{V}_p = T_b T\mathcal{M}.
\end{equation}

Having the connection allows to prescribe a manner of \textit{lifting curves} from the base manifold $T\mathcal{M}$ into the $T_b T\mathcal{M}$.
\textcolor{red}{I need to fix this and understand}. 
A \textit{lift} is the unique horizontal vector $\vec{X}\in T_bT\mathcal{M}$ whose projection is a vector $\vec{x}\in T_q\mathcal{M}$.

\textcolor{red}{fill it}

Let us now define a \textit{connection vector basis} adopted to the aforementioned split of $T_b T\mathcal{M}$ $\{\text{D}/\partial x^A \}:=\{\text{D}/\partial x^{\alpha}, \partial/\partial p^{\alpha} \}$ where 

\textcolor{red}{I did not find where this is derived from... difficult}

\begin{equation}
\frac{\text{D}}{\partial x^{\alpha}}{\partial x^{\alpha}} := \frac{\partial}{\partial x^{\alpha}} - {\Gamma^{\beta}}_{\alpha\gamma}p^{\gamma}\frac{\partial}{\partial p^{\beta}}.
\end{equation}

Similarly a connection can be build for differential forms. 
Using the pull-back $\pi^*$ the dual basis $\{ \text{D}z^{A} \}:=\{\text{d}x^{\alpha}, \text{D}p^{\alpha}\}$ that satisfies 

\begin{equation}
\text{D} = \text{d} p ^{\alpha} + {Gamma^{\alpha}}_{\beta\gamma}p^{\gamma}\text{d}x^{\beta}.
\end{equation}


\paragraph{Metric on $T\mathcal{M}$}


Note that 

\begin{equation}
\frac{\partial^2 \hat{x}^{\mu}}{\partial x^{\nu}\partial x^{\lambda}}p^{\lambda} = {\hat{\Gamma}^{\mu}}_{\delta\gamma}p^{\lambda}\frac{\partial\hat{x}^{\delta}}{\partial x^{\nu}}.
\end{equation}

Let us assume that for any point $b\in T\mathcal{M}$ there exist an open set $TU$, such that $b\in TU$ with a coordinate system on $TU$ that satisfies

\begin{equation}
G_{AB} = (\boldsymbol{\eta}\otimes\boldsymbol{\eta})_{AB},
\end{equation}

where $\boldsymbol{\eta} = \text{diag}(-1, 1, 1, 1)$. 

Let the $\hat{x}^A$ denote the generic coordinate system on $TU$. 
Then the metric in this coordinate system can be expressed as

\begin{align}
\hat{G}_{\mu\nu} &= \frac{\partial \hat{x}^{\alpha}}{\partial x^{\mu}}\frac{\partial \hat{x}^{\beta}}{\partial x^{\nu}}\eta_{\alpha\beta} + \frac{\partial \hat{x}^{\alpha}}{\partial x^{\mu}}{\hat{\Gamma}^{\gamma}}_{\:\:\:\alpha\lambda}p^{\lambda}\frac{\partial \hat{x}^{\beta}}{\partial x^{\nu}}{\hat{\Gamma}^{\delta}}_{\:\:\:\beta\xi}p^{\xi}\eta_{\gamma\delta}; \\
\hat{G}_{\mu\: \nu+4} &= \frac{\partial \hat{x}^{\alpha}}{\partial x^{\mu}}\frac{\partial \hat{x}^{\gamma}}{\partial x^{\nu}}{\hat{\Gamma}^{\beta}}_{\:\:\:\gamma\lambda}p^{\lambda}\eta_{\alpha\beta}; \\
\hat{G}_{\mu+4 \: \nu+4} &= \frac{\partial \hat{x}^{\alpha}}{\partial x^{\mu}}\frac{\partial \hat{x}^{\beta}}{\partial x^{\nu}} \eta_{\alpha\beta}
\end{align}

and the line element 

\begin{align}
dS^2 &= \hat{G}_{AB}d\hat{z}^A d\hat{z}^B = \hat{g}_{\mu\nu}\text{d}\hat{x}^{\mu}\text{d}\hat{x}^{\nu} + \hat{g}_{\mu\nu}[\text{d}p^{\mu} + {\hat{\Gamma}^{\mu}}_{\:\:\:\alpha\beta}p^{\beta}\text{d}x^{\alpha}] [\text{d}p^{\nu} + {\hat{\Gamma}^{\nu}}_{\:\:\:\alpha\beta}p^{\beta}\text{d}x^{\alpha}] \\
&= \hat{g}_{\mu\nu}\text{d}\hat{x}^{\mu}\text{d}\hat{x}^{\nu} + \hat{g}_{\mu\nu}\text{D}\hat{x}^{\mu}\text{D}\hat{x}^{\nu}
\end{align}

It is possible to show that the \textit{determinant} $|\text{det}\boldsymbol{G}| = g^{2}$ as the transformation from the natural frame to the connection frame is \textit{unimodular} \cite{Lindquist:1966}. 
Thus the \textit{volume pseudo-form} on $T\mathcal{M}$ is in the coordiante patch $TU$

\begin{align}
\text{Vol}^8 &:= -g \text{d}x^{0} \wedge \text{d}x^{1} \wedge ... \wedge \text{d}p^{3} := - g\text{d}^{4}x \text{d}^{4}p, \\
&:= -g \text{d}x^{0} \wedge \text{d}x^{0} \wedge ... \wedge \text{D}p^{3} :=-g \text{d}^{4}x\text{D}^4 p
\end{align}

\textcolor{red}{I kinda gave up here and just copied.}


\paragraph{the Liuville theorem}


Let us start by introducing a \textit{cotangent bundle}. 
Let $\mathcal{M}$ be a \textit{differentiable manifold}. 
Similarly to the construction of the tangent bundle, we can make a set of covectors on a given manifold into a vector bundle over $\mathcal{M}$, denoted $T^*\mathcal{M}$ and called \textit{cotangent bundle} of $\mathcal{M}$. 
Similarly we can define a contangent bundle of a tangent one $T^*T\mathcal{M}$. 
The contangent bundle $T^*\mathcal{M}$ is the vector bundle dual to the tangent bundle $T\mathcal{M}$. 

Let us start by defining \textit{Poincar\'e} 1-form on $T\mathcal{M}$, $\underline{\lambda}\in T^* T\mathcal{M}$. 
Consider point $q$ on a manifold $\mathcal{M}$ and a point $A$ associated with $q$ on a tangent bundle $T\mathcal{M}$. 
Let there be a 1-form $\underline{\alpha}\in T^* _q\mathcal{M}$. 
The $\underline{\lambda}$ and $\underline{\alpha}$ are uniquely connected $\underline{\lambda} = \pi^* \alpha$, and the former is called the \textit{Poincar\'e} 1-form. 
In local coordinate patch, $TU$ it is expressed as

\begin{equation}
\underline{\lambda} = p_{\alpha} \text{d}x^{\alpha}.
\end{equation}

the associated vector is 

\begin{equation}
\vec{\lambda} = p^{\alpha} \frac{\text{D}}{\partial x^{\alpha}} = p^{\alpha}\frac{\partial}{\partial x^{\alpha}} - p^{\alpha}{\Gamma^{\beta}}_{\alpha\gamma}p^{\gamma}\frac{\partial}{\partial p^{\beta}},
\end{equation}

is called the $\textit{geodesic flow field}$.

This flow represents a phase-space flow of particles moving along geodesics.

Consider a mass shell, that at a point $q\in U$ can be defined as a set:

\textcolor{red}{remider: I have no idea how is this possible...}

\begin{equation}
\mathcal{S}_m = \big\{ p^{\alpha}\in T_q\mathcal{M}: p_{\mu}p^{\mu}+m^2 =:f(p) = 0 \big\}.
\end{equation}

The normal to the mass-shell is 

\begin{align}
\text{if } m &\neq 0 \hspace{5mm} \underline{\pi}:=\frac{q}{2m}\text{d}f, \hspace{5mm} \text{d}f = \frac{\partial f}{\partial x^{\mu}} + \frac{\partial f}{\partial p^{\mu}}\text{d}p^{\mu} = 2p_{\mu}\text{d}p^{\mu}, \\
\text{if } m &= 0 \hspace{5mm} \underline{\pi}:=\frac{1}{2}\text{d}f
\end{align}

Next, we introduce a unique form $\underline{\nu}$ whose restriction on $T_q\mathcal{M}$ is equal to $\underline{\pi}$. 

\begin{align}
\text{if } m &\neq 0 \hspace{5mm} \underline{\nu} = \frac{1}{m}p_{\alpha}\text{D}p^{\alpha} \\
\text{if } m &= 0 \hspace{5mm} \underline{\nu} = p_{\alpha}\text{D}p^{\alpha}
\end{align} 

Note that $\underline{\nu} = 0$, meaning that the $\underline{\nu}$ is \textit{irrotational}. 
It becomes clear if we re-express it as 

\begin{equation}
\underline{\nu} = \frac{1}{2m}\frac{\text{D}f}{\partial p^{\alpha}}\text{D}p^{\alpha}
\end{equation} 

for massive particle case. For the mass-less the procedure is analogous.

It can be shown that $\underline{\lambda}$ is incompressible \textit{i.e.,} $\text{d}^{\star}\underline{\lambda} =\star \text{d}\star\underline{\lambda} =0$.

In addition, both $\underline{\lambda}$ and $\underline{\nu}$ are \textit{harmonic forms} as 

\begin{equation}
\nabla\underline{\nu} = 0 , \hspace{5mm}
\nabla\underline{\lambda} = [\text{dd}^{\star} + \text{d}^{\star}\text{d}]\underline{\lambda} = 0.
\end{equation}

Let us now consider the density of states in the phase space, of particles moving along the geodeiscs with velocities on the mass shell.

In the previous section we introduced a \textit{flux of the vector field} $\vec{X}$ across $\Sigma$ in \ref{eq:theory:flux_of_flow}, we define the following six-form

\begin{align}
\boldsymbol{\omega} &= \star\big(\underline{\nu}\wedge\underline{\lambda}\big) = i_{\vec{\lambda}} i_{\vec{\nu}}\text{Vol}^8 \\
&= i_{\vec{\lambda}}\Big[i_{\vec{\nu}}\big(\text{Vol}^{4}_{x}\wedge\text{Vol}^{4}_{p}\big)\Big] = i_{\vec{\lambda}} \big[\text{Vol}^{4}_{x}\wedge\text{Vol}^{3}_{p}\big],
\end{align}

where we used the definition of $\text{Vol}^8$. 

The introduced four forms read,

\begin{align}
\text{Vol}_x ^4 &:= \sqrt{-g} \text{d}x^{0} \wedge \text{d}x^{1} \wedge \text{d}x^{2} \wedge \text{d}x^{3}, \\
\text{Vol}_p ^4 &:= \sqrt{-g} \text{D}p^{0} \wedge \text{D}p^{1} \wedge \text{D}p^{2} \wedge \text{D}p^{3}, \\
\text{Vol}_p ^3 &:= i_{\vec{\nu}}\text{Vol}_p ^4,
\end{align}

where the four-forms are on the $TU$ and the latter three-form is on the mass shell $S_m$.

Consider coordinates adopted to the mass-shell, where $\underline{\nu} = (p_0/m)\text{D}p^0$ and $\underline{\nu} = p_0\text{D}p^0$ in the massive and massless cases respectively, the three-form becomes

\begin{equation}
\text{Vol}^3 _p =\frac{\sqrt{-g}}{-p_0}\text{D}p^1\wedge\text{D}p^2\wedge\text{D}p^3
\end{equation}

Now we have a three-form $\text{Vol}^3 _p$ and a four-form $\text{Vol}_x ^4$. 
In the context of the ADM foliation, we split spacetime manifold as $\mathcal{M}=\mathcal{R}\times\Sigma$, with $x^0 = \text{const}$ being constant hypersurfaces with normal $\underline{n} = - \alpha\text{d}x^0$ and $\alpha$ -- the lapse function. We can now simplify the $\boldsymbol{\omega}$, splitting $\text{Vol}_x ^3$ as 

\begin{align}
\text{Vol}_x ^4 &= -\underline{n}\wedge\text{Vol}_x ^3 \hspace{5mm} \text{where,} \\
\text{Vol}_x ^3 &= i_{\vec{n}}\text{Vol}_x ^4 = \sqrt{\gamma}\text{d}x^1\wedge\text{d}x^2\wedge\text{d}x^3
\end{align}

and the $\boldsymbol{\gamma}$ is the three-metric induced on the slices.

The resulted coordinates, adapted to the mass shell and the spacetime foliation read

\begin{align}
\boldsymbol{\omega} &=-(\vec{p}\cdot\vec{n})\frac{1}{-p_0}\sqrt{\gamma}\sqrt{-g}\text{d}x^1\wedge\text{d}x^2\wedge\text{d}x^3 \wedge\text{D}p^2\wedge\text{D}p^2\wedge\text{D}p^3 \\
&= \frac{p^0}{-p_0}|g|\text{d}x^1 \wedge\text{d}x^2\wedge\text{d}x^3 \wedge\text{D}p^2\wedge\text{D}p^2\wedge\text{D}p^3
\end{align}

Now, consider a six-vector, $\delta_i x \delta_i p$ with $i\in\{1,2,3\}$. 
The $\delta_i x$ are tangent vectors to the slice $\Sigma$ and the $\delta_i p$ are tangent to mass shell $S_m$.

The action of $\boldsymbol{\omega}$ on the six-vectors $\delta_1 x$, $\delta_2 x$, $\delta_3 x$, $\delta_1 p$, $\delta_2 p$, $\delta_3 p$ yields

\begin{align}
\boldsymbol{\omega}(\delta_1 x,...,\delta_3 p) =& \frac{p^0}{-p_0}|g|\big[\text{d}x^1\wedge\text{d}x^2\wedge\text{d}x^3\big](\delta_{1}x,\delta_{2}x,\delta_{3}x)\times \\
& \hspace{10mm} \Big[\text{D}p^1\wedge\text{D}p^2\wedge\text{D}p^3\Big](\delta_1 p, \delta_2 p, \delta_3 p) \\
& \hspace{2mm} -\frac{p^0}{-p_0}|g|\big[\text{d}x^1\wedge\text{d}x^2\wedge\text{d}x^3\big](\delta_{1}p,\delta_{2}p,\delta_{3}p)\times \\
& \hspace{10mm} \Big[\text{D}p^1\wedge\text{D}p^2\wedge\text{D}p^3\Big](\delta_1 x, \delta_2 x, \delta_3 x) = \\
=& \frac{p^0}{-p_0}|g|\big[\text{d}x^1\wedge\text{d}x^2\wedge\text{d}x^3\big](\delta_{1}x,\delta_{2}x,\delta_{3}x)\times \\
& \hspace{10mm} \Big[\text{D}p^1\wedge\text{D}p^2\wedge\text{D}p^3\Big](\delta_1 p, \delta_2 p, \delta_3 p) = \\
=& \frac{p^0}{-p_0}|g|\big[\text{d}x^1\wedge\text{d}x^2\wedge\text{d}x^3\big](\delta_{1}x,\delta_{2}x,\delta_{3}x)\times \\
& \hspace{10mm} \Big[\text{d}p^1\wedge\text{d}p^2\wedge\text{d}p^3\Big](\delta_1 p, \delta_2 p, \delta_3 p), \\
\end{align}

where we used that $\text{d}x^i(\delta_j p)=0$ and the relation

\begin{equation}
\text{D}p^{i}(\delta_j p) = \text{d}p^{i}(\delta_j p) - {\Gamma^i}_{\alpha\beta}p^{\alpha}\text{d}x^{\beta}(\delta_j p) = \text{d}p^i(\delta_j p)
\end{equation}

Thus, on the space-like hypersurface $\Sigma$ and on the mass shell we have

\begin{equation}
\boldsymbol{\omega} = \frac{p^0}{-p_0}|g|\text{d}x^1\wedge\text{d}x^2\wedge\text{d}x^3\wedge\text{d}p^1\wedge\text{d}p^2\wedge\text{d}p^3 =: \boldsymbol{\Omega}
\end{equation}

The $\boldsymbol{\Omega}$ can be split as 

\begin{align}
\boldsymbol{\Omega} &= \boldsymbol{\Lambda} \wedge \boldsymbol{\Pi}, \hspace{5mm} \text{where} \\
\boldsymbol{\Lambda} &= p^0 \sqrt{-g}\text{d}x^1\wedge\text{d}x^2\wedge\text{d}x^3 \\
\boldsymbol{\Pi} &=  \frac{1}{-p_0}\text{d}p^1\wedge\text{d}p^2\wedge\text{d}p^3
\end{align}

The defined forms $\boldsymbol{\Lambda}$ and $\boldsymbol{\Pi}$ can be written in a coordinate-independent way at any point $q\in\mathcal{M}$ as 

\begin{equation}
\boldsymbol{\Lambda} = \star_{\mathcal{M}}\underline{\lambda}, \hspace{5mm} \boldsymbol{\Pi} = \star_{T_q\mathcal{M}}\underline{\pi}
\end{equation}

and this are intrinsic forms in $T\mathcal{M}$. 
In addition, the $\boldsymbol{\Lambda}$ and $\boldsymbol{\Pi}$ are the proper geodesics flux
volume form on $\Sigma\in\mathcal{M}$ and mass shell $S_m\in T_q\mathcal{M}$ at a point $q\in U$ respectively.

Let us now consider the geodesic flow $\vec{\lambda}$. 
It generates a "tube" in a phase space, that we limit with $S_1$ and $S_2$ sections. 
Then the flux of points in phase space associated with geodesic flow is $\int_{S}\boldsymbol{\omega}$. 
It is possible to show that the flux satisfies

\begin{equation}
\int_{S_1}\boldsymbol{\omega} = \int_{S_2}\boldsymbol{\omega}
\label{eq:theory:liuville}
\end{equation}

which is the \textit{Liouville’s Theorem} in the relativistic case.

To see that this is indeed the case, consider the \textit{exterior differential} of $\boldsymbol{\omega}$

\begin{equation}
\star\text{d}\omega = \text{d}^{\star}(\underline{\nu}\wedge\underline{\lambda}) = d^{\star}\underline{\nu}\wedge\underline{\lambda} + \underline{\nu}\wedge\text{d}^{\star}\underline{\lambda}.
\end{equation}

The $\text{d}^{\star}=\text{const}=k$ as $\text{dd}^{\star}\underline{\nu}=0$. 
In addition, the $\text{d}^{\star}\underline{\lambda}=0$. 
This allow us to write 

\begin{equation}
\text{d}\omega = -k(\star\lambda).
\end{equation}

We note the $\star\underline{\lambda}$ is the volume form of the hypersurfaces orthogonal to $\vec{\lambda}$. 
Hence, the $\star\underline{\lambda}[\vec{\lambda},...]=0$ along the "tube" in phase space. 

\begin{equation}
\int_S\text{d}\boldsymbol{\omega} = 0.
\end{equation}

the \ref{eq:theory:liuville} is recovered, if we use the \textit{Stoke’s Theorem}, and using the fact that the $\boldsymbol{\omega}$ vanishes along the part of the boundary tangent to $\vec{\lambda}$.

\textcolor{red}{Note that I still have no Idea what I have written. I need to go through the original materal, which I could not find... at least I could not find what I could read and understand. }


\paragraph{The Boltzmann equation}


Let us introduce the phase space version of the \textit{mass flux}, the 6-form representing the number of phase-space trajectories crossing $S$ of the phase tube between $S_1$ and $S_2$ cross sections. 

In the absence of collisions we obtain 

\begin{equation}
\int_{S_1}\boldsymbol{\mu} = \int_{S_2}\boldsymbol{\mu}.
\end{equation}

Remembering that $\boldsymbol{\omega}$ represents the \textit{density of states} in phase space of particles moving along geodesics, we obtain 

\begin{equation}
\boldsymbol{\mu} = F\boldsymbol{\omega},
\end{equation}

where $F$ is \textit{invariant distribution function}, \textcolor{gray}{i.e. $F$ is the Radon-Nikodym derivative of $\boldsymbol{\mu}$ with re $\boldsymbol{\omega}$}.

Consider now that collisions change the number of phase trajectories as 

\begin{equation}
\delta N = \int_{S_2} \boldsymbol{\mu} - \int_{S_1}\boldsymbol{\mu} = \int_S \text{d}\boldsymbol{\mu} = \int_S \text{d}F\wedge\boldsymbol{\omega}.
\end{equation}

where 

\begin{equation}
\text{d}F\wedge\boldsymbol{\omega} = \text{d}F\wedge\star (\underline{\nu}\wedge\underline{\lambda}) = \langle\text{d}F,\underline{\lambda}\rangle\star\underline{\nu} - \langle\text{d}F,\underline{\nu}\rangle\star\underline{\lambda},
\end{equation}

where $\langle\cdot,\cdot\rangle$ is a scalar product between forms and can be written as

\begin{equation}
\langle\boldsymbol{\alpha},\boldsymbol{\beta}\rangle\text{Vol}^8 := \boldsymbol{\alpha}\wedge\star\boldsymbol{\beta},
\end{equation}

and with the $\star\underline{\lambda}=0$ on $S$ we obtain 

\begin{equation}
\delta N = \int_S\langle\text{d}F\underline{\lambda}\rangle\star\underline{\nu} = \int_S\mathcal{C}[F]\star\underline{\nu},
\end{equation}

where we denoted the effect of collisions as

\begin{equation}
\langle\text{d}F\underline{\lambda}\rangle = \mathcal{C}[F].
\end{equation}

This is the \textit{Boltzmann equation}. 

In component from it reads as 

\begin{equation}
p^{\alpha}\frac{\partial F}{\partial x^{\alpha}} - {\Game^{\gamma}}_{\alpha\beta}p^{\alpha}p^{\alpha}\frac{\partial F}{\partial p^{\gamma}} =\mathcal{C}[F].
\end{equation}

In the coordinate system adapted to the equation, when $P^0 = p^0(p^i)$, the equation reads \cite{Cercignani:2002}:

\begin{equation}
p^{\alpha}\frac{\partial F}{\partial x^{\alpha}} - {\Game^{i}}_{\alpha\beta}p^{\alpha}p^{\alpha}\frac{\partial F}{\partial p^{i}} =\mathcal{C}[F].
\end{equation}

Remembering that the $\underline{\lambda}$ is incompressible, we can write

\begin{equation}
\langle\text{d}F,\underline{\lambda}\rangle = \text{d}^{\star}[F,\underline{\lambda}],
\end{equation}

This allows to obtain a conservative formulation of the Boltzmann equation, that indicates the conservation of the number of particles \cite{Cardall:2002bp}

\begin{equation}
\text{d}^{\star}[F\underline{\lambda}] = \mathcal{C}[F].
\end{equation}

Next, we re-introduce the Levi Civita connection $\nabla$ in phase space. 
For incompressible $\underline{\lambda}$ it gives

\begin{equation}
\nabla_A\lambda^A=0,
\end{equation}

while the \textit{Boltzmann equation} becomes 

\begin{equation}
\lambda^A\partial_A F=\mathcal{C}[F]
\end{equation}

and its \textit{conservative} form 

\begin{equation}
\nabla_A[Fp^{A}] = \mathcal{C}[F],
\label{eq:theory:liouvilletheorem}
\end{equation}

or in component form

\begin{equation}
\frac{1}{|g|}\frac{\partial}{\partial x^{\mu}}\Bigg[|g|Fp^{\mu}\Bigg] + \frac{p_0}{|g|}\frac{\partial}{\partial p^{k}}\Bigg[\frac{|g|}{-p_0}{\Gamma^k}_{\alpha\beta}p^{\alpha}p^{\beta}F\Bigg] = \mathcal{C}[F].
\end{equation}



\paragraph{From the Boltzmann Equation to the Euler Equation}



In order to derive from the Boltzmann equation the equations of hydrodynamics, we need to first define the needed varaibles of the kinetic description, such as \textit{mass} and \textit{energy} fluxes. 

We not that the density flux can easly be obtained from $\boldsymbol{\mu}$. 

We write the \textit{mass flow} then as 

\begin{equation}
\boldsymbol{\rho} = \int_{S_m} \boldsymbol{\mu}.
\end{equation}

Recalling the definition of the rest-mass density four-vector $J$, we note that

\begin{equation}
\int_{\Sigma}(-J^{\mu}n_{\mu})\text{Vol}_x ^3 = \int_{\Sigma\times S_{m}} Fp^0\boldsymbol{\Pi}\text{Vol}_x ^3 = \int_{\Sigma\times S_{m}} F\boldsymbol{\Omega} = \int_{\Sigma}\boldsymbol{\rho}
\end{equation}

Thus, the $J$ can be written as 
\textcolor{red}{this is not clear how it was obtain. Must go through again.}

\begin{equation}
J^{\mu} = \int_{S_m}Fp^{\mu}\boldsymbol{\Pi}
\end{equation}

The second moment of the distribution function $F$ in a similar way gives the \textit{stress energy tensor} 

\begin{equation}
T^{\mu\nu} = \int_{S_m} F p^{\mu}p^{\nu}\boldsymbol{\Pi}
\end{equation}

the components of which in the frame comoving with the fluid are

\begin{equation}
T^{\mu\nu} = 
\begin{pmatrix}
E & \vec{F} \\
\vec{F} & \boldsymbol{P} \\
\end{pmatrix}
\end{equation}

where $E$ and $\vec{F}$ are the \textit{energy density} and \textit{flux} respectively, and $\boldsymbol{P}$ is the stress tensor. 

The nature of the collisional operation ultimately defines the equilibrium configuration distribution function $F$ \cite{Cercignani:2002}, and thus the form of the stress-energy tenor. 

Now we use the Liouville Theorem to obtain the equations of hydrodynamics. 
To accomplish that we insert the $\boldsymbol{\Psi}$, a tensorial function of $p^i$, into the theorem, equation \ref{eq:theory:liouvilletheorem} on both sides and integrate with respect to $\boldsymbol{\Pi}$ as

\begin{equation}
\int_{\text{I\!R}}\nabla_{A}[F\lambda^A\boldsymbol{\Psi}]\boldsymbol{\Pi}=\int_{\text{I\!R}}\mathbb{C}[F]\boldsymbol{\Psi}\boldsymbol{\Pi}
\end{equation}

where we used that $\lambda^A\nabla_{A}\boldsymbol{\Psi}=0$ as $\vec{\lambda}$ is the geodesic flow. 

Letting the $F$ decay for large momenta we obtain the \textit{transfer equation} \cite{Israel:1963,Cercignani:2002}:

\begin{equation}
\nabla_{\mu}\int_{\text{I\!R}} F\boldsymbol{\Psi}p^{\mu}\boldsymbol{\Pi} =\int_{\text{I\!R}} \mathcal{C}[F]\boldsymbol{\Psi}\boldsymbol{\Pi},
\label{eq:theory:transferequation}
\end{equation}

\textcolor{red}{check the sources. White intermediate steps}

In a particular case of a simple gas $\Psi$ can be shown to be one of the $\{1,p^0,p^1,p^2,p^3\}$ \cite{Cercignani:2002}. 
Then the right hand side of the transfer equation becomes 

\begin{equation}
\int_{\text{I\!R}} \mathcal{C}[F]\boldsymbol{\Psi}\boldsymbol{\Pi} = 0.
\end{equation}

These $\boldsymbol{\Psi}$ are related to the quantities conserved by the collisional operator and are called \textit{collisional invariants}. 
Choice of $1$ would yield the \textit{mass conservation}, 
while $p^{\mu}$ -- the \textit{energy and momentum conservation}.

Now, having canceled the R.H.S of the eq. \ref{eq:theory:transferequation}, we obtain the conservation laws in a following form

\begin{equation}
\nabla_{\mu}J^{\mu} =0 \hspace{10mm}\nabla_{\nu}T^{\mu\nu} =0
\end{equation}


\subsection{Overview}

We start this section by revisiting the fundamental concepts, such as manifold, tangent and cotangent bundles, with vectors and differential forms defined on them, and operations such ans exterior, Wedge product and Hodge star operator. 

Then we set ourselves a goal to obtain the covariant form of general relativistic hydrodynamics. 
This includes the equations for space-time evolution adopted for use in numerical applications and Euler equations for the (ideal) fluid, which we aim to obtain through the Liuville's theorem and Boltzmann equations.

To derive the Einstein field equations we perform the variation of the so-called Hilbert action, applying the Euler-Lagrange equation. For that we show a quick derivation of Euler-Lagrange equations, using the fact that the fields we are interested in are defined over only a compact domain and that the choice of the variation of coordinates is arbitrary, i.e. $\partial S(\boldsymbol{q}, \nabla \boldsymbol{q}) = 0$. 
Then, in a similar way, the variation of the Hilbert action, yields the Einstein Field equation.

For practical applications it is useful to express the EFE as a initial value boundary problem. 
The Hamiltonian formalism allows to do that, which we briefly review. We then sketch the $3+1$ decomposition procedure, introducing the spacelike foliation, the shift vector $\vec{\beta}$ and lapse function $\alpha$, and extrinsic curvature.
The goal is now to cast the EFE as a set of constraint equations, satisfied on every slice of the foliation, and evolution equations.
%% that allow us to obtain the constraint equations, that has to be satisfied on every hyper-surface of the hypersurface. 
Employing the EFE and Gauss-Codacci equations we write the Hamiltonian density, whose variation with respect to the variables of foliation $\alpha$ and $\beta$ yields constraint equation. 
The evolution equations then are obtained through the variation of the Hamiltinan with respect to the three-metric and momentum.

The obtained ADM system is however not well suited for numerical applications, being only weekly hyperbolic. 
We thus briefly touch on a strongly hyperbolic formulation, the Z4 formulation, that exhibit such usefull for numeric properties as constraint violation dampening (constrain preservation) and its evolution, the CCZ4 formulation that is further adopted for BH evolution. 

The choice of the space-time foliation, the chose of $\alpha$ and $\vec{\beta}$ is left free however. 
Their choice refers to the gauge conditions. 
We recall the most widely used "maximal slicing" and $1+\log$ conditions.



Then we proceed with deriving equations of general relativistc hydrodynamics, aiming to provide a flux-conservative formulation. 
We first define the kinematics of the relativistic fluid, \textit{i.e.,} a covariant description in terms of invariant quantities. 
With this goal in mind we define the rest-mass density vector, whose nullified divergence is the number of particles conservation. 
Then we re-introduce the stress energy tensor ans show that via Bianki identities, its divergence also vanishes.

To discuss the dynamics of the fluid, we first, set its type. 
We consider the fluid with no thermal conductivity and no viscosity, \textit{i.e.}, the perfect fluid. 
In addition to fluid kinematics and the stress-energy tensor, describing its motion, the equation of state is needed. 
Together they from a hyperbolic system of equations that describes the evolution of the fluid in space-time, once the initial data is set.

For the reasons of numerical stability, a special formulation of the equations of general relativistic hydrodynamics is required. 
Such is the Valencia formulation. The main idea is to contract an advection-like equations for fluxes of conserved quantities, from which the primitive quantities can be reconstructed.

To derive these fluxes we first decompose the four velocity into the component parallel to the normal to the hypersurface and a purely spatial part. 
This leads us to the definition of the conserved density. Then we introduce a vector whose zeroth component is just a norm and spatial component which is a \textcolor{gray}{tangent vector to hypersurface}. 
This allows us to write the conserved and primitive quantities of the formulation, as well as the source term. 
\todo{not all quntities in Val.Form. are clear. What is $S_j$ and $\epsilon$}

Then we set an goal of writing the Euler equation for the fluid, based on the phase-spahce decription of it, with a help of Luiville's theorem and general-relativistic Boltzmann equation, the latter of which is derived via geometrical approach.
To do that we first introduce necessary tools, namely vectors, co-vectors, tangent and cotangent bundles.
There are $2n$ components with the first $n$ being coordinates and the second $n$ being impulses. 
In addition, we introduce the coordinate transformation and finally, the metric on the tangent bundle.

Having tools set, we derive the Liuville theorem. In order to do that we write phase-space flow of particles moving along geodesics which is represented by the Poincare 1-form and associated vector.
In addition we define a mass shell, a norm to it and a irrotational form on a tangent bundle. 
Together with the Poincare form it allows us to define the density of the phase-space trajectories which we denote as a Hodge operator of the wedge product of these two forms. 
After some calculations, we obtain that this form can be expressed in a coordinate independent way as a split of two forms, the proper geodesics flux forms on the hypersurface and the mass shell respectively.
By considering the "phase tube" with two crossections, we observe that integrated flux is conserved, which constitutes the Liouville\'s Theorem.

After that we proceed with deriving the Boltzmann equation. 
We start by introducing the number of phase-space trajectories crossing the section of a "phase tube". 
The relation between the number of the phase space trajectories and the density defined above, yields the invariant distribution function. 
Considering the change in the number of particles due to collisions. 
The change in scalar product between the exterior derivative of the distribution function and Poincare 1-form due to collisions constitutes the Boltzmann equation. 
\textcolor{red}{revise this.}. 
Re-introducing the Levi Civita connection in phase space, and taking an advantage of the incompressibility if the poincare 1-form, we obtain a conservative form of the Boltzmann equation.

From Boltzmann equation we can now obtain an equations of hydrodynamics. For that we first redefine the variables of the kinetic description, such as mass and energy fluxes, recalling the definition of the rest-mass density four-vector. 
Similarly how the vector can be now expressed as a first moment of the distribution function, the second moment gives the stress energy tensor. However, we note that the nature of the collisional operation ultimately defines the equilibrium configuration distribution function and thus the form of the stress-energy tenor.

Next we use the Liouville Theorem to obtain the equations of hydrodynamics introducing a tensorial function of the momenta, that is related to the quantities conserved by the collisional operator and are called collisional invariants into the integral form of the theorem. 
Letting the distribution function decay for large momenta we obtain the transfer equation. 
For a simple gas this can be reduced to already familiar equations where divergence of the rest mass vector $J$ and stress energy tensor is zero.
