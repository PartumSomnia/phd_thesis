%% ============================
%%
%% Appendix A
%%
%% ============================

\chapter{Derivation angular momentum and angular momentum flux}
\label{app:ang}
%% \externaldocument{intro}
%% --------------------------------------

Consider and ideal fluid with stress-energy tensor:
\begin{equation}
T_{\mu\nu} = \rho h u_{\mu} u_{\nu} + Pg_{\mu\nu}
\end{equation}
where $\rho$ is the baryon rest-mass density, $h=1+\epsilon + P/\rho$ is the specific enthalpy, $\epsilon$ is the specific internal energy $u^{\mu}$ is the fluid 4-velocity, $P$ is the pressure. 

In the absence of external forces, such as neutrino absorption or vescous heating, the relativistic hydrodynamics equations read

\begin{equation}
\label{eq:1}
\nabla_{\nu}T^{\mu\nu} = 0
\end{equation}

The closure of this equation is achieved through EOS and rest-mass conservation equation. 

Apply $3+1$ decomposition of the equation \ref{eq:1}, taking:

\begin{itemize}
    \item cylindrical coordinates with $t$ $\omega$ $\phi$ coordinates
    \item time stationarity with $t^{\mu} = \partial_t ^{\mu}$ Killing vector
    \item axisymmetry with $\phi^{\mu} = \partial_{\phi} ^{\mu}$ killing vector
\end{itemize}

this implies that metric $g_{\mu\nu}$ does not depend on $t$ or $\phi$. 

Killing vectors can be written as 

\begin{eqnarray}
\nabla_{\mu}\phi_{\nu} + \nabla_{\nu}\phi_{\mu} = 0\\
\nabla_{\mu}t_{\nu} + \nabla_{\nu}t_{\mu} = 0\\
\end{eqnarray}

Under these assumptions, there is a conserved quantity $T_{\mu\nu}n^{\mu}\phi^{\nu}$, (Noether Theorem). 

The conserved quantity associated with $\phi$ symmetry is the \textit{angular momentum}. 

It is possible to show that

\begin{equation}
J = \int j dV = - \int T_{\mu\nu}n^{\mu}\phi^{\nu} dV = -\int \sqrt{\gamma}T_{\mu\nu}n^{\mu}\phi^{\nu} d^3 x
\end{equation}

is conserved. 

This is a special case of Noether's theorem. The proof will also show what \textit{flux of angular momentum} is. \\
First, consider

\begin{equation}
\nabla_{\nu} (T^{\mu\nu}\phi_{\mu}) = \nabla_{\nu}T^{\mu\nu}\phi_{\mu} + T^{\mu\nu}\nabla_{\nu}\phi_{\mu} = 0
\end{equation}

The first term is equal to 0 because of the equation of motion, while the second is zero, because of the killing equation in $3+1$ form:

\begin{equation}
\frac{1}{\sqrt{-g}}\partial_t(T^{\mu 0}\phi_{\mu}\sqrt{-g}) + \frac{1}{\sqrt{-g}}\partial_{i}(T^{\mu i}\phi_{\mu}\sqrt{-g}) = 0
\end{equation}

where $\sqrt{-g}$ is the colume form of the metric. 

Using the $3+1$ identities:

\begin{equation}
\sqrt{-g} = \sqrt{\gamma}\alpha \hspace{5mm} n_{\mu} = -\alpha dt
\end{equation}

where $\alpha$ is the lapse function and $\gamma$ is the determinant of the spatial metric. Using these relations we obtain:

\begin{equation}
\partial_t(T^{\mu\nu}\phi_{\nu}n_{\nu}\sqrt{\gamma}) - \partial_i(\alpha T^{i \nu}\phi_{\nu}\sqrt{\gamma}) = 0
\end{equation}

where the first term is $-j\sqrt{\gamma}$ and the second is the 
\textit{angular momentum flux}. 

Integrating these equations over a given volume $V$, we get 

\begin{equation}
\frac{d}{dt} \int_V T^{\mu\nu}\phi_{\mu}n_{\nu}\sqrt{\gamma} = \int_{\partial V}\alpha T^{i \nu}\phi_{\nu} \hat{r}_i\sqrt{\gamma}
\end{equation}

where $\hat{r}_i$ is the outgoing norm to $V$. 

This equation shows that if $T^{\mu\nu}$ is compactly supported in $V$, the $J$ is conserved. Otherwise $J$ changes according to angular momentum flux. 

\paragraph{Note on the fluid velocity} 

First we introduce the \textit{Lorentz factor} of the fluid with respect to the Eulerian observer. Set $\tau$ to be a Eulerian observer's proper time, $\Vec{u}$ is the fluid $4-$velocity, $\Vec{U}=d\Vec{l}/d\tau$ is the relative velocity of the fluid with respect to the Eulerian observer, whose $4-$velocity is $\Vec{n}$. $\Vec{U}$ is tangent to $\Sigma_t$ hypersurface, and relates to $\Vec{u}$ as $\Vec{u} = W(\Vec{n} + \Vec{U})$.

As $d\tau = W d\tau_0$ and $d\tau_0 = d\tau \Vec{n} + d\Vec{l}$ the Lorentz factor can be expressed as $W = -\Vec{n}\cdot\Vec{u}$ 

Another type of the velocity is \textit{velocity relative to the Eulerian observer}, that is set as displacement $d\Vec{l}$ over proper time $d\tau$, (both relative to Eulerian observer) as $\Vec{U} = d\Vec{l} / d\tau$ and can be expressed as $\Vec{u} = W (\Vec{n} + \Vec{U})$ 

Overall, what is of essence here is that the fluid $4-$velocity van be expressed as 

\begin{equation}
u^{\mu} = (-\Vec{u}\Vec{n})(\Vec{n}+\Vec{v}) = W(\Vec{n} + \vec{v})
\end{equation}

where $\Vec{n}$ is the norm to $Sigma_t$ and $\Vec{v}$ is the $3-$velocity.

\paragraph{Expression for Angular Momentum} 

Consider angular momentum density as 

\begin{align}
\sqrt{\gamma}T_{\mu\nu}n^{\mu}\phi^{\nu} &= \sqrt{\gamma}[\rho h u_{\mu} u_{\nu} + Pg_{\mu\nu}]n^{\mu}\phi^{\nu} \\
&= \sqrt{\gamma}[-\rho h W^2 v_{\phi} + Pg_{\mu\nu}n^{\mu}\phi^{\nu}] \\ 
&= -\sqrt{\gamma}\rho h W^2 v_{\phi}
\end{align}

here we used that $T_{\mu\nu} = \rho h u_{\mu} u_{\nu} + Pg_{\mu\nu}$, and that $u_{\mu}u_{\nu}n^{\mu}\phi^{\nu} = W^2 v_{\phi}$ as the time killing vector leaves only zeroth component, namely $W^2$ and rotation killing vector leaves $\phi$ component of the velocity. 

Thus, the angular momentum density is 

\begin{equation}
j\sqrt{\gamma} = -\sqrt{\gamma} \rho h W^2 v_{\phi}
\end{equation}

\paragraph{Expression for Angular Momentum Flux}  

\begin{equation}
\alpha\sqrt{\gamma}T^i _{\nu}\phi^{\nu} = \alpha\sqrt{\gamma}[\rho h W^2 v^i v_{\phi} + P\delta^i _{\phi}]
\end{equation}

Here the $i$ can be $\omega$, $\phi$ and $z$ components. Thus, in general there exists $3$ components of the flux but only one is left nonzero in light of symmetries:

\begin{align}
\omega &: \alpha\sqrt{\gamma}T^{\omega}_{\nu}\phi^{\nu} = \alpha\sqrt{\gamma}\rho h W^2 v^{\omega}v_{\phi} \\
% \phi &: \alpha\sqrt{\gamma}T^{\phi}_{\nu}\phi^{\nu} = \alpha\sqrt{\gamma}\rho h W^2 v^{\phi}v_{\phi} + P \\
% z &: \alpha\sqrt{\gamma}T^{z}_{\nu}\phi^{\nu} = \alpha\sqrt{\gamma}\rho h W^2 v^{z}v_{\phi}
\end{align}

which summed together give:

% \begin{equation}
%     \alpha\sqrt{\gamma}T^i _{\nu}\phi^{\nu} = \alpha\sqrt{\gamma}\rho h W^2 (v^{\omega}v_{\phi} + v^{\phi}v_{\phi} + P + v^{z}v_{\phi})
% \end{equation}

\begin{equation}
\alpha\sqrt{\gamma}T^i _{\nu}\phi^{\nu} = \alpha\sqrt{\gamma}\rho h W^2 (v^{\omega}v_{\phi})
\end{equation}

\paragraph{Result}: In the integral form 

\begin{equation}
\frac{d}{dt}\int -\sqrt{\gamma} \rho h W^2 v_{\phi} dV = \int \alpha\sqrt{\gamma}\rho h W^2 v^{\omega}v_{\phi} dS
\end{equation}

where the integral on the left-hand-side is taking over the cylindrical volume, $\int \omega d\omega d\phi dz$ and the integral on the right-hand-side is taking over the cylindrical surface. 

\begin{align}
\frac{d}{dt}\int_{V} T^{\mu\nu}\phi_{\mu}n_{\nu}\sqrt{\gamma} &= \int_{\partial V} \alpha T^{i \nu} \phi_{\nu} \hat{r}_i\sqrt{\gamma} \\
&= \int\alpha T^{i \nu} \phi_{\nu}\sqrt{\gamma}\hat{r}_i dS \\
&= \int\Big(\nabla_i \alpha T^{i \nu} \phi_{\nu}\sqrt{\gamma} \Big) dV\\
&= \int\Big(\alpha\sqrt{\gamma}\nabla_i T^i _{\nu}\phi^{\nu}\Big) dV\\
&= \int\alpha\sqrt{\gamma}\Big(\nabla_i(\rho h v^i v_{\phi} + P\delta^i _{\phi})\Big) dV\\
&= \int\alpha\sqrt{\gamma}\Big(\nabla_{\omega}\rho h v^{\omega} v_{\phi} \Big) dV\\
&= \int\alpha\sqrt{\gamma}\frac{\partial}{\partial\omega}(\rho h v^{\omega}v_{\phi}) dV \\
&= \int\alpha\sqrt{\gamma}\rho h \frac{\partial(v^{\omega}v_{\phi})}{\partial\omega}dv \\
&= \int\alpha\sqrt{\gamma}\rho h\Big(v_{\phi}\frac{\partial v^{\omega}}{\partial\omega} + v^{\omega}\frac{\partial v_{\phi}}{\partial\omega}\Big)dV
\end{align}

However, implementing numerically the last equation means to compute $d/d\omega(v^{\omega}v_{\phi})$, which is a 3D array. Thus in the later section we utilise the surface integral over the outer cylindrical surface. 