\newpage 

%
%   This is German summary of the thesis -- translated to German 
%   first part of the conclusion. 
%   For convenience I provide English text (commented) and 
%   deepl translation for each paragraph separately. 
%   PLEASE correct me :) 
%

\textbf{\Large{Zusammenfassung} }
% \vspace*{1cm}2016–2018
% The scope of this thesis is to advance our understanding of \ac{BNS} mergers
% and fundamental physics by using state-of-the-art \ac{NR} simulations with
% advanced physics and \ac{EM} models in tandem with \mm{} observations of 
% \GW{}. 

Das Ziel dieser Arbeit ist es, unser Verständnis von Fusionen binärer Neutronensterne (BNS)
und ihrer fundamentalen Physik durch den Einsatz modernster Numerischer Relativit\"atssimulationen (NR) mit
fortgeschrittener Physik und elektromagentischen Modellen in Verbindung mit Multimessenger Beobachtungen von 
\GW{} voranzustreiben. 

% Considering the \pmerg{} evolution of \ac{BNS} merger remnants, we find an
% overall strong dependency on the system \mr{} and the \ac{EOS}, and on the 
% finite temperature effects in the latter. One of the key affected 
% parameters is the remnant lifetime. We find that models with 
% soft \acp{EOS} or/and large \mr{}s produce short-lived \ac{NS} remnants that 
% collapse within a few ${\sim}10\,$ms after merger. 
% More symmetric models with stiffer \acp{EOS} produce long-lived, possibly 
% stable remnants. The lifetime of the \ac{NS} remnant appears to be correlated with the 
% disk mass for the $q\sim 1$ models, in agreement with previous findings 
% \citep{Radice:2017lry,Radice:2018pdn}.
% Binaries with larger \mr{}s tend to have more massive disks and more massive tidal 
% components of \ac{DE}.

Betrachtet man die Entwicklung des Postfusionsregimes eines BNS-Fusionsüberrestes, so stellt man eine insgesamt starke Abhängigkeit vom Massenverhältnis des Systems und der Neutronenster-Zustandsgleichung sowie von den 
endlichen Temperatureffekten in letzterer fest. Einer der wichtigsten betroffenen 
Parameter ist die Lebensdauer des Überrests. Wir stellen fest, dass Modelle mit 
weicher Zustandsgleichung und/oder großem Massenverhältnis kurzlebige Neutronensternüberreste erzeugen, die 
innerhalb von wenigen ${\sim}10\,$ms nach der Verschmelzung kollabieren. 
Symmetrischere Modelle mit steiferer Zustandsgleichung erzeugen langlebige, möglicherweise 
stabile Überreste. Die Lebensdauer des Fusionsüberrestes scheint mit der 
Akkretionsscheibenmasse für die $q\sim 1$ Modelle zu korrelieren, was mit früheren Erkenntnissen übereinstimmt 
\citep{Radice:2017lry,Radice:2018pdn}.
Binärsysteme mit größeren Massenverhältnisen haben tendenziell massereichere Scheiben und massivere Gezeitenkomponenten von dynamischen Auswürfen.

% The long-term evolution of \pmerg{} \ac{NS} remnants is governed by the accretion, 
% induced by neutrino cooling and viscous stresses, and mass shedding that originates 
% in gravitational and hydrodynamical torques and neutrino reabsorption (heating). 
% Notably, a newly formed \ac{NS} remnant with mass exceeding the 
% maximum of the uniformly rotating configuration, \ac{HMNS}, does not necessarily collapse 
% to a \ac{BH}. Instead, massive winds, such as \ac{SWW}, can efficiently remove the 
% excess in mass (alongside the angular momentum), bringing the \ac{NS} remnant 
% to a rigidly rotating configuration.

Die langfristige Entwicklung des Postfusionsregimes von Neutronensternüberresten wird durch die Akkretion bestimmt, 
die durch Neutrinokühlung und viskose Spannungen hervorgerufen wird, sowie durch Massenauswurf, der 
durch Gravitations- und hydrodynamische Drehmomente und Neutrino-Resorption (Erwärmung) angetrieben ist. 
Bemerkenswerterweise, kollabiert ein neu gebildeter Neutronensternüberrest, der die maximale Masse eines Hypermassiven Neutronensternes in gleichmäßig rotierender Konfiguration übersteigt, nicht unbedingt 
zu einem Schwarzen Loch. Stattdessen können massive Winde, wie Spiralwellenwinde, den 
Überschuss an Masse (zusammen mit dem Drehimpuls) effizient beseitigen und den Neutronensternüberrest 
in eine starr rotierende Konfiguration bringen.

% Considering the matter ejected during and after mergers, we find 
% two distinct types: \ac{DE} and \pmerg{} \ac{SWW}. 
% With respect to the former 
% we augment the analysis of our own models by considering all 
% available \ac{BNS} merger models in the literature with various physics inputs. 
% The statistical analysis of \ac{DE} properties highlights the strong 
% dependency of these properties on neutrino reabsorption. 
% Its inclusion raises the ejecta mass and velocity. 
% Meanwhile, the composition of \ac{DE}  
% from our models, computed with an approximated M0 neutrino scheme, 
% is similar to that found in simulations with more sophisticated neutrino treatment 
% methods \citep{Sekiguchi:2016bjd,Vincent:2019kor}. 
% Taking the largest-to-date set of \ac{BNS} simulations, we link the \ac{DE} properties 
% back to the binary parameters considering a variety of fitting formulae.  
% We update these relations that are very important for \ac{MM} astronomy. 
% We also find that a simple two parameter polynomial, \polql{}, shows a comparable or 
% better statistical performance than other fitting formulae.

Betrachtet man die Materie, die während und nach der Verschmelzung ausgestoßen wird, so findet man 
zwei verschiedene Typen: dynamnischen Auswurf und Spiralwellenwinde im Postfusionsregime. 
In Bezug auf die erste Art 
erweitern wir die Analyse unserer eigenen Modelle unter Berücksichtigung aller 
verfügbaren BNS-Fusionsmodelle in der Literatur mit verschiedenen physikalischen Inputs. 
Die statistische Analyse der Eigenschaften des dynamnischen Auswurfs zeigt eine starke 
Abhängigkeit von der Neutrino-Resorption. 
Ihre Einbeziehung erhöht die Masse und die Geschwindigkeit des Auswurfs. 
Des Weiteren ähnelt die Zusammensetzung des dynamnischen Auswurfs  
aus unseren Modellen, die mit einem angenäherten M0-Neutrino-Schema berechnet wurde, derjenigen, die in Simulationen mit anspruchsvolleren Neutrino-Behandlungsmethoden gefunden wurde \citep{Sekiguchi:2016bjd,Vincent:2019kor}. 
Mithilfe des bisher größten Satzes von BNS-Simulationen verknüpfen wir die Eigenschaften des dynamnischen Auswurfes
mit den binären Systemparametern unter Berücksichtigung einer Vielzahl von Anpassungsformeln.  
Wir aktualisieren diese Beziehungen, die für die Multimessenger-Astronomie sehr wichtig sind. 
Wir finden auch, dass ein einfaches Zwei-Parameter-Polynom, \polql{}, eine vergleichbare oder 
bessere statistische Leistung zeigt als andere Anpassungsformeln.

% In cases where the \pmerg{} remnant is long-lived, we identify a new ejecta component, 
% \ac{SWW}. These winds are driven by energy and angular momentum injected into the 
% disk by a remnant which is subjected to bar-mode and one-armed dynamical instabilities.
% We find that within the simulation time, up to ${\sim}100$~ms, \ac{SWW} do not 
% saturate unless the \ac{NS} remnant collapses to a \ac{BH}.
% \ac{SWW} have a broad distribution in electron fraction that is on average higher
% than that of \ac{DE}. 
% \ac{SWW} have narrow distribution in velocity and can unbind
% ${\sim}0.1{-}0.5\, \Msun$ within a second.

In Fällen, in denen der Fusionsüberrest langlebig ist, identifizieren wir eine neue Auswurfskomponente, die 
Spiralwellenwinde. Diese Winde werden durch Energie und Drehimpuls angetrieben, die vom Überrest, der der Bar-Mode und einarmigen dynamischen Instabilitäten unterworfen ist, in die Scheibe injiziert werden.
Wir stellen fest, dass  die Spiralwellenwinde innerhalb der Simulationszeit, bis zu ${\sim}100$~ms, nicht 
sättigen, es sei denn, der Neutronensternüberrest kollabiert zu einem Schwarzen Loch.
Spiralwellenwinde haben eine breite Verteilung des Elektronenanteils, der im Durchschnitt höher ist
als in dynamnischen Auswürfen. 
Spiralwellenwinde haben eine enge Geschwindigkeitsverteilung und können innerhalb einer Sekunde
${\sim}0.1\textsc{-}0.5\, \Msun$ aus dem System extrahieren.

% A part of \ac{SWW}, channeled along the polar axis and exhibiting the highest
% electron fraction, we identify as \nwind{}. Contrary to other studies of 
% neutrino-driven outflows \citep[\eg][]{Dessart:2008zd,Perego:2014fma,Fujibayashi:2020dvr},
% \nwind{} in our simulations saturate shortly after merger.
% Notably, steady state \nwind{} are generally referred to outflows 
% that emerge on a timescale,  hundreds of milliseconds longer than ours. 
% Additionally, it is plausible that the approximated neutrino reabsorption 
% scheme used in our simulations is insufficient in this case. 
% Long-term simulations employing more advanced 
% neutrino transport schemes are required to assess the properties of \nwind.
% Additionally, the effects of magnetization are important for polar outflows 
% \citep{Siegel:2017nub,Metzger:2018uni,Fernandez:2018kax,Miller:2019dpt,Mosta:2020hlh}.
% Our simulations, however, do not include magnetic fields. 

Ein Teil von Spiralwellenwinden, der entlang der polaren Achse kanalisiert ist und den höchsten
Elektronenanteil aufweist, identifizieren wir als Neutrino-getriebenen Wind. Im Gegensatz zu anderen Studien über 
Neutrino-getriebene Ausströmungen \citep[\eg][]{Dessart:2008zd,Perego:2014fma,Fujibayashi:2020dvr}, sättigen Neutrino-getriebene Winde in unseren Simulationen kurz nach der Fusion.
Insbesondere beziehen sich stationäre, Neutrino-getriebene Winde im Allgemeinen auf Abflüsse,
die auf einer Zeitskala entstehen, die Hunderte von Millisekunden länger als unsere ist. 
Darüber hinaus ist es plausibel, dass das in unseren Simulationen verwendete approximierte Neutrino-Reabsorptions 
Schema, in diesem Fall unzureichend ist. 
Langfristige Simulationen mit fortschrittlicheren 
Neutrinotransportschemata sind erforderlich, um die Eigenschaften von Neutrino-getriebenen Winden zu beurteilen.
Außerdem sind die Auswirkungen der Magnetisierung für polare Ausströmungen wichtig 
\citep{Siegel:2017nub,Metzger:2018uni,Fernandez:2018kax,Miller:2019dpt,Mosta:2020hlh}.
Unsere Simulationen beinhalten jedoch keine Magnetfelder. 

% We assess the outcome of \rproc{} \nuc{} in the ejected matter via the precomputed 
% parameterized model, based on the \ac{NRN} \texttt{SkyNet} \citep{Lippuner:2015gwa}.
% The \rproc{} yields in \ac{DE} depend strongly on the binary \mr{}, 
% with large amounts of lanthanides and actinides produced in high-$q$ cases.
% Models with the highest \mr{}s, that undergo \ac{PC}, show 
% actinides abundances in their \ac{DE} similar to solar.
% Binaries with $q \sim 1$ produce less neutron-rich \ac{DE} and the 
% final abundances show a significant fraction of lighter elements. 

Wir bewerten das Ergebnis von der r-Prozess Nukleosynthese in der ausgeworfenen Materie anhand eines vorberechneten, 
parametrisierten Modells, das auf dem Nuclear Reaction Network \texttt{SkyNet} \citep{Lippuner:2015gwa} basiert.
Die r-Prozess-Ergebnisse in den dynamnischen Auswürfen hängen stark vom binären Massenverhältnis ab, 
wobei in Fällen mit hohem $q$ große Mengen an Lanthaniden und Aktiniden produziert werden.
Modelle mit den höchsten Massenverhältnissen, die einen unverzüglichen Kollaps durchlaufen, zeigen 
Aktinidenhäufigkeiten in ihren dynamnischen Auswürfen, die denen des Sonnensystems ähneln.
Binärsysteme mit $q \sim 1$ erzeugen weniger neutronenreiche dynamnische Auswürfe und die 
Endhäufigkeiten weisen einen signifikanten Anteil leichterer Elemente auf.

% If the \pmerg{} remnant is long-lived, the final \rproc{} abundances in 
% total ejecta (that include \ac{DE} and \ac{SWW}) show large amounts 
% of both heavy and light elements. 
% The abundance pattern in these ejecta is similar to solar, down to $A\simeq 100$.
% This result further emphasizes   the importance of \ac{BNS} mergers in cosmic 
% chemical evolution.

Wenn der Fusionsüberrest langlebig ist, zeigen die endgültigen r-Prozess-Häufigkeiten im 
Gesamtauswurf (dynamische Auswürfe und Spiralwellenwinde eingeschlossen) große Mengen 
von schweren und leichten Elementen. 
Das Häufigkeitsmuster in diesen Auswürfen ist ähnlich wie das im Sonnensystem, bis hinunter zu $A\simeq 100$.
Dieses Ergebnis unterstreicht die Bedeutung von BNS-Fusionen für die kosmische 
chemischen Entwicklung.

% Considering the thermal emission from the decay of \rproc{} elements in ejecta, \ac{kN}, 
% from our models, we find that, when spherically symmetric \ac{kN} models are 
% considered \citep{Villar:2017wcc}, none of our models can explain 
% the \AT{} bolometric light curves.
% However, when anisotranisotropic multi-components \ac{kN} models are considered, 
% that take into account properties and geometry of ejecta, 
% certain key features of \AT{} are recovered.
% Specifically, we find that the early blue emission can be explained 
% when both \ac{DE} and \ac{SWW} are considered, and when the \pmerg{} remnant is long-lived.

Betrachtet man die thermische Emission aus dem Zerfall von r-Prozess-Elementen im Auswurf, eine so genannte Kilonova (kN), 
aus unseren Modellen, stellen wir fest, dass, wenn sphärisch symmetrische kN-Modelle betrachtet werden 
\citep{Villar:2017wcc}, keines unserer Modelle 
die bolometrischen Lichtkurven von \AT{} erklären kann.
Wenn jedoch anisotrope Multikomponenten kN-Modelle berücksichtigt werden, 
die die Eigenschaften und die Geometrie der Auswürfe berücksichtigen, 
werden bestimmte Schlüsselmerkmale von \AT{} wiederhergestellt.
Wir stellen insbesondere fest, dass die frühe blaue Emission erklärt werden kann, wenn sowohl dynamische Auswürfe als auch Spiralwellenwinde berücksichtigt werden und wenn der Fusionsüberrest langlebig ist.

% High electron fraction material was also shown to be present in outflows 
% from \ac{BH}-torus systems and thus does not necessarily require a long-lived remnant 
% \citep{Fujibayashi:2020qda}.
% The late time red kilonova component requires massive, ${\sim}20\%$ of the disk mass, 
% low-$Y_e$ outflows. Such outflows can be driven by viscous processes and nuclear recombination 
% on a timescale of seconds \citep[\eg][]{Metzger:2008av}.

Es wurde auch gezeigt, dass Material mit hohem Elektronenanteil in Ausströmungen 
von Schwarzes Loch-Torus-Systemen vorhanden ist und daher nicht unbedingt einen langlebigen Überrest erfordert 
\citep{Fujibayashi:2020qda}.
Die spätzeitliche rote Kilonova-Komponente erfordert massive, ${\sim}20\%$ der Scheibenmasse, 
Ausströmungen mit niedrigem $Y_e$. Solche Ausströmungen können durch viskose Prozesse und nukleare Rekombination 
auf einer Zeitskala von Sekunden \citep[\eg][]{Metzger:2008av} angetrieben werden.

% Considering the synchrotron afterglow from the interaction between ejecta and the 
% \ac{ISM}, we find that the recently observed change in the afterglow of \GRB{} 
% $10^3\,$days after merger can be explained by the \ac{kN} afterglow produced by ejecta in 
% ab-initio \ac{NR} \ac{BNS} simulations targeted to \GW{}. 
% Specifically, models with moderately stiff \acp{EOS} and moderately large \mr{}s, 
% that produce a mild amount of fast ejecta, are favored.
% This provides a new avenue to constrain properties of \GW{}. 

Unter Berücksichtigung des Synchrotron-Nachleuchtens durch die Wechselwirkung des Auswurfs mit dem 
Interstellaren Medium stellen wir fest, dass die kürzlich beobachtete Veränderung des Nachleuchtens von \GRB{} 
$1000\,$Tage nach der Verschmelzung durch das kN-Nachleuchten erklärt werden kann, das durch 
ab-initio NR BNS-Simulationen berechnet wurde, die auf \GW{} ausgerichtet sind. 
Insbesondere Modelle mit mäßig steifer Zustandsgleichung und mäßig großem Massenvehältnis,
die eine geringe Menge an schnellem Auswurf produzieren, werden bevorzugt.
Dies bietet eine neue Möglichkeit, die Eigenschaften von \GW{} einzuschränken. 
