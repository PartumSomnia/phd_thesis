\documentclass[11pt,a4paper,headinclude=true,DIV=14,BCOR=8mm,chapterprefix,listof=totoc,twoside,openright,abstracton]{scrbook}

\usepackage[headsepline]{scrpage2}
\usepackage[utf8]{inputenc}
\usepackage{geometry}
\usepackage{amssymb}
\usepackage{amsthm}
\usepackage{enumerate}
\usepackage{graphicx}
\usepackage{float}
\usepackage[intlimits]{amsmath}
% \usepackage{siunitx}
% \usepackage{color}
\usepackage{xcolor}
\usepackage{verbatim}
\usepackage{appendix}
\usepackage{hyperref}
\usepackage{hyperref}
\usepackage{mathtools}
% \usepackage[style=authoryear]{biblatex}
\usepackage{natbib}
% \usepackage{newtxtext}
% \usepackage{newtxmath}
% \usepackage{harvard}
\setcitestyle{aysep={}} 
\bibliographystyle{apalike}
\usepackage{xr}
\usepackage{wrapfig}
% \bibliographystyle{agsm}
%\usepackage{feynmf}
%\usepackage{tensor}
\usepackage[framemethod=tikz]{mdframed} % for a block of text

\setlength{\parindent}{0pt}
\geometry{a4paper, tmargin=3cm, bmargin=3cm, lmargin=3cm, rmargin=3cm, headheight=3em, headsep=2em, footskip=1cm}

\setcitestyle{citesep={,}}

\newcommand{\todo}[1]{\textcolor{red}{$\blacksquare$ TODO: #1}} 

\newmdenv[linecolor=cyan,backgroundcolor=cyan!20]{sidenote}


\geometry{a4paper, tmargin=2cm, bmargin=2cm, lmargin=1cm, rmargin=1cm, headheight=2em, headsep=2em, footskip=1cm}

\title{PhD thesis}
\author{Vsevolod Nedora}
\date{today}

\begin{document}
    
    \maketitle

%% --------------- 
%%
%% Theory
%%
%% ---------------

\chapter{General-Relativistic Hydrodynamics}

This chapter is meant to sketch several important parts of the mathematical background. We focus on the aspects relevant for the tools and methods employed in out discussion. We do not aim to provide a comprehensive overview. 
The chapter is divided into \todo{list the parts and their content}

\begin{sidenote}
    \textbf{Note on the exterior algebra} \\
    \textit{Understanding the exterior product, Wedge product} \\
    If $\phi$ and $\psi$ are the 2-forms given for example as 
    \begin{equation}
        \phi = x dx - y dy \hspace{5mm} \text{and} \hspace{5mm}\psi = z dx + x dz
    \end{equation}
    Then the exterior produce is given by 
    \begin{align}
        \phi\wedge\psi &= (x dx - y dy)\wedge(zdx + xdz) = \\
        &=xzdxdx+x^2dxdz-yzdydx-yxdydz= \\
        &=yzdxdy + x^2 dx dz - xydydz
    \end{align}
    as $dxdx=0$ and $dydx=-dxdy$. The product of two 1-forms is a 2-form.
    In general, the wedge product of a$p$-form and $q$-form is a $(p+q)$-form. \\
    
    Next, consider a surface $\mathcal{M}$ and two 1-forms on it $\phi$ and $\psi$ Then the wedge product is 
    \begin{equation}
        (\phi\wedge\psi)(v,w)=\phi(v)\psi(w) - \phi(w)\psi(v)
    \end{equation}
    for any $v$ and $w$ tangent vectors to $\mathcal{M}$. \\
    
    The central idea in exterior algebra is that the operations are designed to create the permutational antisymmetry. Let the $dx_i$ be the basis 1-from $\omega_j$ are the orbitrary $p$-form (of order $p_j$), $a$ and $b$ are arbitrary scalars. Then the wedge product is defined to have properties:
    
    \begin{align}
        (a\omega_1+b\omega_2)\wedge\omega_3 &= a\omega_1\wedge\omega_3+b\omega_2\wedge\omega_3 \hspace{5mm} (p_1 = p_2), \\
        (\omega_1\wedge\omega_2)\wedge\omega_3 &= \omega_1\wedge(\omega_2\wedge\omega_3), \hspace{5mm} a(\omega_1\wedge\omega_2) =  (a\omega_1)\wedge\omega_2\\
        dx_i\wedge dx_j &= -dx_j\wedge dx_i
    \end{align}
    Thus, any arbitrary differential form can be reduced to a coefficient multiplying $dx_i$ or a wedge produce of the generic form 
    \begin{equation}
        dx_i\wedge dx_j \wedge...\wedge dx_p
    \end{equation}
    with the properties allowing to put all coefficients together as 
    \begin{equation}
        a dx_1 \wedge b dx_2 = - a(b dx_2 \wedge dx_1) = -ab(dx_2 \wedge dx_1) = ab(dx_1 \wedge dx_2)
    \end{equation}
    
    The exterior or wedge product acts on tangent vectors. The $\wedge$ or two tangent vectors $\boldsymbol{u}\wedge\boldsymbol{v}$, ($\boldsymbol{u}, \boldsymbol{v}\in T_p(\mathcal{M})$) is an antisymmetric tensor product that in addition to bilinarity requires antisymmetry. 
    \begin{align}
        \boldsymbol{v} =& v^1e_1 + v^2 e_2 + v^3 e_3 \\
        \boldsymbol{u} =& u^1e_1 + u^2 e_2 + u^3 e_3 \\
        \boldsymbol{v}\wedge\boldsymbol{u} =& (v^1u^1 - v^2u^1)(e_1\wedge e_2) + \\
        & + (v^1u^3 - v^3u^1)(e_1\wedge e_1) + \\
        & + (v^2u^3 - v^3u^2)(e_2\wedge e_1)
    \end{align}
    mimicing the behaviour of the cross product. However, this can easly be extended to higher dimensions. \\
    Important, that the resulting object of $\boldsymbol{v}\wedge\boldsymbol{u}$ does not belong to $T_p M$. It is called and alternating bivector and is an element of the vector space $\Lambda^2 T_p (\mathcal{M})$ ,that is called -- second exterior power of $T_p \mathcal{M}$. \\
    Generally one obtains $\Lambda^k T_p (\mathcal{M})$ that is a linear subspace of $T_p ^k (\mathcal{M})$\\
    Consider a cotangent space $T_p ^* \mathcal{M}$. The exterior product on this space is copativle with wedge product on $T_p\mathcal{M}$ and is usually denoted with the same symbol and yeilds: $(\boldsymbol{\alpha}\wedge\boldsymbol{\beta})\in\Lambda^2 T_p ^* \mathcal{M}$.
\end{sidenote}

\begin{sidenote}
    \textbf{Differentia form} \\
    Used for multivariable calculus independent of coordinates. Used for integrands over curves, manifolds. For example, differential form can be used to define a volume element as $f(x,y,z)dx \wedge dy \wedge dz$.\\
    Albegra of diff.forms is organized to reflect the orientation of the domain of integration. For instance: the exterior product $d$ that converts $k$-from into $k+1$-form. This operation is similar to the divergence and the curl of a vector field. \\
    Differential 1-forms are naturally dual to vector field on a manifold. Pairing is done via inner product. \\
    If there are two manifolds, then the albegra of diff.forms and their exterior derivatives is preserved by the \textit{pullblack} under the smooth function. This allows geometrically invariant information to be moved from one space to another via the pullback. \\
    Let $\mathcal{M}$ be an orientated $m$-dimentional manifold and $\mathcal{M}'$ is the same manifold with the opposite orientation and $\omega$ is an $m$-form, then 
    \begin{equation}
        \int_{\mathcal{M}}\omega = -\int_{\mathcal{M}'}\omega.
    \end{equation}
    
    The \textit{exterior algebra} is used to make the notion of an oriented density precise.
    The basic $1$-forms are \textit{differentials} ofthe coordiantes $dx^1,...,dx^n$. Each of them is a \textit{covector} that measures a small displacement in the corresponding coordinate direction. A general $1$-form thus is the combination  of these differentials 
    \begin{equation}
        f_1dx^1\cdot\cdot\cdot f_ndx^n
    \end{equation}
    where $f_k=f_k(x^1,...,x^n)$ are functions of all the coordiantes. \\
    Wedge product is similar to cross product, and is used to be higher diff. forms out of lower ones, as the cross product in vector calculus. \\
    
    The \textit{Exterior derivative}, operator $d$, which is a generalisation of a differential of a function. Let $\omega=fdx^I$ being a simple $k$-form. Then its exterior derivative $d\omega$ is a $(k+1)$-form set by taking differential of the doefficient functions
    \begin{equation}
        d\omega = \sum_{i=1}^n \frac{\partial f}{\partial x^i}dx^i \wedge dx^I
    \end{equation}
    Thus a deferential form, lets say, differential 2-form is called an exterior derivative $da$ of $a=\sum_{j=1}^{n}f_j dx^j$. It is given by
    \begin{equation}
        da = \sum_{j=1}^n df_j \wedge dx^j = \sum_{i,j=1}^n \frac{\partial f_j}{\partial x^i}dx^i\wedge dx^j.
    \end{equation}
    Overall, the $da=0$ is required for a function $f$ such that $a=df$.
    
    On as mooth manifold $\mathcal{M}$ the differential from of degree $k$ is a smooth section of the $k$th exterior power of the cotangent bundle of $\mathcal{M}$. Then, the set of all the $k-$forms on $\mathcal{M}$ is a \textit{vector space} $\Omega^k(\mathcal{M})$. The formal definition then stands. At any point $p\in \mathcal{M}$ a $k-$form $\beta$ defines an element 
    \begin{equation}
        \beta_p\in\Lambda^kT^* _p \mathcal{M}
    \end{equation}
    where $T_p\mathcal{M}$is the tangent space tp $\mathcal{M}$ at $p$. The $T^* _p \mathcal{M}$ is its dual space. Thus, $\beta$ is also a linear functional such that $\beta_p:\Lambda^k T_p \mathcal{M}\rightarrow I\!R$
\end{sidenote}

\begin{sidenote}
    \textbf{Differential forms on a Reimannian maniforld} \\
    There metric defines a fiber-wise isomorphism of the tangent and cotangent spaces. This allows to convert vector fields to covector field and vice versa. It also allows the definition of the \textit{Hodge star operator}.
    
    Hodge star operator $\star$ is a linear map, defined on the exterior algebra of a finite-dimensional oriented vector space endowed with a nongegenerate symmetric bilinear form. Applying the operator to the element of the algebra produces the \textit{Hodge dual} of the element. \\
    
    Example. Consider a 3D Euclidean space. Let there be an orientated plane, that is presented by the exteriour product $\wedge$, of two basis vectors. Then its Hodge dual is the normal vector biven by the cross product. \\
    The Hodge operator $\star$ is a one-to-one mapping of $k-$ to $(n-k)$-vectors.\\
    
    The $\star$ can be applied to the cotangent bundle of a pseudo-reimanian manifold -- to differential $k$-forms. This allows the definition of a differential as a Hodge adjoint of the exteior derivative. 
    
    Formal definition. Let $V$ be a $n$-dimensional vector space with nondegenerate symmetric bilinear form $\langle\cdot,\cdot\rangle$ -- the inner product. This induces an inner product on $k-$vectors $\alpha,\beta\in\Lambda^k V$ for $0\leq k \leq n$ by defining it on decomposable $k$-vectors $\alpha = \alpha_1\wedge\cdots\wedge\alpha_k$ and $\beta=\beta_1\wedge\cdots\wedge\beta_k$
    ...
    \\
    The Hodge star operator is a linear operator on the exterior algebra of $V$, mapping $k$-vectors to $(n-k)$-vectors for $0\leq k \leq n$. It has following property that defines it completely
    \begin{equation}
        \alpha\wedge(\star\beta) = \langle\alpha,\beta\rangle\omega 
    \end{equation}
    for every pair of $k-$vectors $\alpha\beta\in\Lambda^kV$ Here the $\omega\in\Lambda^nV$ is the unit $n-$vector defined in terms of an oriented orthonormal basis $\{e_1,...,e_n\}$ of $V$ as
    \begin{equation}
        \omega := e_1 \wedge \cdots \wedge e_n.
    \end{equation}
    
    Dually in the space $\Lambda^n V^*$ of $n-forms$, the dual $\omega$ is the colume form $\textbf{det}$, the function whose value on $v_1\wedge\cdots\wedge v_n$ is the determinant of the $n\times n$ matric assembled from the column vectors of $v_i$ in $e_i$ coordinates. Thus the dual difinition is 
    \begin{equation}
       \text{det}(\alpha\wedge\star\beta) = \langle\alpha,\beta\rangle.
    \end{equation}
    or ecvilalently 
    \begin{align}
        \alpha =& \alpha_1\wedge\cdots\wedge\alpha_k \\
        \beta =& \beta_1\wedge\cdots\wedge\beta_k \\
        \star\beta =& \beta_1 ^{\star} \wedge\cdots\wedge \beta_{n-k} ^ {\star} \\
        \text{det}(\alpha_1\wedge\cdots\wedge\alpha_k\wedge\beta_1 ^{\star}\wedge\cdots\wedge\beta_{n-k}^{\star}) =& \text{det}(\langle\alpha_i,\beta_j\rangle)
    \end{align}
    
    \textit{Geometrical interpetation.}  -- cannot understand
    \textit{Examples}
    Consider 2D with normalized Euclidian metric and orientation given by ordering $(x,y)$. The Hodge star on $k-$forms is given by 
    
    \begin{align}
        \star 1 &= dx \wedge dy \\
        \star dx &= dy \\
        \star dy &= -dx \\
        \star(dx \wedge dy) &= 1.
    \end{align}
    
    Consider a more complex example. A plane that can be regarded as a vector space with a standard sesquilinear form as the metric. There the Hodge operator has a properiy that it is invariant under the holomorphic cahges of cooridantes. Consider $z = x + iy$ holomorphic function of $w=u + iv$. Then in the new coordiantes 
    
    \begin{align}
        \alpha &= pdx +qdy \\
        \star \alpha &= -q dx + p dy
    \end{align}
    
    \textit{3D} \\
    Here the $\star$ can be regarded as a correspondence between vectors and bivectors. Thus in Eucledian $\boldsymbol{R}^3$ with basis $dx,dy,dz$, basis of oneforms one finds
    \begin{align}
        \star dx =& dy\wedge dz \\
        \star dy =& dz\wedge dx \\
        \star dz =& dx \wedge dy \\
    \end{align}
    The relation to the exteriour and cross producs are:
    \begin{equation}
        \star(\boldsymbol{u}\wedge\boldsymbol{v})=\boldsymbol{u}\times\boldsymbol{v}, \hspace{5mm}\star(\boldsymbol{u}\times\boldsymbol{v}) = \boldsymbol{u}\wedge\boldsymbol{v}
    \end{equation}
    Thus in 3D the $\star$ provides and isomorphism between vectors and bivectors, so each axial vector $\boldsymbol{a}$ is associated with the bivector $\boldsymbol{A}$ as $\boldsymbol{A} = \star\boldsymbol{a}$ and $\boldsymbol{a} = \star\boldsymbol{A}$. It can also mean a correspondance betweeen the axis and ifenitesicmal rotation around the axis with the speed equal to the length of the axis vector.
    %% One can see that mapping $\star:V\rightarrow\Lambda^2V\subset V\otimes V$
    Consider a tensor $dx \otimes dy$that correesponds to the matrix with one $dx$ row and $dy$ column. The wedge $dx\wedge dy = dx\otimes dy - dy\otimes dx$ is a 3 by 3 \textit{skew-symmetric matrix} with all 0 excet $01$ and $10$ components that are 1. So the $\wedge$ operator turns $\boldsymbol{v} = adx + bdy + cdz$ into $\star\boldsymbol{v}\approx$ 3x3 matrix with 0 on diaoganals. \\
    
    \textit{4D}\\
    Here $\star$ acts as an endomorphism of the second exteriour power, mapping 2-forms into 2-forms.Consider Minkowski space time with signature $(+---)$ and coordinates $(t,x,y,z)$
    
    \begin{align}
        \star dt &= dx \wedge dy \wedge dz \\
        \star dx &= dt \wedge dy \wedge dz \\
        \star dy &= -dt \wedge dx \wedge dz \\ 
        \star dz &= dt \wedge dx \wedge dy 
    \end{align}
    
    \textit{Wedge on manifold}
    For an $n-$dimensional oriented pseudo-Reimannian manifold $\mathcal{M}$ we apply the construction such that to each cotangent vector space $T^* _p \mathcal{M}$ and its exterior powers $\Lambda^k T_p ^* \mathcal{M}$and hence to all differential $k-$forms $\xi\in\Omega^k(\mathcal{M})=\Gamma(\Lambda^k T^* \mathcal{M})$, the global sections of the bundle $\Lambda^k T^*\mathcal{M}\rightarrow \mathcal{M}$. The Reimannian metric induces inner product on $\Lambda^k T_p ^* \mathcal{M}$ at each point $p\in\mathcal{M}$. We define the Hodge dual of a $k-$form $\xi$ defining $\star\xi$ as a unique $(n-k)$-form satisfying
    \begin{equation}
        \eta\wedge\star\xi = \langle\eta,\xi\rangle\omega
    \end{equation}
    for every $k-$form $\eta$ where $\langle\eta,\xi\rangle$ is a real value function on $\mathcal{M}$ and the volume form $\omega$ is induced by the Reimannian metric.
    
    \textit{In coordiante form}
    Consider an orthonormal basis $\{ \frac{\partial}{\partial x_1}, \cdots,\frac{\partial}{\partial x_n} \}$ the a tangent space $V=T_p\mathcal{M}$. And its dual basis $\{ dx_1, ..., dx_n \}$ in $V^* = T_p ^*\mathcal{M}$, with the metric matrix $g_{ij} = \big(\langle\frac{\partial}{\partial x_i},\frac{\partial}{\partial x_j}\rangle\big)$ and its inverse matrix $g^{ij} = \big(\langle dx_i, dx_j \rangle\big)$. The Hodge dual of a decomposable $k$-form is them 
    \begin{equation}
        \star(dx^{i_1}\wedge\cdots\wedge dx^{i_k}) = \frac{\sqrt{|\text{det}[g_{ab}]|}}{(n-k)!}g^{i_1 j_1}\cdots g^{i_k j_k} \epsilon_{j_1 ... j_n} dx^{j_{k+1}}\wedge\cdots\wedge dx^{j_n}
    \end{equation}
    
\end{sidenote}

\section{The Cauchy Problem in General Relativity}

In this section we briefly recall the initial-value formulation of the Einstein equations of general relativity through the following steps. We start by introducing notations and the basics of GR. We summarize the Einstein field equations. Then we continue with how EFE can be split in a set of evolutionary equations and constraints. For that we focus on the Arnowitt, Deser and Misner, or ADM, formalism. In the end we comment on the stability of the ADM equations, on the need for strongly-hyperbolic formulations of the EFE, and on the choice of gauge conditions commonly used to
evolve spacetimes with singularities. This overview is based in \cite{Arnowitt:1962hi,Landau:1982dva,Wald:1984,Misner:1973,Baumgarte:2002jm}, which we refer to for more detained discussion.

\subsection{Euler-Lagrange equations}

We consider a spacetime defined by the real smooth manifold $\mathcal{M}$ and Lorentzian metric $\boldsymbol{g}$ on $\mathcal{M}$ of signature (-,+,+,+). The $\nabla$ denotes the affine connection associated with $\boldsymbol{g}$, the Levi-Civita connection. \\
We use the convention that all Greek indices lie in $\{0, 1, 2, 3\}$ and Lower case Latin indices $\{1, 2, 3\}$. \\
The $\nabla\boldsymbol{T}$ denotes the covariant derivative of a tensor $\boldsymbol{T}$ and $\nabla_{\boldsymbol{u}}\boldsymbol{T}$ -- covariant derivative along a given vector field $\boldsymbol{u}$.\\
The scalar product of two vectors then 
\begin{equation}
    \boldsymbol{a}\cdot\boldsymbol{b}:=g_{\mu\nu}a^{\mu}b^{\nu}
\end{equation}
The action of a linear form on a vector however is represented as 
\begin{equation}
    \langle\boldsymbol{\omega},\boldsymbol{\upsilon}\rangle=\omega_{\mu}\upsilon^{\mu}
\end{equation}

Let the $\boldsymbol{\alpha}$ be the totally antisymmetric symbol that expresses through coordinates $x^{\mu}$ as
\begin{equation}
    \boldsymbol{\alpha} = dx^0 \wedge dx^1 \wedge dx^2 \wedge dx^3,
\end{equation}
where $\wedge$ denotes exterior product. Then, proper volume pseudo-form of the spacetime is

\begin{equation}
    \boldsymbol{\varepsilon} = \sqrt{-g}\boldsymbol{\alpha},
\end{equation}
where $g$ denotes the determinant of the spacetime metric. \\

In GR, the spacetime is represented by Lorentzian manifold $\mathcal{M}$ and $g$, the Loretzian metric. \\

The action principle of the Lagrangian field theory on the spacetime $(\mathcal{M}; \boldsymbol{g})$ is
\begin{equation}
    S(\boldsymbol{q}, \nabla\boldsymbol{q}) = \int_{\mathcal{M}}\boldsymbol{\alpha}\mathcal{L}(\boldsymbol{q}, \nabla\boldsymbol{q}),
\end{equation}
where $\boldsymbol{q}$ are a set of generalized coordinates for the fields described by the theory, $\nabla$ is the Levi-Civita connection, $\mathcal{L}$ is a scalar density of a scalar quantity $\lambda$ as $\lambda(\boldsymbol{q},\nabla\boldsymbol{q})$. 

Varying the action with respect to the $\boldsymbol{q}$
\begin{equation}
    \delta S(\boldsymbol{q}, \nabla\boldsymbol{q}) = \delta\int\boldsymbol{\alpha}\mathcal{L}(\boldsymbol{q}, \nabla\boldsymbol{q}) = \int\boldsymbol{\alpha}\Big(\frac{\partial\mathcal{L}}{\partial\boldsymbol{q}}\delta\boldsymbol{q}+\frac{\partial\mathcal{L}}{\partial(\nabla\boldsymbol{q})}\delta\nabla\boldsymbol{q}\Big)
\end{equation}

As $\delta$ and $\nabla$ commute, and partially integrating $\nabla$, we obtain

\begin{equation}
    \partial S(\boldsymbol{q}, \nabla\boldsymbol{q}) = \int\boldsymbol{\alpha}\Big(\frac{\mathcal{L}}{\partial\boldsymbol{q}}-\nabla\frac{\partial \mathcal{L}}{\partial(\nabla\boldsymbol{q})}\Big)\delta\boldsymbol{q} + \int_{\mathcal{M}}\boldsymbol{\alpha}\nabla\Big(\frac{\partial\mathcal{L}}{\partial(\nabla\boldsymbol{q})}\delta\boldsymbol{q}\Big)
\end{equation}

The last term is a boundary term and in order to vanish we impose boundary condition. Assume that the fields are defined over only a compact domain. \\
As the choice of $\partial\boldsymbol{q}$ is arbitrary, the 

\begin{equation}
    \partial S(\boldsymbol{q}, \nabla\boldsymbol{q}) = 0
\end{equation}

and the Euler-Lagrange equations are

\begin{equation}
    \frac{\partial \mathcal{L}}{\partial\boldsymbol{q}} - \nabla\Big(\frac{\partial\mathcal{L}}{\partial(\nabla\boldsymbol{q})}\Big) = 0
    \label{eq:theory:eulerlagrange}
\end{equation}

%% ----------------------------------------------- 
\subsection{The Hilbert Action}

The Einstein–Hilbert action allows to obtain an Einstein field equations through ad principle of least action. Here we briefly underline the procedure.

Introduce action that describes the graviatational field, and a matter field $\mathcal{L}_m$:
\begin{align}
    S_g &= \int\frac{1}{2\kappa}R\epsilon, \\
    S_m &= \int\mathcal{L}_{m}\epsilon,
\end{align}
where $R$ is the Ricci scalar and $\kappa$ is the  Einstein's constant. \\

The full action then:
\begin{equation}
    S = \int\Big(\frac{1}{2\kappa}R+\mathcal{L}_m\Big)\epsilon
\end{equation}

The action principle dicatates, that $\delta S = 0$  with respect to the inverse metric $g^{\mu\nu}$. 

\begin{equation}
    \int\Bigg[\frac{1}{2\kappa}\Big(\frac{\delta R}{\delta g^{\mu\nu}}+\frac{R}{\sqrt{-g}}\frac{\delta\sqrt{-g}}{\delta g^{\mu\nu}}\Big) + \frac{1}{\sqrt{-g}}\frac{\delta(\sqrt{-g}\mathcal{L}_m)}{\delta g^{\mu\nu}}\Bigg]\delta g^{\mu\nu}\epsilon
\end{equation}

Owing to the arbitrariness of $\delta g^{\mu\nu}$, the integrant must be zero. 

\begin{equation}
    \frac{\delta R}{\delta g^{\mu\nu}} + \frac{R}{\sqrt{-g}}\frac{\delta\sqrt{-g}}{\delta g^{\mu\nu}} = -2\kappa\frac{1}{\sqrt{-g}}\frac{\delta(\sqrt{-g}\mathcal{L}_m)}{\delta g^{\mu\nu}} = -\frac{2\kappa}{\sqrt{-g}}\frac{\delta S_m}{\delta g_{\mu\nu}} := \kappa T_{\mu\nu},
    \label{eq:theory:action1}
\end{equation}
where we introduced the stress-energy tensor $T_{\mu\nu}$ and te matter action $S_m$ for future use. \\

\todo{this matter action is used in deriving the $T_{\mu} ^{\nu}$ i the invariant fluid formalisn}

The continuation of this deriviation requires taking variation of the Riccia scalar $R$ and the determinantof the metric $\sqrt{-g}$. As this is a length procedure, we provide here the result. 

\begin{equation}
    \frac{\delta R}{\delta g^{\mu\nu}} = R_{\mu\nu},
    \label{eq:theory:deltaR}
\end{equation}
where the $R_{\mu\nu}$ is the Ricci curvature tensor.

\begin{equation}
    \frac{1}{\sqrt{-g}}\frac{\delta\sqrt{-g}}{\delta g^{\mu\nu}} = -\frac{1}{2}g_{\mu\nu}.
    \label{eq:theory:deltagmuny}
\end{equation}

Substituting Eq. \ref{eq:theory:deltaR} and Eq. \ref{eq:theory:deltagmuny} into equation of motion Eq.  \ref{eq:theory:action1} we obtain the Einstein's field equation 

\begin{equation}
    R_{\mu\nu} -\frac{1}{2}g_{\mu\nu}R=8\pi T_{\mu\nu},
    \label{eq:theory:EFE}
\end{equation}
where in the geometrized unit system, \textit{i.e} $c=G=1$, the $\kappa=8\pi$.

%% ----------------
\subsection{3+1 Decomposition of Einstein field equations}

The Einstein field equations (\ref{eq:theory:EFE}) represent a set of 10 non-linear partial differential equations. These equations can be defeined on a while metric $\mathcal{M}$ or a domain $\Omega\subset\mathcal{M}$, where in the latter, the boundary conditions on $\partial\Omega$ are required. \\
It is convenient to chose a null hyersurface $\Sigma\subset\mathcal{M}$ on which to define the initial data, from which the evolution of space-time begins. This, however, requires the spacetime to be strongly hyperbolic, meaning that the foliation $\mathcal{M}=\Sigma\times\mathbb{R}$ is allowed. This foliation can be understood as splitting the spacetime into a set of spacelike hypersurfaces $\Sigma_t$. 


\subsubsection{Spacelike Foliations}
Let the $t$ be the global smooth functions such that, 

\begin{equation}
    \Sigma_{\tau} = \{x^{\alpha}\in\mathcal{M}: t(x^{\alpha})=\tau\},
\end{equation}

and let $\vec{t}$ be a vector such that $\langle\nabla t, \vec{t}\rangle = 1$. This the $t$ can be seen as a "function that advances time" and $\vec{t}$ as a "flow of time" vector field. Continuing the analogy, the rate at which a given tensor quantity $\boldsymbol{q}$ changes between hypersurfaces $\Sigma_t$ is given by the Lie derivative of the $\boldsymbol{q}$ along the vector $\vec{t}$. \\

Consider two hypersurfaces $\Sigma_t$ and $\Sigma_{t+dt}$. A transition from one to another can be decomposed into the part tangent to the hypersurface $\Sigma_{t+dt}$ and expressed in a form of a vector $\vec{\beta}$ and a pert normal to the hypersurface $\Sigma_t$ and expressed as a $\alpha \vec{n}$, where $\vec{n}$ is a unit vector, normal to the $\Sigma_t$ in the diretion to $\Sigma_{t+dt}$. Then, the vector $\vec{t}$ can be written as 

\begin{equation}
    \vec{t} = \alpha\vec{n}+\vec{\beta}.
\end{equation}

$\vec{\beta}$ is called shift vector and $\alpha$ is called lapse-function. \\

The spacetime metric $\boldsymbol{g}$ can be decomposed into a spatial, Riemannian metric $\boldsymbol{\gamma}$  as $\boldsymbol{\gamma} = \boldsymbol{g} + \underline{n} \otimes \underline{n} $, where $\underline{n}$ is the 1-form associated to the vector $\vec{n}$. The Levi-Civita connection can be computed by projecting the $\nabla$ on the space tangent to the hypersurface $\Sigma_t$.

There are exist coordinates that are adapted to the foliation, namely $\{t, x^i\}$ with $\vec{\partial}_i\cdot \vec{n} = 0$. In these coordiantes the $\nabla t = dt$ and $\vec{t} = \vec{\partial}_t$. 

The connection between $\boldsymbol{g}$ and $\boldsymbol{\gamma}$ is $g_{\mu\nu}=\vec{\partial}_{\mu}\cdot\vec{\partial}_{\nu} $ and can be expressed in terms of $\alpha$ and $\vec{\beta}$ as

\begin{align}
    \text{spatial components: } g_{ik}&=\vec{\partial}_{i}\cdot\vec{\partial}_{j} =\gamma_{ik}, \\
    \text{time component: } g_{tt} &= \vec{\partial}_{t}\cdot\vec{\partial}_{t} = \vec{t}\cdot\vec{t} = - (\alpha^2-\vec{\beta}\cdot\vec{\beta}), \\
    \text{mixed components: } g_{ti} &= \vec{\partial}_{t}\cdot\vec{\partial}_{i} = \vec{t}\cdot\vec{\partial}_i = (\alpha\vec{n}+\vec{\beta})\cdot\vec{\partial}_i=\beta_i,
\end{align}
we we made use of $\vec{\beta}$ being the spatial vector, \textit{i.e} $\vec{\beta}\cdot\vec{\beta}=\gamma_{ik}\beta^i\beta^k$.

The line-element can be thus written as
\begin{equation}
    ds^2 = -(\alpha^2-\beta_i\beta^i)dt^2 +2\beta_i dx^i dt + \gamma_{ik} dx^i dx^k.
\end{equation}

\subsubsection{Ex-curse: Hamiltonian Field Theory}

First we recall the generalized coordinates $\boldsymbol{q}$ and their covariant derivatives $\nabla\boldsymbol{q}$. \\
In light of the spacetime decomposition discussed above, we divide the $\boldsymbol{\alpha}$ into the time $dt$ and spatial parts represented by the antisymmetric symbol ${^{(3)}\boldsymbol{\alpha}}$ as 

\begin{equation}
    \boldsymbol{\alpha} = dx^0 \wedge dx^1 \wedge dx^2 \wedge dx^3 = dt \wedge {^{(3)}\boldsymbol{\alpha}}.
\end{equation}

Next, we introduce the "time derivative" as a Lie derivative along the vector field $\vec{t}$ as 

\begin{equation}
    \dot{\boldsymbol{q}} := \mathcal{L}_{\vec{t}}\boldsymbol{q}.
\end{equation}

As the $\Lambda(\boldsymbol{q}, \nabla\boldsymbol{q})$ is the Lagrangian density, a conjugate momentum can be defined as 

\begin{equation}
    \boldsymbol{p} := \frac{\partial\Lambda}{\partial\dot{\boldsymbol{q}}},
\end{equation}

Assuming that $\boldsymbol{p}$ and $\nabla\boldsymbol{q}$ can be expressed as a function of $\boldsymbol{q}$ and $\boldsymbol{p}$, inspired by the Legendre transformation, we define the Hamiltonian and its density density as

\begin{align}
    \mathcal{H} &= \boldsymbol{p}\cdot\dot{\boldsymbol{q}} - \mathcal{L}(\boldsymbol{q}, \nabla\boldsymbol{q}) \\
    H &= \int_{\Sigma}\mathcal{H}{^{(3)}\boldsymbol{\alpha}}
\end{align}

Additionally we define the quantity 

\begin{equation}
    J = \int_{0}^{t}H(\boldsymbol{q},\boldsymbol{p})dt = \int_{0}^{t}dt\int_{\Sigma}\mathcal{H}(\boldsymbol{q},\boldsymbol{p}){^{(3)}\boldsymbol{\alpha}} = \int_{0}^{t}dt\int_{\Sigma}{^{(3)}\boldsymbol{\alpha}}\Big(\boldsymbol{p}\cdot\dot{\boldsymbol{q}} - \mathcal{L}(\boldsymbol{q},\nabla\boldsymbol{q})\Big).
\end{equation}

Consider the variation of the $J$ with respect to the $\delta\boldsymbol{p}$ and $\delta\boldsymbol{q}$ as

\begin{equation}
    \delta J = \int_{0}^{t}\delta H(\boldsymbol{q},\boldsymbol{p})dt = \int_{0}^{t}dt (\dot{\boldsymbol{q}}\delta\boldsymbol{p}+\boldsymbol{p}\delta\dot{\boldsymbol{q}}) - \int_{0}^{t}dt\delta\Lambda(\boldsymbol{q}, \nabla\boldsymbol{q}).
\end{equation}

Consider the last term, the variation of the Lagrangian 

\begin{equation}
    \delta\Lambda = \int_{\Sigma}{^{(3)}\boldsymbol{\alpha}}\Bigg[\frac{\delta\Lambda}{\delta\dot{\boldsymbol{q}}}\delta\dot{\boldsymbol{q}}+\frac{\delta\Lambda}{\delta\boldsymbol{q}}\delta\boldsymbol{q}\Bigg],
\end{equation}

The first term in the square brackets can be reduced to $\boldsymbol{p}\delta\dot{\boldsymbol{q}}$, suingthe definition of the conjugate momentum. The second term can be treated, applying the Euler-Lagrange equations (\ref{eq:theory:eulerlagrange}). These manipulations result in

\begin{equation}
    \delta\Lambda = \int_{0}^{t}dt\int_{\Sigma}{^{(3)}\boldsymbol{\alpha}}(\boldsymbol{p}\delta\dot{\boldsymbol{q}} + \dot{\boldsymbol{p}}\delta\boldsymbol{q}).
\end{equation}

Thus we obtain that 

\begin{equation}
    \int_{0}^{t} \delta H(\boldsymbol{q},\boldsymbol{p})dt =   \int_{0}^{t}dt\int_{\Sigma}{^{(3)}\boldsymbol{\alpha}}(\dot{\boldsymbol{q}}\cdot\delta\boldsymbol{p}-\dot{\boldsymbol{p}}\cdot\delta\boldsymbol{q}),
\end{equation}

and as $\delta\boldsymbol{p}$ and $\delta\boldsymbol{p}$ are arbitrary, the Hamilton equations read

\begin{equation}
    \dot{\boldsymbol{q}}=\frac{\delta H}{\delta\boldsymbol{p}}, \hspace{5mm} \dot{\boldsymbol{p}} = -\frac{\delta H}{\delta\boldsymbol{q}}.
    \label{eq:theory:hamiltoneqs}
\end{equation}

The Hamiltonian formalism can be used to redirive the field-equations in a from that once the initial data is specified on a hypersurface $\Sigma_0$ for $\boldsymbol{q}$ and $\boldsymbol{p}$, the equations (\ref{eq:theory:hamiltoneqs}) would govern whole the evolution.


\subsubsection{Extrinsic Curvature and Constraint equations}

We define the \textit{extrinsic curvature} of a $D-1$-suface $\Sigma_t\subset\mathcal{M}$ at a point $\mathcal{P}\in\Sigma_t$ as mapping $\boldsymbol{K}$ such that $\boldsymbol{K}(\boldsymbol{\upsilon})=-\nabla_{\boldsymbol{\upsilon}}\boldsymbol{n}$. Note, that the $\boldsymbol{K}$ thus does not depend on $\alpha$ and $\vec{\beta}$, it is a purely spatial tensor. The components of the extrinsic curvature are \\

\begin{equation}
    K_{\mu\nu} = -{\gamma^{\alpha}}_{\mu}\nabla_{\boldsymbol{u}}^{\alpha} n_{\nu} = -\frac{1}{2}\mathcal{L}_{\vec{n}}\gamma_{\mu\nu},
    \label{eq:theory:extrcurvdef}
\end{equation}
where $\mathcal{L}_{\vec{n}}$ is the Lie derivative along the vector field $\vec{n}$. \\
From the (\ref{eq:theory:extrcurvdef}) the extrinsic curvature can be interprated as a "speed of the $\vec{n}$ during the parallel transport along the hypersurface $\Sigma_t$".

Codazzi equations relate the $4D$ Ricci tensor to the extrinsic curvature as

\begin{equation}
    D_{\beta}K-D_{\alpha}{K^{\alpha}}_{\beta}=R_{\gamma\delta}n^{\delta}{\gamma^{\gamma}}_{\beta},
    \label{eq:theory:formomentum}
\end{equation}

here $K$ is a trace of the tensor $\boldsymbol{K}$. \\

Gauss equation realtes the $3D$ Riemann tensor $^3{R_{\alpha\beta\gamma}}^{\delta}$ to the $4D$ one and the $\boldsymbol{K}$ as

\begin{equation}
    ^3{R_{\alpha\beta\gamma}}^{\delta} = {\gamma^{\mu}}_{\alpha}{\gamma^{\nu}}_{\beta}{\gamma^{\lambda}}_{\gamma}{\gamma^{\delta}}_{\sigma}{R_{\mu\nu\lambda}}^{\delta}-K_{\alpha\gamma}{K_{\beta}}^{\delta}+K_{\beta\gamma}{K^{\delta}}_{\alpha}.
    \label{eq:theory:forhamiltconst}
\end{equation}

The \textit{momentum constraint} thus cab be obtained by substituting the (\ref{eq:theory:EFE}) into  (\ref{eq:theory:formomentum}) which yields

\begin{equation}
    D_{\beta}K-D_{\alpha}{K^{\alpha}}_{\beta} = -8\pi{\gamma^{\alpha}}_{\beta} n^{\gamma}T_{\alpha\gamma}=:8\pi j_{\beta},
    \label{eq:theory:momconstraint}
\end{equation}
where $j^{\alpha}$ is the ADM momentum density. \\

The \textit{Hamiltonian constrant} can be obtained by substituting EFE (\ref{eq:theory:EFE}) into the (\ref{eq:theory:forhamiltconst}), yielding 

\begin{equation}
    ^3 R+ K^2 - K_{\alpha\beta}K^{\alpha\beta} = 2G^{\alpha\beta}n_{\alpha}n_{\beta} = 16\pi n_{\alpha}n_{\beta} T^{\alpha\beta} =: 16\pi E,
    \label{eq:theory:hamilconstraint}
\end{equation}
where $E$ is the ADM energy density. \\

The obtained constraint equations represent a set of elliptic equations that must be satisfied on every hyprsurface $\Sigma_i$ of the foliation. It is however, possible to show that Eistein equations preserve the constraints, meaning that if they are satisfied at the initial slice $\Sigma_0$ they will be satisfied at any time in the future. 





\subsubsection{The Hamiltonian Formulation of the Einstein Equations}

Here we briefly sketch to path of derivation of the Einstein field equations in the Hamiltonian framework. We will elude most of the intimidate and computationally extensive steps, as well as derivation of the boundary terms. For this we refer to \cite{Poisson:2004}.\\
First it is useful to note that determinant of the three-metric $\sqrt{\gamma}$ can be expressed as $\sqrt{\gamma}=\sqrt{-g}/\alpha$. The $p$ is the trace of the canonical momentum $\boldsymbol{p}$.

Now, consider the scalar curvature, R

\begin{align}
    G_{\mu\nu} &= R_{\mu\nu} - \frac{1}{2}Rg_{\mu\nu} \\
    -Rg_{\mu\nu}n^{\nu}n^{\mu} &= 2(G_{\mu\nu} n^{\nu}n^{\mu}-R_{\mu\nu}n^{\mu}n^{\mu})\\
    -Rn_{\mu}n^{\mu}& = 2(G_{\mu\nu}n^{\nu}n^{\mu} - R_{\mu\nu}n^{\mu}n^{\mu}) \\
    R &= 2(G_{\mu\nu}n^{\mu}n^{\nu} - R_{\mu\nu}n^{\mu}n^{\nu}).
\end{align}

From the Gauss-Codacci equation (\ref{eq:theory:momconstraint}), which relates the spatial curvature $^{(3)}R$ to the spacetime curvature $R$, we have the following constraint
relationship

\begin{equation}
    2G_{\mu\nu}n^{\mu}n^{\nu} = {^{(3)}R} + K^2 - K_{\mu\nu}K^{\mu\nu}.
\end{equation}

The $R_{\mu\nu}n^{\mu}n^{\nu})$ can be expressed as a combination of extrinsic curvature and total divergences as 

From the definition of the Ricci tensor $R_{\mu\nu}$, we have:

\begin{align}
    R_{\mu\nu} &= {R_{\mu\gamma\nu}}^{\gamma} \\
    R_{\mu\nu}n^{\mu}n^{\nu} &= {R_{\mu\gamma\nu}}^{\gamma} \\
    &= -(\nabla_{\mu}\nabla_{\gamma} - \nabla_{\gamma}\nabla_{\mu})n^{\gamma}n^{\nu} \\
    &= n^{\mu}(\nabla_{\mu}\nabla_{\gamma} - \nabla_{\gamma}\nabla_{\nu})n^{\gamma} \\
    &= (\nabla_{\mu}n^{\mu})(\nabla_{\gamma}n^{\gamma}) - \nabla_{\mu}(n^{\mu}\nabla_{\gamma}n^{\gamma}) - (\nabla_{\gamma}n^{\mu})(\nabla_{\mu}n^{\gamma}) + \nabla_{\gamma}(n^{\mu}\nabla_{\mu}n^{\gamma}) \\
    &= K^2 - K_{\mu\gamma}K^{\mu\gamma} - \nabla_{\mu}(n^{\mu}\nabla_{\gamma}n^{\gamma}) + \nabla_{\gamma}(n^{\mu}\nabla_{\mu}n^{\gamma})
\end{align}

In case of variations with compact support, that we are interested in, the total divergences. last two terms, can be neglected. Then the result is

\begin{equation}
    R_{\mu\nu}n^{\mu}n^{\nu}= K^2 - K_{\mu\nu}K^{\mu\nu}.
    \label{eq:theory:rmunu_as_func_k}
\end{equation}

Using the fact that $\sqrt{\gamma}=\sqrt{-g}/\alpha$ and the (\ref{eq:theory:rmunu_as_func_k}) we obtain the Lagrangian density in terms of the variables of the hypersurface:

\begin{align}
    \Lambda &= \sqrt{-g}R \\
    &= \alpha\sqrt{\gamma}R \\
    &= 2\alpha\sqrt{\gamma}(G_{\mu\nu}n^{\mu}n^{\nu} - R_{\mu\nu}n^{\mu}n^{\nu})\\ 
    &= 2\alpha\sqrt{\gamma}\Big(\frac{1}{2}[{^{(3)}R} - K_{\mu\nu}K^{\mu\nu} + K^2] - K^2 - K_{\mu\nu}K^{\mu\nu}\Big)
\end{align}

Together with the contribution from matter fields, we obtain

\begin{equation}
    \Lambda = \Lambda_g+\Lambda_m= \frac{1}{16\pi}\alpha({^{(3)}R} + K_{\mu\nu}K^{\mu\nu} - K^2)\sqrt{\gamma}+\Lambda_m
\end{equation}

Next we note that the extrinsic curvature of a
surface $\Sigma$ is defined as $K_{\mu\nu} = \nabla_{\mu}n_{\nu}$. \\
To relate $K_{\mu\nu}$ to the metric, we make use of the following property of Lie derivatives:

\begin{align}
    \mathcal{L}_{\vec{n}}g_{\mu\nu} &= n^{\gamma}\nabla_{\gamma}g_{\mu\nu} + g_{\gamma\nu}\nabla_{\mu}\upsilon^{\gamma} + g_{\mu\gamma}\nabla_{\nu}\upsilon^{\gamma} \\
    &= \nabla_{\mu}n_{\nu}+\nabla_{\nu}\upsilon_{\nu} \\
    &=2\nabla_{\mu}n_{\nu}
\end{align}

where the second line holds when $\nabla_{\gamma}\mu$ is the natural derivative operator corresponding to the metric $g_{\mu\nu}$ and the third line holds because $K_{\mu\nu}$ is symmetric.

Substituting this into our definition of $K_{\mu\nu}$,

\begin{align}
    K_{\mu\nu} &= -\frac{1}{2}\mathcal{L}_{\vec{\vec{n}}}g_{\mu\nu} \\
    &= -\frac{1}{2}\mathcal{L}_{\vec{\vec{n}}}(\gamma_{\mu\nu}-n_{\mu}n_{\nu}) \\
    &= -\frac{1}{2}\mathcal{L}_{\vec{\vec{n}}}\gamma_{\mu\nu} \\
    &= -\frac{1}{2}[n^{\gamma}\nabla_{\gamma}\gamma_{\mu\nu} + \gamma_{\gamma\nu}\nabla_{\mu}\upsilon^{\nu} + h_{\mu\gamma}\nabla_{\nu}\upsilon^{\gamma}] \\
    &= -\frac{1}{2\alpha}[\alpha n^{\gamma}\nabla_{\gamma}\gamma_{\mu\nu} + \gamma_{\gamma\nu}\nabla_{\mu}\alpha\upsilon^{\nu} + h_{\mu\gamma}\nabla_{\nu}\alpha\upsilon^{\gamma}] \\
    &= -\frac{1}{2\alpha}{\gamma_{\mu}}^{\gamma}{\gamma_{\nu}}^{\delta}[\mathcal{L}_{\vec{t}}\gamma_{\gamma\delta}-\mathcal{L}_{\vec{\beta}}\gamma_{\gamma\delta}] \\
    &= -\frac{1}{2\alpha}{\gamma_{\mu}}^{\gamma}{\gamma_{\nu}}^{\delta}[\partial_t\gamma_{\mu\nu}-D_{\mu}\beta_{\nu}-D_{\nu}\beta_{\mu}]
\end{align}

and on the hypersurface $\Sigma$ the projection operators are not needed. So we obtain

\begin{equation}
    K_{\mu\nu} = -\frac{1}{2}\mathcal{L}_{\vec{n}}\gamma_{\mu\nu}=-\frac{1}{2\alpha}(\partial_t\gamma_{\mu\nu}-D_{\mu}\beta_{\nu}-D_{\nu}\beta_{\mu})
\end{equation}

which us to express the canonical momentum $p^{\mu\nu}$ as

\begin{align}
    p^{\mu\nu} &= \frac{\partial\Lambda}{\partial\dot{\gamma}_{\mu\nu}} \\
    &= -\frac{\sqrt{\gamma}}{16\pi}\alpha\Bigg[\frac{\partial {^{(3)}R}}{\partial\dot{\gamma}_{\mu\nu}} + \frac{\partial(K_{\mu\nu}K^{\mu\nu})}{\partial\dot{\gamma}_{\mu\nu}} - \frac{\partial K^2}{\partial\dot{\gamma}_{\mu\nu}}\Bigg] \\
    &= \frac{\sqrt{\gamma}}{16\pi}(K\gamma^{\mu\nu} - K^{\mu\nu}),
\end{align}
where 
\begin{equation}
    \frac{\partial K_{\mu\nu}}{\partial \dot{\gamma}_{\mu\nu}} = \frac{1}{2\alpha}, \hspace{5mm} \frac{\partial {^{(3)}R}}{\partial \dot{\gamma}_{\mu\nu}} = 0, \hspace{5mm}\frac{\partial K^2}{\partial \dot{\gamma}_{\mu\nu}} = \frac{\gamma^{\mu\nu}K}{\alpha}
\end{equation}

assuming that there is no explicit dependency of the $\Lambda$ on $dot{\gamma}_{\mu\nu}$.

Since, $\alpha$ and $\vec{\beta}$ are related to the the gauge freedom, as there are many ways manifold $\mathcal{M}$ can be split into hypersurfaces, the momenta associated with these function and vector is zero. 

Thus, the Hamiltonian density is

\begin{align}
    \mathcal{H} &= p^{\mu\nu}\dot{\gamma}_{\mu\nu} - \Lambda \\
    &= -\sqrt{\gamma}\alpha{^{(3)}R} + \frac{\alpha}{\sqrt{\gamma}}\Big[p^{\mu\nu}p_{\mu\nu}-\frac{1}{2}p^2\Big] + 2p^{\mu\nu} D_{\mu}\beta_{\mu} -\Lambda_m \\
%    &=  \frac{\sqrt{\gamma}}{16\pi}\Bigg\{\alpha\Big[-{^{(3)}R}+h^{-1}p^{\mu\nu}p_{\mu\nu}-\frac{1}{2}h^{-1}p^2\Big] - 2\beta_{\nu}\big[D_{\mu}(h^{-1/2}p^{\mu\nu})\big] + D_{\mu}(h^{-1/2}\beta_{\nu}p^{\mu\nu})\Bigg\} \\
    &= \frac{\sqrt{\gamma}}{16\pi}\Bigg\{\alpha\Big[ -{^{(3)}R} + \gamma^{-1}p^{\mu\nu}p_{\mu\nu}-\frac{1}{2}\gamma^{-1}p^2\Big] +  2\beta_{\nu}\Big[D_{\mu}(\gamma^{-1/2}p^{\mu\nu})\Big] - 2D_{\mu}(\gamma^{-1/2}\beta_{\nu}p^{\mu\nu}) \Bigg\} - \Lambda_m,
\end{align}
where we restored the correct $16\pi$ factor in the last line.

As the we consider variations with compact suppot, the last boundary term, can be neglected. \\

Now we consider the variation of the matter action $S_m$ with respect to the $\alpha$ and $\vec{\beta}$

\begin{align}
    \frac{\delta S_m}{\delta \alpha} &=-\alpha\frac{\delta S_m}{\delta g_{00}} = -\alpha\sqrt{-g}T^{00} = -\alpha^2\sqrt{\gamma}T^{00} = -\sqrt{\gamma}T^{\mu\nu}n_{\mu}n_{\nu} \\
    \frac{\delta S_m}{\delta \beta_{\mu}} &= \frac{\delta S_m}{\delta g_{\mu 0}} =\frac{1}{2}\sqrt{-g}T^{\mu 0} = -\frac{1}{2} \sqrt{\gamma}T^{\mu\nu}n_{\nu}.
\end{align}

As the variation of the Hamiltonian $H$ with respect to a quantity with vanishing canonical momentum is zero, we obtain two equations 

\begin{align}
    \frac{\delta H}{\delta \alpha} &= 0 = -{^{(3)}R} + \gamma^{-1}p^{\mu\nu}p_{\mu\nu}-\frac{1}{2}\gamma^{-1}p^2 + 16\pi T^{\mu\nu}n_{\mu}n_{\nu} \\
    \frac{\delta H}{\delta \beta_{\mu}} &= 0 = - D_{\mu}(\gamma^{-1/2}p^{\mu\nu}) + 8\pi{\gamma^{\mu}}_{\nu}n_{\gamma}T^{\nu\gamma}.
    \label{eq:theory:hamiltonianvariation}
\end{align}


Note, that the $\delta H / \delta\beta_{\mu}$ is actually a Frech\'et differential $dH$, $\delta \beta_{\mu}$, which is writes as
\begin{equation}
    \langle dH,\delta\beta \rangle = \delta\beta_{\mu}\big[-D_{\nu}(\gamma^{-1/2}p^{\mu\nu})+8\pi n_{\gamma}T^{\mu\nu}\big], 
\end{equation}
containing $\delta\beta_{\mu}$ which is spatial. Thus only the spatial part is being constrained in the equation above. To account for that the procector ${\gamma^{\mu}}_{\nu}$ is added to the $\delta H/\delta \beta_{\mu}$. \\

The pair of equations (\ref{eq:theory:hamiltonianvariation}) is in fact the constraint equations derived before, namely the (\ref{eq:theory:momconstraint}) and (\ref{eq:theory:hamilconstraint}), and as we now see, they are related to the coordinate freedom of $\mathcal{M}$ decomposition and a coodrinate freedom on hypersurfaces. \\

Proceeding with the Hamiltinan formalism we note that equation \ref{eq:theory:hamiltoneqs} leads to the evolution equations for the three-metric, assuming that $\Lambda$ explicitly does not depend on the momentum

\begin{equation}
    \dot{\gamma}_{\mu\nu} =\frac{\delta H}{\delta p^{\mu\nu}} = 2\gamma^{-1/2}\alpha\big(p_{\mu\nu}-\frac{1}{2}\gamma_{\mu\nu}p\big) - D_{\nu}\beta_{\mu}-D_{\mu}\beta_{\nu}
%    -2D_{(\mu}\beta_{\nu)},
    \label{eq:theory:_adm_metric_evo}
\end{equation}

The evolution equations for the canonical momentum can read

\begin{align}
    \dot{p}^{\mu\nu} = -\frac{\delta H}{\delta \gamma_{\mu\nu}} = &+ \alpha\gamma^{1/2}\big({^{(3)}R}^{\mu\nu}-\frac{1}{2}{^{(3)}R\gamma^{\mu\nu}}\big) \\
    & - \frac{1}{2}\alpha\gamma^{-1/2}\gamma^{\mu\nu}\big(p_{\gamma\delta}p^{\gamma\delta}-\frac{1}{2}p^2\big) \\
    & + 2\alpha\gamma^{-1/2}\big(p^{\mu\gamma}{p^{\nu}}_{\gamma}-\frac{1}{2}pp^{\mu\nu}\big) \\
    & - \gamma^{1/2}\big(D^{\mu}D^{\nu}\alpha-\gamma^{\mu\nu}D^{\gamma}D_{\gamma}\alpha\big) \\
    & - \gamma^{1/2}D_{\gamma}\big(\gamma^{-1/2}\beta^{\gamma}p^{\mu\nu}\big) \\
    &+ 2p^{\gamma(\mu}D_{\gamma}\beta^{\nu)} + 8\pi \alpha \gamma^{1/2}S^{\mu\nu},
    \label{eq:theory:_adm_mom_evo}
\end{align}
where $A_{(\mu\nu)} = 0.5(A_{\mu\nu}+A_{\nu\mu})$ the convention was used. \\

where $S^{\mu\nu}={\gamma^{\mu}}_{\alpha}{\gamma^{\nu}}_{\beta}T^{\alpha\beta}$. 
Taking the variation of the matter field we noted that 


\begin{equation}
    \frac{\delta S}{\delta \gamma_{ik}} = \frac{\delta S_m}{\delta g_{ik}} = \frac{1}{2}\sqrt{-g}T^{ik}
\end{equation}

The set of equations (\ref{eq:theory:hamiltonianvariation}), (\ref{eq:theory:_adm_metric_evo}) and (\ref{eq:theory:_adm_mom_evo}) comprise the ADM system. A more widely used from of these equations is in turns of $\gamma_{ij}$ and $K_{ij}$ that reads

\begin{align}
    (\partial_t - \mathcal{L}_{\vec{\beta}})\gamma_{ik} &= -2\alpha K_{ik}; \\
    (\partial_t - \mathcal{L}_{\vec{\beta}})K_{ik} &= -D_{i}D_{k}\alpha + \alpha\big(R_{ik} - 2K_{ij}{K^j}_k+KK_{ik}\big) - 8\pi\alpha\big(S_{ik} - \frac{1}{2}\gamma_{ik}(S-E)\big); \\
    {^{(3)}R} + K^2 - K_{ik}K^{ik} &= 16\pi E; \\
    D_{i}K-D_{k}{K^k}_i &= 8\pi j_i,
    \label{eq:theory:adm}
\end{align}
where $S = \gamma^{ij}S_{ij}$.
These equations constitute the IVP for Einstein field equations and are known as ADM equations. The last two equations are the constraint equations. They determine how to set the initial data on the hypersurface $\Sigma_0$, via prescribing the three-metric and extrinsic curvature. The first two equations then govern the evolution.

\todo{make sure that the coefficients in formuals are consistent, $16\pi$ might me missing or $-$}
\todo{Makse sure that $\Lambda$ stands for largangian density and $\mathcal{L}$ for lie derivative}

\subsubsection{Strongly Hyperbolic Formulations of the Einstein Equations}

It has been shown, that the ADM system of equations in its original form (\ref{eq:theory:adm}) is only weekly hyperbolic \cite{Baumgarte:2002jm}. It was shown that in such system the errors tend to couple with zero-velocity modes \cite{Alcubierre:1999rt}.  \\
In an attempt to mitigate this problem, different formulations of the Einstein equations as initial-value problem were created. In particular, the the generalized-harmonic formulation \cite{Friedrich:1985,Lindblom:2005qh,Lindblom:2009}, the BSSNOK formulation, derived by Baumgarte, Shapiro, Shibata, Nakamura, Oohara and Kojima \cite{Nakamura1987,Shibata:1995we,Baumgarte:1998te} and and the Z4 formulation \cite{Bona:2003fj,Bernuzzi:2009ex,Ruiz:2010qj,Weyhausen:2011cg,Alic:2011gg}. We do not attempt to elaborate on any of these formations and only aim to emphasize that a search for a new and better formulations of Einstein equations for numerical applications is ongoing. We limit ourselves to sketching only the conformal-covariant variant of the Z4 formulation, also known as Z4c. The numerical implementation of this formulation was used to obtain the results discussed in this thesis. 


\subsubsection*{The CCZ4 Formulation}

The idea behind the Z4 formulation is to derive a set of evolution equations that is free from the zero-speed modes of the original ADM and thus -- strongly-hyperbolic. This is achieved by not explicitly enforcing the constraints and treating the deviation from them as an dependent variable $Z_{\mu}$. The $Z_{\mu}$ is also called the Z4 four-vector.

One starts with the covariant Lagrangian
\begin{equation}
    \Lambda = g^{\mu\nu}[R_{\mu\nu} + 2\nabla_{\mu}Z_{\nu}]\sqrt{g} + \Lambda_m,
\end{equation}

and applying Palatini-type variational principle \cite{Bona:2010is}, obtains an evolution equations

\begin{equation}
    R_{\mu\nu} + \nabla_{\mu}Z_{\nu} + \nabla_{\nu}Z_{\mu}=8\pi\Big(T_{\mu\nu} - \frac{1}{2}Tg_{\mu\nu}\Big),
    \label{eq:theory:z4fieldeq}
\end{equation}

and two sets of constraint equations

\begin{equation}
    \nabla_{\rho} g^{\mu\nu} = 0, 
    \label{eq:theory:z4connect}
\end{equation}

and

\begin{equation}
    Z_{\mu} = 0,
\end{equation}

where the latter is called an algebraic constraint. If its derivative vanishes, it is equivalent to imposing the ADM momentum and Hamiltonian constraints \cite{Bona:2009}. 

The Einstein field equations themselves are recovered from (\ref{eq:theory:z4connect}) and (\ref{eq:theory:z4fieldeq}) when the algebraic constraint is satisfied. \\

The Z4 system preserves the constraint, $\partial_t (Z_{\mu})= 0$. This allows to obtain the solution of the Einstein equations. 

However, the numerical solution of the system of equations introduces error, that leads to a constraint violation during the evolution. To mitigate this problem the Z4 system is further modified to enforce the dampening of the constraint violation propagation \cite{Gundlach:2005eh}.

A new version of Z4 was recently introduced by \cite{Alic:2011gg}. It incorporates the constraint-damping properties of the original Z4 and also allows for a better black hole treatment via \textit{moving-puncture}, that will be discussed later. 
The CCZ4 system reads 

\begin{align}
    \partial_{t}\widetilde{\gamma}_{ij} = & -2\alpha\widetilde{A}_{ij}^{\text{TF}} + 2\widetilde{\gamma}_{k(i}\partial_{j)}\beta^k - \frac{2}{3}\widetilde{\gamma}_{ij}\partial_k \beta^k + \beta^k\partial_k\widetilde{\gamma}_{ij}, \\
    \partial_{t}\widetilde{A}_{ij}^{\text{TF}} = & \phi^2\big[-\nabla_i\nabla_j\alpha + \alpha\big({^{(3)}R}_{ij}+\nabla_{i}Z_{j} + \nabla_{j}Z_{i}- 8\pi S_{ij}\big)\big]^{\text{TF}} \\
    & + \alpha\widetilde{A}_{ij}(K-2\Theta)-2\alpha\widetilde{A}_{il}{\widetilde{A}^l}_{j} + 2\widetilde{A}_{k(i}\partial_{j)}\beta^{k} \\
    & -\frac{2}{3}\widetilde{A}_{ij}\partial_{k}\beta^{k} + \beta^{k}\partial_{k}\widetilde{A}_{ij} \\
    \partial_{t} \phi = & \frac{1}{3}\alpha\phi K - \frac{1}{3}\phi\partial_{k}\beta^{k} + \beta^{k}\partial_{k}\phi \\
    \partial_{t}K = &-\nabla^{i}\nabla_{i}\alpha + \alpha\big({^{(3)}R} + 2\nabla_{i}Z^{i} + K^2 - 2\Theta K\big) + \beta^{j}\partial_{j}K \\
    & - 3\alpha\kappa_1(1+\kappa_2)\Theta + 4\pi\alpha (S-3E) \\
    \partial_{t}\Theta = &\frac{1}{2}\alpha\Big(R + 2\nabla_{i}Z^{i} - \widetilde{A}_{ij}\widetilde{A}^{ij} + \frac{2}{3}K^2 - 2\Theta K\Big) - Z^{i}\partial_{i}\alpha \\
    & + \beta^{k}\partial_{k}\Theta - \alpha\kappa_1(2 + \kappa_2)\Theta - 8\pi\alpha E \\
    \partial_{t}\hat{\Gamma}^j = & 2\alpha\Bigg({\widetilde{\Gamma}^i}_{jk}\widetilde{A}^{ij} - 3\widetilde{A}^{ij}\frac{\partial_{j}\phi}{\phi} -\frac{2}{3}\widetilde{\gamma}^{ij}\partial_{j}K\Bigg) + 2\widetilde{\gamma}^{ki}\Big(\alpha\partial_{k}\Theta - \Theta\partial_{k}\alpha - \frac{2}{3}\alpha K Z_{k}\Big) \\
    & - 2\widetilde{A}^{ij}\partial_{j}\alpha + \widetilde{\gamma}^{kl}\partial_{k}\partial_{l}\beta^{i} + \frac{1}{3} \widetilde{\gamma}^{ik}\partial_{k}\partial_{l}\beta^{l} + \frac{2}{3}\widetilde{\Gamma}^i\partial_{k}\beta^{k} \\
    & - \widetilde{\Gamma}^k\partial_{k}\beta^{i} + 2\kappa_3\Big(\frac{2}{3}\widetilde{\gamma}^{ij}Z_{j}\partial_{k}\beta^{k} - \widetilde{\gamma}^{jk}Z_{j}\partial_{k}\beta^{i}\Big) + \beta^{k}\partial_{k}\hat{\Gamma}^i \\
    & -2\alpha\kappa_1\widetilde{\gamma}^{ij}Z_{j}- 16\pi\alpha\widetilde{\gamma}^{ij}S_j,
\end{align}

where $\Theta:=n_{\mu}Z^{\mu}=\alpha Z^0$, the $\widetilde{\Gamma}^i:=\widetilde{\gamma}^{jk}{\widetilde{\Gamma}^i}_{jk} = \widetilde{\gamma}^{ij}\widetilde{\gamma}^{kl}\partial_{l}\widetilde{\gamma}_{jk}$ and $\hat{\Gamma}:=\widetilde{\Gamma}^i + 2\widetilde{\gamma}^{ij}Z_j$, constants $\kappa_1$ and $\kappa_2$ are related to the constraint damping terms, the $\kappa_3$ is the additional constant for further adjustments, the The three-dimensional Ricci tensor ${^{(3)})R}_{ij}$ is split into conformal part $\widetilde{R_{ij}^{\phi}}$ and the $\widetilde{R_{ij}}$ that contains the derivatives of the conformal metric

\begin{align}
    \widetilde{R_{ij}} &= -\frac{1}{2}\widetilde{\gamma}^{lm}\partial_{l}\partial_{m}\widetilde{\gamma}_{ij} + \widetilde{\gamma}_{k(i}\partial_{j)}\widetilde{\Gamma}_{(ij)k} + \widetilde{\gamma}^{lm}\big[2\widetilde{\Gamma}^{k}_{l(i}\widetilde{\Gamma}_{j)km} + \widetilde{\Gamma}^{k}_{im}\widetilde{\Gamma}_{kjl}\big] \\
    \widetilde{R_{ij}}^{\phi} &= \frac{1}{\phi^2}\big[\phi\big(\widetilde{\nabla}_{i}\widetilde{\nabla}_{j}\phi + \widetilde{\gamma}_{ij}\widetilde{\nabla}^{l}\phi\widetilde{\nabla}_{l}\phi\big) - 2\widetilde{\gamma}_{ij}\widetilde{\nabla}^{l}\phi\widetilde{\nabla}_{l}\phi\big]
\end{align}

And as one sees, the ecolution of $Z_i$ is now included in $\hat{\Gamma}$. 
\todo{understand the conformal stuff and add some steps to show how the ccz4 was made}

\subsubsection{Gauge conditions}

During the discussion of the original ADM system, the choice of the lapse function, \textit{i.e} slicing condition, and shift vector \textit{i.e} spatial gauge condition was left open. The right choice however, is crutual for the stable evolution and in itself presents a broad and rapidly evolving subject. Here we are going to discuss only the gauge that is relevant for our work. 

\paragraph{Slicing condition} One of the widely used conditions is so called 'maximal slicing' that sets $K=0$, which in turn results in the equation

\begin{equation}
    D^{i}D_{i}\alpha = \alpha\big[K_{ij}K^{ij} + 4\pi(e+S)\big].
\end{equation}

This conditions has an advantage of being \textit{singularity-avoiding}. For example, it was shown that in the case of Schwarzschild black hole, the $\alpha$ goes to zero at a finite distance from singularity \cite{Geyer:1995}. However implementation of this condition in from of a elliptic equations is computationally expensive.   \\
A class of slicing conditions in form of hyperbolic equations that are more favorable from numerical standpoint and that reproduces the desired behavior of the maximal scicing was proposed in \cite{Bona:1994dr}. It is read 

\begin{equation}
(\partial_t - \beta^i\partial_i)\alpha = \alpha^2 f(\alpha)K
\label{eq:theory:gauge_onepluslog}
\end{equation}

which in CCZ4 reads 

\begin{equation}
    (\partial_t - \beta^i \partial_i )\alpha = \alpha^2 f(\alpha)(K-2\Theta)
\end{equation}

where $f(\alpha)$ is a positive function. For many numerical applications, including those that are discussed in this work, the "1 + log" slicing is adopted, the $\beta_i=0$. Then, integrating equation (\ref{eq:theory:gauge_onepluslog}) yields 

\begin{equation}
    \alpha = 1 + \log\gamma
\end{equation}

This condition is numerically more favorable and as $f\rightarrow\infty$ in the vicinity of a singularity, allows to treat black holes well like maximal slicing \cite{Baumgarte:2002jm}.

\todo{add/modify some text.}

\paragraph{Spatial gauge conditions}

The requirements for the gauge are similar as in the case of the $\alpha$, namely hyperbolicity and minimization of numerical distortions for more stable evolution.  

One of the widely used shift conditions is so called \textit{Gamma driver} condition \cite{Alcubierre:2002kk}, 

\begin{align}
    \partial_t\beta^i &= \frac{3}{4}\alpha B^i, \\
    \partial_t B^i &= \partial_t\widetilde{\Gamma}^i - \eta B^i,
\end{align}

where $\eta$ is a dumping coefficient. \\

This gauge condition tries to decrease the coordinate stretching that occur in the vicinity of a singularity. It was shown to be effective in numerical applications, in particular for a single moving black hole. However it has a zero-speed mode, that can amplify the numerical errors and destabilize the system \cite{vanMeter:2006vi}.

A modified \textit{Gamma driver}, gauge that does not have zero or small speed modes:

\begin{align}
    (\partial_t - \beta^j\partial_j)\beta^i &= \frac{3}{4}B^i \\
    (\partial_t - \beta^j\partial_j)B^i &= (\partial_t - \beta^j\partial_j)\widetilde{\Gamma}^i-\eta\beta^i,
\end{align}

was proposed by \cite{vanMeter:2006vi} and was applied to study binary black holes by \cite{Campanelli:2005dd}.

\subsection{The Equations of General-Relativistic Hydrodynamics}

In this section we discuss the equations of general relativistic hydrodynamics. We consider the fluid on a Lorentzian manifold and how its flow affects the spacetime. \\ 

The topics that we are going to touch are:
\begin{itemize}
    \item fluid kinematics,
    \item equations of motion for perfect fluids (assuming that there is no thermal conduction or viscosity)
    \item the “Valencia formulation” of the hydrodynamic equations.
\end{itemize}

We note that the following description is very brief and is based on the following works: \cite{Misner:1973},\cite{Schutz:2009a},\cite{Gourgoulhon:2006bn},\cite{Andersson:2006nr},\cite{Rezzolla:2013} to which we refer the reader for more details.  

\subsubsection{Kinematics of a Relativistic Fluid}

In Newtonian physics, a fluid is an "entity" whose dynamics is described by flows of quantities such as energy density, mass, momentum density. However, in general and special relativity, the these quantities are not well defined and depend on the observer. In other words, different observers perceive the the same fluid being in different thermodynamic state. Hence, a description of the fluid dynamics in relativity requires a new formulation, a formulation in which a fluid is not represented by a scalar and vector fields, that are observer-dependent, but implicitly by a "flow" in spacetime. These are \textit{flux-conservative formulations} of hydrodynamics.

Consider the classical mass density, a scalar $\rho$, usually defined as total umber of particles $N$ of rest-mass $m$ in the volume $V$. Then, the total mass is given by

\begin{equation}
    \int_V \rho \text{d}^3x = m\int_V n \text{d}^3 x = mN.
\end{equation}

However, while the number of particles $N$ would be the same regardless of the observer, the $\text{d}^3x$ would be measured differently by observers moving in relation to each other. Hence, the $n$ would differ. One of the solutions is to chose a frame of reference, say comoving with the fluid and define the $\rho$ there. However, this would hinder our ability to generalize to other reference frames.\\ A better soution is to construct a \textit{covariant description in terms of invariant quantities}. 
 
We start by defining the flow of the fluid density in space-time, the 3 pseudo-form $\boldsymbol{\rho}$ that on any three dimensional submanifold describes the flow of mass transverse to the submanifold as

\begin{equation}
    \int_{\Sigma} \boldsymbol{\rho},
\end{equation}

where $\Sigma$ be a spacelike hypersurface,  $\vec{n}$ -- the future-oriented normal vector. This is the density measured by an observer with 4-velocity $\vec{n}$. 

To define a mass flow measured by an Eulerian observer across any spacelike surface $\Omega\subset\Sigma$, we need to construct a two-form $\boldsymbol{\rho}(\vec{n}, \cdot, \cdot)$ given by the interior product between the 3 pseudo-form $\boldsymbol{\rho}$ and $\vec{n}$. Then the mass flow is 

\begin{equation}
\int_{\Omega} i_{\vec{n}}\boldsymbol{\rho}.
\end{equation}

The conservation of the number of particles of the fluid is expressed by the vanishing exterior product of the density form, i.e. $\text{d}\boldsymbol{\rho}=0$, or in an integral form 

\begin{equation}
\int_{\partial\Omega} \boldsymbol{\rho} = \int_{\Omega}\text{d}\boldsymbol{\rho} = 0,
\end{equation}

that reads as the following: the net flow across any sufficiently regular surface $\partial\Omega$ enclosing a four-dimensional open set $\Omega\subset\mathcal{M}$ is zero.

Next we define a flux. First, let us reintroduce the volume pseudo-form

\begin{equation}
\text{Vol}_x ^4 = \sqrt{-g}dx^0 \wedge dx^1 \wedge dx^2 \wedge dx^3,
\end{equation}

where $g$ is the determinant of the spacetime metric. \\
On a on the submanifold $\Sigma$, the intrinsic volume then would be defined as 

\begin{equation}
\text{Vol}_x ^3 = i_{\vec{n}} \text{Vol}_x ^4.
\end{equation}

A flux of a vector field can be described by a three-form, for which on a pseudo-Riemannian manifold there exist a vector field associated with it.

A vector field associated with density is called \textit{rest-mass density four-vector} and is denoted by $\vec{j}$.

It is constructed from the one-form by rasing indexes, $\vec{j} = {^{\#}\underline{j}}$. The one-form $\underline{j}$ is obtained as $\underline{j}\star\boldsymbol{\rho}$, where $\star$ is the Hodge dual operator (see \textit{e.g.,} \cite{Frankel:1982dva}). 

Then if the $\boldsymbol{\rho} = i_{\vec{j}}\text{Vol}_x ^4$ the flux of $\vec{j}$ can be shown as 

\begin{equation}
\int_{\Sigma} \boldsymbol{\rho} = - \int_{\Sigma}\vec{j}\cdot\vec{n}\text{Vol}_x ^3,
\end{equation}

where $\vec{n}$ is the future-oriented unit-timelike normal to $\Sigma$.


\textcolor{gray}{
[Direct copy... maybe not needed] More generally the flux associated with a flow defined by a vector field, $\vec{X}$, across a hypersurface, $\Sigma$, transverse to it and with normal $\vec{\nu}$ (with appropriate sign depending on the signature of the metric and on $\Sigma$), is given 
\begin{equation}
\int_{\Sigma} \star\underline{X} = \int_{\Sigma}i_{\vec{X}}\text{Vold}^n = \int_{\sigma}i_{\vec{X}}\big[\underline{\nu}\wedge\text{Vol}^{n-1}\big] = \int_{\Sigma}\vec{X}\cdot\vec{\nu}\text{Vol}^{n-1}
\label{eq:theory:flux_of_flow}
\end{equation}
}
\textcolor{red}{this piece is used in Liuille theorem though}

We note that $\vec{j}$ is time like (or null). It is given by the the flux of particles across any future-oriented spacelike hypersurface is positive (or zero). If $\vec{j}$ is timelike, there exists a unique decomposition 

\begin{equation}
\vec{j} = \rho \vec{u},
\label{eq:theory:defofjandu}
\end{equation}
where the scalar $\rho$ can be seen as density in the comoving frame and unit-timelike vector $\vec{u}$ as a fluid four-velocity.\\

The divergence of vecotor $j$ then gives a familiar mass conservation expression

\begin{equation}
0 = \nabla_{\mu}j^{\mu} = \frac{1}{\sqrt{-g}}\partial_{\mu}[\sqrt{-g}\rho u^{\mu}].
\label{eq:theory:nablamu_jmu}
\end{equation}

\textcolor{gray}{Similarly energy and momentum of a fluid can be defined, using the Cartan formalism... but this is a PAIN! and is done to show that div(T)=0 is not really energy/momentum conservation...}

Next, let us introduce the mixed tensor $\boldsymbol{T}$. Since the three-forms are equivalent to vectors, we can define a flow of the $\nu$ momentum across the volume element orthogonal to $dx^{\mu}$ as 

\begin{equation}
{T^{\mu}}_{\nu}=\boldsymbol{T}(dx^{\mu},\partial_{\nu}).
\end{equation}

${T^{\mu}}_{\nu}$ is the stress energy tensor that was already introduced earlier \ref{eq:theory:action1}. 

Note, that if the Einstein equation are satisfied the Bianchi identities dictate that the $\nabla_{\mu}{T^{\mu}}_{\nu}$ must vanish as

\begin{equation}
\nabla_{\mu}{T^{\mu}}_{\nu} = 0= \frac{1}{\sqrt{-g}}\partial_{\mu}(\sqrt{-g}{T^{\mu}}_{\nu}) - {\Gamma^{\alpha}}_{\mu\nu}{T^{\mu}}_{\alpha}.
\label{eq:theory:nablamu_tmunu}
\end{equation}

However, this statement does not imply the conservation of the energy and momentum of the fluid in a general sense. The conservation of the $\nu$-momentum requires $\vec{\partial}_{\nu}$ to be a Killing vector.


\todo{define somewhere an eulerian observer}

\subsubsection{Dynamics of a Relativistic Fluid}

In the previous subsection we have introduced the fluid kinematic, and defined the important quantities such as mass, energy and momentum and their "conservation" in \ref{eq:theory:nablamu_jmu} and \ref{eq:theory:nablamu_tmunu}.

In this thesis we consider only the \textit{perfect fluid}, meaning that in the co-moving frame, there is not heat conduction and there is no viscosity. \todo{actually we do have a viscous part -- you have to add this...}. The former criterion implies that the fluid is in local thermodynamic equilibrium (LTE). The latter however requires more explanation. There is still no consensus on the correct mathematical formulation, especially with respect to the numerical applications, of the viscous and/or thermally conducting fluids in general-relativity (see e.g., \cite{Andersson:2006nr} and references therein). \textcolor{blue}{however in recent youers there have been some progress GRELS models and David's implementation I must add!}. \\

Consider a stress-energy tensor of a perfect fluid in the comoving frame with the fluid. To construct it, we return to the fluid's four velocity $\vec{u}$ from (\ref{eq:theory:defofjandu}). If $e_{i}$ is the basis vector, the scalar product $\vec{u}\cdot\vec{e}_i=0$ and $\vec{e_i}\cdot\vec{e}_k = \delta_{ik}$. then the orthonormal tetrad $\{\vec{u},\vec{e}\}$ is comoving with the fluid, and the $\{\underline{u},\underline{e}^i\}$ is the dual basis. \\
Tensor $\boldsymbol{T}$ is the stress-energy tensor with the following components: 

\begin{itemize}
    \item $\boldsymbol{T}(\underline{u}, \vec{u})$ energy-density in the rest-frame of the fluid, the scalar $e$
    \item $\boldsymbol{T}(\underline{u}, \vec{e}_i) = 0$ represent the energy flowing transverse to the four-velocity, which we set to $0$ in the absence of the heat-conduction.
    \item $\boldsymbol{T}(\underline{e}^i, \vec{e}_k) = 0$ represent the $k$ component of the force exchanged across the surface element orthogonal to $\underline{e}_i$.
\end{itemize}

Taking into account that the $\boldsymbol{T}$ must be invariant with respect to the rotations of the $\{\vec{e}_i\}$ and that the viscosity is not included, force exchange can be effectively desibed by a scalar $p$, that we call pressure as

\begin{equation}
    \boldsymbol{T}(\underline{e}^i,\vec{e}_k) = p {\delta^i}_k,
\end{equation}

Combining the aforementioned description of the components of $\boldsymbol{T}$ we get

\begin{equation}
\boldsymbol{T} = (e + p)\vec{u}\otimes \underline{u} + p\boldsymbol{\delta}.
\end{equation}

Defining the enthalpy of the fluid as $h = 1 + \epsilon = p/\rho$, where $\epsilon$ is the specific internal energy, we rewrite $\boldsymbol{T}$ as 

\begin{equation}
\boldsymbol{T} = \rho h \vec{u}\otimes\underline{u} + p\boldsymbol{\delta}
\label{eq:theory:stressenergytensor}
\end{equation}

In addition to the fluid kinematics (eqs. \ref{eq:theory:nablamu_jmu} and \ref{eq:theory:nablamu_tmunu}) and the description of motion (eq. \ref{eq:theory:stressenergytensor}), the relation between the pressure, internal energy and density is needed to fully describe the dynamics of the fluid. This relation is usually called the equation of state.

The commonly adopted equations (EoS) of state are the the ideal-gas, or gamma-law EoS $\rho = (\Gamma-1)\rho\epsilon$, where $\Gamma$ is the polytropic index of the gas, the polytropic EoS $p = K\rho^{\Gamma}$ and the microphysical equation of state \todo{that you need to discuss more..., as we use only it.}

Combined with an EoS, equations \ref{eq:theory:adm}, \ref{eq:theory:nablamu_jmu}, \ref{eq:theory:nablamu_tmunu} and \ref{eq:theory:stressenergytensor} form a hyperbolic
system of equations that can be evolved, once initial data is prescribed. The complete evolution of spacetime and the dynamics of the matter requires initial data to be set on the Cauchy surface.

\subsubsection{Conservative Formulations}

In the pioneering works of May and White \cite{May:1966} and Wilson \cite{Wilson:1972} the equations of general relativistic hydrodynamics were solved using the finite-difference (FD) schemes after casting them a from of non-linear advection-like equations. To avoid excessive oscillations at shocks a combination of upwinding and artificial-viscosity methods was employed. This however led to severl limitations, such as difficulty with tunning the artificial viscosity to still allow shocks to develop, and the limit on a flows being only mildly relativistic \cite{Font:2008fka}. \\
A next big advancement in the numerical relativistic hydrodynamics was made after the non-conservative nature of the Wilson’s approach was pointed out \cite{Marti:1991wi} and the conservation formulation was developed. 

An important example of the conservation formulation that is adopted to $3 + 1$ formalism is the "Valencia formulation" \cite{Banyuls:1997} that can be represented as following

\begin{equation}
    \frac{\partial\boldsymbol{F}^{0}(\boldsymbol{u})}{\partial t} + \frac{\partial\boldsymbol{F}^{i}(\boldsymbol{u})}{\partial x^{i}} = \boldsymbol{S}(\boldsymbol{u})
    \label{eq:theory:valencia_formalism}
\end{equation}

where $u$ is a “vector” of \textit{primitive quantities}, such as the rest-mass density or the specific internal energy, $\boldsymbol{F}^0$ is a “vector” of \textit{conserved quantities} and $\boldsymbol{F}^i$ and $\boldsymbol{S}$ are their fluxes and sources respectively. \\

This formulation allowed to study ultra-relativistic flows and resolve shocks without spurious oscillations and without need for artificial viscosity.

It was shown to be especially well suited for use with numerical methods that take into account the conservation laws. These are the finite-volume (FV) FD high-resolution shock capturing (HRSC) methods, that will be discussed in Chapter \ref{chapter:num_methods} \\

Many recent advancements in numerical relativistic hydrodynamics and magnetohydrodynamics (MHD) have relied on these methods (\textit{e.g.,} \cite{Giacomazzo:2010bx} [274]\cite{Rezzolla:2011da} and references therein \todo{add recolla/bernuzzi/radice/shibata}).

There are other conservative formulations and methods (see \textit{e.g.,} \cite{Papadopoulos:1999kt}). However, we will limit our focus to the "Valencia formulation". \\

To begin we split the four-velocity $\vec{u}$ into the component parallel to the normal vector $\vec{n}$ and a purely spatial component as

\begin{equation}
    \vec{u} = (-\vec{u} \cdot \vec{n})(\vec{n} + \vec{\upsilon}),
\end{equation}

where naturally the Lorentz factor, measured by theEulerian observer $W = (-\vec{u}\cdot\vec{n})$ emerges, and the $\upsilon$ is the fluid three-velocity measured by the Eulerian observer, 

\begin{equation}
    \vec{\upsilon} = \frac{\vec{u}}{W} -\vec{n},
\end{equation}

components of which are

\begin{equation}
    \upsilon^i = \frac{u^i}{W}+ \frac{\beta^i}{\alpha}, \hspace{10mm} \upsilon_i= \frac{u_{i}}{W}.
\end{equation}

Divergence of the rest-mass density four-vector $j$, (\ref{eq:theory:nablamu_jmu}) can easily be cast as 

\begin{eqnarray}
    0 = \nabla_{\mu}j^{\mu} = \frac{1}{\sqrt{-g}}\partial_{t}[\sqrt{\gamma}\rho W] + \frac{1}{\sqrt{-g}}\partial_{i}[\sqrt{\gamma}\rho(\alpha \upsilon^{i} - \beta^{i})]
\end{eqnarray}

where $D=\rho W = -\vec{j}\cdot \vec{n}$ is the conserved density.

To write the energy and momentum equations we note that for any vector field $\vec{p} $ \cite{Rezzolla:2013}, 

\begin{equation}
    \nabla_{\mu}[{T^{\mu}}_{\nu}p^{\nu}].
\end{equation}

To obtain the Valencia formulation we set $\vec{p}$ to have zeroth component $-\vec{n}$ and spatial components $\vec{\partial}_i$. Then the

\begin{itemize}
    \item ${T^0}_{\nu}p^{\nu}$ represent the conserved quantities,
    \item ${T^i}_{\nu}p^{\nu}$ are associated fluxes,
    \item ${T^{\mu}}_{\nu}p^{\nu}$ are sources
\end{itemize}

with the former being 

\begin{equation}
    S_{i} = \alpha {T^0}_{\nu}(\partial_i)^{\nu}=-\boldsymbol{T}(\vec{n},\vec{\partial}_i), \hspace{10mm} E = -\alpha{T^0}_{\nu}n^{\nu} = \boldsymbol{T}(\vec{n},\vec{n})
\end{equation}

for numerical reasons we will replace the total internal energy density $E$ with $\tau = E-D$, where $D$ is the rest mass density. This is done to avid errors emerging due to $E$ being much smaller then $D$. 

Now we can combine the obtained expressions for the conserved quantities, associated fluxes and sources with eq. (\ref{eq:theory:valencia_formalism}) and obtain

\begin{equation}
    \frac{1}{\sqrt{-g}}\Big[\frac{\partial\sqrt{\gamma}\boldsymbol{F}^{0}(\boldsymbol{u})}{\partial t} + \frac{\partial\sqrt{-g}\boldsymbol{F}^{i}(\boldsymbol{u})}{\partial x^i}\Big] = \boldsymbol{S}(\boldsymbol{u}),
\end{equation}

where primitive quantities being

\begin{equation}
    \boldsymbol{u} = [\rho,\: \upsilon_i,\: \epsilon],
\end{equation}

conserved quantities: 

\begin{equation}
    \boldsymbol{F}^0(\boldsymbol{u}) = [D,\: S_j,\: \tau] = [\rho W,\: \rho h W^2 \upsilon_j,\: \rho h W^2 - p - \rho W],
\end{equation}

associated fluxes

\begin{equation}
    \boldsymbol{F}^i(\boldsymbol{u})=\Bigg[D\Big(\upsilon^{i}-\frac{\beta^i}{\alpha}\Big),\: S_{j}\Big(\upsilon^{i}-\frac{\beta^i}{\alpha}\Big)+p{\delta^i}_j ,\: \tau\Big(\upsilon^{i}-\frac{\beta^i}{\alpha}+p\upsilon^i\Big)\Bigg]
\end{equation}

and sources 

\begin{equation}
    \boldsymbol{S}(\boldsymbol{u}) = \Bigg[0,\: T^{\mu\nu}\Big(\frac{\partial g_{\nu j}}{\partial x^{\mu}} - \Gamma^{\delta}_{\nu\mu}g_{\delta j}\Big),\: \alpha\Big(T^{\mu 0}\frac{\partial\log\alpha}{\partial x^{\mu}}-T^{\mu\nu}\Gamma^{0}_{\nu\mu}\Big)\Bigg]^T
\end{equation}

The from of the obtained general relativistic hydrodynamics equations resemble the one of the Newtonian gas dynamics. If the latter is adopted for numerical solutions. \\
There are however several complications. In particular there is no explicit inverse relation between the primitive quantities and the conserved ones. Thus one has to resort to the root-finding algorithms to reconstruct them (More on this in later chapters). In addition, it was pointed out that the $W$ cpuples the equation for the momenta in different direction \cite{Pons:2000,Rezzolla:2002ra,Rezzolla:2002cc,Aloy:2006rd}. This leads to the fact that the dynamics of the shock wave can be affected by the non-zero tangential velocity. Hence, the increased complexity if he problem of GR hydrodynamics \cite{Mignone:2005ns,Zhang:2005qy}.


\subsection{The General-Relativistic Boltzmann Equation}

In special relativity the Boltzmann equation was expressed by Synge \cite{Synge:1957}. Later Chernikov \cite{Chernikov:1962} and Tauber and Weinberg \cite{Tauber:1961} proposed its extension to the general relativity. \\
The list of applications of the Boltzmann equation was limited to the relativistic gas at first \cite{Israel:1963}. Later the list was supplemented by transient relativistic thermodynamics \cite{Israel:1979wp}, radiative transfer \cite{Lindquist:1966}, core-collapse supernovae \cite{Bruenn:1985} and others (see \textit{e.g.}, \cite{Cercignani:2002} and references therein). \\

Different formulations of the general relativistic Boltzmann equation exists in the literature. Lindquist \cite{Lindquist:1966} and Ehlers \cite{Ehlers:1971} proposed a geometrical interpretation. Later, a formulation based on Riemannian structure of tangent bundles was proposed by Sasaki \cite{Sasaki:1958,Sasaki:1962}. In addition, Debbasch and van Leuuwen \cite{Debbasch:2009a,Debbasch:2009b} recently provided a detailed derivation, albeit strongly focused on the algebraic aspects while eluding simple geometrical interpretation. \\
Here we recall the detailed derivation of the general relativistic Boltzmann equation, using modern differential geometry notation by Radice. 

\textcolor{red}{This.Is.Tough. Pure math. Copied from David + his sources.}

\subsubsection{The geometry of the tangent bundle}

Let the $\mathcal{M}$ be $4$ dimensional differential manifold such that $(\mathcal{M},\: g_{\alpha\beta})$ form the $C^2$ spacetime. The set of tangent vectors of $\mathcal{M}$ constitutes \textit{tangent bundle} of $\mathcal{M}$, the we denote as $T\mathcal{M}$. The set of all unit vectors of $\mathcal{M}$ constitute the \textit{subbundle} of $T\mathcal{M}$. \\
\textcolor{gray}{incompressible vector field}\\
\textit{Every Killing vector field of $\mathcal{M}$ is in incompressible vector field}

\paragraph{Extended transformation and extended tensors}

Let the $T\mathcal{M}$ be the set of all the tangent vectors of $\mathcal{M}$. The $T\mathcal{M}$ has a natural topology, bundle structure with $\mathcal{M}$ and the base - linear vector space $E^i$. We call $T\mathcal{M}$ the \textit{tangent bundle} of $\mathcal{M}$. Natural projection, or a projection map $\pi:\: T\mathcal{M}\rightarrow\mathcal{M}$.  \\

Let $U$ be a coordinate neighborhood, or a coordinate patch of $\mathcal{M}$ with $n$ variables $x^{\alpha}$ as coordinates. Then, every tangent vector of $\mathcal{M}$ at a point $p\in U$ with $2n$ variables $(x^i,\upsilon^{\alpha})$. Here $x^{\alpha}$ are coordinates of $p$ with respect to the coordinate patch ${x^{\alpha}}$ and $\upsilon^{\alpha}$ are components of a tangent vector in the natural frame that constitutes by the vectors $\partial/\partial x^4$ at $q$. Thus, the vector $\vec{p}$ at $q$ can be written as:

\begin{equation}
    \vec{p} = p^{\alpha}\frac{\partial}{\partial^{\alpha}}
\end{equation}

and its dual as 

\begin{equation}
    \underline{p} = p_{\alpha}dx^{\alpha}:=g_{\alpha\beta}p^{\beta}dx^{\alpha}
\end{equation}

In addition we introduce a coordiante patch $TU$, $\{z^A\}$, where $A$ runs from $0$ to $7$ of $T\mathcal{M}$ as 

\begin{equation}
    z^{\alpha} = z^{\alpha}, \hspace{10mm} z^{\alpha+4} = p^{\alpha}.
\end{equation}

Now, let the $U(x^{\alpha})$ and $\hat{U}(\hat{x}^{\alpha})$ be the two coordinate patches of $\mathcal{M}$ such that $U\cap\hat{U}$ is not empty. Then the intersection of the coordinate patches is also not empty. 
for every coordinate transformation of $\mathcal{M}$, there is a corresponding matrix $\frac{\partial \hat{x}^{\alpha}}{\partial x^{\beta}}$.
The coordinate transformation is then

\begin{equation}
    \hat{x}^{\mu} = \hat{x}^{\mu}(x), \hspace{5mm} \hat{p}^{\mu} = \frac{\partial\hat{x}^{\mu}}{\partial x^{\nu}}p^{\nu}
\end{equation}

which denotes the extended transforation of the $\hat{x}^{\mu} = \hat{x}^{\mu}(x)$. \\


The corresponding Jacobian matrix is 
\renewcommand\arraystretch{1.6} %% it stretches the matrix
\begin{equation}
\frac{\partial\hat{z}^A}{\partial z^B} = 
    \begin{pmatrix}
    \frac{\partial\hat{x}^{\alpha}}{\partial x^{\beta}} & 0 \\
    \frac{\partial^2\hat{x}^{\alpha}}{\partial x^{\beta} \partial x^{\gamma}}p^{\gamma} & \frac{\partial\hat{x}^{\alpha}}{\partial x^{\beta}} 
    \end{pmatrix}
\end{equation}
\renewcommand\arraystretch{1.0}




\paragraph{Vectors on $T\mathcal{M}$}

As we will need to introduce connections on a tangent bundle, here we discuss the double tangent bundle, ot a second tangent bundle. Since $T\mathcal{M}$ is a vector bundle on its own right, its tangent bundle has the secondary vector bundle structure $TT\mathcal{M}$. Let point $b\in TU$ and $T_b T\mathcal{M}$ be the tangent space to $T\mathcal{M}$ at $b$. \\
Given a vector $\partial/\partial x^{\alpha}$ at a point $b$, it can be "pushed forward" to the point on the $TT\mathcal{M}$ by means of so called \textit{differential of} $\pi$, whitten as $\pi_*$ \cite{Frankel:2002}.
On a natural basis the push-forward acts as 

\begin{equation}
    \pi_*\Big[\frac{\partial}{\partial x^{\alpha}}\Big] = \frac{\partial}{\partial x^{\alpha}}, \hspace{5mm} \pi_* \Big[\frac{\partial}{\partial p^{\alpha}}\Big] = 0,
\end{equation}

and the pull back 

\begin{equation}
    \pi^* {\text d} x^{\alpha} = {\text d} x^{\alpha}.
\end{equation}

Consider a vector field $\vec{X} \ in TT\mathcal{M}$  in a vicinity of the point $b$, which is associated with the point $q$ of $\mathcal{M}$ and vector $\vec{x}\in T_{q}\mathcal{M}$. Let $b{\lambda}$ be the flow of $b$ generated by $\vec{X}$. The $b(\lambda)$ is associate with $q(\lambda)$, the one parameter family of points of $\mathcal{M}$. The $b(\lambda)$ is also associated with $\vec{x}(\lambda)$ the one parameter family of vectors on $T\mathcal{M}$.  \\

The vector field $\vec{X}$ is called \textit{vertical} if the $q(\lambda)\in\mathcal{M}$ are constant along the flow. Similarly, the vector field $\vec{X}$ is called \textit{horizontal} if $\vec{x}(\lambda)\in T_p \mathcal{M}$ is "constant" along the flow, meaning that $\vec{x}(\lambda)$ is just $\vec{x}$ that is \textit{parallel transported} to $q(\lambda)$. \\
As there is no unique way to perform a parallel transport, the linear connection $\nabla$ on $\mathcal{M}$ has to be chosen. This choice is akin choosing two vector spaces $\mathcal{O}_b$ and $\mathcal{V}_b$ of the horizontal and vectical vectors respectively at each point $b$ that the direct sum of these spaces yields

\begin{equation}
    \mathcal{O}_b\oplus \mathcal{V}_p = T_b T\mathcal{M}.
\end{equation}

Having the connection allows to prescribe a manner of lifting curves from the base manifold $T\mathcal{M}$ into the $T_b T\mathcal{M}$ \textcolor{red}{I need to fix this and understand}. A lift is the unique horizontal vector $\vec{X}\in T_bT\mathcal{M}$ whose projection is a vector $\vec{x}\in T_q\mathcal{M}$.\\ 
\textcolor{red}{fill it}
Let us now define a \textit{connection vector basis} adopted to the aforementioned split of $T_b T\mathcal{M}$ $\{\text{D}/\partial x^A \}:=\{\text{D}/\partial x^{\alpha}, \partial/\partial p^{\alpha} \}$ where 

\textcolor{red}{I did not find where this is derived from... difficult}

\begin{equation}
    \frac{\text{D}}{\partial x^{\alpha}}{\partial x^{\alpha}} := \frac{\partial}{\partial x^{\alpha}} - {\Gamma^{\beta}}_{\alpha\gamma}p^{\gamma}\frac{\partial}{\partial p^{\beta}}.
\end{equation}

Similarly a connection can be build for differential forms. Using the pull-back $\pi^*$ the dual basis $\{ \text{D}z^{A} \}:=\{\text{d}x^{\alpha}, \text{D}p^{\alpha}\}$ that satisfies 

\begin{equation}
    \text{D} = \text{d} p ^{\alpha} + {Gamma^{\alpha}}_{\beta\gamma}p^{\gamma}\text{d}x^{\beta}.
\end{equation}

\paragraph{Metric on $T\mathcal{M}$}

Note that 

\begin{equation}
    \frac{\partial^2 \hat{x}^{\mu}}{\partial x^{\nu}\partial x^{\lambda}}p^{\lambda} = {\hat{\Gamma}^{\mu}}_{\delta\gamma}p^{\lambda}\frac{\partial\hat{x}^{\delta}}{\partial x^{\nu}}.
\end{equation}

Let us assume that for any point $b\in T\mathcal{M}$ there exist an open set $TU$, such that $b\in TU$ with a coordinate system on $TU$ that satisfies

\begin{equation}
    G_{AB} = (\boldsymbol{\eta}\otimes\boldsymbol{\eta})_{AB},
\end{equation}

where $\boldsymbol{\eta} = \text{diag}(-1, 1, 1, 1)$. \\
Let the $\hat{x}^A$ denote the generic coordinate system on $TU$. Then the metric in this coordinate system can be expressed as

\begin{align}
    \hat{G}_{\mu\nu} &= \frac{\partial \hat{x}^{\alpha}}{\partial x^{\mu}}\frac{\partial \hat{x}^{\beta}}{\partial x^{\nu}}\eta_{\alpha\beta} + \frac{\partial \hat{x}^{\alpha}}{\partial x^{\mu}}{\hat{\Gamma}^{\gamma}}_{\:\:\:\alpha\lambda}p^{\lambda}\frac{\partial \hat{x}^{\beta}}{\partial x^{\nu}}{\hat{\Gamma}^{\delta}}_{\:\:\:\beta\xi}p^{\xi}\eta_{\gamma\delta}; \\
    \hat{G}_{\mu\: \nu+4} &= \frac{\partial \hat{x}^{\alpha}}{\partial x^{\mu}}\frac{\partial \hat{x}^{\gamma}}{\partial x^{\nu}}{\hat{\Gamma}^{\beta}}_{\:\:\:\gamma\lambda}p^{\lambda}\eta_{\alpha\beta}; \\
    \hat{G}_{\mu+4 \: \nu+4} &= \frac{\partial \hat{x}^{\alpha}}{\partial x^{\mu}}\frac{\partial \hat{x}^{\beta}}{\partial x^{\nu}} \eta_{\alpha\beta}
\end{align}

and the line element 

\begin{align}
    dS^2 &= \hat{G}_{AB}d\hat{z}^A d\hat{z}^B = \hat{g}_{\mu\nu}\text{d}\hat{x}^{\mu}\text{d}\hat{x}^{\nu} + \hat{g}_{\mu\nu}[\text{d}p^{\mu} + {\hat{\Gamma}^{\mu}}_{\:\:\:\alpha\beta}p^{\beta}\text{d}x^{\alpha}] [\text{d}p^{\nu} + {\hat{\Gamma}^{\nu}}_{\:\:\:\alpha\beta}p^{\beta}\text{d}x^{\alpha}] \\
    &= \hat{g}_{\mu\nu}\text{d}\hat{x}^{\mu}\text{d}\hat{x}^{\nu} + \hat{g}_{\mu\nu}\text{D}\hat{x}^{\mu}\text{D}\hat{x}^{\nu}
\end{align}

It is possible to show that the determinant $|\text{det}\boldsymbol{G}| = g^{2}$ as the transformation from the natural frame to the connection frame is unimodular \cite{Lindquist:1966}. Thus the volume pseudo-form on $T\mathcal{M}$ is in the coordiante patch $TU$

\begin{align}
    \text{Vol}^8 &:= -g \text{d}x^{0} \wedge \text{d}x^{1} \wedge ... \wedge \text{d}p^{3} := - g\text{d}^{4}x \text{d}^{4}p, \\
    &:= -g \text{d}x^{0} \wedge \text{d}x^{0} \wedge ... \wedge \text{D}p^{3} :=-g \text{d}^{4}x\text{D}^4 p
\end{align}

\textcolor{red}{I kinda gave up here and just copied.}


\paragraph{the Liuville theorem}

Let us start by introducing a \textit{cotangent bundle}. Let $\mathcal{M}$ be a differentiable manifold. Similarly to the construction of the tangent bundle, we can make a set of covectors on a given manifold into a vector bundle over $\mathcal{M}$, denoted $T^*\mathcal{M}$ and called \textit{cotangent bundle} of $\mathcal{M}$.  Similarly we can define a contangent bundle of a tangent one $T^*T\mathcal{M}$. The contangent bundle $T^*\mathcal{M}$ is the vector bundle dual to the tangent bundle $T\mathcal{M}$. 

Let us start by defining \textit{Poincar\'e} 1-form on $T\mathcal{M}$, $\underline{\lambda}\in T^* T\mathcal{M}$. Consider point $q$ on a manifold $\mathcal{M}$ and a point $A$ associated with $q$ on a tangent bundle $T\mathcal{M}$. Let there be a 1-form $\underline{\alpha}\in T^* _q\mathcal{M}$. The $\underline{\lambda}$ and $\underline{\alpha}$ are uniquely connected $\underline{\lambda} = \pi^* \alpha$, and the former is called the \textit{Poincar\'e} 1-form. In local coordinate patch, $TU$ it is expressed as

\begin{equation}
    \underline{\lambda} = p_{\alpha} \text{d}x^{\alpha}.
\end{equation}

the associated vector is 

\begin{equation}
    \vec{\lambda} = p^{\alpha} \frac{\text{D}}{\partial x^{\alpha}} = p^{\alpha}\frac{\partial}{\partial x^{\alpha}} - p^{\alpha}{\Gamma^{\beta}}_{\alpha\gamma}p^{\gamma}\frac{\partial}{\partial p^{\beta}},
\end{equation}

is called the $\textit{geodesic flow field}$. \\
This flow represents a phase-space flow of particles moving along geodesics. \\

Consider a mass shell, that at a point $q\in U$can be defined as a set:\\
\textcolor{red}{remider: I have no idea how is this possible...}

\begin{equation}
    \mathcal{S}_m = \big\{ p^{\alpha}\in T_q\mathcal{M}: p_{\mu}p^{\mu}+m^2 =:f(p) = 0 \big\}.
\end{equation}

The normal to the mass-shell is 

\begin{align}
    \text{if } m &\neq 0 \hspace{5mm} \underline{\pi}:=\frac{q}{2m}\text{d}f, \hspace{5mm} \text{d}f = \frac{\partial f}{\partial x^{\mu}} + \frac{\partial f}{\partial p^{\mu}}\text{d}p^{\mu} = 2p_{\mu}\text{d}p^{\mu}, \\
    \text{if } m &= 0 \hspace{5mm} \underline{\pi}:=\frac{1}{2}\text{d}f
\end{align}

Next, we introduce a unique form $\underline{\nu}$ whose restriction on $T_q\mathcal{M}$ is equal to $\underline{\pi}$. 

\begin{align}
    \text{if } m &\neq 0 \hspace{5mm} \underline{\nu} = \frac{1}{m}p_{\alpha}\text{D}p^{\alpha} \\
    \text{if } m &= 0 \hspace{5mm} \underline{\nu} = p_{\alpha}\text{D}p^{\alpha}
\end{align} 

Note that $\underline{\nu} = 0$, meaning that the $\underline{\nu}$ is irrotational. It becomes clear if we re-express it as 

\begin{equation}
    \underline{\nu} = \frac{1}{2m}\frac{\text{D}f}{\partial p^{\alpha}}\text{D}p^{\alpha}
\end{equation} 

for massive particle case. For the mass-less the procedure is analogous. \\

It can be shown that $\underline{\lambda}$ is ncompressible \textit{i.e.,} $\text{d}^{\star}\underline{\lambda} =\star \text{d}\star\underline{\lambda} =0$. \\

In addition, both $\underline{\lambda}$ and $\underline{\nu}$ are harmonic forms as 

\begin{equation}
    \nabla\underline{\nu} = 0 , \hspace{5mm}
    \nabla\underline{\lambda} = [\text{dd}^{\star} + \text{d}^{\star}\text{d}]\underline{\lambda} = 0.
\end{equation}

Let us now consider the density of states in the phase space, of particles moving along tgeodeiscs with velocities on the mass shell. \\ 
In the previous section we introduced a flux of the vector field $\vec{X}$ across $\Sigma$ in \ref{eq:theory:flux_of_flow}, we define the following six-form

\begin{align}
    \boldsymbol{\omega} &= \star\big(\underline{\nu}\wedge\underline{\lambda}\big) = i_{\vec{\lambda}} i_{\vec{\nu}}\text{Vol}^8 \\
    &= i_{\vec{\lambda}}\Big[i_{\vec{\nu}}\big(\text{Vol}^{4}_{x}\wedge\text{Vol}^{4}_{p}\big)\Big] = i_{\vec{\lambda}} \big[\text{Vol}^{4}_{x}\wedge\text{Vol}^{3}_{p}\big],
\end{align}

where we used the definition of $\text{Vol}^8$. \\
The introduced four forms read,

\begin{align}
    \text{Vol}_x ^4 &:= \sqrt{-g} \text{d}x^{0} \wedge \text{d}x^{1} \wedge \text{d}x^{2} \wedge \text{d}x^{3}, \\
    \text{Vol}_p ^4 &:= \sqrt{-g} \text{D}p^{0} \wedge \text{D}p^{1} \wedge \text{D}p^{2} \wedge \text{D}p^{3}, \\
    \text{Vol}_p ^3 &:= i_{\vec{\nu}}\text{Vol}_p ^4,
\end{align}

where the four-forms are on the $TU$ and the latter three-form is on the mass shell $S_m$. \\

Consider coordinates adopted to the mass-shell, where $\underline{\nu} = (p_0/m)\text{D}p^0$ and $\underline{\nu} = p_0\text{D}p^0$ in the massive nad massless cases respectively, the three-form becomes

\begin{equation}
    \text{Vol}^3 _p =\frac{\sqrt{-g}}{-p_0}\text{D}p^1\wedge\text{D}p^2\wedge\text{D}p^3
\end{equation}

Now we have a three-form $\text{Vol}^3 _p$ and a four-form $\text{Vol}_x ^4$. In the context of the ADM foliation, we split spacetime manifold as $\mathcal{M}=\mathcal{R}\times\Sigma$, with $x^0 = \text{const}$ being constant hypersurfaces with normal $\underline{n} = - \alpha\text{d}x^0$ and $\alpha$ -- the lapse function. We can now simplify the $\boldsymbol{\omega}$, splitting $\text{Vol}_x ^3$ as 

\begin{align}
    \text{Vol}_x ^4 &= -\underline{n}\wedge\text{Vol}_x ^3 \hspace{5mm} \text{where,} \\
    \text{Vol}_x ^3 &= i_{\vec{n}}\text{Vol}_x ^4 = \sqrt{\gamma}\text{d}x^1\wedge\text{d}x^2\wedge\text{d}x^3
\end{align}

and the $\boldsymbol{\gamma}$ is the three-metric induced on the slices.  \\
The resulted coordinates, adapted to the mass shell and the spacetime foliation read

\begin{align}
    \boldsymbol{\omega} &=-(\vec{p}\cdot\vec{n})\frac{1}{-p_0}\sqrt{\gamma}\sqrt{-g}\text{d}x^1\wedge\text{d}x^2\wedge\text{d}x^3 \wedge\text{D}p^2\wedge\text{D}p^2\wedge\text{D}p^3 \\
    &= \frac{p^0}{-p_0}|g|\text{d}x^1 \wedge\text{d}x^2\wedge\text{d}x^3 \wedge\text{D}p^2\wedge\text{D}p^2\wedge\text{D}p^3
\end{align}

Now, consider a six-vector, $\delta_i x \delta_i p$ with $i\in\{1,2,3\}$. The $\delta_i x$ are tangent vectors to the slice $\Sigma$ and the $\delta_i p$ are tangent to mass shell $S_m$. \\
The action of $\boldsymbol{\omega}$ on the six-vectors $\delta_1 x$, $\delta_2 x$, $\delta_3 x$, $\delta_1 p$, $\delta_2 p$, $\delta_3 p$ yields

\begin{align}
    \boldsymbol{\omega}(\delta_1 x,...,\delta_3 p) =& \frac{p^0}{-p_0}|g|\big[\text{d}x^1\wedge\text{d}x^2\wedge\text{d}x^3\big](\delta_{1}x,\delta_{2}x,\delta_{3}x)\times \\
    & \hspace{10mm} \Big[\text{D}p^1\wedge\text{D}p^2\wedge\text{D}p^3\Big](\delta_1 p, \delta_2 p, \delta_3 p) \\
    & \hspace{2mm} -\frac{p^0}{-p_0}|g|\big[\text{d}x^1\wedge\text{d}x^2\wedge\text{d}x^3\big](\delta_{1}p,\delta_{2}p,\delta_{3}p)\times \\
    & \hspace{10mm} \Big[\text{D}p^1\wedge\text{D}p^2\wedge\text{D}p^3\Big](\delta_1 x, \delta_2 x, \delta_3 x) = \\
    =& \frac{p^0}{-p_0}|g|\big[\text{d}x^1\wedge\text{d}x^2\wedge\text{d}x^3\big](\delta_{1}x,\delta_{2}x,\delta_{3}x)\times \\
    & \hspace{10mm} \Big[\text{D}p^1\wedge\text{D}p^2\wedge\text{D}p^3\Big](\delta_1 p, \delta_2 p, \delta_3 p) = \\
    =& \frac{p^0}{-p_0}|g|\big[\text{d}x^1\wedge\text{d}x^2\wedge\text{d}x^3\big](\delta_{1}x,\delta_{2}x,\delta_{3}x)\times \\
    & \hspace{10mm} \Big[\text{d}p^1\wedge\text{d}p^2\wedge\text{d}p^3\Big](\delta_1 p, \delta_2 p, \delta_3 p), \\
\end{align}

where we used that $\text{d}x^i(\delta_j p)=0$ and the relation

\begin{equation}
    \text{D}p^{i}(\delta_j p) = \text{d}p^{i}(\delta_j p) - {\Gamma^i}_{\alpha\beta}p^{\alpha}\text{d}x^{\beta}(\delta_j p) = \text{d}p^i(\delta_j p)
\end{equation}

Thus, on the space-like hypersurface $\Sigma$ and on the mass shell we have

\begin{equation}
    \boldsymbol{\omega} = \frac{p^0}{-p_0}|g|\text{d}x^1\wedge\text{d}x^2\wedge\text{d}x^3\wedge\text{d}p^1\wedge\text{d}p^2\wedge\text{d}p^3 =: \boldsymbol{\Omega}
\end{equation}

The $\boldsymbol{Omega}$ can be split as 

\begin{align}
    \boldsymbol{\Omega} &= \boldsymbol{\Lambda} \wedge \boldsymbol{\Pi}, \hspace{5mm} \text{where} \\
    \boldsymbol{\Lambda} &= p^0 \sqrt{-g}\text{d}x^1\wedge\text{d}x^2\wedge\text{d}x^3 \\
    \boldsymbol{\Pi} &=  \frac{1}{-p_0}\text{d}p^1\wedge\text{d}p^2\wedge\text{d}p^3
\end{align}

The defined forms $\boldsymbol{\Lambda}$ and $\boldsymbol{\Pi}$ can be written in a coordinate-independent way at any point $q\in\mathcal{M}$ as 

\begin{equation}
    \boldsymbol{\Lambda} = \star_{\mathcal{M}}\underline{\lambda}, \hspace{5mm} \boldsymbol{\Pi} = \star_{T_q\mathcal{M}}\underline{\pi}
\end{equation}

and this are intrinsic forms in $T\mathcal{M}$. In addition, the $\boldsymbol{\Lambda}$ and $\boldsymbol{\Pi}$ are the proper geodesics flux
volume form on $\Sigma\in\mathcal{M}$ and mass shell $S_m\in T_q\mathcal{M}$ at a point $q\in U$ respectively. \\

Let us now consider the geodesic flow $\vec{\lambda}$. It generates a "tube" in a phase space, that we limit with $S_1$ and $S_2$ sections. Then the flux of points in phase space associated with geodesic flow is $\int_{S}\boldsymbol{\omega}$. It is possible to show that the flux satisfies

\begin{equation}
    \int_{S_1}\boldsymbol{\omega} = \int_{S_2}\boldsymbol{\omega}
    \label{eq:theory:liuville}
\end{equation}

which is the \textit{Liouville’s Theorem} in the relativistic case. \\

To see taht this is indeed the case, consider the exterior differential of $\boldsymbol{\omega}$

\begin{equation}
    \star\text{d}\omega = \text{d}^{\star}(\underline{\nu}\wedge\underline{\lambda}) = d^{\star}\underline{\nu}\wedge\underline{\lambda} + \underline{\nu}\wedge\text{d}^{\star}\underline{\lambda}.
\end{equation}

The $\text{d}^{\star}=\text{const}=k$ as $\text{dd}^{\star}\underline{\nu}=0$. In addition, the $\text{d}^{\star}\underline{\lambda}=0$. This allow us to write 

\begin{equation}
    \text{d}\omega = -k(\star\lambda).
\end{equation}

We note the $\star\underline{\lambda}$ is the volume form of the hypersurfaces orthogonal to $\vec{\lambda}$. Hence, the $\star\underline{\lambda}[\vec{\lambda},...]=0$ along the "tube" in phase space. 

\begin{equation}
    \int_S\text{d}\boldsymbol{\omega} = 0.
\end{equation}

the \ref{eq:theory:liuville} is recovered, if we use the Stoke’s Theorem, and using the fact that the $\boldsymbol{\omega}$ vanishes along the part of the boundary tangent to $\vec{\lambda}$.

\textcolor{red}{Note that I still have no Idea what I have written. I need to go through the original materal, which I could not find... at least I could not find what I could read and understand. }

\subsubsection{The Boltzmann equation}

Let us introduce the phase space version of the mass flux, the 6-form representing the number of phase-space trajectories crossing $S$ of the phase tube between $S_1$ and $S_2$ cross sections. \\
In the absence of collisions we obtain 

\begin{equation}
    \int_{S_1}\boldsymbol{\mu} = \int_{S_2}\boldsymbol{\mu}.
\end{equation}

Remembering that $\boldsymbol{\omega}$ represents the density of states in phase space of particles moving along geodesics, we obtain 

\begin{equation}
    \boldsymbol{\mu} = F\boldsymbol{\omega},
\end{equation}

where $F$ is \textit{invariant distribution function}, \textcolor{gray}{i.e. F is the Radon-Nikodym derivative of $\boldsymbol{\mu}$ with re $\boldsymbol{\omega}$}. \\

Consider now that collisions change the number of phase trajectories as 

\begin{equation}
    \delta N = \int_{S_2} \boldsymbol{\mu} - \int_{S_1}\boldsymbol{\mu} = \int_S \text{d}\boldsymbol{\mu} = \int_S \text{d}F\wedge\boldsymbol{\omega}.
\end{equation}

where 

\begin{equation}
    \text{d}F\wedge\boldsymbol{\omega} = \text{d}F\wedge\star (\underline{\nu}\wedge\underline{\lambda}) = \langle\text{d}F,\underline{\lambda}\rangle\star\underline{\nu} - \langle\text{d}F,\underline{\nu}\rangle\star\underline{\lambda},
\end{equation}

where $\langle\cdot,\cdot\rangle$ is a scalar product between forms and can be written as

\begin{equation}
    \langle\boldsymbol{\alpha},\boldsymbol{\beta}\rangle\text{Vol}^8 := \boldsymbol{\alpha}\wedge\star\boldsymbol{\beta},
\end{equation}

and with the $\star\underline{\lambda}=0$ on $S$ we obtain 

\begin{equation}
    \delta N = \int_S\langle\text{d}F\underline{\lambda}\rangle\star\underline{\nu} = \int_S\mathcal{C}[F]\star\underline{\nu},
\end{equation}

where we denoted the effect of collisions as

\begin{equation}
    \langle\text{d}F\underline{\lambda}\rangle = \mathcal{C}[F].
\end{equation}

This is the Boltzmann equation. In component from it reads as 

\begin{equation}
    p^{\alpha}\frac{\partial F}{\partial x^{\alpha}} - {\Game^{\gamma}}_{\alpha\beta}p^{\alpha}p^{\alpha}\frac{\partial F}{\partial p^{\gamma}} =\mathcal{C}[F].
\end{equation}

In the coordinate system adapted to the equation, when $P^0 = p^0(p^i)$, the equation reads \cite{Cercignani:2002}:

\begin{equation}
p^{\alpha}\frac{\partial F}{\partial x^{\alpha}} - {\Game^{i}}_{\alpha\beta}p^{\alpha}p^{\alpha}\frac{\partial F}{\partial p^{i}} =\mathcal{C}[F].
\end{equation}

Remembering that that the $\underline{\lambda}$ is incompressible, we can write

\begin{equation}
    \langle\text{d}F,\underline{\lambda}\rangle = \text{d}^{\star}[F,\underline{\lambda}],
\end{equation}

This allows to obtain a conservative formulation of the Boltzmann equation, that indicates the conservation of the number of particles \cite{Cardall:2002bp}

\begin{equation}
    \text{d}^{\star}[F\underline{\lambda}] = \mathcal{C}[F].
\end{equation}

Next, we re-introduce the Levi Civita connection $\nabla$ in phase space. For incompressible $\underline{\lambda}$ it gives

\begin{equation}
    \nabla_A\lambda^A=0,
\end{equation}

while the Boltzmann equation becomes 

\begin{equation}
    \lambda^A\partial_A F=\mathcal{C}[F]
\end{equation}

and its conservative form 

\begin{equation}
    \nabla_A[Fp^{A}] = \mathcal{C}[F],
    \label{eq:theory:liouvilletheorem}
\end{equation}

or in component form

\begin{equation}
    \frac{1}{|g|}\frac{\partial}{\partial x^{\mu}}\Bigg[|g|Fp^{\mu}\Bigg] + \frac{p_0}{|g|}\frac{\partial}{\partial p^{k}}\Bigg[\frac{|g|}{-p_0}{\Gamma^k}_{\alpha\beta}p^{\alpha}p^{\beta}F\Bigg] = \mathcal{C}[F].
\end{equation}

\subsubsection{From the Boltzmann Equation to the Euler Equation}

In order to derive from the Boltzmann equation the equations of hydrodynamics, we need to first define the needed varaibles of the kinetic description, such as mass and energy fluxes. We not that the density flux can easly be obtained from $\boldsymbol{\mu}$. We write the mass flow then as 

\begin{equation}
    \boldsymbol{\rho} = \int_{S_m} \boldsymbol{\mu}.
\end{equation}

Recalling the definition of the rest-mass denisty four-vector $J$, we note that

\begin{equation}
    \int_{\Sigma}(-J^{\mu}n_{\mu})\text{Vol}_x ^3 = \int_{\Sigma\times S_{m}} Fp^0\boldsymbol{\Pi}\text{Vol}_x ^3 = \int_{\Sigma\times S_{m}} F\boldsymbol{\Omega} = \int_{\Sigma}\boldsymbol{\rho}
\end{equation}

Thus, the $J$ can be written as 
\textcolor{red}{this is not clear how it was obtain. Must go through again.}

\begin{equation}
J^{\mu} = \int_{S_m}Fp^{\mu}\boldsymbol{\Pi}
\end{equation}

The second moment of the distribution function $F$ in a similar way gives the stress energy tensor 

\begin{equation}
    T^{\mu\nu} = \int_{S_m} F p^{\mu}p^{\nu}\boldsymbol{\Pi}
\end{equation}

the components of which in the frame comoving with the fluid are

\begin{equation}
    T^{\mu\nu} = 
    \begin{pmatrix}
    E & \vec{F} \\
    \vec{F} & \boldsymbol{P} \\
    \end{pmatrix}
\end{equation}

where $E$ and $\vec{F}$ are the energy density and flux respectively, and $\boldsymbol{P}$ is the stress tensor. 

The nature of the collisional operation ultimately defines the equilibrium configuration distribution function $F$ \cite{Cercignani:2002}, and thus the form of the stress-energy tenor. 

Now we use the Liouville Theorem to obtain the equations of hydrodynamics. 
To accomplish that we insert the $\boldsymbol{\Psi}$, a tensorial funcition of $\pagebreak^i$, into the theorem, eqiation \ref{eq:theory:liouvilletheorem} on both sides and integrate with respect to $\boldsymbol{\Pi}$ as

\begin{equation}
    \int_{\text{I\!R}}\nabla_{A}[F\lambda^A\boldsymbol{\Psi}]\boldsymbol{\Pi}=\int_{\text{I\!R}}\mathbb{C}[F]\boldsymbol{\Psi}\boldsymbol{\Pi}
\end{equation}

where we used that $\lambda^A\nabla_{A}\boldsymbol{\Psi}=0$ as $\vec{\lambda}$ is the geodesic flow. \\

Letting the $F$ decay for large momenta we obtain the transfer equation \cite{Israel:1963,Cercignani:2002}:

\begin{equation}
    \nabla_{\mu}\int_{\text{I\!R}} F\boldsymbol{\Psi}p^{\mu}\boldsymbol{\Pi} =\int_{\text{I\!R}} \mathcal{C}[F]\boldsymbol{\Psi}\boldsymbol{\Pi},
    \label{eq:theory:transferequation}
\end{equation}

\textcolor{red}{check the sources. White intermediate steps}

In a particular case of a simple gas $\Psi$ can be shown to be one of the $\{1,p^0,p^1,p^2,p^3\}$ \cite{Cercignani:2002}. Then the right hand side of the transfer equation becomes 

\begin{equation}
    \int_{\text{I\!R}} \mathcal{C}[F]\boldsymbol{\Psi}\boldsymbol{\Pi} = 0.
\end{equation}

These $\boldsymbol{\Psi}$ are related to the quantities conserved by the collisional operator and are called \textit{collisional invariants}. Choice of $1$ would yield the mass conservation, while $p^{\mu}$ -- the energy and momentum conservation. \\

Now, having cancelled the R.H.S of the eq. \ref{eq:theory:transferequation}, we obtain the conservation laws in a following form

\begin{equation}
    \nabla_{\mu}J^{\mu} =0 \hspace{10mm}\nabla_{\nu}T^{\mu\nu} =0
\end{equation}

\subsection{Overview}

We start this section by revisiting the fundamental concepts, such as manifold, tangent and cotangent bundles, with vectors and differential forms defined on them, and operations such ans exterior, Wedge product and Hodge star operator. \\

Then we set ourselves a goal to obtain the general form of general relativistic hydrodynamics. This includes the equations for space-time evolution adopted adopted for use in numerical applications and Euler equations for the fluid, which we aim to obtain through the Liuville's theorem and Boltzmann equations. \\

To derive the Einstein field equations we perform the variation of the so-called Hilbert action, applying the Euler-Lagrange equation. Thus we start by briefly deriving the Euler-Lagrange equations, using the fact that the fields we are interested in are defined over only a compact domain and that the choice of the variation of coordinates is arbitrary, i.e. $\partial S(\boldsymbol{q}, \nabla \boldsymbol{q}) = 0$. Then, in a similar way, the variation of the Hilbert action, yields the Einstein Field equation. \\

For practical applications it is useful to express the EFE as a initial value boundary problem. The hamiltonian formalism allows to do that, which we briefly review. We then sketch the $3+1$ decomposition procedure, introducing the spacelike foliation and extrinsic curvature that allow us to obtain the constraint equations, that has to be satisfied on every hyper-surface of the hypersurface. Then, emplying the EFE and Gauss-Codacci equations we write the Hamiltonian density, whose variation with respcet to the variables of folliation $\alpha$ and $\beta$ yileds constraint equation. The evolution equations then are obtained through the variation of the Hamiltinan with respect to the three-metric and momentum. \\

The obtained ADM system is however not well suited for numerical applications, being only weekly hyperbolic. We thus briefly touch on a strongly hyperbolic formulation, the Z4 formulation, that exhibit such usefull for numercs properties as constraint violation dumpening (constrain preservation) and its evolution, the CCZ4 formulation that is furhter adopted for BH evolution. \\

After, we briefly touch on the gauge conditions, as in the 3+1 we are left with the freedom on how to do the foliation, namely, chosing the lapse function and shift vector. \\

Then we proceed with deriving equations of general relativistc hydrodynamics, aiming to provide a flux-conservative formulation. We first define the kinematics of the relativistc fluid, \textit{i.e.,} a covariant description in terms of invariant quantities. With this goal in mind we define the rest-mass density vector, whose divergence give the number of particles conservation. Then we re-introduce the stress energy tensor ans show that via Bianki identities, its divergence alse vanishes. \\

To discuss the dynmaics of the fluid, we first, set its type. We consider the fluid that hs and thermal conductivity and no viscosity, \textit{i.e.}, the perfect fluid. In addition to fluid kinematics and the stress-energy tensor, describing tis motion, we discuss an equation of state. Together they from a hyperbolic ssytem of equations that describes the evolution of the fluid in space-time, once the initial data is set.\\

For the reasosns of numerical stability, a special formulation of the equations of general relativistic hydrodynamics is required. Such is the Valencia formulation. The main idea is to constract an advection-like equation for fluxes of conserved quantites, from which the promitive quantities can be reconstructed.\\ 

To derive these fluxes we first decompose the four velocity into the component parallel to the normal to the hypersurface and a purely spatial part. This leads us to the defention of the conserved density. Then we introduce a vector whose zeroth component is just a norm and spatial component which is a \textcolor{gray}{tangent vector to hypersurface}. This allows us to write the conserved and primitive qiantities of the formulatio, as well as the sourve term. \todo{not all quntities in Val.Form. are clear. What is $S_j$ and $\epsilon$} \\

Next we consider geometrical approach to the general-relativistic boltzmann equation \textcolor{gray}{still not sure why though}. To do that we first introduce necessary tools, namely vectors and tensors that are needed to define the phase-space. There there are $2n$ components with the first $n$ being coordinates and the second $n$ being impulses. In addition, we introduce the coordinate trnaformation and finally, the metric on the tangent bundle. \\

Having tools set, we derive the Liuville theorem. In order to do that we write phase-space flow of particles moving along geodesics which is represented by the Poincare 1-form and associated vecotr. In addition we define a mass shell, a norm to it and a irrotational form on a tangent bundle. Together with the poincare form it allows us to define the denisty of the phase-space trjectories which we denote ad a Hodge operator of the wedge product of these two forms. After some calculations, we obtain that this form can be expressed in a coordiante independed way as a split of two froms, the proper geodeiscs flux froms on the hypersurface and the mass shell respectively. By considering the "phase tube" with two crossections, we arrive that itegrated flux is conserved, which constitutes the Liouville\'s Theorem. \\

After that we proceed with deriving the Boltzmann equation. Which we start by intoducing the number of phase-space trajectories crossing the section of a "phase tube". The relation between the number of the phase space trajectoies and the density defined above, yilds the invariant distribution function. Considering the change in the number of particles due to collisions. The change in scalar product between the exteriour derivative of the distribution function and poincare 1-form due to collisions constitudes the Boltzmann equation. \textcolor{gray}{revise this.}. Re-introducing the Levi Civita connection in phase space, and taking an advantage of the incompressibility if the poincare 1-form, we obtain a conservative form of the Boltzmann equation. \\

From Boltzmann equation we can now obtain an equations of hydrodynamics. For that we firs redefine the variables of the kinetic description, such as mass and energy fluxes, recalling the definition of the rest-mass density four-vector. Similarly how the vector can be now exprressed as a first moment of the distribution function, the second moment gives the sress energy tensor. However, we note that the nature of the collisional operation ultimately denes the equilibrium conguration distribution function and thus the form of the stress-energy tenor. \\

Next we use the Liouville Theorem to obtain the equations of hydrodynamics introducing a tensorial function of the momenta, that is related to the quantities conserved by the collisional operator and are called collisional invariants into the integral form onf the theorem. Letting the distribution function decay for large momenta we obtain the transfer equation. For as simple gas this can be reduced to already familiar equations where divergence of the rest mass vector J and stress energy tensor is zero. 


\chapter{Numerical Approximation of Conservation Laws}

In this chapter we aim to briefly remark on the theory of the numerical approximation of conservational laws. Owing to its key importance in physics, there is quite an extensive amount of literature concerning the topic. We therefore select the aspects that are of relevance to this work, namely high-order, state-of-the-art numerical methods for the solution of conservation laws. For the sake of compactness we would refrain from stating complete descritions of these schemes, focusing on key ideas behind them, their advantages and drawbacks in the context of general-relativistic hydrodynamics. \\

This chapter is structured as following. In Section \ref{sec:theory:conserv_laws:theorback} we state the basics behind the theory of conservation laws and their numerical approximation. In Section \textcolor{red}{[??]} we give a brief overview of the Godunov-like finite-volume schemes. Next, in Section \textcolor{red}{[??]} we focus on the high-resolution shock-capturing (HRSC) finite-difference schemes. \textcolor{gray}{Finally, in Section [??????] we present discontinuous Galerkin methods.}

\textcolor{red}{note that here section is capital 'S'}

\section{Theoretical Background}
\label{sec:theory:conserv_laws:theorback}

In this section we briefly recall the basics behind the mathematical theory of conservational laws and subsequently, their numerical application. We start by defining the weak and entropic solutions, and present certain results regarding existence and uniqueness of these solutions for conservation laws. We then proceed with reviewing numerical approximation to conservation laws, as well as concepts of consistency, stability and convergence, \textcolor{gray}{briefly stating the Lax-Richtmeyer theorem}. Then we conclude with reviewing the extension to the case of non-linear equations. This section is based on the descriptions provided in \cite{LeVeque:1992,Tadmor1998}, and we refer the reader to these sources for more in-depth discussion. \\

\subsection{Conservation Laws}

Let us consider the consercation laws in the following form

\begin{align}
    \partial_t\boldsymbol{u} + \nabla\cdot\boldsymbol{f}(\boldsymbol{u}) = 0, \hspace{10mm} &(t,x)\in \text{I\!R}_{+}\times\text{I\!R}^d , \\
    \boldsymbol{u}(0, x) = \boldsymbol{u}(x), \hspace{18mm} &x\in \text{ I\!R},
    \label{eq:theory:conservlaws}
\end{align}

where $\boldsymbol{u}$ is the vector of $m$ unknowns, $\boldsymbol{f}=(\boldsymbol{\boldsymbol{f}^1,...,\boldsymbol{f}^m})$ is a $d$-dimensional flux and $\boldsymbol{u_0}\in\big[L^{\infty}(\text{I\!R}^d)\big]^m$ is the initial data. \\

Investigations of the system \ref{eq:theory:conservlaws} showed irrespective of the initial data, the solution can develop discontinuities (shocks) in a finite time. Thus, the system should be viewed in distribution formalism. There, if for all test functions $\upsilon\in C_0 ^1 (\text{I\!R}^{d+1})$ and $i=1,2,...,m$  we obtain

\begin{equation}
    \int_{0}^{\infty}\text{d}t\int_{\text{I\!R}^d}\big[u^i \partial_t\upsilon + \boldsymbol{f}^i(\boldsymbol{u})\cdot\nabla\upsilon\big]\text{d}x = \int_{\text{I\!R}^d} u_0 ^i \upsilon \text{d}x,
\end{equation}

then, the vector $\boldsymbol{u}\in\big[\text{I\!R}_{+}\times\text{I\!R}^d\big]^m$ is a \textit{weak solution} of \ref{eq:theory:conservlaws}.\\

It can be shown that that multiple weak solutions are allowed even for a scalar conservation law. Let us then consider the concept of the entropic solution. A convex function \textcolor{red}{what is it?}, $\eta(\boldsymbol{u})$, is said to be an entropy function if its Hesian \textcolor{red}{what is it???}, $\nabla_{\boldsymbol{u}}^2\eta$, symmetrizes the spatial Jacobian, $\nabla_{\boldsymbol{u}}f^i$ \textcolor{red}{WHAT IS IT?!},

\begin{equation}
    \nabla_{\boldsymbol{u}}^2\eta\cdot\nabla_{\boldsymbol{u}}\boldsymbol{f}^i = [\nabla_{\boldsymbol{u}}\boldsymbol{f}^i]^{T}\cdot\nabla_{\boldsymbol{u}}^2\eta, \hspace{10mm} i = 1, ... , m.
\end{equation}

Here we infer the compatibility relation, introducing the entropy flux $\boldsymbol{\psi} = (\boldsymbol{\psi}^1,...,\boldsymbol{\psi}^m)$, as 

\begin{equation}
    [\nabla_{\boldsymbol{u}}\eta]^T\cdot\nabla_{\boldsymbol{u}}\boldsymbol{f}^i = [\nabla_{\boldsymbol{u}}\boldsymbol{\psi}^i]^T, \hspace{15mm} i = 1,...,m.
\end{equation}

The pair $\eta\boldsymbol{\psi}$ is referred to as \textit{entropy pair}. \\

Then, the \textit{entropic solution} is a weak solution that for any $(\eta\boldsymbol{\psi})$, in the sense of   admits

\begin{equation}
    \partial_t\eta(\boldsymbol{u}) + \nabla\cdot\boldsymbol{\psi}(\boldsymbol{u})\leq 0.
\end{equation}

Concerning distributions, the \textit{entropic solution} requires that for any positive test function $\upsilon\in C_0 ^1 (\text{I\!R}_+\times\text{I\!R}^d)$ that 

\begin{equation}
    \int_{\text{I\!R}_+\times\text{I\!R}^d} \big[\eta(\boldsymbol{u})\partial_t\upsilon + \boldsymbol{\psi}(\boldsymbol{u}) \cdot \nabla\upsilon \big]\text{d}t\text{d}x = 0
\end{equation}

It is possible to show \cite{LeVeque:1992}, that this condition in a scalar cast is equivalent to making characteristic line impinged into shock waves. This constitutes the the process that resulted in a shock forming is irreversible and that the time-symmetry is not longer applies. \\

Considering the scalar case, where $m=1$, it is possible to prove the existence and uniqueness of the entropic solution under very general conditions \cite{Kruzkov:1970}. It s is also can be extended to measure-valued solutions \cite{DiPerna:1985} and to the case of conservation laws on manifolds \cite{Benartzi:2007}. \\

On the other hand, uniqueness and stability of entropic solutions is not well understood in case of systems of conservation laws, as not even the existence of $(\eta\boldsymbol{\psi})$ for the general system of equation has been proven. In \cite{Chen:2009} a novel approach has been employed, based on divergence-measure vector fields. For one dimensional
Riemann problem, where the initial data in a form 


\begin{equation}
    \boldsymbol{u}_0(x) = 
    \begin{dcases}
        \boldsymbol{u}^L, \hspace{5mm} \text{if } x<0; \\
        \boldsymbol{u}^R, \hspace{5mm} \text{if } x>0.
    \end{dcases}
\end{equation}

it allowed to prove the existence, uniqueness and stability of the entropic solution of the Euler equations for a classical ideal-gas \cite{Chen:2003}. \\

However the mere exestiance of the weak solution to the Riemann problem for general equation of state is not guaranteed \cite{Curtis:1972}. (for more recent results regarding classical Euler equations, see the review \cite{Chen:2006}). \\

On the other hand, using Glimm’s method \cite{Glimm:1965} in the existence of solutions to the Riemann problem was shown in the relativistic case but for the ultrarelativistic equation of state \cite{Smoller:1993}. \\

For \textit{strictly hyperbolic systems} \textit{i.e.,} when $\nabla_{\boldsymbol{u}}\boldsymbol{f}$ has a complete set of real eigenvalues and eigenvectors, the existence of weak solutions was proven in case when the initial data having small enough initial jump \cite{Lax:1957}.

\subsection{Consistency, Stability and Convergence}

Now we consider how the conservation laws can be treated numerically. We limit the discussion to the $m=1$ case, as the non-linear theory is well established only for the scalar fields. Thus we define a problem as 

\begin{align}
    \partial_t u + \nabla\cdot\boldmath(u) = 0&, \hspace{10mm} (t,x) \in I\!R_{+}\times I\! R^d \\ 
    u(0, x) = u_o(x)&, \hspace{15mm} x\in I\!R^{d},
\end{align}

where $u$ is now just a scalar function. \\

Let us introduce the following notation. The form of equations \ref{eq:theory:conservlaws} allows us to view the solution to this system in a form of a "curve" in an infinite dimensional vector sapce $L^{\infty}I\!(R^d)$, or as a sequence of bounded functions $u(t,\cdot)\in L^{\infty}(I\! R^d)$, that we can consider only being functions of time $u(t)$ with values in the vector space. Owing to the curve being bounded in $L^{\infty}(I\!R^d)$, the sequence of bounded functions read $u(t)\in L^{\infty}[I\!R_{+};L^{\infty}(I\!R^d)]$. Then the system  \ref{eq:theory:conservlaws} can be seen as a system of ordinary differential equations (ODEs), which can be represented as 

\begin{equation}
    \frac{\text{d}u(t)}{\text{d}t} = \mathcal{L}[u(t)], \hspace{10mm} u(0) = u_0,
    \label{eq:theory:conservlawsode}
\end{equation}

where we associate the operator $\mathcal{L}(\cdot)$ with the $-\nabla\cdot\boldsymbol{f}(\cdot)$. It is important to remember, that as the $u(t)$ is not a smooth function of time, the equation \ref{eq:theory:conservlawsode} should be considered from a point of view of distributions. In addition, we note that stating $u(0) = u_0$ is in all mathematical rigor is a restriction of general functions $L^{\infty}[I\!R_{+};L^{\infty}(I\!R^d)]$ to the set of zero Lebesgue measure, which is not define. For the purpose of keeping the discssion brief we refer to the \cite{Kruzkov:1970} for the related discussion. For the numerical applications we further restrict $u(t)$ to be a smooth function of time, and leave the mathematical subtleties out of discussion.  \\

Transitioning from general formulation \ref{eq:theory:conservlaws} to one adopted for a scalar function $u(t)$ \ref{eq:theory:conservlawsode}, we can introduce the numerical approximation, that is depended in a discrimination parameter $\Delta$. This approximation reads

\begin{equation}
    \frac{\text{d}u^{\Delta}(t)}{\text{d}t} = L^{\Delta}[u^{\Delta}(t)], \hspace{10mm} u^{\Delta}(0) = P^{\Delta}[u_0],
\end{equation}

where $u^{\Delta}$ and $L^{\Delta}$ are approximations of $u$ and $\mathcal{L}$, \textit{i.e.,} $u^{\delta}\approxeq u u$, $L^{\Delta}\approxeq \mathcal{L}$ and $P^{\Delta}$ is a projection operator. However, as the error associated with it is negable in comparison with other errors arizing in discritisation of conservation laws, we will ignore it, effectively assinging that $u^{\Delta}(0) = u_0$. One of such errors is the trucaction error. \\
The \textit{ local truncation error} can be defined as 

\begin{equation}
    r^{\Delta} = L^{\Delta}[u(t)] - \mathcal{L}[u(t)],
\end{equation}

where $u(t)$ is the exact solution to \ref{eq:theory:conservlaws}. We call a numerical scheme \textit{consistent} if the $r^{\Delta}\rightarrow 0$ when $\Delta\rightarrow 0$ in a given norm for all possible initial data $u_0$. Note, however, that the choice of norm is problem- and method- dependent and may limit the allowable initial data. \\

We then call a scheme to be of order $r$ if 

\begin{equation}
    || r^{\Delta}(t) || = \mathcal{O}(\Delta^r).
\end{equation}

\begin{sidenote}
    \textbf{Infimum and Supremum} \\
    Im math, the \textit{infimum (inf)} of a subset $S$ of a partially ordered set $T$ is the greatest element of $T$ that is less than or equal to all elements of $S$. \\
    The \textit{supremum (sup)} of a subset $S$ of a partially ordered set $T$ is the least element in $T$ that is greater than or equatl to all elements of $S$. The upper bound of a subset $S$ of a partially ordered set (P,$\leq$) is an element $b$ of $P$ such that $b\geq x$ for all $x$ in $S$. If a supremum of a subset $S$ exhists it is unique. The supremum of a subset $S$ of partially ordered set $P$ does not necessearly belongs to $S$. But if it does, it is the maximum, or the greatest element of $S$. \\
    Examples of suprema.
    The supremum of a set of real numbers of $\{1,2,3\}$ is $3$. The number $4$ is an upper bound but it is not the least upper bound and hence not the supremum. \\
    \begin{align}
        \sup\{x\in\text{I\!R}|0<x<1\} = 1. \\
        \sup\{(-1)^n - 1/n | n = 1,2,3,... \} = 1. \\
        \sup\{x\in\mathbb{Q} | x^2 < 2\} = \sqrt{2}
    \end{align}
\end{sidenote}

A scheme is considered to be \textit{stable} if the norm $L^{\Delta}$ is limited 

\begin{equation}
    |||L^{\Delta}||| := \sup \frac{||L^{\Delta}||}{||\upsilon||}\leq C,
\end{equation}

where $C\geq 0$ is a constant independent of $\upsilon$. \\
A scheme is considered to be \textit{convergent} if 

\begin{equation}
    \lim_{\Delta\rightarrow 0} ||u^{\Delta}(t)-u(t)|| = 0, \hspace{10mm} \text{a.e. } t\in \text{I\!R}_{+}.
\end{equation}

The relation between the consistency, stability and convergence is given by the Lax-Richtmeyer equivalence theorem \cite{Lax:1956}. It states that the numerical approximation of well-posed problems is convergent if and only if the scheme is stable and consistent. In addition, consider a scheme of the order $r$, then

\begin{equation}
    || u^{\Delta}(t) - u(t) || = \mathcal{O}(\Delta^r).
\end{equation}

However, in the non-linear case, the \textit{non-linear stability} is required in addition to the stability and consistency are to assure convergence. 

\subsection{Non-Linear Equations and Non-Linear Stability}











%[15]\cite{Aloy:2006rd}
%[36]\cite{Banyuls:1997}
%[40]\cite{Benartzi:2007}
%[67]\cite{Bruenn:1985}
%[77]\cite{Cardall:2002bp}
%[79]\cite{Cercignani:2002}
%[81]\cite{Chen:2006}
%[82]\cite{Chen:2003}
%[83]\cite{Chen:2009}
%[84]\cite{Chernikov:1962}
%[108]\cite{Debbasch:2009a}
%[109]\cite{Debbasch:2009b}
%[114]\cite{DiPerna:1985}
%[124]\cite{Ehlers:1971}
%[130]\cite{Font:2008fka}
%[146]\cite{Giacomazzo:2010bx}
%[147]\cite{Glimm:1965}
%[173]\cite{Israel:1979wp}
%[174]\cite{Israel:1963}
%[188]\cite{Kruzkov:1970}
%[192]\cite{Lax:1957}
%[193]\cite{Lax:1956}
%[196]\cite{LeVeque:1992}
%[201]\cite{Lindquist:1966}
%[210]\cite{Marti:1991wi}
%[222]\cite{Curtis:1972}
%[224]\cite{Mignone:2005ns}
%[248]\cite{Papadopoulos:1999kt}
%[257]\cite{Pons:2000}
%[272]\cite{Rezzolla:2002ra} 
%[273]\cite{Rezzolla:2002cc}
%[274]\cite{Rezzolla:2011da}
%[275]\cite{Rezzolla:2013}
%[284]\cite{Sasaki:1958}
%[285]\cite{Sasaki:1962}
%[304]\cite{Smoller:1993}
%[311]\cite{Synge:1957}
%[312]\cite{Tadmor1998}
%[315]\cite{Tauber:1961}
%[321]\cite{Thorne:1981}
%[339]\cite{Zhang:2005qy}








\chapter{Numerical methods}
\label{chapter:num_methods}

%% --------------- 
%%
%% References
%%
%% ---------------

\newpage

\bibliography{references}

\end{document}