\documentclass[11pt,a4paper,headinclude=true,DIV=14,BCOR=8mm,chapterprefix,listof=totoc,twoside,openright,abstracton]{scrbook}

\usepackage[headsepline]{scrpage2}
\usepackage[utf8]{inputenc}
\usepackage{geometry}
\usepackage{amssymb}
\usepackage{amsthm}
\usepackage{enumerate}
\usepackage{graphicx}
\usepackage{float}
\usepackage[intlimits]{amsmath}
% \usepackage{siunitx}
\usepackage{color}
\usepackage{verbatim}
\usepackage{appendix}
\usepackage{hyperref}
\usepackage{hyperref}
% \usepackage[style=authoryear]{biblatex}
\usepackage{natbib}
% \usepackage{newtxtext}
% \usepackage{newtxmath}
% \usepackage{harvard}
\setcitestyle{aysep={}} 
\bibliographystyle{apalike}
\usepackage{xr}
\usepackage{wrapfig}
% \bibliographystyle{agsm}
%\usepackage{feynmf}

\setlength{\parindent}{0pt}
\geometry{a4paper, tmargin=3cm, bmargin=3cm, lmargin=3cm, rmargin=3cm, headheight=3em, headsep=2em, footskip=1cm}

\setcitestyle{citesep={,}}

\newcommand{\todo}[1]{\textcolor{red}{$\blacksquare$ TODO: #1}} 

\geometry{a4paper, tmargin=2cm, bmargin=2cm, lmargin=1cm, rmargin=1cm, headheight=2em, headsep=2em, footskip=1cm}

\title{PhD thesis}
\author{Vsevolod Nedora}
\date{today}

\begin{document}
    
    \maketitle

%% --------------- 
%%
%% Theory
%%
%% ---------------

\chapter{General-Relativistic Hydrodynamics}

This chapter is meant to sketch several important parts of the mathematical background. We focus on the aspects relevant for the tools and methods employed in out discussion. We do not aim to provide a comprehensive overview. 
The chapter is divided into \todo{list the parts and their content}

\section{The Cauchy Problem in General Relativity}

In this section we briefly recall the initial-value formulation of the Einstein equations of general relativity through the following steps. We start by introducing notations and the basics of GR. We summarize the Einstein field equations. Then we continue with how EFE can be split in a set of evolutionary equations and constraints. For that we focus on the Arnowitt, Deser and Misner, or ADM, formalism. In the end we comment on the stability of the ADM equations, on the need for strongly-hyperbolic formulations of the EFE, and on the choice of gauge conditions commonly used to
evolve spacetimes with singularities. This overview is based in \cite{Arnowitt:1962hi,Landau:1982dva,Wald:1984,Misner:1973,Baumgarte:2002jm}, which we refer to for more detained discussion.

\subsection{Euler-Lagrange equations}

Let the $\alpha$ be the totally antisymmetric symbol that expresses through coordinates $x^{\mu}$ as
\begin{equation}
    \alpha = dx^0 \wedge dx^1 \wedge dx^2 \wedge dx^3,
\end{equation}
where $\wedge$ denotes exterior product. Then, proper volume pseudo-form of the spacetime is

\begin{equation}
    \varepsilon = \sqrt{-g}\alpha,
\end{equation}
where $g$ denotes the determinant of the spacetime metric. \\

In GR, the spacetime is represented by Lorentzian manifold $\mathcal{M}$ and $g$, the Loretzian metric. \\

The action principle of the Lagrangian field theory on the spacetime $(\mathcal{M}; \boldsymbol{g})$ is
\begin{equation}
    S(\boldsymbol{q}, \nabla\boldsymbol{q}) = \int_{\mathcal{M}}\alpha\mathcal{L}(\boldsymbol{q}, \nabla\boldsymbol{q}),
\end{equation}
where $\boldsymbol{q}$ are a set of generalized coordinates for the fields described by the theory, $\nabla$ is the Levi-Civita connection, $\mathcal{L}$ is a scalar density of a scalar quantity $\lambda$ as $\lambda(\boldsymbol{q},\nabla\boldsymbol{q})$. 

Varying the action with respect to the $\boldsymbol{q}$
\begin{equation}
    \delta S(\boldsymbol{q}, \nabla\boldsymbol{q}) = \delta\int\alpha\mathcal{L}(\boldsymbol{q}, \nabla\boldsymbol{q}) = \int\alpha\Big(\frac{\partial\mathcal{L}}{\partial\boldsymbol{q}}\delta\boldsymbol{q}+\frac{\partial\mathcal{L}}{\partial(\nabla\boldsymbol{q})}\delta\nabla\boldsymbol{q}\Big)
\end{equation}

As $\delta$ and $\nabla$ commute, and partially integrating $\nabla$, we obtain

\begin{equation}
    \partial S(\boldsymbol{q}, \nabla\boldsymbol{q}) = \int\alpha\Big(\frac{\mathcal{L}}{\partial\boldsymbol{q}}-\nabla\frac{\partial \mathcal{L}}{\partial(\nabla\boldsymbol{q})}\Big)\delta\boldsymbol{q} + \int_{\mathcal{M}}\alpha\nabla\Big(\frac{\partial\mathcal{L}}{\partial(\nabla\boldsymbol{q})}\delta\boldsymbol{q}\Big)
\end{equation}

The last term is a boundary term and in order to vanish we impose boundary condition. Assume that the fields are defined over only a compact domain. \\
As the choice of $\partial\boldsymbol{q}$ is arbitrary, the 

\begin{equation}
    \partial S(\boldsymbol{q}, \nabla\boldsymbol{q}) = 0
\end{equation}

and the Euler-Lagrange equations are

\begin{equation}
    \frac{\partial \mathcal{L}}{\partial\boldsymbol{q}} - \nabla\Big(\frac{\partial\mathcal{L}}{\partial(\nabla\boldsymbol{q})}\Big) = 0
\end{equation}

% ----------------------------------------------- 
\subsection{The Hilbert Action}

The Einstein–Hilbert action allows to obtain an Einstein field equations through ad principle of least action. Here we briefly underline the procedure.

Introduce action that describes the graviatational field, and a matter field $\mathcal{L}_m$:
\begin{align}
    S_g &= \int\frac{1}{2\kappa}R\epsilon, \\
    S_m &= \int\mathcal{L}_{m}\epsilon,
\end{align}
where $R$ is the Ricci scalar and $\kappa$ is the  Einstein's constant. \\

The full action then:
\begin{equation}
    S = \int\Big(\frac{1}{2\kappa}R+\mathcal{L}_m\Big)\epsilon
\end{equation}

The action principle dicatates, that $\delta S = 0$  with respect to the inverse metric $g^{\mu\nu}$. 

\begin{equation}
    \int\Bigg[\frac{1}{2\kappa}\Big(\frac{\delta R}{\delta g^{\mu\nu}}+\frac{R}{\sqrt{-g}}\frac{\delta\sqrt{-g}}{\delta g^{\mu\nu}}\Big) + \frac{1}{\sqrt{-g}}\frac{\delta(\sqrt{-g}\mathcal{L}_m)}{\delta g^{\mu\nu}}\Bigg]\delta g^{\mu\nu}\epsilon
\end{equation}

Owing to the arbitrariness of $\delta g^{\mu\nu}$, the integrant must be zero. 

\begin{equation}
    \frac{\delta R}{\delta g^{\mu\nu}} + \frac{R}{\sqrt{-g}}\frac{\delta\sqrt{-g}}{\delta g^{\mu\nu}} = -2\kappa\frac{1}{\sqrt{-g}}\frac{\delta(\sqrt{-g}\mathcal{L}_m)}{\delta g^{\mu\nu}} := \kappa T_{\mu\nu},
    \label{eq:theory:action1}
\end{equation}
where we introduced the stress-energy tensor $T_{\mu\nu}$.

The continuation of this deriviation requires taking variation of the Riccia scalar $R$ and the determinantof the metric $\sqrt{-g}$. As this is a length procedure, we provide here the result. 

\begin{equation}
    \frac{\delta R}{\delta g^{\mu\nu}} = R_{\mu\nu},
    \label{eq:theory:deltaR}
\end{equation}
where the $R_{\mu\nu}$ is the Ricci curvature tensor.

\begin{equation}
    \frac{1}{\sqrt{-g}}\frac{\delta\sqrt{-g}}{\delta g^{\mu\nu}} = -\frac{1}{2}g_{\mu\nu}.
    \label{eq:theory:deltagmuny}
\end{equation}

Substituting Eq. \ref{eq:theory:deltaR} and Eq. \ref{eq:theory:deltagmuny} into equation of motion Eq.  \ref{eq:theory:action1} we obtain the Einstein's field equation 

\begin{equation}
    R_{\mu\nu} -\frac{1}{2}g_{\mu\nu}R=8\pi T_{\mu\nu},
\end{equation}
where in the geometrized unit system, \textit{i.e} $c=G=1$, the $\kappa=8\pi$.

\subsection{3+1 Decomposition of Spacetime}

 
%% --------------- 
%%
%% References
%%
%% ---------------

\bibliography{references}

\end{document}