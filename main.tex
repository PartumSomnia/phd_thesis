\documentclass[11pt,a4paper,headinclude=true,DIV=14,BCOR=8mm,chapterprefix,listof=totoc,twoside,openright,abstracton]{scrbook}

\usepackage[headsepline]{scrpage2}
\usepackage[utf8]{inputenc}
\usepackage{geometry}
\usepackage{amssymb}
\usepackage{amsthm}
\usepackage{enumerate}
\usepackage{graphicx}
\usepackage{float}
\usepackage[intlimits]{amsmath}
% \usepackage{siunitx}
% \usepackage{color}
\usepackage{xcolor}
\usepackage{verbatim}
\usepackage{appendix}
\usepackage{hyperref}
\usepackage{hyperref}
% \usepackage[style=authoryear]{biblatex}
\usepackage{natbib}
% \usepackage{newtxtext}
% \usepackage{newtxmath}
% \usepackage{harvard}
\setcitestyle{aysep={}} 
\bibliographystyle{apalike}
\usepackage{xr}
\usepackage{wrapfig}
% \bibliographystyle{agsm}
%\usepackage{feynmf}
%\usepackage{tensor}

\setlength{\parindent}{0pt}
\geometry{a4paper, tmargin=3cm, bmargin=3cm, lmargin=3cm, rmargin=3cm, headheight=3em, headsep=2em, footskip=1cm}

\setcitestyle{citesep={,}}

\newcommand{\todo}[1]{\textcolor{red}{$\blacksquare$ TODO: #1}} 

\geometry{a4paper, tmargin=2cm, bmargin=2cm, lmargin=1cm, rmargin=1cm, headheight=2em, headsep=2em, footskip=1cm}

\title{PhD thesis}
\author{Vsevolod Nedora}
\date{today}

\begin{document}
    
    \maketitle

%% --------------- 
%%
%% Theory
%%
%% ---------------

\chapter{General-Relativistic Hydrodynamics}

This chapter is meant to sketch several important parts of the mathematical background. We focus on the aspects relevant for the tools and methods employed in out discussion. We do not aim to provide a comprehensive overview. 
The chapter is divided into \todo{list the parts and their content}

\section{The Cauchy Problem in General Relativity}

In this section we briefly recall the initial-value formulation of the Einstein equations of general relativity through the following steps. We start by introducing notations and the basics of GR. We summarize the Einstein field equations. Then we continue with how EFE can be split in a set of evolutionary equations and constraints. For that we focus on the Arnowitt, Deser and Misner, or ADM, formalism. In the end we comment on the stability of the ADM equations, on the need for strongly-hyperbolic formulations of the EFE, and on the choice of gauge conditions commonly used to
evolve spacetimes with singularities. This overview is based in \cite{Arnowitt:1962hi,Landau:1982dva,Wald:1984,Misner:1973,Baumgarte:2002jm}, which we refer to for more detained discussion.

\subsection{Euler-Lagrange equations}

We consider a spacetime defined by the real smooth manifold $\mathcal{M}$ and Lorentzian metric $\boldsymbol{g}$ on $\mathcal{M}$ of signature (-,+,+,+). The $\nabla$ denotes the affine connection associated with $\boldsymbol{g}$, the Levi-Civita connection. \\
We use the convection that all Greek indices lie in $\{0, 1, 2, 3\}$ and Lower case Latin indices $\{1, 2, 3\}$. \\
The $\nabla\boldsymbol{T}$ denotes the covariant derivative of a tensor $\boldsymbol{T}$ and $\nabla_{\boldsymbol{u}}\boldsymbol{T}$ -- covariant derivative along a given vector field $\boldsymbol{u}$.\\
The scalar product of two vectors then 
\begin{equation}
    \boldsymbol{a}\cdot\boldsymbol{b}:=g_{\mu\nu}a^{\mu}b^{\nu}
\end{equation}
The action of a linear form on a vector however is represented as 
\begin{equation}
    \langle\boldsymbol{\omega},\boldsymbol{\upsilon}\rangle=\omega_{\mu}\upsilon^{\mu}
\end{equation}

Let the $\boldsymbol{\alpha}$ be the totally antisymmetric symbol that expresses through coordinates $x^{\mu}$ as
\begin{equation}
    \boldsymbol{\alpha} = dx^0 \wedge dx^1 \wedge dx^2 \wedge dx^3,
\end{equation}
where $\wedge$ denotes exterior product. Then, proper volume pseudo-form of the spacetime is

\begin{equation}
    \boldsymbol{\varepsilon} = \sqrt{-g}\boldsymbol{\alpha},
\end{equation}
where $g$ denotes the determinant of the spacetime metric. \\

In GR, the spacetime is represented by Lorentzian manifold $\mathcal{M}$ and $g$, the Loretzian metric. \\

The action principle of the Lagrangian field theory on the spacetime $(\mathcal{M}; \boldsymbol{g})$ is
\begin{equation}
    S(\boldsymbol{q}, \nabla\boldsymbol{q}) = \int_{\mathcal{M}}\boldsymbol{\alpha}\mathcal{L}(\boldsymbol{q}, \nabla\boldsymbol{q}),
\end{equation}
where $\boldsymbol{q}$ are a set of generalized coordinates for the fields described by the theory, $\nabla$ is the Levi-Civita connection, $\mathcal{L}$ is a scalar density of a scalar quantity $\lambda$ as $\lambda(\boldsymbol{q},\nabla\boldsymbol{q})$. 

Varying the action with respect to the $\boldsymbol{q}$
\begin{equation}
    \delta S(\boldsymbol{q}, \nabla\boldsymbol{q}) = \delta\int\boldsymbol{\alpha}\mathcal{L}(\boldsymbol{q}, \nabla\boldsymbol{q}) = \int\boldsymbol{\alpha}\Big(\frac{\partial\mathcal{L}}{\partial\boldsymbol{q}}\delta\boldsymbol{q}+\frac{\partial\mathcal{L}}{\partial(\nabla\boldsymbol{q})}\delta\nabla\boldsymbol{q}\Big)
\end{equation}

As $\delta$ and $\nabla$ commute, and partially integrating $\nabla$, we obtain

\begin{equation}
    \partial S(\boldsymbol{q}, \nabla\boldsymbol{q}) = \int\boldsymbol{\alpha}\Big(\frac{\mathcal{L}}{\partial\boldsymbol{q}}-\nabla\frac{\partial \mathcal{L}}{\partial(\nabla\boldsymbol{q})}\Big)\delta\boldsymbol{q} + \int_{\mathcal{M}}\boldsymbol{\alpha}\nabla\Big(\frac{\partial\mathcal{L}}{\partial(\nabla\boldsymbol{q})}\delta\boldsymbol{q}\Big)
\end{equation}

The last term is a boundary term and in order to vanish we impose boundary condition. Assume that the fields are defined over only a compact domain. \\
As the choice of $\partial\boldsymbol{q}$ is arbitrary, the 

\begin{equation}
    \partial S(\boldsymbol{q}, \nabla\boldsymbol{q}) = 0
\end{equation}

and the Euler-Lagrange equations are

\begin{equation}
    \frac{\partial \mathcal{L}}{\partial\boldsymbol{q}} - \nabla\Big(\frac{\partial\mathcal{L}}{\partial(\nabla\boldsymbol{q})}\Big) = 0
    \label{eq:theory:eulerlagrange}
\end{equation}

%% ----------------------------------------------- 
\subsection{The Hilbert Action}

The Einstein–Hilbert action allows to obtain an Einstein field equations through ad principle of least action. Here we briefly underline the procedure.

Introduce action that describes the graviatational field, and a matter field $\mathcal{L}_m$:
\begin{align}
    S_g &= \int\frac{1}{2\kappa}R\epsilon, \\
    S_m &= \int\mathcal{L}_{m}\epsilon,
\end{align}
where $R$ is the Ricci scalar and $\kappa$ is the  Einstein's constant. \\

The full action then:
\begin{equation}
    S = \int\Big(\frac{1}{2\kappa}R+\mathcal{L}_m\Big)\epsilon
\end{equation}

The action principle dicatates, that $\delta S = 0$  with respect to the inverse metric $g^{\mu\nu}$. 

\begin{equation}
    \int\Bigg[\frac{1}{2\kappa}\Big(\frac{\delta R}{\delta g^{\mu\nu}}+\frac{R}{\sqrt{-g}}\frac{\delta\sqrt{-g}}{\delta g^{\mu\nu}}\Big) + \frac{1}{\sqrt{-g}}\frac{\delta(\sqrt{-g}\mathcal{L}_m)}{\delta g^{\mu\nu}}\Bigg]\delta g^{\mu\nu}\epsilon
\end{equation}

Owing to the arbitrariness of $\delta g^{\mu\nu}$, the integrant must be zero. 

\begin{equation}
    \frac{\delta R}{\delta g^{\mu\nu}} + \frac{R}{\sqrt{-g}}\frac{\delta\sqrt{-g}}{\delta g^{\mu\nu}} = -2\kappa\frac{1}{\sqrt{-g}}\frac{\delta(\sqrt{-g}\mathcal{L}_m)}{\delta g^{\mu\nu}} = -\frac{2\kappa}{\sqrt{-g}}\frac{\delta S_m}{\delta g_{\mu\nu}} := \kappa T_{\mu\nu},
    \label{eq:theory:action1}
\end{equation}
where we introduced the stress-energy tensor $T_{\mu\nu}$ and te matter action $S_m$ for future use. \\

\todo{this matter action is used in deriving the $T_{\mu} ^{\nu}$ i the invariant fluid formalisn}

The continuation of this deriviation requires taking variation of the Riccia scalar $R$ and the determinantof the metric $\sqrt{-g}$. As this is a length procedure, we provide here the result. 

\begin{equation}
    \frac{\delta R}{\delta g^{\mu\nu}} = R_{\mu\nu},
    \label{eq:theory:deltaR}
\end{equation}
where the $R_{\mu\nu}$ is the Ricci curvature tensor.

\begin{equation}
    \frac{1}{\sqrt{-g}}\frac{\delta\sqrt{-g}}{\delta g^{\mu\nu}} = -\frac{1}{2}g_{\mu\nu}.
    \label{eq:theory:deltagmuny}
\end{equation}

Substituting Eq. \ref{eq:theory:deltaR} and Eq. \ref{eq:theory:deltagmuny} into equation of motion Eq.  \ref{eq:theory:action1} we obtain the Einstein's field equation 

\begin{equation}
    R_{\mu\nu} -\frac{1}{2}g_{\mu\nu}R=8\pi T_{\mu\nu},
    \label{eq:theory:EFE}
\end{equation}
where in the geometrized unit system, \textit{i.e} $c=G=1$, the $\kappa=8\pi$.

%% ----------------
\subsection{3+1 Decomposition of Einstein field equations}

The Einstein field equations (\ref{eq:theory:EFE}) represent a set of 10 non-linear partial differential equations. These equations can be defeined on a while metric $\mathcal{M}$ or a domain $\Omega\subset\mathcal{M}$, where in the latter, the boundary conditions on $\partial\Omega$ are required. \\
It is convenient to chose a null hyersurface $\Sigma\subset\mathcal{M}$ on which to define the initial data, from which the evolution of space-time begins. This, however, requires the spacetime to be strongly hyperbolic, meaning that the foliation $\mathcal{M}=\Sigma\times\mathbb{R}$ is allowed. This foliation can be understood as splitting the spacetime into a set of spacelike hypersurfaces $\Sigma_t$. 

\subsubsection{Spacelike Foliations}
Let the $t$ be the global smooth functions such that, 

\begin{equation}
    \Sigma_{\tau} = \{x^{\alpha}\in\mathcal{M}: t(x^{\alpha})=\tau\},
\end{equation}

and let $\vec{t}$ be a vector such that $\langle\nabla t, \vec{t}\rangle = 1$. This the $t$ can be seen as a "function that advances time" and $\vec{t}$ as a "flow of time" vector field. Continuing the analogy, the rate at which a given tensor quantity $\boldsymbol{q}$ changes between hypersurfaces $\Sigma_t$ is given by the Lie derivative of the $\boldsymbol{q}$ along the vector $\vec{t}$. \\

Consider two hypersurfaces $\Sigma_t$ and $\Sigma_{t+dt}$. A transition from one to another can be decomposed into the part tangent to the hypersurface $\Sigma_{t+dt}$ and expressed in a form of a vector $\vec{\beta}$ and a pert normal to the hypersurface $\Sigma_t$ and expressed as a $\alpha \vec{n}$, where $\vec{n}$ is a unit vector, normal to the $\Sigma_t$ in the diretion to $\Sigma_{t+dt}$. Then, the vector $\vec{t}$ can be written as 

\begin{equation}
    \vec{t} = \alpha\vec{n}+\vec{\beta}.
\end{equation}

$\vec{\beta}$ is called shift vector and $\alpha$ is called lapse-function. \\

The spacetime metric $\boldsymbol{g}$ can be decomposed into a spatial, Riemannian metric $\boldsymbol{\gamma}$  as $\boldsymbol{\gamma} = \boldsymbol{g} + \underline{n} \otimes \underline{n} $, where $\underline{n}$ is the 1-form associated to the vector $\vec{n}$. The Levi-Civita connection can be computed by projecting the $\nabla$ on the space tangent to the hypersurface $\Sigma_t$.

There are exist coordinates that are adapted to the foliation, namely $\{t, x^i\}$ with $\vec{\partial}_i\cdot \vec{n} = 0$. In these coordiantes the $\nabla t = dt$ and $\vec{t} = \vec{\partial}_t$. 

The connection between $\boldsymbol{g}$ and $\boldsymbol{\gamma}$ is $g_{\mu\nu}=\vec{\partial}_{\mu}\cdot\vec{\partial}_{\nu} $ and can be expressed in terms of $\alpha$ and $\vec{\beta}$ as

\begin{align}
    \text{spatial components: } g_{ik}&=\vec{\partial}_{i}\cdot\vec{\partial}_{j} =\gamma_{ik}, \\
    \text{time component: } g_{tt} &= \vec{\partial}_{t}\cdot\vec{\partial}_{t} = \vec{t}\cdot\vec{t} = - (\alpha^2-\vec{\beta}\cdot\vec{\beta}), \\
    \text{mixed components: } g_{ti} &= \vec{\partial}_{t}\cdot\vec{\partial}_{i} = \vec{t}\cdot\vec{\partial}_i = (\alpha\vec{n}+\vec{\beta})\cdot\vec{\partial}_i=\beta_i,
\end{align}
we we made use of $\vec{\beta}$ being the spatial vector, \textit{i.e} $\vec{\beta}\cdot\vec{\beta}=\gamma_{ik}\beta^i\beta^k$.

The line-element can be thus written as
\begin{equation}
    ds^2 = -(\alpha^2-\beta_i\beta^i)dt^2 +2\beta_i dx^i dt + \gamma_{ik} dx^i dx^k.
\end{equation}

\subsubsection{Extrinsic Curvature and Constraint equations}

We define the \textit{extrinsic curvature} of a $D-1$-suface $\Sigma_t\subset\mathcal{M}$ at a point $\mathcal{P}\in\Sigma_t$ as mapping $\boldsymbol{K}$ such that $\boldsymbol{K}(\boldsymbol{\upsilon})=-\nabla_{\boldsymbol{\upsilon}}\boldsymbol{n}$. Note, that the $\boldsymbol{K}$ thus does not depend on $\alpha$ and $\vec{\beta}$, it is a purely spatial tensor. The components of the extrinsic curvature are \\

\begin{equation}
    K_{\mu\nu} = -{\gamma^{\alpha}}_{\mu}\nabla_{\boldsymbol{u}}^{\alpha} n_{\nu} = -\frac{1}{2}\mathcal{L}_{\vec{n}}\gamma_{\mu\nu},
    \label{eq:theory:extrcurvdef}
\end{equation}
where $\mathcal{L}_{\vec{n}}$ is the Lie derivative along the vector field $\vec{n}$. \\
From the (\ref{eq:theory:extrcurvdef}) the extrinsic curvature can be interprated as a "speed of the $\vec{n}$ during the parallel transport along the hypersurface $\Sigma_t$".

Codazzi equations relate the $4D$ Ricci tensor to the extrinsic curvature as

\begin{equation}
    D_{\beta}K-D_{\alpha}{K^{\alpha}}_{\beta}=R_{\gamma\delta}n^{\delta}{\gamma^{\gamma}}_{\beta},
    \label{eq:theory:formomentum}
\end{equation}

here $K$ is a trace of the tensor $\boldsymbol{K}$. \\

Gauss equation realtes the $3D$ Riemann tensor $^3{R_{\alpha\beta\gamma}}^{\delta}$ to the $4D$ one and the $\boldsymbol{K}$ as

\begin{equation}
    ^3{R_{\alpha\beta\gamma}}^{\delta} = {\gamma^{\mu}}_{\alpha}{\gamma^{\nu}}_{\beta}{\gamma^{\lambda}}_{\gamma}{\gamma^{\delta}}_{\sigma}{R_{\mu\nu\lambda}}^{\delta}-K_{\alpha\gamma}{K_{\beta}}^{\delta}+K_{\beta\gamma}{K^{\delta}}_{\alpha}.
    \label{eq:theory:forhamiltconst}
\end{equation}

The \textit{momentum constraint} thus cab be obtained by substituting the (\ref{eq:theory:EFE}) into  (\ref{eq:theory:formomentum}) which yields

\begin{equation}
    D_{\beta}K-D_{\alpha}{K^{\alpha}}_{\beta} = -8\pi{\gamma^{\alpha}}_{\beta} n^{\gamma}T_{\alpha\gamma}=:8\pi j_{\beta},
    \label{eq:theory:momconstraint}
\end{equation}
where $j^{\alpha}$ is the ADM momentum density. \\

The \textit{Hamiltonian constrant} can be obtained by substituting EFE (\ref{eq:theory:EFE}) into the (\ref{eq:theory:forhamiltconst}), yielding 

\begin{equation}
    ^3 R+ K^2 - K_{\alpha\beta}K^{\alpha\beta} = 2G^{\alpha\beta}n_{\alpha}n_{\beta} = 16\pi n_{\alpha}n_{\beta} T^{\alpha\beta} =: 16\pi E,
    \label{eq:theory:hamilconstraint}
\end{equation}
where $E$ is the ADM energy density. \\

The obtained constraint equations represent a set of elliptic equations that must be satisfied on every hyprsurface $\Sigma_i$ of the foliation. It is however, possible to show that Eistein equations preserve the constraints, meaning that if they are satisfied at the initial slice $\Sigma_0$ they will be satisfied at any time in the future. 

\subsubsection{Hamiltonian Field Theory}

First we recall the generalized coordinates $\boldsymbol{q}$ and their covariant derivatives $\nabla\boldsymbol{q}$. \\
In light of the spacetime decomposition discussed above, we divide the $\boldsymbol{\alpha}$ into the time $dt$ and spatial parts represented by the antisymmetric symbol ${^{(3)}\boldsymbol{\alpha}}$ as 

\begin{equation}
    \boldsymbol{\alpha} = dx^0 \wedge dx^1 \wedge dx^2 \wedge dx^3 = dt \wedge {^{(3)}\boldsymbol{\alpha}}.
\end{equation}

Next, we introduce the "time derivative" as a Lie derivative along the vector field $\vec{t}$ as 

\begin{equation}
    \dot{\boldsymbol{q}} := \mathcal{L}_{\vec{t}}\boldsymbol{q}.
\end{equation}

As the $\Lambda(\boldsymbol{q}, \nabla\boldsymbol{q})$ is the Lagrangian density, a conjugate momentum can be defined as 

\begin{equation}
    \boldsymbol{p} := \frac{\partial\Lambda}{\partial\dot{\boldsymbol{q}}},
\end{equation}

Assuming that $\boldsymbol{p}$ and $\nabla\boldsymbol{q}$ can be expressed as a function of $\boldsymbol{q}$ and $\boldsymbol{p}$, inspired by the Legendre transformation, we define the Hamiltonian and its density density as

\begin{align}
    \mathcal{H} &= \boldsymbol{p}\cdot\dot{\boldsymbol{q}} - \mathcal{L}(\boldsymbol{q}, \nabla\boldsymbol{q}) \\
    H &= \int_{\Sigma}\mathcal{H}{^{(3)}\boldsymbol{\alpha}}
\end{align}

Additionally we define the quantity 

\begin{equation}
    J = \int_{0}^{t}H(\boldsymbol{q},\boldsymbol{p})dt = \int_{0}^{t}dt\int_{\Sigma}\mathcal{H}(\boldsymbol{q},\boldsymbol{p}){^{(3)}\boldsymbol{\alpha}} = \int_{0}^{t}dt\int_{\Sigma}{^{(3)}\boldsymbol{\alpha}}\Big(\boldsymbol{p}\cdot\dot{\boldsymbol{q}} - \mathcal{L}(\boldsymbol{q},\nabla\boldsymbol{q})\Big).
\end{equation}

Consider the variation of the $J$ with respect to the $\delta\boldsymbol{p}$ and $\delta\boldsymbol{q}$ as

\begin{equation}
    \delta J = \int_{0}^{t}\delta H(\boldsymbol{q},\boldsymbol{p})dt = \int_{0}^{t}dt (\dot{\boldsymbol{q}}\delta\boldsymbol{p}+\boldsymbol{p}\delta\dot{\boldsymbol{q}}) - \int_{0}^{t}dt\delta\Lambda(\boldsymbol{q}, \nabla\boldsymbol{q}).
\end{equation}

Consider the last term, the variation of the Lagrangian 

\begin{equation}
    \delta\Lambda = \int_{\Sigma}{^{(3)}\boldsymbol{\alpha}}\Bigg[\frac{\delta\Lambda}{\delta\dot{\boldsymbol{q}}}\delta\dot{\boldsymbol{q}}+\frac{\delta\Lambda}{\delta\boldsymbol{q}}\delta\boldsymbol{q}\Bigg],
\end{equation}

The first term in the square brackets can be reduced to $\boldsymbol{p}\delta\dot{\boldsymbol{q}}$, suingthe definition of the conjugate momentum. The second term can be treated, applying the Euler-Lagrange equations (\ref{eq:theory:eulerlagrange}). These manipulations result in

\begin{equation}
    \delta\Lambda = \int_{0}^{t}dt\int_{\Sigma}{^{(3)}\boldsymbol{\alpha}}(\boldsymbol{p}\delta\dot{\boldsymbol{q}} + \dot{\boldsymbol{p}}\delta\boldsymbol{q}).
\end{equation}

Thus we obtain that 

\begin{equation}
    \int_{0}^{t} \delta H(\boldsymbol{q},\boldsymbol{p})dt =   \int_{0}^{t}dt\int_{\Sigma}{^{(3)}\boldsymbol{\alpha}}(\dot{\boldsymbol{q}}\cdot\delta\boldsymbol{p}-\dot{\boldsymbol{p}}\cdot\delta\boldsymbol{q}),
\end{equation}

and as $\delta\boldsymbol{p}$ and $\delta\boldsymbol{p}$ are arbitrary, the Hamilton equations read

\begin{equation}
    \dot{\boldsymbol{q}}=\frac{\delta H}{\delta\boldsymbol{p}}, \hspace{5mm} \dot{\boldsymbol{p}} = -\frac{\delta H}{\delta\boldsymbol{q}}.
    \label{eq:theory:hamiltoneqs}
\end{equation}

The Hamiltonian formalism can be used to redirive the field-equations in a from that once the initial data is specified on a hypersurface $\Sigma_0$ for $\boldsymbol{q}$ and $\boldsymbol{p}$, the equations (\ref{eq:theory:hamiltoneqs}) would govern whole the evolution.

\subsubsection{The Hamiltonian Formulation of the Einstein Equations}

Here we briefly sketch to path of derivation of the Einstein field equations in the Hamiltonian framework. We will elude most of the intimidate and computationally extensive steps, as well as derivation of the boundary terms. For this we refer to \cite{Poisson:2004}.\\
First it is useful to note that determinant of the three-metric $\sqrt{\gamma}$ can be expressed as $\sqrt{\gamma}=\sqrt{-g}/\alpha$. The $p$ is the trace of the canonical momentum $\boldsymbol{p}$.

Now, consider the scalar curvature, R

\begin{align}
    G_{\mu\nu} &= R_{\mu\nu} - \frac{1}{2}Rg_{\mu\nu} \\
    -Rg_{\mu\nu}n^{\nu}n^{\mu} &= 2(G_{\mu\nu} n^{\nu}n^{\mu}-R_{\mu\nu}n^{\mu}n^{\mu})\\
    -Rn_{\mu}n^{\mu}& = 2(G_{\mu\nu}n^{\nu}n^{\mu} - R_{\mu\nu}n^{\mu}n^{\mu}) \\
    R &= 2(G_{\mu\nu}n^{\mu}n^{\nu} - R_{\mu\nu}n^{\mu}n^{\nu}).
\end{align}

From the Gauss-Codacci equation (\ref{eq:theory:momconstraint}), which relates the spatial curvature $^{(3)}R$ to the spacetime curvature $R$, we have the following constraint
relationship

\begin{equation}
    2G_{\mu\nu}n^{\mu}n^{\nu} = {^{(3)}R} + K^2 - K_{\mu\nu}K^{\mu\nu}.
\end{equation}

The $R_{\mu\nu}n^{\mu}n^{\nu})$ can be expressed as a combination of extrinsic curvature and total divergences as 

From the definition of the Ricci tensor $R_{\mu\nu}$, we have:

\begin{align}
    R_{\mu\nu} &= {R_{\mu\gamma\nu}}^{\gamma} \\
    R_{\mu\nu}n^{\mu}n^{\nu} &= {R_{\mu\gamma\nu}}^{\gamma} \\
    &= -(\nabla_{\mu}\nabla_{\gamma} - \nabla_{\gamma}\nabla_{\mu})n^{\gamma}n^{\nu} \\
    &= n^{\mu}(\nabla_{\mu}\nabla_{\gamma} - \nabla_{\gamma}\nabla_{\nu})n^{\gamma} \\
    &= (\nabla_{\mu}n^{\mu})(\nabla_{\gamma}n^{\gamma}) - \nabla_{\mu}(n^{\mu}\nabla_{\gamma}n^{\gamma}) - (\nabla_{\gamma}n^{\mu})(\nabla_{\mu}n^{\gamma}) + \nabla_{\gamma}(n^{\mu}\nabla_{\mu}n^{\gamma}) \\
    &= K^2 - K_{\mu\gamma}K^{\mu\gamma} - \nabla_{\mu}(n^{\mu}\nabla_{\gamma}n^{\gamma}) + \nabla_{\gamma}(n^{\mu}\nabla_{\mu}n^{\gamma})
\end{align}

In case of variations with compact support, that we are interested in, the total divergences. last two terms, can be neglected. Then the result is

\begin{equation}
    R_{\mu\nu}n^{\mu}n^{\nu}= K^2 - K_{\mu\nu}K^{\mu\nu}.
    \label{eq:theory:rmunu_as_func_k}
\end{equation}

Using the fact that $\sqrt{\gamma}=\sqrt{-g}/\alpha$ and the (\ref{eq:theory:rmunu_as_func_k}) we obtain the Lagrangian density in terms of the variables of the hypersurface:

\begin{align}
    \Lambda &= \sqrt{-g}R \\
    &= \alpha\sqrt{\gamma}R \\
    &= 2\alpha\sqrt{\gamma}(G_{\mu\nu}n^{\mu}n^{\nu} - R_{\mu\nu}n^{\mu}n^{\nu})\\ 
    &= 2\alpha\sqrt{\gamma}\Big(\frac{1}{2}[{^{(3)}R} - K_{\mu\nu}K^{\mu\nu} + K^2] - K^2 - K_{\mu\nu}K^{\mu\nu}\Big)
\end{align}

Together with the contribution from matter fields, we obtain

\begin{equation}
    \Lambda = \Lambda_g+\Lambda_m= \frac{1}{16\pi}\alpha({^{(3)}R} + K_{\mu\nu}K^{\mu\nu} - K^2)\sqrt{\gamma}+\Lambda_m
\end{equation}

Next we note that the extrinsic curvature of a
surface $\Sigma$ is defined as $K_{\mu\nu} = \nabla_{\mu}n_{\nu}$. \\
To relate $K_{\mu\nu}$ to the metric, we make use of the following property of Lie derivatives:

\begin{align}
    \mathcal{L}_{\vec{n}}g_{\mu\nu} &= n^{\gamma}\nabla_{\gamma}g_{\mu\nu} + g_{\gamma\nu}\nabla_{\mu}\upsilon^{\gamma} + g_{\mu\gamma}\nabla_{\nu}\upsilon^{\gamma} \\
    &= \nabla_{\mu}n_{\nu}+\nabla_{\nu}\upsilon_{\nu} \\
    &=2\nabla_{\mu}n_{\nu}
\end{align}

where the second line holds when $\nabla_{\gamma}\mu$ is the natural derivative operator corresponding to the metric $g_{\mu\nu}$ and the third line holds because $K_{\mu\nu}$ is symmetric.

Substituting this into our definition of $K_{\mu\nu}$,

\begin{align}
    K_{\mu\nu} &= -\frac{1}{2}\mathcal{L}_{\vec{\vec{n}}}g_{\mu\nu} \\
    &= -\frac{1}{2}\mathcal{L}_{\vec{\vec{n}}}(\gamma_{\mu\nu}-n_{\mu}n_{\nu}) \\
    &= -\frac{1}{2}\mathcal{L}_{\vec{\vec{n}}}\gamma_{\mu\nu} \\
    &= -\frac{1}{2}[n^{\gamma}\nabla_{\gamma}\gamma_{\mu\nu} + \gamma_{\gamma\nu}\nabla_{\mu}\upsilon^{\nu} + h_{\mu\gamma}\nabla_{\nu}\upsilon^{\gamma}] \\
    &= -\frac{1}{2\alpha}[\alpha n^{\gamma}\nabla_{\gamma}\gamma_{\mu\nu} + \gamma_{\gamma\nu}\nabla_{\mu}\alpha\upsilon^{\nu} + h_{\mu\gamma}\nabla_{\nu}\alpha\upsilon^{\gamma}] \\
    &= -\frac{1}{2\alpha}{\gamma_{\mu}}^{\gamma}{\gamma_{\nu}}^{\delta}[\mathcal{L}_{\vec{t}}\gamma_{\gamma\delta}-\mathcal{L}_{\vec{\beta}}\gamma_{\gamma\delta}] \\
    &= -\frac{1}{2\alpha}{\gamma_{\mu}}^{\gamma}{\gamma_{\nu}}^{\delta}[\partial_t\gamma_{\mu\nu}-D_{\mu}\beta_{\nu}-D_{\nu}\beta_{\mu}]
\end{align}

and on the hypersurface $\Sigma$ the projection operators are not needed. So we obtain

\begin{equation}
    K_{\mu\nu} = -\frac{1}{2}\mathcal{L}_{\vec{n}}\gamma_{\mu\nu}=-\frac{1}{2\alpha}(\partial_t\gamma_{\mu\nu}-D_{\mu}\beta_{\nu}-D_{\nu}\beta_{\mu})
\end{equation}

which us to express the canonical momentum $p^{\mu\nu}$ as

\begin{align}
    p^{\mu\nu} &= \frac{\partial\Lambda}{\partial\dot{\gamma}_{\mu\nu}} \\
    &= -\frac{\sqrt{\gamma}}{16\pi}\alpha\Bigg[\frac{\partial {^{(3)}R}}{\partial\dot{\gamma}_{\mu\nu}} + \frac{\partial(K_{\mu\nu}K^{\mu\nu})}{\partial\dot{\gamma}_{\mu\nu}} - \frac{\partial K^2}{\partial\dot{\gamma}_{\mu\nu}}\Bigg] \\
    &= \frac{\sqrt{\gamma}}{16\pi}(K\gamma^{\mu\nu} - K^{\mu\nu}),
\end{align}
where 
\begin{equation}
    \frac{\partial K_{\mu\nu}}{\partial \dot{\gamma}_{\mu\nu}} = \frac{1}{2\alpha}, \hspace{5mm} \frac{\partial {^{(3)}R}}{\partial \dot{\gamma}_{\mu\nu}} = 0, \hspace{5mm}\frac{\partial K^2}{\partial \dot{\gamma}_{\mu\nu}} = \frac{\gamma^{\mu\nu}K}{\alpha}
\end{equation}

assuming that there is no explicit dependency of the $\Lambda$ on $dot{\gamma}_{\mu\nu}$.

Since, $\alpha$ and $\vec{\beta}$ are related to the the gauge freedom, as there are many ways manifold $\mathcal{M}$ can be split into hypersurfaces, the momenta associated with these function and vector is zero. 

Thus, the Hamiltonian density is

\begin{align}
    \mathcal{H} &= p^{\mu\nu}\dot{\gamma}_{\mu\nu} - \Lambda \\
    &= -\sqrt{\gamma}\alpha{^{(3)}R} + \frac{\alpha}{\sqrt{\gamma}}\Big[p^{\mu\nu}p_{\mu\nu}-\frac{1}{2}p^2\Big] + 2p^{\mu\nu} D_{\mu}\beta_{\mu} -\Lambda_m \\
%    &=  \frac{\sqrt{\gamma}}{16\pi}\Bigg\{\alpha\Big[-{^{(3)}R}+h^{-1}p^{\mu\nu}p_{\mu\nu}-\frac{1}{2}h^{-1}p^2\Big] - 2\beta_{\nu}\big[D_{\mu}(h^{-1/2}p^{\mu\nu})\big] + D_{\mu}(h^{-1/2}\beta_{\nu}p^{\mu\nu})\Bigg\} \\
    &= \frac{\sqrt{\gamma}}{16\pi}\Bigg\{\alpha\Big[ -{^{(3)}R} + \gamma^{-1}p^{\mu\nu}p_{\mu\nu}-\frac{1}{2}\gamma^{-1}p^2\Big] +  2\beta_{\nu}\Big[D_{\mu}(\gamma^{-1/2}p^{\mu\nu})\Big] - 2D_{\mu}(\gamma^{-1/2}\beta_{\nu}p^{\mu\nu}) \Bigg\} - \Lambda_m,
\end{align}
where we restored the correct $16\pi$ factor in the last line.

As the we consider variations with compact suppot, the last boundary term, can be neglected. \\

Now we consider the variation of the matter action $S_m$ with respect to the $\alpha$ and $\vec{\beta}$

\begin{align}
    \frac{\delta S_m}{\delta \alpha} &=-\alpha\frac{\delta S_m}{\delta g_{00}} = -\alpha\sqrt{-g}T^{00} = -\alpha^2\sqrt{\gamma}T^{00} = -\sqrt{\gamma}T^{\mu\nu}n_{\mu}n_{\nu} \\
    \frac{\delta S_m}{\delta \beta_{\mu}} &= \frac{\delta S_m}{\delta g_{\mu 0}} =\frac{1}{2}\sqrt{-g}T^{\mu 0} = -\frac{1}{2} \sqrt{\gamma}T^{\mu\nu}n_{\nu}.
\end{align}

As the variation of the Hamiltonian $H$ with respect to a quantity with vanishing canonical momentum is zero, we obtain two equations 

\begin{align}
    \frac{\delta H}{\delta \alpha} &= 0 = -{^{(3)}R} + \gamma^{-1}p^{\mu\nu}p_{\mu\nu}-\frac{1}{2}\gamma^{-1}p^2 + 16\pi T^{\mu\nu}n_{\mu}n_{\nu} \\
    \frac{\delta H}{\delta \beta_{\mu}} &= 0 = - D_{\mu}(\gamma^{-1/2}p^{\mu\nu}) + 8\pi{\gamma^{\mu}}_{\nu}n_{\gamma}T^{\nu\gamma}.
    \label{eq:theory:hamiltonianvariation}
\end{align}


Note, that the $\delta H / \delta\beta_{\mu}$ is actually a Frech\'et differential $dH$, $\delta \beta_{\mu}$, which is writes as
\begin{equation}
    \langle dH,\delta\beta \rangle = \delta\beta_{\mu}\big[-D_{\nu}(\gamma^{-1/2}p^{\mu\nu})+8\pi n_{\gamma}T^{\mu\nu}\big], 
\end{equation}
containing $\delta\beta_{\mu}$ which is spatial. Thus only the spatial part is being constrained in the equation above. To account for that the procector ${\gamma^{\mu}}_{\nu}$ is added to the $\delta H/\delta \beta_{\mu}$. \\

The pair of equations (\ref{eq:theory:hamiltonianvariation}) is in fact the constraint equations derived before, namely the (\ref{eq:theory:momconstraint}) and (\ref{eq:theory:hamilconstraint}), and as we now see, they are related to the coordinate freedom of $\mathcal{M}$ decomposition and a coodrinate freedom on hypersurfaces. \\

Proceeding with the Hamiltinan formalism we note that equation \ref{eq:theory:hamiltoneqs} leads to the evolution equations for the three-metric, assuming that $\Lambda$ explicitly does not depend on the momentum

\begin{equation}
    \dot{\gamma}_{\mu\nu} =\frac{\delta H}{\delta p^{\mu\nu}} = 2\gamma^{-1/2}\alpha\big(p_{\mu\nu}-\frac{1}{2}\gamma_{\mu\nu}p\big) - D_{\nu}\beta_{\mu}-D_{\mu}\beta_{\nu}
%    -2D_{(\mu}\beta_{\nu)},
    \label{eq:theory:_adm_metric_evo}
\end{equation}

The evolution equations for the canonical momentum can read

\begin{align}
    \dot{p}^{\mu\nu} = -\frac{\delta H}{\delta \gamma_{\mu\nu}} = &+ \alpha\gamma^{1/2}\big({^{(3)}R}^{\mu\nu}-\frac{1}{2}{^{(3)}R\gamma^{\mu\nu}}\big) \\
    & - \frac{1}{2}\alpha\gamma^{-1/2}\gamma^{\mu\nu}\big(p_{\gamma\delta}p^{\gamma\delta}-\frac{1}{2}p^2\big) \\
    & + 2\alpha\gamma^{-1/2}\big(p^{\mu\gamma}{p^{\nu}}_{\gamma}-\frac{1}{2}pp^{\mu\nu}\big) \\
    & - \gamma^{1/2}\big(D^{\mu}D^{\nu}\alpha-\gamma^{\mu\nu}D^{\gamma}D_{\gamma}\alpha\big) \\
    & - \gamma^{1/2}D_{\gamma}\big(\gamma^{-1/2}\beta^{\gamma}p^{\mu\nu}\big) \\
    &+ 2p^{\gamma(\mu}D_{\gamma}\beta^{\nu)} + 8\pi \alpha \gamma^{1/2}S^{\mu\nu},
    \label{eq:theory:_adm_mom_evo}
\end{align}
where $A_{(\mu\nu)} = 0.5(A_{\mu\nu}+A_{\nu\mu})$ the convention was used. \\

where $S^{\mu\nu}={\gamma^{\mu}}_{\alpha}{\gamma^{\nu}}_{\beta}T^{\alpha\beta}$. 
Taking the variation of the matter field we noted that 


\begin{equation}
    \frac{\delta S}{\delta \gamma_{ik}} = \frac{\delta S_m}{\delta g_{ik}} = \frac{1}{2}\sqrt{-g}T^{ik}
\end{equation}

The set of equations (\ref{eq:theory:hamiltonianvariation}), (\ref{eq:theory:_adm_metric_evo}) and (\ref{eq:theory:_adm_mom_evo}) comprise the ADM system. A more widely used from of these equations is in turns of $\gamma_{ij}$ and $K_{ij}$ that reads

\begin{align}
    (\partial_t - \mathcal{L}_{\vec{\beta}})\gamma_{ik} &= -2\alpha K_{ik}; \\
    (\partial_t - \mathcal{L}_{\vec{\beta}})K_{ik} &= -D_{i}D_{k}\alpha + \alpha\big(R_{ik} - 2K_{ij}{K^j}_k+KK_{ik}\big) - 8\pi\alpha\big(S_{ik} - \frac{1}{2}\gamma_{ik}(S-E)\big); \\
    {^{(3)}R} + K^2 - K_{ik}K^{ik} &= 16\pi E; \\
    D_{i}K-D_{k}{K^k}_i &= 8\pi j_i,
    \label{eq:theory:adm}
\end{align}
where $S = \gamma^{ij}S_{ij}$.
These equations constitute the IVP for Einstein field equations and are known as ADM equations. The last two equations are the constraint equations. They determine how to set the initial data on the hypersurface $\Sigma_0$, via prescribing the three-metric and extrinsic curvature. The first two equations then govern the evolution.

\todo{make sure that the coefficients in formuals are consistent, $16\pi$ might me missing or $-$}
\todo{Makse sure that $\Lambda$ stands for largangian density and $\mathcal{L}$ for lie derivative}

\subsubsection{Strongly Hyperbolic Formulations of the Einstein Equations}

It has been shown, that the ADM system of equations in its original form (\ref{eq:theory:adm}) is only weekly hyperbolic \cite{Baumgarte:2002jm}. It was shown that in such system the errors tend to couple with zero-velocity modes \cite{Alcubierre:1999rt}.  \\
In an attempt to mitigate this problem, different formulations of the Einstein equations as initial-value problem were created. In particular, the the generalized-harmonic formulation \cite{Friedrich:1985,Lindblom:2005qh,Lindblom:2009}, the BSSNOK formulation, derived by Baumgarte, Shapiro, Shibata, Nakamura, Oohara and Kojima \cite{Nakamura1987,Shibata:1995we,Baumgarte:1998te} and and the Z4 formulation \cite{Bona:2003fj,Bernuzzi:2009ex,Ruiz:2010qj,Weyhausen:2011cg,Alic:2011gg}. We do not attempt to elaborate on any of these formations and only aim to emphasize that a search for a new and better formulations of Einstein equations for numerical applications is ongoing. We limit ourselves to sketching only the conformal-covariant variant of the Z4 formulation, also known as Z4c. The numerical implementation of this formulation was used to obtain the results discussed in this thesis. 


\subsubsection*{The CCZ4 Formulation}

The idea behind the Z4 formulation is to derive a set of evolution equations that is free from the zero-speed modes of the original ADM and thus -- strongly-hyperbolic. This is achieved by not explicitly enforcing the constraints and treating the deviation from them as an dependent variable $Z_{\mu}$. The $Z_{\mu}$ is also called the Z4 four-vector.

One starts with the covariant Lagrangian
\begin{equation}
    \Lambda = g^{\mu\nu}[R_{\mu\nu} + 2\nabla_{\mu}Z_{\nu}]\sqrt{g} + \Lambda_m,
\end{equation}

and applying Palatini-type variational principle \cite{Bona:2010is}, obtains an evolution equations

\begin{equation}
    R_{\mu\nu} + \nabla_{\mu}Z_{\nu} + \nabla_{\nu}Z_{\mu}=8\pi\Big(T_{\mu\nu} - \frac{1}{2}Tg_{\mu\nu}\Big),
    \label{eq:theory:z4fieldeq}
\end{equation}

and two sets of constraint equations

\begin{equation}
    \nabla_{\rho} g^{\mu\nu} = 0, 
    \label{eq:theory:z4connect}
\end{equation}

and

\begin{equation}
    Z_{\mu} = 0,
\end{equation}

where the latter is called an algebraic constraint. If its derivative vanishes, it is equivalent to imposing the ADM momentum and Hamiltonian constraints \cite{Bona:2009}. 

The Einstein field equations themselves are recovered from (\ref{eq:theory:z4connect}) and (\ref{eq:theory:z4fieldeq}) when the algebraic constraint is satisfied. \\

The Z4 system preserves the constraint, $\partial_t (Z_{\mu})= 0$. This allows to obtain the solution of the Einstein equations. 

However, the numerical solution of the system of equations introduces error, that leads to a constraint violation during the evolution. To mitigate this problem the Z4 system is further modified to enforce the dampening of the constraint violation propagation \cite{Gundlach:2005eh}.

A new version of Z4 was recently introduced by \cite{Alic:2011gg}. It incorporates the constraint-damping properties of the original Z4 and also allows for a better black hole treatment via \textit{moving-puncture}, that will be discussed later. 
The CCZ4 system reads 

\begin{align}
    \partial_{t}\widetilde{\gamma}_{ij} = & -2\alpha\widetilde{A}_{ij}^{\text{TF}} + 2\widetilde{\gamma}_{k(i}\partial_{j)}\beta^k - \frac{2}{3}\widetilde{\gamma}_{ij}\partial_k \beta^k + \beta^k\partial_k\widetilde{\gamma}_{ij}, \\
    \partial_{t}\widetilde{A}_{ij}^{\text{TF}} = & \phi^2\big[-\nabla_i\nabla_j\alpha + \alpha\big({^{(3)}R}_{ij}+\nabla_{i}Z_{j} + \nabla_{j}Z_{i}- 8\pi S_{ij}\big)\big]^{\text{TF}} \\
    & + \alpha\widetilde{A}_{ij}(K-2\Theta)-2\alpha\widetilde{A}_{il}{\widetilde{A}^l}_{j} + 2\widetilde{A}_{k(i}\partial_{j)}\beta^{k} \\
    & -\frac{2}{3}\widetilde{A}_{ij}\partial_{k}\beta^{k} + \beta^{k}\partial_{k}\widetilde{A}_{ij} \\
    \partial_{t} \phi = & \frac{1}{3}\alpha\phi K - \frac{1}{3}\phi\partial_{k}\beta^{k} + \beta^{k}\partial_{k}\phi \\
    \partial_{t}K = &-\nabla^{i}\nabla_{i}\alpha + \alpha\big({^{(3)}R} + 2\nabla_{i}Z^{i} + K^2 - 2\Theta K\big) + \beta^{j}\partial_{j}K \\
    & - 3\alpha\kappa_1(1+\kappa_2)\Theta + 4\pi\alpha (S-3E) \\
    \partial_{t}\Theta = &\frac{1}{2}\alpha\Big(R + 2\nabla_{i}Z^{i} - \widetilde{A}_{ij}\widetilde{A}^{ij} + \frac{2}{3}K^2 - 2\Theta K\Big) - Z^{i}\partial_{i}\alpha \\
    & + \beta^{k}\partial_{k}\Theta - \alpha\kappa_1(2 + \kappa_2)\Theta - 8\pi\alpha E \\
    \partial_{t}\hat{\Gamma}^j = & 2\alpha\Bigg({\widetilde{\Gamma}^i}_{jk}\widetilde{A}^{ij} - 3\widetilde{A}^{ij}\frac{\partial_{j}\phi}{\phi} -\frac{2}{3}\widetilde{\gamma}^{ij}\partial_{j}K\Bigg) + 2\widetilde{\gamma}^{ki}\Big(\alpha\partial_{k}\Theta - \Theta\partial_{k}\alpha - \frac{2}{3}\alpha K Z_{k}\Big) \\
    & - 2\widetilde{A}^{ij}\partial_{j}\alpha + \widetilde{\gamma}^{kl}\partial_{k}\partial_{l}\beta^{i} + \frac{1}{3} \widetilde{\gamma}^{ik}\partial_{k}\partial_{l}\beta^{l} + \frac{2}{3}\widetilde{\Gamma}^i\partial_{k}\beta^{k} \\
    & - \widetilde{\Gamma}^k\partial_{k}\beta^{i} + 2\kappa_3\Big(\frac{2}{3}\widetilde{\gamma}^{ij}Z_{j}\partial_{k}\beta^{k} - \widetilde{\gamma}^{jk}Z_{j}\partial_{k}\beta^{i}\Big) + \beta^{k}\partial_{k}\hat{\Gamma}^i \\
    & -2\alpha\kappa_1\widetilde{\gamma}^{ij}Z_{j}- 16\pi\alpha\widetilde{\gamma}^{ij}S_j,
\end{align}

where $\Theta:=n_{\mu}Z^{\mu}=\alpha Z^0$, the $\widetilde{\Gamma}^i:=\widetilde{\gamma}^{jk}{\widetilde{\Gamma}^i}_{jk} = \widetilde{\gamma}^{ij}\widetilde{\gamma}^{kl}\partial_{l}\widetilde{\gamma}_{jk}$ and $\hat{\Gamma}:=\widetilde{\Gamma}^i + 2\widetilde{\gamma}^{ij}Z_j$, constants $\kappa_1$ and $\kappa_2$ are related to the constraint damping terms, the $\kappa_3$ is the additional constant for further adjustments, the The three-dimensional Ricci tensor ${^{(3)})R}_{ij}$ is split into conformal part $\widetilde{R_{ij}^{\phi}}$ and the $\widetilde{R_{ij}}$ that contains the derivatives of the conformal metric

\begin{align}
    \widetilde{R_{ij}} &= -\frac{1}{2}\widetilde{\gamma}^{lm}\partial_{l}\partial_{m}\widetilde{\gamma}_{ij} + \widetilde{\gamma}_{k(i}\partial_{j)}\widetilde{\Gamma}_{(ij)k} + \widetilde{\gamma}^{lm}\big[2\widetilde{\Gamma}^{k}_{l(i}\widetilde{\Gamma}_{j)km} + \widetilde{\Gamma}^{k}_{im}\widetilde{\Gamma}_{kjl}\big] \\
    \widetilde{R_{ij}}^{\phi} &= \frac{1}{\phi^2}\big[\phi\big(\widetilde{\nabla}_{i}\widetilde{\nabla}_{j}\phi + \widetilde{\gamma}_{ij}\widetilde{\nabla}^{l}\phi\widetilde{\nabla}_{l}\phi\big) - 2\widetilde{\gamma}_{ij}\widetilde{\nabla}^{l}\phi\widetilde{\nabla}_{l}\phi\big]
\end{align}

And as one sees, the ecolution of $Z_i$ is now included in $\hat{\Gamma}$. 
\todo{understand the conformal stuff and add some steps to show how the ccz4 was made}

\subsubsection{Gauge conditions}

During the discussion of the original ADM system, the choice of the lapse function, \textit{i.e} slicing condition, and shift vector \textit{i.e} spatial gauge condition was left open. The right choice however, is crutual for the stable evolution and in itself presents a broad and rapidly evolving subject. Here we are going to discuss only the gauge that is relevant for our work. 

\paragraph{Slicing condition} One of the widely used conditions is so called 'maximal slicing' that sets $K=0$, which in turn results in the equation

\begin{equation}
    D^{i}D_{i}\alpha = \alpha\big[K_{ij}K^{ij} + 4\pi(e+S)\big].
\end{equation}

This conditions has an advantage of being \textit{singularity-avoiding}. For example, it was shown that in the case of Schwarzschild black hole, the $\alpha$ goes to zero at a finite distance from singularity \cite{Geyer:1995}. However implementation of this condition in from of a elliptic equations is computationally expensive.   \\
A class of slicing conditions in form of hyperbolic equations that are more favorable from numerical standpoint and that reproduces the desired behavior of the maximal scicing was proposed in \cite{Bona:1994dr}. It is read 

\begin{equation}
(\partial_t - \beta^i\partial_i)\alpha = \alpha^2 f(\alpha)K
\label{eq:theory:gauge_onepluslog}
\end{equation}

which in CCZ4 reads 

\begin{equation}
    (\partial_t - \beta^i \partial_i )\alpha = \alpha^2 f(\alpha)(K-2\Theta)
\end{equation}

where $f(\alpha)$ is a positive function. For many numerical applications, including those that are discussed in this work, the "1 + log" slicing is adopted, the $\beta_i=0$. Then, integrating equation (\ref{eq:theory:gauge_onepluslog}) yields 

\begin{equation}
    \alpha = 1 + \log\gamma
\end{equation}

This condition is numerically more favorable and as $f\rightarrow\infty$ in the vicinity of a singularity, allows to treat black holes well like maximal slicing \cite{Baumgarte:2002jm}.

\todo{add/modify some text.}

\paragraph{Spatial gauge conditions}

The requirements for the gauge are similar as in the case of the $\alpha$, namely hyperbolicity and minimization of numerical distortions for more stable evolution.  

One of the widely used shift conditions is so called \textit{Gamma driver} condition \cite{Alcubierre:2002kk}, 

\begin{align}
    \partial_t\beta^i &= \frac{3}{4}\alpha B^i, \\
    \partial_t B^i &= \partial_t\widetilde{\Gamma}^i - \eta B^i,
\end{align}

where $\eta$ is a dumping coefficient. \\

This gauge condition tries to decrease the coordinate stretching that occur in the vicinity of a singularity. It was shown to be effective in numerical applications, in particular for a single moving black hole. However it has a zero-speed mode, that can amplify the numerical errors and destabilize the system \cite{vanMeter:2006vi}.

A modified \textit{Gamma driver}, gauge that does not have zero or small speed modes:

\begin{align}
    (\partial_t - \beta^j\partial_j)\beta^i &= \frac{3}{4}B^i \\
    (\partial_t - \beta^j\partial_j)B^i &= (\partial_t - \beta^j\partial_j)\widetilde{\Gamma}^i-\eta\beta^i,
\end{align}

was proposed by \cite{vanMeter:2006vi} and was applied to study binary black holes by \cite{Campanelli:2005dd}.

\subsection{The Equations of General-Relativistic Hydrodynamics}

In this section we discuss the equations of general relativistic hydrodynamics. We consider the fluid on a Lorentzian manifold and how its flow affects the spacetime. \\ 

The topics that we are going to touch are:
\begin{itemize}
    \item fluid kinematics,
    \item equations of motion for perfect fluids (assuming that there is no thermal conduction or viscosity)
    \item the “Valencia formulation” of the hydrodynamic equations.
\end{itemize}

We note that the following description is very brief and is based on the following works: \cite{Misner:1973},\cite{Schutz:2009a},\cite{Gourgoulhon:2006bn},\cite{Andersson:2006nr},\cite{Rezzolla:2013} to which we refer the reader for more details.  

\subsubsection{Kinematics of a Relativistic Fluid}

In Newtonian physics, a fluid is an "entity" whose dynamics is described by flows of quantities such as energy density, mass, momentum density. However, in general and special relativity, the these quantities are not well defined and depend on the observer. In other words, different observers perceive the the same fluid being in different thermodynamic state. Hence, a description of the fluid dynamics in relativity requires a new formulation, a formulation in which a fluid is not represented by a scalar and vector fields, that are observer-dependent, but implicitly by a "flow" in spacetime. These are \textit{flux-conservative formulations} of hydrodynamics.

Consider the classical mass density, a scalar $\rho$, usually defined as total umber of particles $N$ of rest-mass $m$ in the volume $V$. Then, the total mass is given by

\begin{equation}
    \int_V \rho \text{d}^3x = m\int_V n \text{d}^3 x = mN.
\end{equation}

However, while the number of particles $N$ would be the same regardless of the observer, the $\text{d}^3x$ would be measured differently by observers moving in relation to each other. Hence, the $n$ would differ. One of the solutions is to chose a frame of reference, say comoving with the fluid and define the $\rho$ there. However, this would hinder our ability to generalize to other reference frames.\\ A better soution is to construct a \textit{covariant description in terms of invariant quantities}. 
 
We start by defining the flow of the fluid density in space-time, the 3 pseudo-form $\boldsymbol{\rho}$ that on any three dimensional submanifold describes the flow of mass transverse to the submanifold as

\begin{equation}
    \int_{\Sigma} \boldsymbol{\rho},
\end{equation}

where $\Sigma$ be a spacelike hypersurface,  $\vec{n}$ -- the future-oriented normal vector. This is the density measured by an observer with 4-velocity $\vec{n}$. 

To define a mass flow measured by an Eulerian observer across any spacelike surface $\Omega\subset\Sigma$, we need to construct a two-form $\boldsymbol{\rho}(\vec{n}, \cdot, \cdot)$ given by the interior product between the 3 pseudo-form $\boldsymbol{\rho}$ and $\vec{n}$. Then the mass flow is 

\begin{equation}
\int_{\Omega} i_{\vec{n}}\boldsymbol{\rho}.
\end{equation}

The conservation of the number of particles of the fluid is expressed by the vanishing exterior product of the density form, i.e. $\text{d}\boldsymbol{\rho}=0$, or in an integral form 

\begin{equation}
\int_{\partial\Omega} \boldsymbol{\rho} = \int_{\Omega}\text{d}\boldsymbol{\rho} = 0,
\end{equation}

that reads as the following: the net flow across any sufficiently regular surface $\partial\Omega$ enclosing a four-dimensional open set $\Omega\subset\mathcal{M}$ is zero.

Next we define a flux. First, let us reintroduce the volume pseudo-form

\begin{equation}
\text{Vol}_x ^4 = \sqrt{-g}dx^0 \wedge dx^1 \wedge dx^2 \wedge dx^3,
\end{equation}

where $g$ is the determinant of the spacetime metric. \\
On a on the submanifold $\Sigma$, the intrinsic volume then would be defined as 

\begin{equation}
\text{Vol}_x ^3 = i_{\vec{n}} \text{Vol}_x ^4.
\end{equation}

A flux of a vector field can be described by a three-form, for which on a pseudo-Riemannian manifold there exist a vector field associated with it.

A vector field associated with density is called \textit{rest-mass density four-vector} and is denoted by $\vec{j}$.

It is constructed from the one-form by rasing indexes, $\vec{j} = {^{\#}\underline{j}}$. The one-form $\underline{j}$ is obtained as $\underline{j}\star\boldsymbol{\rho}$, where $\star$ is the Hodge dual operator (see \textit{e.g.,} \cite{Frankel:1982dva}). 

Then if the $\boldsymbol{\rho} = i_{\vec{j}}\text{Vol}_x ^4$ the flux of $\vec{j}$ can be shown as 

\begin{equation}
\int_{\Sigma} \boldsymbol{\rho} = - \int_{\Sigma}\vec{j}\cdot\vec{n}\text{Vol}_x ^3,
\end{equation}

where $\vec{n}$ is the future-oriented unit-timelike normal to $\Sigma$.


\textcolor{gray}{
[Direct copy... maybe not needed] More generally the flux associated with a flow defined by a vector field, $\vec{X}$, across a hypersurface, $\Sigma$, transverse to it and with normal $\vec{\nu}$ (with appropriate sign depending on the signature of the metric and on $\Sigma$), is given 
\begin{equation}
\int_{\Sigma} \star\underline{X} = \int_{\Sigma}i_{\vec{X}}\text{Vold}^n = \int_{\sigma}i_{\vec{X}}\big[\underline{\nu}\wedge\text{Vol}^{n-1}\big] = \int_{\Sigma}\vec{X}\cdot\vec{\nu}\text{Vol}^{n-1}
\end{equation}
}

We note that $\vec{j}$ is time like (or null). It is given by the the flux of particles across any future-oriented spacelike hypersurface is positive (or zero). If $\vec{j}$ is timelike, there exists a unique decomposition 

\begin{equation}
\vec{j} = \rho \vec{u},
\label{eq:theory:defofjandu}
\end{equation}
where the scalar $\rho$ can be seen as density in the comoving frame and unit-timelike vector $\vec{u}$ as a fluid four-velocity.\\

The divergence of vecotor $j$ then gives a familiar mass conservation expression

\begin{equation}
0 = \nabla_{\mu}j^{\mu} = \frac{1}{\sqrt{-g}}\partial_{\mu}[\sqrt{-g}\rho u^{\mu}].
\label{eq:theory:nablamu_jmu}
\end{equation}

\textcolor{gray}{Similarly energy and momentum of a fluid can be defined, using the Cartan formalism... but this is a PAIN! and is done to show that div(T)=0 is not really energy/momentum conservation...}

Next, let us introduce the mixed tensor $\boldsymbol{T}$. Since the three-forms are equivalent to vectors, we can define a flow of the $\nu$ momentum across the volume element orthogonal to $dx^{\mu}$ as 

\begin{equation}
{T^{\mu}}_{\nu}=\boldsymbol{T}(dx^{\mu},\partial_{\nu}).
\end{equation}

${T^{\mu}}_{\nu}$ is the stress energy tensor that was already introduced earlier \ref{eq:theory:action1}. 

Note, that if the Einstein equation are satisfied the Bianchi identities dictate that the $\nabla_{\mu}{T^{\mu}}_{\nu}$ must vanish as

\begin{equation}
\nabla_{\mu}{T^{\mu}}_{\nu} = 0= \frac{1}{\sqrt{-g}}\partial_{\mu}(\sqrt{-g}{T^{\mu}}_{\nu}) - {\Gamma^{\alpha}}_{\mu\nu}{T^{\mu}}_{\alpha}.
\label{eq:theory:nablamu_tmunu}
\end{equation}

However, this statement does not imply the conservation of the energy and momentum of the fluid in a general sense. The conservation of the $\nu$-momentum requires $\vec{\partial}_{\nu}$ to be a Killing vector.


\todo{define somewhere an eulerian observer}

\subsubsection{Dynamics of a Relativistic Fluid}

In the previous subsection we have introduced the fluid kinematic, and defined the important quantities such as mass, energy and momentum and their "conservation" in \ref{eq:theory:nablamu_jmu} and \ref{eq:theory:nablamu_tmunu}.

In this thesis we consider only the \textit{perfect fluid}, meaning that in the co-moving frame, there is not heat conduction and there is no viscosity. \todo{actually we do have a viscous part -- you have to add this...}. The former criterion implies that the fluid is in local thermodynamic equilibrium (LTE). The latter however requires more explanation. There is still no consensus on the correct mathematical formulation, especially with respect to the numerical applications, of the viscous and/or thermally conducting fluids in general-relativity (see e.g., \cite{Andersson:2006nr} and references therein). \textcolor{blue}{however in recent youers there have been some progress GRELS models and David's implementation I must add!}. \\

Consider a stress-energy tensor of a perfect fluid in the comoving frame with the fluid. To construct it, we return to the fluid's four velocity $\vec{u}$ from (\ref{eq:theory:defofjandu}). If $e_{i}$ is the basis vector, the scalar product $\vec{u}\cdot\vec{e}_i=0$ and $\vec{e_i}\cdot\vec{e}_k = \delta_{ik}$. then the orthonormal tetrad $\{\vec{u},\vec{e}\}$ is comoving with the fluid, and the $\{\underline{u},\underline{e}^i\}$ is the dual basis. \\
Tensor $\boldsymbol{T}$ is the stress-energy tensor with the following components: 

\begin{itemize}
    \item $\boldsymbol{T}(\underline{u}, \vec{u})$ energy-density in the rest-frame of the fluid, the scalar $e$
    \item $\boldsymbol{T}(\underline{u}, \vec{e}_i) = 0$ represent the energy flowing transverse to the four-velocity, which we set to $0$ in the absence of the heat-conduction.
    \item $\boldsymbol{T}(\underline{e}^i, \vec{e}_k) = 0$ represent the $k$ component of the force exchanged across the surface element orthogonal to $\underline{e}_i$.
\end{itemize}

Taking into account that the $\boldsymbol{T}$ must be invariant with respect to the rotations of the $\{\vec{e}_i\}$ and that the viscosity is not included, force exchange can be effectively desibed by a scalar $p$, that we call pressure as

\begin{equation}
    \boldsymbol{T}(\underline{e}^i,\vec{e}_k) = p {\delta^i}_k,
\end{equation}

Combining the aforementioned description of the components of $\boldsymbol{T}$ we get

\begin{equation}
\boldsymbol{T} = (e + p)\vec{u}\otimes \underline{u} + p\boldsymbol{\delta}.
\end{equation}

Defining the enthalpy of the fluid as $h = 1 + \epsilon = p/\rho$, where $\epsilon$ is the specific internal energy, we rewrite $\boldsymbol{T}$ as 

\begin{equation}
\boldsymbol{T} = \rho h \vec{u}\otimes\underline{u} + p\boldsymbol{\delta}
\label{eq:theory:stressenergytensor}
\end{equation}

In addition to the fluid kinematics (eqs. \ref{eq:theory:nablamu_jmu} and \ref{eq:theory:nablamu_tmunu}) and the description of motion (eq. \ref{eq:theory:stressenergytensor}), the relation between the pressure, internal energy and density is needed to fully describe the dynamics of the fluid. This relation is usually called the equation of state.

The commonly adopted equations (EoS) of state are the the ideal-gas, or gamma-law EoS $\rho = (\Gamma-1)\rho\epsilon$, where $\Gamma$ is the polytropic index of the gas, the polytropic EoS $p = K\rho^{\Gamma}$ and the microphysical equation of state \todo{that you need to discuss more..., as we use only it.}

Combined with an EoS, equations \ref{eq:theory:adm}, \ref{eq:theory:nablamu_jmu}, \ref{eq:theory:nablamu_tmunu} and \ref{eq:theory:stressenergytensor} form a hyperbolic
system of equations that can be evolved, once initial data is prescribed. The complete evolution of spacetime and the dynamics of the matter requires initial data to be set on the Cauchy surface.

\subsubsection{Conservative Formulations}

In the pioneering works of May and White \cite{May:1966} and Wilson \cite{Wilson:1972} the equations of general relativistic hydrodynamics were solved using the finite-difference (FD) schemes after casting them a from of non-linear advection-like equations. To avoid excessive oscillations at shocks a combination of upwinding and artificial-viscosity methods was employed. This however led to severl limitations, such as difficulty with tunning the artificial viscosity to still allow shocks to develop, and the limit on a flows being only mildly relativistic \cite{Font:2008fka}. \\
A next big advancement in the numerical relativistic hydrodynamics was made after the non-conservative nature of the Wilson’s approach was pointed out \cite{Marti:1991wi} and the conservation formulation was developed. 

An important example of the conservation formulation that is adopted to $3 + 1$ formalism is the "Valencia formulation" \cite{Banyuls:1997} that can be represented as following

\begin{equation}
    \frac{\partial\boldsymbol{F}^{0}(\boldsymbol{u})}{\partial t} + \frac{\partial\boldsymbol{F}^{i}(\boldsymbol{u})}{\partial x^{i}} = \boldsymbol{S}(\boldsymbol{u})
    \label{eq:theory:valencia_formalism}
\end{equation}

where $u$ is a “vector” of \textit{primitive quantities}, such as the rest-mass density or the specific internal energy, $\boldsymbol{F}^0$ is a “vector” of \textit{conserved quantities} and $\boldsymbol{F}^i$ and $\boldsymbol{S}$ are their fluxes and sources respectively. \\

This formulation allowed to study ultra-relativistic flows and resolve shocks without spurious oscillations and without need for artificial viscosity.

It was shown to be especially well suited for use with numerical methods that take into account the conservation laws. These are the finite-volume (FV) FD high-resolution shock capturing (HRSC) methods, that will be discussed in Chapter \ref{chapter:num_methods} \\

Many recent advancements in numerical relativistic hydrodynamics and magnetohydrodynamics (MHD) have relied on these methods (\textit{e.g.,} \cite{Giacomazzo:2010bx} [274]\cite{Rezzolla:2011da} and references therein \todo{add recolla/bernuzzi/radice/shibata}).

There are other conservative formulations and methods (see \textit{e.g.,} \cite{Papadopoulos:1999kt}). However, we will limit our focus to the "Valencia formulation". \\

To begin we split the four-velocity $\vec{u}$ into the component parallel to the normal vector $\vec{n}$ and a purely spatial component as

\begin{equation}
    \vec{u} = (-\vec{u} \cdot \vec{n})(\vec{n} + \vec{\upsilon}),
\end{equation}

where naturally the Lorentz factor, measured by theEulerian observer $W = (-\vec{u}\cdot\vec{n})$ emerges, and the $\upsilon$ is the fluid three-velocity measured by the Eulerian observer, 

\begin{equation}
    \vec{\upsilon} = \frac{\vec{u}}{W} -\vec{n},
\end{equation}

components of which are

\begin{equation}
    \upsilon^i = \frac{u^i}{W}+ \frac{\beta^i}{\alpha}, \hspace{10mm} \upsilon_i= \frac{u_{i}}{W}.
\end{equation}

Divergence of the rest-mass density four-vector $j$, (\ref{eq:theory:nablamu_jmu}) can easily be cast as 

\begin{eqnarray}
    0 = \nabla_{\mu}j^{\mu} = \frac{1}{\sqrt{-g}}\partial_{t}[\sqrt{\gamma}\rho W] + \frac{1}{\sqrt{-g}}\partial_{i}[\sqrt{\gamma}\rho(\alpha \upsilon^{i} - \beta^{i})]
\end{eqnarray}

where $D=\rho W = -\vec{j}\cdot \vec{n}$ is the conserved density.

To write the energy and momentum equations we note that for any vector field $\vec{p} $ \cite{Rezzolla:2013}, 

\begin{equation}
    \nabla_{\mu}[{T^{\mu}}_{\nu}p^{\nu}].
\end{equation}

To obtain the Valencia formulation we set $\vec{p}$ to have zeroth component $-\vec{n}$ and spatial components $\vec{\partial}_i$. Then the

\begin{itemize}
    \item ${T^0}_{\nu}p^{\nu}$ represent the conserved quantities,
    \item ${T^i}_{\nu}p^{\nu}$ are associated fluxes,
    \item ${T^{\mu}}_{\nu}p^{\nu}$ are sources
\end{itemize}

with the former being 

\begin{equation}
    S_{i} = \alpha {T^0}_{\nu}(\partial_i)^{\nu}=-\boldsymbol{T}(\vec{n},\vec{\partial}_i), \hspace{10mm} E = -\alpha{T^0}_{\nu}n^{\nu} = \boldsymbol{T}(\vec{n},\vec{n})
\end{equation}

for numerical reasons we will replace the total internal energy density $E$ with $\tau = E-D$, where $D$ is the rest mass density. This is done to avid errors emerging due to $E$ being much smaller then $D$. 

Now we can combine the obtained expressions for the conserved quantities, associated fluxes and sources with eq. (\ref{eq:theory:valencia_formalism}) and obtain

\begin{equation}
    \frac{1}{\sqrt{-g}}\Big[\frac{\partial\sqrt{\gamma}\boldsymbol{F}^{0}(\boldsymbol{u})}{\partial t} + \frac{\partial\sqrt{-g}\boldsymbol{F}^{i}(\boldsymbol{u})}{\partial x^i}\Big] = \boldsymbol{S}(\boldsymbol{u}),
\end{equation}

where primitive quantities being

\begin{equation}
    \boldsymbol{u} = [\rho,\: \upsilon_i,\: \epsilon],
\end{equation}

conserved quantities: 

\begin{equation}
    \boldsymbol{F}^0(\boldsymbol{u}) = [D,\: S_j,\: \tau] = [\rho W,\: \rho h W^2 \upsilon_j,\: \rho h W^2 - p - \rho W],
\end{equation}

associated fluxes

\begin{equation}
    \boldsymbol{F}^i(\boldsymbol{u})=\Bigg[D\Big(\upsilon^{i}-\frac{\beta^i}{\alpha}\Big),\: S_{j}\Big(\upsilon^{i}-\frac{\beta^i}{\alpha}\Big)+p{\delta^i}_j ,\: \tau\Big(\upsilon^{i}-\frac{\beta^i}{\alpha}+p\upsilon^i\Big)\Bigg]
\end{equation}

and sources 

\begin{equation}
    \boldsymbol{S}(\boldsymbol{u}) = \Bigg[0,\: T^{\mu\nu}\Big(\frac{\partial g_{\nu j}}{\partial x^{\mu}} - \Gamma^{\delta}_{\nu\mu}g_{\delta j}\Big),\: \alpha\Big(T^{\mu 0}\frac{\partial\log\alpha}{\partial x^{\mu}}-T^{\mu\nu}\Gamma^{0}_{\nu\mu}\Big)\Bigg]^T
\end{equation}

The from of the obtained general relativistic hydrodynamics equations resemble the one of the Newtonian gas dynamics. If the latter is adopted for numerical solutions. \\
There are however several complications. In particular there is no explicit inverse relation between the primitive quantities and the conserved ones. Thus one has to resort to the root-finding algorithms to reconstruct them (More on this in later chapters). In addition, it was pointed out that the $W$ cpuples the equation for the momenta in different direction \cite{Pons:2000,Rezzolla:2002ra,Rezzolla:2002cc,Aloy:2006rd}. This leads to the fact that the dynamics of the shock wave can be affected by the non-zero tangential velocity. Hence, the increased complexity if he problem of GR hydrodynamics \cite{Mignone:2005ns,Zhang:2005qy}.


\subsection{The General-Relativistic Boltzmann Equation}

In special relativity the Boltzmann equation was expressed by Synge \cite{Synge:1957}. Later Chernikov \cite{Chernikov:1962} and Tauber and Weinberg \cite{Tauber:1961} proposed its extension to the general relativity. \\
The list of applications of the Boltzmann equation was limited to the relativistic gas at first \cite{Israel:1963}. Later the list was supplemented by transient relativistic thermodynamics \cite{Israel:1979wp}, radiative transfer \cite{Lindquist:1966}, core-collapse supernovae \cite{Bruenn:1985} and others (see \textit{e.g.}, \cite{Cercignani:2002} and references therein). \\

Different formulations of the general relativistic Boltzmann equation exists in the literature. Lindquist \cite{Lindquist:1966} and Ehlers \cite{Ehlers:1971} proposed a geometrical interpretation. Later, a formulation based on Riemannian structure of tangent bundles was proposed by Sasaki \cite{Sasaki:1958,Sasaki:1962}. In addition, Debbasch and van Leuuwen \cite{Debbasch:2009a,Debbasch:2009b} recently provided a detailed derivation, albeit strongly focused on the algebraic aspects while eluding simple geometrical interpretation. \\
Here we recall the detailed derivation of the general relativistic Boltzmann equation, using modern differential geometry notation by Radice. 

\textcolor{red}{This.Is.Tough. Pure math. Copied from David + his sources.}

\subsubsection{The geometry of the tangent bundle}

Let the $\mathcal{M}$ be $4$ dimensional differential manifold such that $(\mathcal{M},\: g_{\alpha\beta})$ form the $C^2$ spacetime. The set of tangent vectors of $\mathcal{M}$ constitutes \textit{tangent bundle} of $\mathcal{M}$, the we denote as $T\mathcal{M}$. The set of all unit vectors of $\mathcal{M}$ constitute the \textit{subbundle} of $T\mathcal{M}$. \\
\textcolor{gray}{incompressible vector field}\\
\textit{Every Killing vector field of $\mathcal{M}$ is in incompressible vector field}

\paragraph{Extended transformation and extended tensors}

Let the $T\mathcal{M}$ be the set of all the tangent vectors of $\mathcal{M}$. The $T\mathcal{M}$ has a natural topology, bundle structure with $\mathcal{M}$ and the base - linear vector space $E^i$. We call $T\mathcal{M}$ the \textit{tangent bundle} of $\mathcal{M}$. Natural projection, or a projection map $\pi:\: T\mathcal{M}\rightarrow\mathcal{M}$.  \\

Let $U$ be a coordinate neighborhood, or a coordinate patch of $\mathcal{M}$ with $n$ variables $x^{\alpha}$ as coordinates. Then, every tangent vector of $\mathcal{M}$ at a point $p\in U$ with $2n$ variables $(x^i,\upsilon^{\alpha})$. Here $x^{\alpha}$ are coordinates of $p$ with respect to the coordinate patch ${x^{\alpha}}$ and $\upsilon^{\alpha}$ are components of a tangent vector in the natural frame that constitutes by the vectors $\partial/\partial x^4$ at $q$. Thus, the vector $\vec{p}$ at $q$ can be written as:

\begin{equation}
    \vec{p} = p^{\alpha}\frac{\partial}{\partial^{\alpha}}
\end{equation}

and its dual as 

\begin{equation}
    \underline{p} = p_{\alpha}dx^{\alpha}:=g_{\alpha\beta}p^{\beta}dx^{\alpha}
\end{equation}

In addition we introduce a coordiante patch $TU$, $\{z^A\}$, where $A$ runs from $0$ to $7$ of $T\mathcal{M}$ as 

\begin{equation}
    z^{\alpha} = z^{\alpha}, \hspace{10mm} z^{\alpha+4} = p^{\alpha}.
\end{equation}

Now, let the $U(x^{\alpha})$ and $\hat{U}(\hat{x}^{\alpha})$ be the two coordinate patches of $\mathcal{M}$ such that $U\cap\hat{U}$ is not empty. Then the intersection of the coordinate patches is also not empty. 
for every coordinate transformation of $\mathcal{M}$, there is a corresponding matrix $\frac{\partial \hat{x}^{\alpha}}{\partial x^{\beta}}$.
The coordinate transformation is then

\begin{equation}
    \hat{x}^{\mu} = \hat{x}^{\mu}(x), \hspace{5mm} \hat{p}^{\mu} = \frac{\partial\hat{x}^{\mu}}{\partial x^{\nu}}p^{\nu}
\end{equation}

which denotes the extended transforation of the $\hat{x}^{\mu} = \hat{x}^{\mu}(x)$. \\


The corresponding Jacobian matrix is 
\renewcommand\arraystretch{1.6} %% it stretches the matrix
\begin{equation}
\frac{\partial\hat{z}^A}{\partial z^B} = 
    \begin{pmatrix}
    \frac{\partial\hat{x}^{\alpha}}{\partial x^{\beta}} & 0 \\
    \frac{\partial^2\hat{x}^{\alpha}}{\partial x^{\beta} \partial x^{\gamma}}p^{\gamma} & \frac{\partial\hat{x}^{\alpha}}{\partial x^{\beta}} 
    \end{pmatrix}
\end{equation}
\renewcommand\arraystretch{1.0}




\paragraph{Vectors on $T\mathcal{M}$}

As we will need to introduce connections on a tangent bundle, here we discuss the double tangent bundle, ot a second tangent bundle. Since $T\mathcal{M}$ is a vector bundle on its own right, its tangent bundle has the secondary vector bundle structure $TT\mathcal{M}$. Let point $b\in TU$ and $T_b T\mathcal{M}$ be the tangent space to $T\mathcal{M}$ at $b$. \\
Given a vector $\partial/\partial x^{\alpha}$ at a point $b$, it can be "pushed forward" to the point on the $TT\mathcal{M}$ by means of so called \textit{differential of} $\pi$, whitten as $\pi_*$ \cite{Frankel:2002}.
On a natural basis the push-forward acts as 

\begin{equation}
    \pi_*\Big[\frac{\partial}{\partial x^{\alpha}}\Big] = \frac{\partial}{\partial x^{\alpha}}, \hspace{5mm} \pi_* \Big[\frac{\partial}{\partial p^{\alpha}}\Big] = 0,
\end{equation}

and the pull back 

\begin{equation}
    \pi^* {\text d} x^{\alpha} = {\text d} x^{\alpha}.
\end{equation}

Consider a vector field $\vec{X} \ in TT\mathcal{M}$  in a vicinity of the point $b$, which is associated with the point $q$ of $\mathcal{M}$ and vector $\vec{x}\in T_{q}\mathcal{M}$. Let $b{\lambda}$ be the flow of $b$ generated by $\vec{X}$. The $b(\lambda)$ is associate with $q(\lambda)$, the one parameter family of points of $\mathcal{M}$. The $b(\lambda)$ is also associated with $\vec{x}(\lambda)$ the one parameter family of vectors on $T\mathcal{M}$.  \\

The vector field $\vec{X}$ is called \textit{vertical} if the $q(\lambda)\in\mathcal{M}$ are constant along the flow. Similarly, the vector field $\vec{X}$ is called \textit{horizontal} if $\vec{x}(\lambda)\in T_p \mathcal{M}$ is "constant" along the flow, meaning that $\vec{x}(\lambda)$ is just $\vec{x}$ that is \textit{parallel transported} to $q(\lambda)$. \\
As there is no unique way to perform a parallel transport, the linear connection $\nabla$ on $\mathcal{M}$ has to be chosen. This choice is akin choosing two vector spaces $\mathcal{O}_b$ and $\mathcal{V}_b$ of the horizontal and vectical vectors respectively at each point $b$ that the direct sum of these spaces yields

\begin{equation}
    \mathcal{O}_b\oplus \mathcal{V}_p = T_b T\mathcal{M}.
\end{equation}

Having the connection allows to prescribe a manner of lifting curves from the base manifold $T\mathcal{M}$ into the $T_b T\mathcal{M}$ \textcolor{red}{I need to fix this and understand}. A lift is the unique horizontal vector $\vec{X}\in T_bT\mathcal{M}$ whose projection is a vector $\vec{x}\in T_q\mathcal{M}$.\\ 
\textcolor{red}{fill it}
Let us now define a \textit{connection vector basis} adopted to the aforementioned split of $T_b T\mathcal{M}$ $\{\text{D}/\partial x^A \}:=\{\text{D}/\partial x^{\alpha}, \partial/\partial p^{\alpha} \}$ where 

\begin{equation}
    \frac{\text{D}}{den}{\partial x^{\alpha}} := \frac{\partial}{\partial x^{\alpha}} - {\Gamma^{\beta}}_{\alpha\gamma}p^{\gamma}\frac{\partial}{\partial p^{\beta}}.
\end{equation}

Similarly a connection can be build for differential forms. Using the pull-back $\pi^*$ the dual basis $\{ \text{D}z^{A} \}:=\{\text{d}x^{\alpha}, \text{D}p^{\alpha}\}$ that satisfies 

\begin{equation}
    \text{D} = \text{d} p ^{\alpha} + {Gamma^{\alpha}}_{\beta\gamma}p^{\gamma}\text{d}x^{\beta}.
\end{equation}

\paragraph{Metric on $T\mathcal{M}$}

Note that 

\begin{equation}
    \frac{\partial^2 \hat{x}^{\mu}}{\partial x^{\nu}\partial x^{\lambda}}p^{\lambda} = {\hat{\Gamma}^{\mu}}_{\delta\gamma}p^{\lambda}\frac{\partial\hat{x}^{\delta}}{\partial x^{\nu}}.
\end{equation}

Let us assume that for any point $b\in T\mathcal{M}$ there exist an open set $TU$, such that $b\in TU$ with a coordinate system on $TU$ that satisfies

\begin{equation}
    G_{AB} = (\boldsymbol{\eta}\otimes\boldsymbol{\eta})_{AB},
\end{equation}

where $\boldsymbol{\eta} = \text{diag}(-1, 1, 1, 1)$. \\
Let the $\hat{x}^A$ denote the generic coordinate system on $TU$. Then the metric in this coordinate system can be expressed as

\begin{align}
    \hat{G}_{\mu\nu} &= \frac{\partial \hat{x}^{\alpha}}{\partial x^{\mu}}\frac{\partial \hat{x}^{\beta}}{\partial x^{\nu}}\eta_{\alpha\beta} + \frac{\partial \hat{x}^{\alpha}}{\partial x^{\mu}}{\hat{\Gamma}^{\gamma}}_{\:\:\:\alpha\lambda}p^{\lambda}\frac{\partial \hat{x}^{\beta}}{\partial x^{\nu}}{\hat{\Gamma}^{\delta}}_{\:\:\:\beta\xi}p^{\xi}\eta_{\gamma\delta}; \\
    \hat{G}_{\mu\: \nu+4} &= \frac{\partial \hat{x}^{\alpha}}{\partial x^{\mu}}\frac{\partial \hat{x}^{\gamma}}{\partial x^{\nu}}{\hat{\Gamma}^{\beta}}_{\:\:\:\gamma\lambda}p^{\lambda}\eta_{\alpha\beta}; \\
    \hat{G}_{\mu+4 \: \nu+4} &= \frac{\partial \hat{x}^{\alpha}}{\partial x^{\mu}}\frac{\partial \hat{x}^{\beta}}{\partial x^{\nu}} \eta_{\alpha\beta}
\end{align}

and the line element 

\begin{align}
    dS^2 &= \hat{G}_{AB}d\hat{z}^A d\hat{z}^B = \hat{g}_{\mu\nu}\text{d}\hat{x}^{\mu}\text{d}\hat{x}^{\nu} + \hat{g}_{\mu\nu}[\text{d}p^{\mu} + {\hat{\Gamma}^{\mu}}_{\:\:\:\alpha\beta}p^{\beta}\text{d}x^{\alpha}] [\text{d}p^{\nu} + {\hat{\Gamma}^{\nu}}_{\:\:\:\alpha\beta}p^{\beta}\text{d}x^{\alpha}] \\
    &= \hat{g}_{\mu\nu}\text{d}\hat{x}^{\mu}\text{d}\hat{x}^{\nu} + \hat{g}_{\mu\nu}\text{D}\hat{x}^{\mu}\text{D}\hat{x}^{\nu}
\end{align}

It is possible to show that the determinant $|\text{det}\boldsymbol{G}| = g^{2}$ as the transformation from the natural frame to the connection frame is unimodular \cite{Lindquist:1966}. Thus the volume pseudo-form on $T\mathcal{M}$ is in the coordiante patch $TU$

\begin{align}
    \text{Vol}^8 &:= -g \text{d}x^{0} \wedge \text{d}x^{1} \wedge ... \wedge \text{d}p^{3} := - g\text{d}^{4}x \text{d}^{4}p, \\
    &:= -g \text{d}x^{0} \wedge \text{d}x^{0} \wedge ... \wedge \text{D}p^{3} :=-g \text{d}^{4}x\text{D}^4 p
\end{align}

\textcolor{red}{I kinda gave up here and just copied.}

%[130]\cite{Font:2008fka}
%[36]\cite{Banyuls:1997}
%[210]\cite{Marti:1991wi}
%[146]\cite{Giacomazzo:2010bx}
%[274]\cite{Rezzolla:2011da}
%[248]\cite{Papadopoulos:1999kt}
%[275]\cite{Rezzolla:2013}
%[257]\cite{Pons:2000}
%[272]\cite{Rezzolla:2002ra} 
%[273]\cite{Rezzolla:2002cc}
%[15]\cite{Aloy:2006rd}
%[224]\cite{Mignone:2005ns}
%[339]\cite{Zhang:2005qy}
%[311]\cite{Synge:1957}
%[315]\cite{Tauber:1961}
%[84]\cite{Chernikov:1962}
%[174]\cite{Israel:1963}
%[173]\cite{Israel:1979wp}
%[201]\cite{Lindquist:1966}
%[321]\cite{Thorne:1981}
%[67]\cite{Bruenn:1985}
%[79]\cite{Cercignani:2002}
%[108]\cite{Debbasch:2009a}
%[109]\cite{Debbasch:2009b}
%[124]\cite{Ehlers:1971}
%[284]\cite{Sasaki:1958}
%[285]\cite{Sasaki:1962}



\chapter{Numerical methods}
\label{chapter:num_methods}

%% --------------- 
%%
%% References
%%
%% ---------------

\newpage

\bibliography{references}

\end{document}